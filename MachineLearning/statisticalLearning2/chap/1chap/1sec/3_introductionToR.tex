\paragraph{Basic command}
R uses function to perform operations.\\
\begin{itemize}
  \item \emph{?funcname} will always cause R to open a new help file
    window
  \item \emph{ls()} allows us to look at a list of the objects such
    as data and functions that we have saved.
  \item \emph{rm()} can be used to delete any that we don't want.
  \item \emph{matrix()} can be used to create matrix of number. Example
  : \textit{matrix(c(1,2,3,4), 2, 2, byrow=TRUE)}
  \item \emph{rnorm()} function generates a vector of random normal
    variables. Example: \textit{rnorm(50, mean=50, sd=.1)}
  \item \emph{cor()} compute the correlation between 2 sets of numbers.
    Example: \textit{cor(x,y)}
  \item \emph{set.seed()} function takes an (arbitrary) integer 
    argument and allows to reproduce the exact same set of random
    numbers.
 \end{itemize}
 \paragraph{Graphics}
 \begin{itemize}
   \item \emph{plot()} function is the primary way to plot a data.
     Example: \textit{plot(x, y, xlab=``xName'', ylab=``yName'',
     main=``Plot of X vs Y'')}
   \item \emph{pdf()} is used to create pdf and \emph{dev.off()}
     indicates that we are done creating the plot.\\Example:\textit{
     pdf(``Figure.pdf'')\\plot(x, y, col=``green'')\\dev.off()}
   \item \emph{seq()} is used to create a sequence of numbers.
     Example: \textit{seq(-pi, pi, length=50)}
   \item \emph{contour()} function produces a contour plot in order to
     represent 3-dimensional data.\\Example: \textit{y=x\\
     f=outer(x,y,function(x,y)cos(y)/(1+x\^ 2))\\contour(x,y,f)\\
   contour(x,y,f,nlevels=45,add=T)\\fa=(f-t(f))/2\\contour(x,y,fa,
 nlevels=15)}
 \item \emph{image()} function works the same way as \emph{contour()},
   except that it produces a color-coded plot whose colors depend on
   the $z$ value.
 \item \emph{persp()} can be used to produce a 3-dimensional plot. The
   \emph{theta} and \emph{phi} control the angles at which the plot is
   viewed.\\\textit{image(x,y,fa)\\persp(x,y,fa)\\persp(x,y,fa,
   theta=30)\\persp(x,y,fa,theta=30,phi=20)}
 \end{itemize}
 \paragraph{Indexing Data}
 For a matrix of length $4\times 4$:\\
 $A[c(1,3),c(2,4)]$, $A[,1:2]$
 \paragraph{Loading Data}
 \begin{itemize}
   \item \emph{read.table()} is one of the primary ways to data
     importing.
   \item \emph{write.table()} to export data.
 \end{itemize}
 \paragraph{Additional Graphical and Numerical Summaries}
 \begin{itemize}
   \item \emph{attach()} function in order to make the variables in
     this data frame available by name.
   \item \emph{as.foctor()} function converts quantitative variables
     into qualitative variables
   \item \emph{hist()} function can be used to plot a histogram
   \item \emph{identify()} provides interactive method for identifying
     the value for a particular variable for point on a plot.
   \item \emph{summary()} produces a numerical summary of each variable
     in a particular data set.
   \item \emph{savehistory()} allows to save a record of all of the
     commands that we typed in the most recent session.
   \item \emph{loadhistory()} allows to load history of previous time.
 \end{itemize}
