\paragraph{Conceptual}
\begin{enumerate}
 \item
   \begin{itemize}
     \item[(a)] The number of predictors being small, and sample size
being extremely high, and knowing the expected test MSE $\E{\left( 
y_{0}-\widehat{f}\left( x_{0} \right) \right)^{2}}=\V{\widehat{f}\left(
x_{0} \right)}+\left[ Biais\left( \widehat{f}\left( x_{0} \right)
\right) \right]^{2}+\V{\epsilon}$ which refers to the average test MSE.
\\A great number of observations results an increase of the variance.
Then to compensate this increasing we should use a unflexible method,
because unflexible methods decrease the variance.\\Then we avoid the
overfitting risk.
\item[(b)] It is the opposite case, now the low amount number of
observations results to a high probability to make mistakes on our
approximate function $f$. This means that biais increases, therefore
we should use a flexible methods therewith to componsate the biais
increasing. However the variance will increase but it's not a problem
because the number of observation is small.
\item[(c)] If the relationship between predictors and response is
highly non-linear we need a method with a high level of flexibility.
Then the challange is to approximate the real life problem therefore we
must reduce the biais.
\item[(d)] We have no control on the $\epsilon$ so I do not know.
   \end{itemize}
 \item
   \begin{itemize}
     \item[(a)] Our predictors are almost all quantitative, there is
just the industry predictor which is qualitative however we can assign
a number in function of the industry.\\Then it looks like a regression
problem. We are most interested in inference.$n=500,p=4$
\item[(b)] Our predictors are all quantitative but response is
qualitative.\\Then it looks like a classification problem. We are most
interested in prediction $n=20,p=14$
\item[(b)] Our predictors are all quantitative and the response 
searched is quantitative.\\Then it looks like a
regression problem. We are most interested in prediction $n=?,p=4$
   \end{itemize}
 \end{enumerate}
