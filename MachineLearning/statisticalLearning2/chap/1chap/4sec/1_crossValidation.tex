\paragraph{The Validation Set Approach}
It involves randomly \tB{dividing the available set of observations 
into 2 parts}:
\begin{enumerate}
	\item \emph{\tR{training set}}
	\item \emph{\tR{validation set}}
\end{enumerate}
Pitfall: 
\begin{itemize}
	\item the \sB{validation estimate of the test error rate can be highly variable}.
	\item \sB{Only a subset of the observations are used to fit}
		the model.
\end{itemize}

\begin{python}
import sklearn
from sklearn.model_selectioon import LeaveOneOut,\
KFold
\end{python}

\paragraph{Leave-One-Out Cross-Validation}
Instead of creating 2 subsets of comparable size, \tR{a single
observations $(x_{1},y_{1})$ is used for the validation} set and the
remaining observations $\left\{ x_{i},y_{i} \right\}_{2\leq i\leq n}$ 
make up the training set.\\
The LOOCV estimate for the test MSE is the average of these $n$ test
error estimates:

\begin{center}
\encB{$CV_{n}=\dfrac{1}{n}\su{{i=1}}{n}MSE_{i}$}
\end{center}

LOOCV approach tends not to overestimate the test error rate as much
as the validation set approach does.\\
With least squares linear or polynomial regression, \sB{an amazing
shortcut makes the cost of LOOCV the same as that of a single model 
fit}!
\begin{center}
\encV{$
CV_{n}=\dfrac{1}{n}\su{{i=1}}{n}\left( \dfrac{y_{i}-\overline{y}_{i}}{1-h_{i}} \right)^{2}
$}
\end{center}
$h_{i}$ is the leverage (that reflects the amount that an observations
influence its own fit).\\
LOOCV is a very general method, and can be used with any kind of
predictive modeling.

\begin{python}
X = [1, 2, 3, 4]
loo = LeaveOneOut()
for train, test in loo.split(X):
    print(train, test)
\end{python}

\paragraph{k-Fold Cross-Validation}
This approach \tB{involves randomly dividing the set of observations 
into $k$ groups (folds), of approximately equal size}. The \tB{first
fold is treated as a validation set}, and the method is fit on the 
remaining $k-1$ folds.
\begin{center}
\encV{$
CV_{n}=\dfrac{1}{k}\su{{i=1}}{k}MSE_{i}
$}
\end{center}

When we examine real data, \tB{we do not know the True test MSE, and
so it is difficult to determine the accuracy of the cross-validation estimate}.\\

When we perform cross-validation, our goal might to be \tB{determine
how well a given statistical learning procedure can be expected to 
perform on independent data}; \tB{in this case, the actual estimate of 
the test MSE is of interest}. But \tB{other times we are interested 
only in the location of the \emph{minimum point in the estimated test}
MSE curve}.\\

Despite the fact that they sometimes underestimate the true test MSE,
\tR{all of the CV curves come close to identifying, the flexibility
level corresponding to the smallest test MSE}.
\begin{python}
X = [1, 2, 3, 4]
kf = KFold(n_splits=2)
for train, test in kf.split(X):
    print(train, test)
\end{python}

\paragraph{Biais-Variance Trade-Off for k-Fold Cross-Validation} 
\tB{Validation set approach can lead to overestimates of the test error
rate}, since the training set used to fit the response contains only
half the observations of the entire data set. Likewise \tB{LOOCV will
give approximately unbiased estimates of the test error}.\\

And \tB{performing $k$-fold CV for, say, $k=5$ or $k=10$ will lead to
an intermediate level of biais}, since each training set contains 
$\frac{(k-1)n}{k}$ observations fewer than in the LOOCV approach, but
substantially more than in the validation set approach.\\


It turns out that \tB{LOOCV has higher variance than does $k$-fold CV
with k < n}. Why ?
We are averaging the outputs of $n$ fitted models, each of which is 
trained on an almost identical set of observations; therefore, these
outputs are highly (positively) correlated with each other.
In contrast, when we perform $k$-fold CV with $k<n$ we are averaging
the outputs of $k$ fitted models that are somewhat less correlated with
each other, since the overlap between the training sets in each model 
is smaller.\\

Since the mean of many highly correlated quantities has high variance 
than does the mean of many quantities that are not as highly 
correlated, the test error estimate resulting from LOOCV tends to have
higher variance than does the test error estimate resulting from
$k$-fold CV.

\paragraph{Cross-Validation on Classification Problems} Instead to use
the MSE to quantify test error, we rather use the number of 
misclassified observations.
\begin{center}
\encV{$
CV_{n}=\dfrac{1}{n}\su{{i=1}}{n}Err_{i}
$}
\end{center}
where $Err_{i}=I(y_{i}\neq\hat{y}_{i})$
