\begin{enumerate}
	\item \emph{Is there a relationship between advertising sales
		and budget?}
		\begin{itemize}
			\item fitting a multiple regression model
			\item $F-statistic$ can be used to determine if
				or not we should reject the null 
				hypothesis, in this case a low 
				$p-value$ corresponding to the 
				$F-statistic$ indicates a clear
				evindence of a relationship.
		\end{itemize}
	\item \emph{How strong is the relationship?}
		\begin{itemize}
			\item RSE
			\item $R^{2}$
		\end{itemize}
	\item \emph{Which media contribute to sales?}
		Examine the $p-values$ associated with each 
		predictor's $t-statistic$
	\item \emph{How large is the effect of each medium on sales?}
		If confidence intervals are narrow and far from $0$ 
		then it provides evindence that exists relationship 
		between predictors.\\The collinearity can cause a wide
		confidence interval, so we could use \emph{VIF} to know
		if there is collinearity.
	\item \emph{How accurately can we predict future sales?}
		\begin{itemize}
			\item To predict an individual response $Y=
				f\left(X\right)+\epsilon$
				we use a \emph{predictor interval}
			\item To predict the average response $f\left(
				X\right)$, we use a confidence interval
		\end{itemize}
	\item \emph{Is the relationship linear?}
		To observe residual plots.
	\item \emph{Is there synergy among the advertising media?}
		In the presence of non-additive relationships we can
		introduce an \emph{interaction term} in the regression.
	\ldots
 \end{enumerate}
