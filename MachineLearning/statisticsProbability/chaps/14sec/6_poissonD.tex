\paragraph{Use case}
To express the probability of a given number of events occurring in a fixed interval
of time or space if these events occur with a known constant mean rate.

\paragraph{Definition}
\begin{center}
	$\forall x\in \mathbb{N}, f(x)= \frac{e^{-\lambda}\lambda^{x}}{x!}\Rightarrow 
	X\hookrightarrow\mathcal{P}(\lambda)$
\end{center}
$\lambda\in\mathbb{R}_{+}^{*}$

\paragraph{Assumptions}
\begin{itemize}
	\item $k$ is the number of times an event occurs in an interval and 
		$k\in\mathbb{N}$
	\item Events are mutually independent.
	\item The average rate is constant.
	\item 2 events cannot occur at exactly the same instant.
\end{itemize}

\paragraph{Theorem}
\begin{center}
$\begin{cases}
	\mu_{X}=\lambda\\
	\sigma_{X}^{2} = \lambda\\
	M(t)=e^{\lambda(e^{t}-1)}
\end{cases}$
\end{center}
