\paragraph{Definition}
A continuous random variable $X$ is said to have a $t$-distribution 
with $\nu$ degrees of freedom if its probability density function is of
the form:
$$
f(x;\nu) = 
\dfrac{\Gamma\left(\frac{\nu+1}{2}\right)}{\sqrt{\pi\nu}~\Gamma\left(\frac{\nu}{2}\right)\left(1+\frac{x^{2}}{\nu}\right)^{\left(\frac{\nu+1}{2}\right)}}, -\infty\leq x\leq\infty
$$
where $\nu > 0$. If $X$ has a $t$-distribution with $\nu$ degrees of 
freedom, then we denote it by writing $X\hookrightarrow t(\nu)$\\
The distribution is a generalization of the Cauchy distribution and the
normal distribution:
$$
\left\{
\begin{array}{ll}
\nu=1 \Rightarrow \forall x\in \mathbb{R} & f(x;\nu) = \dfrac{1}{\pi(1+x^{2})}\\
\nu\rightarrow\infty \Rightarrow \forall x\in \mathbb{R} & \lm{\nu}{\infty} f(x;\nu) = \dfrac{1}{2\pi}e^{-\frac{1}{2}x^{2}}
\end{array}
\right.
$$
\paragraph{Properties}
\subparagraph{Expected value and Variance}
If the random variable $X$ has a $t$-distribution with $\nu$ degrees of
freedom, then:
$$
\mathbb{E}(X) = 
	\left\{
	\begin{array}{ll}
	0 & \mbox{if } \nu \geq 2\\
	DNE & \mbox{if } \nu = 1
	\end{array}
	\right.
$$
$$
\mathbb{V}(X) = 
	\left\{
	\begin{array}{ll}
		\dfrac{\nu}{\nu-2} & \mbox{if } \nu \geq 3\\
		DNE & \mbox{if } \nu \in \inter{1}{2} 
	\end{array}
	\right.
$$
\subparagraph{Normal and Chi-Squared distribution}
$$
\begin{cases}
Z\hookrightarrow N(0,1)\\
U\hookrightarrow\chi^{2}(\nu)\\
Z\text{ and }U\text{ independents}
\end{cases}
\Rightarrow
W=\dfrac{Z}{\sqrt{\frac{U}{\nu}}}\hookrightarrow t(\nu)
$$
\subparagraph{Normal variable sample}
$$
\begin{cases}
X\hookrightarrow N(\mu,\sigma^{2})\\
\prth{X}{i}{1}{n}\text{ a sample of the population }X
\end{cases}
\Rightarrow
\dfrac{\overline{X}-\mu}{\frac{S}{\sqrt{n}}}\hookrightarrow t(n-1)
$$
