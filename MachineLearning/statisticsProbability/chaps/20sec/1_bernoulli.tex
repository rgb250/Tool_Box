\paragraph{Definition}
A discrete bivariate random variable $(X, Y )$ is said to have
the bivariate Bernoulli distribution if its joint probability density is of the
form
\begin{center}
	$f(x,y)=
	\begin{cases}	
		\dfrac{1}{x!y!(1-x-y)}p_{1}^{x}p_{2}^{y}(1-p_{1}-p_{2})^{1-x-y}\Leftarrow (x,y)\in\left\{ 0,1 \right\}^{2}\\
		0\Leftarrow (x,y)\not\in\left\{ 0,1 \right\}^{2} 
	\end{cases}$	
	$\Rightarrow \left( X,Y \right)\hookrightarrow BER(p_{1},p_{2})
	$
\end{center}
where $0<p_{1},p_{2}, p_{1}+p_{2}<1$ and $x+y\leq 1$
\paragraph{Properties}
\begin{center}
	$
	\begin{cases}
	\E{X}=p_{1}\\
	\E{Y}=p_{2}\\
	\V{X}=p_{1}(1-p_{1})\\
	\V{Y}=p_{2}(1-p_{2})\\
	Cov{X,Y}=-p_{1}p_{2}\\
	M(s,t)=1-p_{1}-p_{2}+p_{1}e^{s}+p_{2}e^{t}
	\end{cases}
	$
\end{center}
\paragraph{Conditional Properties}
\begin{center}
	$
	\begin{cases}
	\Ec{X=x}{Y}=\dfrac{p_{2}(1-x)}{1-p_{1}}\\
	\Ec{Y=y}{X}=\dfrac{p_{1}(1-y)}{1-p_{2}}\\
	\Vc{X=x}{Y}=\dfrac{p_{2}(1-p_{1}-p_{2})(1-x)}{(1-p_{1})^{2}}\\
	\Vc{Y=y}{X}=\dfrac{p_{1}(1-p_{1}-p_{2})(1-y)}{(1-p_{2})^{2}}
	\end{cases}
	$
\end{center}
