\paragraph{Interval estimator \& Interval estimate}
Let $\prth{X}{i}{1}{n}$ be a random sample from $X\hookrightarrow f(x;\theta)$.\\
The \emph{interval estimator} of $\theta$ is a pair of statistics $L=
L(\prth{X}{i}{1}{n})\text{ and }U=U(\prth{X}{i}{1}{n})\text{ with }
L\leq U$ such that if $\prth{x}{i}{1}{n}$ is a set of sample data, then
$\theta$ belongs to the interval $\left[L(\prth{X}{i}{1}{n}),U(\prth{X}{i}{1}{n})\right]$\\
The interval $[l,u]$ will be denoted as an interval estimate of $\theta$ of $\theta$ whereas the random interval [L,U] will denote the 
interval estimator of $\theta$

\paragraph{Confidence interval}
$
\begin{cases}
	X\hookrightarrow f(x;\theta)\\
	\prth{X}{i}{1}{n}\text{ a random sample from }X\\
	\Prob{L\leq\theta\leq U}=1-\alpha
\end{cases}\\
\Rightarrow
$ The interval estimator of $\theta$ is called a $100(1-\alpha)\%$ 
\emph{confidence interval} for $\theta$
