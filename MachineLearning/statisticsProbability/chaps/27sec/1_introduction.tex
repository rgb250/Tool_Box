\paragraph{Statistical hypothesis}
is a conjecture about the distribution $f(x;\theta)$ of a population
$X$.\\
This conjecture is usually about the parameter $\theta$.

\paragraph{Simple and Composite hypotheses}
H completely specifies the density $f(x;\theta)$ of the population 
$\Rightarrow$ H is \emph{simple hypothesis}\\
H does not completely specify the density $f(x;\theta)$ of the
population $\Rightarrow$ H is \emph{composite hypothesis}

\paragraph{Null and alternative hypothesis}
Null hypothesis $H_{0}$ is to be tested, and correspond to the idea
that an observed difference is due to chance.\\
Alternative hypothesis $H_{a} = \overline{H_{0}}$, corresponds to the
idea that the observed difference is real.

\paragraph{Hypothesis test}
It is used to measure the difference between the data and what is
expected on the null hypothesis.\\
Let $X\hookrightarrow f(x;\theta)\text{ and }\prth{X}{i}{1}{n}$ a 
sample from $X$ and $C$ a Borel set in $\mathbb{R}^{n}$
It is an ordered sequence: $\left(\prth{X}{i}{1}{n};H_{0},H_{\alpha},C\right)$\\
The set $C$ is called the \emph{critical region} in the hypothesis 
test. The critical region is obtained using a \emph{test statistics}
$W\left(\prth{X}{i}{1}{n}\right)$. If the outcome of 
$\prth{X}{i}{1}{n}$ turns out to be an element of $C$ then we decide to
accept $H_{\alpha}$ otherwise we accept $H_{0}$
