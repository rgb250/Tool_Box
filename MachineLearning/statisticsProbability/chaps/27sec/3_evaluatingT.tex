\begin{tabular}{|c|c|c|}
	\hline
	& $H_{0}$ is true & $H_{0}$ is false\\
	\hline
	Accept $H_{0}$ & Correct Decision & Type $II$ Error\\
	\hline
	Reject $H_{0}$ & Type $I$ Error & Correct Decision\\
	\hline
\end{tabular}

\paragraph{Significance level}
It is denoted by $\alpha = \Prob{\text{Type I Error}}$ corresponds to
the probability of getting a test statistic as extreme as, or more 
extreme than the observed one. This probability is computed on the 
basis that the $H_{0}$ is true.

\paragraph{P-value}
Is the probability of getting a big test statistic, assuming the null
hypothesis to be right.
\paragraph{Probability of type II error}
It is denoted by $\beta = \ProbC{H_{0}\text{ is false}}{\text{Accept }H_{0}} = \ProbC{H_{\alpha}\text{ is true}}{\text{Accept }H_{0}}$ 

\paragraph{The power function}
It is the function $\pi : \Omega\rightarrow [0,1]$ defined by:
$
\pi(\theta)=
\begin{cases}
	\Prob{\text{Type I Error}}\\
	1-\Prob{\text{Type II Error}}
\end{cases}
$

\paragraph{A test of level $\delta$} Given $\delta\in[0,1]$
$\max_{\theta\in\Omega_{0}}\pi(\theta)\leq\delta$
\paragraph{Test of size $\delta$} Given $\delta\in[0,1]$
$\max_{\theta\in\Omega_{0}}\pi(\theta)=\delta$
\paragraph{Uniformly most powerful} 
Let $T$ be a test procedure for testing the null hypothesis.
For any test $W$ of level $\delta,\\ \forall\theta\in\Omega_{0}, \pi_{T}(\theta)\geq\pi_{W}(\theta)$ 

\paragraph{Theorem Neyman-Perarson}
$
\begin{cases}
	X\hookrightarrow f(x;\theta)\\
	\prth{X}{i}{1}{n}\text{ sample from }X\\
	L\left(\theta;\prth{x}{i}{1}{n}=\prd{{i=1}}{n}f(x_{i};\theta)\right)\text{ likelihood function of the sample }
\end{cases}\\
\Rightarrow
C=\left\{\prth{x}{i}{1}{n}|\frac{L\left(\theta_{0},\prth{x}{i}{1}{n}\right)}{L\left(\theta_{\alpha},\prth{x}{i}{1}{n}\right)}\right\}\leq k
$ for $k\in\mathbb{R}_{+}^{*}$ is best of its size for testing:\\
$H_{0}: \theta = \theta_{0}$ against $H_{a}:\theta = \theta_{a}$
