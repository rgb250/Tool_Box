\paragraph{Gamma distribution}
The \emph{gamma} function $\Gamma(z)$ is a generalization of the notion of 
factorial, for $z\in\mathbb{R}_{+}^{*}$:\\
\begin{center}
	$\Gamma(z)=\Su{0}{\infty}x^{z-1}e^{-x}dx$
\end{center}
\subparagraph{Definition}

\begin{center}
	$f(x)\begin{cases}\frac{1}{\Gamma(\alpha)\theta^{\alpha}}x^{\alpha-1}e^{-\frac{x}{\theta}}\Rightarrow x\in\mathbb{R}_{+}^{*}\\0 \Leftarrow \overline{x\in\mathbb{R}_{+}^{*}}\end{cases}\Rightarrow X\hookrightarrow \mathcal{G}(\theta, \alpha)$
\end{center}
\subparagraph{Theorem}
$X\hookrightarrow\mathcal{\theta, \alpha}\Rightarrow
\begin{cases}
	\E{X}=\theta\alpha\\
	\V{X}=\theta^{2}\alpha\\
	M(t)=\left( \frac{1}{1-\theta t}^{\alpha}\right)\text{ if }t<\frac{1}{\theta} 
\end{cases}
$
\paragraph{Exponential distribution}
\subparagraph{Definition}
\begin{center}
	$f(x)=
	\begin{cases}
		\frac{1}{\theta}e^{-\frac{x}{\theta}}\Leftarrow x >0\\
		0\Leftarrow x\leq 0
	\end{cases}$
\end{center}
Most of the information about an exponential distribution can be obtained 
from the gamma distribution
%\subparagraph{}<++>
\paragraph{Chi-square distribution with $r$ degrees of freedom}
\subparagraph{Definition}
\begin{center}
	$f(x)=
	\begin{cases}
		\frac{1}{\Gamma{\frac{r}{2}}2^{\frac{r}{2}}}e^{-\frac{x}{2}}\Leftarrow x\in\mathbb{R}_{+}^{*}\\
		0\Leftarrow x\leq 0
	\end{cases} \Rightarrow X\hookrightarrow \chi_{2}(r)$
\end{center}
\paragraph{$n\text{-Erlang}$}
\subparagraph{Definition}
\begin{center}
	$f(x)=
	\begin{cases}
		\lambda e^{-\lambda x}\frac{(\lambda x)^{n-1}}{(n-1)!}\Leftarrow x\in\mathbb{R}_{+}^{*}\\
		0\Leftarrow x\leq 0
	\end{cases} \Rightarrow X\hookrightarrow \emph{n-Erlang}(\lambda)$
\end{center}
with $\lambda > 0$
\paragraph{Unified distribution}
\subparagraph{Definition}
\begin{center}
	$f(x)=
	\begin{cases}
		\dfrac{\alpha}{\theta^{\alpha^{\psi}}\Gamma\left( \alpha^{\psi}+1\right)}x^{\alpha -1}e^{\frac{-x^{-\left( \alpha^{\psi }- \alpha -1\right)}}{\theta}}\Leftarrow x\in\mathbb{R}_{+}^{*}\\
		0\Leftarrow x\leq 0
	\end{cases} \Rightarrow X\hookrightarrow \emph{n-Erlang}(\lambda)$
\end{center}
with $\lambda > 0$
\subparagraph{Weibull distribution}
For $\alpha = 1$ we get Weibull distribution\\
\begin{center}
$\begin{cases}\E{X}=\theta^{\frac{1}{\alpha}}\Gamma\left(1+\frac{1}{\alpha}\right)\\
	\V{X}=\theta^{\frac{2}{\alpha}}\left(\Gamma\left(1+\frac{2}{\alpha}\right)-\left(1+\frac{1}{\alpha}\right)^{2}\right)\end{cases}$
\end{center}
The Weibull distribution provides probabilistic models for life-length data of components or systems.
