\documentclass[a4paper, 10pt]{report}  
% documentclass options:
%	10pt | 11pt | 12pt       --> font size
%	a4paper | letterpaper    --> paper size & format
% 	draft                    --> draft mode
% 	onecolumn | twocolumn    --> multiple columns
%	felqn | leqno            --> formula-specific
%	landscape                --> landscape mode
%	onside | twoside         --> single or double sided documents
%	notitlepage | title page --> titlepage behavior
%	openright | openany      --> chapter opening page
% book can be replaced with:
%	article --> short document that does not contain parts & chapters
%	report  --> quite long document containing chapters
%	book    --> real book as in litterature
%	letter  --> cover letter, etc..
%	slides  --> ONLY use with beamer

%% Page size, margins
\usepackage[margin=1in]{geometry}

%% For accented character printing
\usepackage[T1]{fontenc} % T1 --> support characters of european languages

%% For format
\usepackage[english]{babel}

%% Bibliography
\usepackage[backend=biber]{biblatex}          % --> import biblatex package using the helper program 'biber' that sorts the bibliography entries and provide all the relevant information to package 'biblatex'.
\addbibresource{ml_notes.bib}                 % --> import the bibliography file

%% For handling characters from Latin Modern family 
\usepackage{lmodern}

%% Adaptive table
\usepackage{tabularx}

%% Handle hypertext links without altering KOMA library
\usepackage{scrhack}

%% Make table of content clickable
\usepackage[pdfpagelabels]{hyperref} % pdfpagelabels --> change the (invisible) numbering style of the title page
\hypersetup{
    colorlinks,
    citecolor=black,
    filecolor=black,
    linkcolor=black,
    urlcolor=black
}

%% Adapt spelling rules depending of the language
\usepackage[english]{babel}

%% Handle command without precising the number of arguments.
\usepackage{xargs}

%% Libraries of mathematical purposes.
\usepackage{amsmath}  % --> correctly print mathematical characters
\usepackage{amssymb}  % --> extended symbol collection
\usepackage{amsthm}   % --> theorem like-structures
\usepackage{bm}       % --> bold font in mathematical mode.
\usepackage{bbm}      % --> bold fond for numbers
\usepackage{blkarray} % --> indexing (rows & columns) matrices
\usepackage{dsfont}   % --> to access to the symbol of mathematics sets via \mathbb

%% Alphabetical indexed list.
\usepackage{enumitem}

%% Multirow option
\usepackage{multirow}

%% Libraries to display number depending on the chose typography.
\usepackage{siunitx} % --> Scientific notation
%\sisetup{locale= FR,exponent-product=.} % --> French Adaptation for example 5,6.10²
%\DeclareSIUnit\year{ann\'{e}ee}         % --> French Adaptation for year unit

%% Handle image position
\usepackage{float}      % --> to envelop content that cannot be broken over a page
\usepackage{subcaption} % --> captions below images or sub images


%% Formatting
\usepackage{xcolor}         % --> use color
\usepackage{color,soulutf8} % --> underlining
\definecolor{bleudefrance}{rgb}{.19, .55, .91}
\definecolor{pakistangreen}{rgb}{.0, .4, .0}
\definecolor{rossocorsa}{rgb}{0.83, 0.0, 0.0}
\definecolor{persimmon}{rgb}{0.93, 0.35, 0.0}
\definecolor{deepblue}{rgb}{0,0,0.5}
\definecolor{deepred}{rgb}{0.6,0,0}
\definecolor{deepgreen}{rgb}{0,0.5,0}

%% Anotation
\usepackage{todonotes}
\newcommand{\Moi}[1]{\todo[color = teal!40]{#1}}
\newcommand{\Cnsl}[1]{\todo[color = pakistangreen!40]{#1}}
\newcommand{\MeG}[1]{\todo[color = rossocorsa!40]{#1}}

%% Redundant notation:
\newcommandx{\hb}[1]{\widehat{\beta_{#1}}}
\newcommandx{\prth}[3]{\left( #1_{#2} \right)_{1\leq #2 \leq #3}}
\newcommandx{\prtH}[4]{\left( #1_{#2} \right)_{#3\leq #2 \leq #4}}

%% Square results
\newcommand{\fr}[1]{\fcolorbox{rossocorsa}{white}{#1}}
\newcommand{\frB}[1]{\fcolorbox{bleudefrance}{white}{#1}}
\newcommand{\frG}[1]{\fcolorbox{pakistangreen}{white}{#1}}
\newcommand{\frN}[1]{\fcolorbox{black}{white}{#1}}

%% Color underlining
\newcommandx{\uN}[1]{\setulcolor{black}\ul{#1}}
\newcommandx{\uR}[1]{\setulcolor{rossocorsa}\ul{#1}}
\newcommandx{\uB}[1]{\setulcolor{bleudefrance}\ul{#1}}
\newcommandx{\uG}[1]{\setulcolor{pakistangreen}\ul{#1}}
\newcommandx{\uT}[1]{\setulcolor{teal}\ul{#1}}
\newcommandx{\uO}[1]{\setulcolor{persimmon}\ul{#1}}

%% Text in color 
\newcommandx{\tR}[1]{\textcolor{rossocorsa}{#1}}
\newcommandx{\tB}[1]{\textcolor{bleudefrance}{#1}}
\newcommandx{\tG}[1]{\textcolor{pakistangreen}{#1}}
\newcommandx{\tT}[1]{\textcolor{teal}{#1}}
\newcommandx{\tO}[1]{\textcolor{persimmon}{#1}}

%% Mathematics writting | symbols
\newcommandx{\inter}[2]{\left[\![#1, #2]\!\right]}                % --> interval
\newcommandx{\lm}[2]{{\displaystyle \lim_{#1 \rightarrow #2}}}    % --> limit
\DeclareMathOperator*{\argmax}{arg\,max}                          % --> argmax
\DeclareMathOperator*{\argmin}{arg\,min}                          % --> argmin
\newcommandx{\Su}[2]{{\displaystyle \int_{#1}^{#2}}}              % --> integral
\newcommandx{\su}[2]{{\displaystyle \sum_{#1}^{#2}}}              % --> sum (sigma)
\newcommandx{\prd}[2]{{\displaystyle \prod_{#1}^{#2}}}            % --> product (pi)
\newcommandx{\prob}[1]{\mathbb{P}\left(#1\right)}                 % --> probability P(A)
\newcommandx{\probC}[2]{\mathbb{P}\left(#1|#2\right)} % --> conditionnal probability
\newcommandx{\Prob}[1]{\mathbb{P}\left(\left\{#1 \right\}\right)} % --> probability P({A})
\newcommandx{\ProbC}[2]{\mathbb{P}_{\left\{#1\right\}}\left(\left\{#2\right\}\right)} % --> conditionnal probability
\newcommandx{\E}[1]{\mathbb{E}\left(#1\right)}                    % --> expecting
\newcommandx{\V}[1]{\mathbb{V}\left(#1\right)}                    % --> variance
\newcommandx{\norm}[1]{\left\lVert#1\right\rVert}                 % --> norm symbol
\newcommandx{\sP}[2]{\langle\left. #1\right\vert #2 \rangle}      % --> dot product




\title{Machine Learning Methods}
\author{Siger}


\begin{document}
\pagenumbering{Alph} % change the (invisible) numbering style of the title page
\maketitle

Si Dieu est infini, alors je suis une partie de Dieu sinon je serai sa limite\ldots

\pagenumbering{arabic} % change the (invisible) numbering style of the title page

\tableofcontents

\part{Collect and Pre-process Data}
\section{Data cleaning}  \cite{omar_elgabry_the_ultimate_guide_to_data_cleaning}
\subsection{Data Quality}
\section{Validity}
\begin{itemize}
    \item \textbf{Data-Type Constraints}: for a given column a fixed data-type must be associated with.
    \item \textbf{Range Constraints}: only a range of values should be taken.
    \item \textbf{Mandatory Constraints}: some columns cannot be empty.
    \item \textbf{Unique Constraints}: across a given dataset a field or a combination of variables.
    \item \textbf{Foreign-key constraints}: a foreign key column cannot have a value that
        does not exist in the primary key.
    \item \textbf{Regular expression patterns}: text fields that have to follow a given 
        alphanumerical pattern.
    \item \textbf{Cross-field validation}: consistency of values, for example considering
        a given man, his birth date have to be older than his death date.
\end{itemize}

\section{Accuracy}
The degree to which the data is close to the true value.

\section{Completeness}
The degree to which the all the required data is known.

\section{Consistency}
The degree to which the data is consistent, within the same data set or across multiple 
data sets.

\section{Uniformity}
The degree to which the data is specified using the same unit of measure.


\subsection{The workflow}
\section{Inspection} 
Detect unexpected behavior in the data.
\begin{itemize}
    \item \textbf{Data profiling}: summary statistics about the data, see \href{https://github.com/ydataai/ydata-profiling}{\textit{ydata-profiling}} in Python.
    \item \textbf{Visualizations}: visualize the data using statistical metrics, see
        \href{https://plotly.com/python/}{\textit{plotly}}
    \item \textbf{Software packages}: to note and check the constraints regarding the data
        see \href{https://pypi.org/project/pydeequ/}{\textit{pydeequ}}
\end{itemize}

\section{Cleaning} 
Fix or remove anomalies discovered in the above phase.
\begin{itemize}
    \item \textbf{Irrelevant Data}: ask to the expert what can be the unnecessary columns,
        check them and remove them if they are not useful.
    \item \textbf{Duplicates}
    \item \textbf{Type conversion}: make sure the appropriate data type is associated with
        a given column.
    \item \textbf{Syntax errors}: white spaces, pad strings \dots
    \item \textbf{Standardize}: same unit across the dataset, same pattern for text.
    \item \textbf{Scaling/Transformation}: in order to compare different scores for 
        example.
    \item \textbf{Normalization}: useful for some statistical methods.
    \item \textbf{Missing values}: 
        \begin{itemize}
            \item Drop: only if the missing values in a column rarely and randomly occur.
            \item Impute: many methods,  
                \begin{itemize}
                    \item \textit{mean} is relevant when data is not skewed
                    \item \textit{median} otherwise
                    \item \textit{hot-deck} imputes from a randomly selected similar record
                    \item \textit{polynomial interpolation} approximate the data with a polynomial 
                    \item \textit{linear regression} imputes from prediction 
                    \item \textit{k-nearest} imputes depending on a set of closet observations
                \end{itemize}
            \item Flag: let the missing value as it is.
        \end{itemize}
    \item Outliers: Remove outliers only if they are harmful for the chosen model.
    \item In-record \& cross-datasets errors: fix non-consistent situations like married 
        kids, quantity being different of the one when we compute using other columns.
\end{itemize}

\section{Verifying}
Check correctness of the cleaning phase.
\section{Reporting} 
Report about changes made, using one of the software summarising the data quality for 
example.



\part{Statistics}
\chapter{Fundamental probability concepts}
\section{Basic probability properties}
\subsection{What is a probability?}
It is a \tB{mathematical measure of the uncertainty} of a given event.

\paragraph{Objectivist interpretation \cite{wiki:probability_interpretation}}: assigns
numbers describing some objective state, \tB{\textit{Frequentist} interpretation claiming
that the probability of a random event is quantified by the relative frequency in a given
experiment}. 

\paragraph{Subjectivist interpretation \cite{wiki:probability_interpretation}}: assigns
numbers quantifying the degree of belief that a given event occurs. \textit{Bayesian} 
interpretation uses expert knowledge considered as subjective and represented by the 
prior, as well as experimental data represented by the likelihood. The normalized product
of the 2 above quantity is the \tB{posterior probability distribution containing both 
expert knowledge and experimental data}.


\subsection{Properties}
\paragraph{Event and its opposite}
$\Prob{A} + \Prob{\overline{A}} = 1$
\paragraph{Not necessary mutually exclusive events}
$\Prob{A\cup B} = \Prob{A} + \Prob{B} - \Prob{A\cap B}$
\paragraph{Independent events}
$\Prob{A\cap B} = \Prob{A} \times \Prob{B}$
\paragraph{Conditional Probability}
$\probC{A}{B} = \dfrac{\prob{A\cap B}}{\prob{B}}$
\paragraph{Law of Total Probability}
$
\begin{cases}
    \left(B_{i}\right)_{1\leq i\leq n}\text{: partition of a sample }\mathcal{S}\\
    \forall i\in \inter{1}{n},~\Prob{B_{i}} \neq 0
\end{cases} 
\Rightarrow \prob{A} = \su{i=1}{n}\prob{B_{i}}\probC{A}{B_{i}}
$


\paragraph{Bayes' Theorem}
Using \tB{Law of Total Probability}:\\
$
\begin{cases}
    \left(B_{i}\right)_{1\leq i\leq n}\text{: partition of a sample }\mathcal{S}\\
    \forall i\in \inter{1}{n},~\Prob{B_{i}} \neq 0
\end{cases} 
\Rightarrow \probC{B_{i}}{A} = \dfrac{\prob{B_{i}}\times\probC{A}{B_{i}}}{\su{k=1}{n}\prob{B_{k}}\probC{A}{B_{k}}}
$


\subsection{Moments}
They are certain quantitative measures related to the shape of the function's graph.
\cite{wiki:moments_mathematics}
\paragraph{$n^{th}$ moments of a random variable:}
The $n^{th}$ moment about the origin of a random variable $X$ as denoted by
$E\left( X^{n} \right)$, is defined to be:
\begin{center}
	$\E{X^{n}}=
	\begin{cases}
		\su{x\in R_{X}}{{}}x^{n}f(x)\text{ if }X\text{ is discrete}\\
		\Su{-\infty}{\infty}x^{n}f(x)dx\text{ if }X\text{ is continuous}
	\end{cases}$
\end{center}

\paragraph{Expected value:}
The expected value of a random variable $X$ as denoted by
$E\left( X \right)$, is defined to be:
\begin{center}
	$\E{X}=
	\begin{cases}
		\su{x\in R_{X}}{{}}xf(x)\text{ if }X\text{ is discrete}\\
		\Su{-\infty}{\infty}xf(x)dx\text{ if }X\text{ is continuous}
	\end{cases}$
\end{center}
After normalized this moment by total mass we have the \tB{center of mass}.


\paragraph{Variance :}
Let $X$ be a random variable with mean $\mu_{X}$. The variance of $X$ 
denoted by $\V{X}$ or $\sigma_{X}^{2}$ is defined by:
\begin{center}
	$\V{X}=\E{\left[ X-\mu_{X} \right]^{2}}$
\end{center}
After normalized this moment by total mass we have the \tB{moment of inertia}.\\
If $X$ is a random variable with mean $\mu_{X}$ and variance $\sigma_{X}^{2}$ then:
\begin{center}
$\sigma_{X}^{2}=\E{X^{2}}-\mu_{X}^{2}$
\end{center}
And:
\begin{center}
	$\V{aX+b}=a^{2}\V{X}$
\end{center}

\paragraph{Skewness and Kurtosis}
\begin{itemize}
    \item \tB{Skewness}: $\E{\left[ \dfrac{X-\mu_{X}}{\sigma_{X}} \right]^{3}}$, indicates
        the direction (negative $\rightarrow$ left tail is longer, positive $\rightarrow$ 
        right tail is longer) and relative magnitude of a distribution's \uB{deviation from
        the normal distribution}.
    \item \tB{Kurtosis}: $\E{\left[ \dfrac{X-\mu_{X}}{\sigma_{X}} \right]^{4}}$, \uB{measures
        the outliers}, data within one standard deviation will not contribute a lot to the 
        kurtosis values conversely data exceeding one standard deviation will contribute a
        lot because of the fourth power.
\end{itemize}


\subsection{Asymptotic properties}

\paragraph{Chebychev inequality}
allows to find an estimate of the area between the
values $\mu-k\sigma\text{ and }\mu+k\sigma$ for some given $k\neq 0$, showing
that the area under $f(x)$ on the interval $\left[\mu-k\sigma, \mu+k\sigma
\right]$ is at least $1-k^{2}$.\\
Let $X$ be a random variable with probability density function $f(x)$. If
$\mu$ and $\sigma>0$ are the mean and standard deviation of $X$ then:
\begin{center}
	\fr{$\Prob{|X-\mu|<k\sigma}\geq 1-\frac{1}{k^{2}}$}
\end{center}
\begin{figure}[H]
	\begin{center}
		\includegraphics[width=.5\textwidth]{chapters/2_statistics/01_fundamental_probability_concepts/1_images/1_ineqChebyGraph.png}
	\end{center}
	\caption{Illustration of Chebychev inequality}
	\label{fig:fig2.01.1}
\end{figure}



\paragraph{Markov inequality}
\begin{center}
		$X\neq\underline{0}\Rightarrow \Prob{X\geq t}\leq \dfrac{\E{X}}{t}$
\end{center}
\paragraph{Theorem weak law of large numbers:}
Let $\prth{X}{i}{1}{n}\text{: independent \& identically distributed RV}$
\begin{center}
    \fr{$\forall \epsilon \in \mathbb{R}_{+}: \lm{n}{\infty}\Prob{\left|\overline{S}_{n}-\mu\right|
            \geq \epsilon}=0$ with $\overline{S}_{n}=\dfrac{1}{n}\su{ {i=1}}{n}X_{i}$}
\end{center}
\paragraph{Convergence in probability}
Suppose $\prth{X}{i}{1}{n}$ is a sequence of random variables de-
fined on a sample space S. The sequence ``converges in probability'' to the
random variable X if, for any $\epsilon>0$
\begin{center}
	$\lm{n}{\infty}\Prob{\left|X_{n}-X\right|<\epsilon}=1$
\end{center}
\paragraph{Convergence almost surely}
Suppose the RV $X\text{ and }\prth{X}{i}{1}{n}$ is a sequence of random variables de-
fined on a sample space S. The sequence $X_{n}(\omega)$ ``converges almost surely'' to $X(\omega)$ if
\begin{center}
	$\Prob{w\in S|\lm{n}{\infty}X_{n}(\omega)=X(\omega)}=1$
\end{center}
\paragraph{Properties}
\begin{itemize}
	\item For a Bernoulli distribution, $\overline{S}_{n}\text{ converges in probability to }p$
	\item For a Normal distribution, $\overline{S}_{n}\text{ converges almost surely to }\mu$
\end{itemize}


\subsection{Central Limit Theorem}
The central limit therorem (Lindeberg-Levy Theorem) states that for any
population distribution, the distribution of the standardized sample mean
is approximately standard normal with better approximations obtained with
the larger sample size.
\begin{center}
    \tR{$\begin{cases}\prth{X}{i}{1}{n}\hookrightarrow ?(\mu,\sigma^{2})\\ n\rightarrow\infty
        \end{cases} \Rightarrow \dfrac{\overline{X}-\mu}{\frac{\sigma}{\sqrt{n}}} \hookrightarrow 
    \mathcal{N}(0, 1)$}
\end{center}
\subsection{Convergence in distribution}
Consider $X$ with its cumulative density function $F$ and $\prth{X}{i}{1}{n}$ with their 
cdf $\left( F_{i} \right)_{1\leq i\leq n}$:
\begin{center}
	$\lm{n}{\infty}F_{n}(x)=F(x)\Rightarrow X_{n}\text{ "converges in distribution" to }X$
\end{center}
\subsection{Lévy Continuity Theorem}
\begin{center}
	$
	\begin{cases}	
		\prth{X}{i}{1}{n}\text{RV}\\
		\prth{F}{i}{1}{n}\text{distribution functions}\\
		\left( M_{X_{i}} \right)_{1\leq i\leq n}\text{moment generating function}
	\end{cases}$
	$\forall t\in [-h,h]\lm{n}{\infty}M_{X_{n}}(t)=M_{X}(t)\Rightarrow
	\lm{n}{\infty}F_{n}(x)=F(x)
	$
\end{center}

\section{Bivariate case}
\paragraph{Joint probability density function}
Let $\left(X,Y\right):\left(\Omega_{X},\Omega_{Y}\right)\rightarrow\left(R_{X},R_{Y}\right)$ and $f:R_{X}\times R_{Y}\rightarrow \mathbb{R}$
\begin{center}
	$\forall (x,y)\in R_{X}\times R_{Y},f(x,y)=\Prob{X=x,Y=y}\Leftrightarrow\text{ f is the joint probability density function for }X\text{ and }Y$
\end{center}

\paragraph{Marginal probability density function}
Let for all $(x,y)\in R_{X}\times R_{Y}$: $f(x,y)$ be the joint probability density of $X$ and $Y$
\begin{center}
$\begin{cases}
	f_{X}(x)=\Su{{-\infty}}{\infty}f(x,y)dy\text{is the marginal probability density of }X\\
	f_{Y}(y)=\Su{{-\infty}}{\infty}f(x,y)dx\text{is the marginal probability density of }Y
\end{cases}$
\end{center}
\paragraph{Joint cumulative probability distribution function}
Let $F:\mathbb{R}^{2}\rightarrow\mathbb{R}$
\begin{center}
	$\forall (x,y)\in \mathbb{R}^{2},F(x,y)=\Prob{X\leq x,Y\leq y}=\Su{ {-\infty}}{y}\Su{ {-\infty}}{x}f(u,v)dudv\Leftrightarrow\text{ F is the joint cumulative probability density function for }X\text{ and }Y$
\end{center}
From the fundamental theorem of calculus:
$f(x,y)=\dfrac{\partial^{2}F(x,y)}{\partial x\partial y}$


\paragraph{Conditional expectation}
The conditional mean of $X$ given $Y=y$ is defined as:
\begin{center}
\fr{
$
\E{X|y}=
\begin{cases}
\su{{x\in R_{X}}}{}xg(x/y)\Leftarrow X\text{ discrete}\\
\Su{{-\infty}}{\infty}xg(x/y)dx\Leftarrow X\text{ continuous}
\end{cases}
$}
\end{center}

Properties:
\begin{center}
	$\begin{cases}
        \mathbb{E}_{X}\left(\mathbb{E}_{y|x}\left(Y|X\right)\right)=
        \mathbb{E}_{y}\left(Y\right)\\
	\E{Y|\left\{ X=x \right\}}=\mu_{Y}+\rho\dfrac{\sigma_{Y}}{\sigma_{X}}(x-\mu_{X})
	\end{cases}$	
\end{center}


\paragraph{Conditional Variance}
\begin{center}
	$\begin{cases}
	\V{Y|x}=\E{Y^{2}|x}-\E{Y|x}^{2}\\
	\mathbb{E}_{x}\left( \V{Y|X}=(1-\rho^{2})\V{Y} \right)
	\end{cases}$
\end{center}


\paragraph{Conditional Independence}
$X\perp Y[Z \Leftrightarrow \prob{X,Y|Z}=\prob{X|Z}\prob{Y|Z}$

\paragraph{The chain rule of conditional probabilities}
Any joint probability distribution over many random variables may be decomposed into conditional
distribution over only one variable:
\begin{center}
    $\prob{x_{1:T}} = \prob{x_{1}}\prd{t=2}{T}\prob{x_{t}|x_{1:t-1}}$
\end{center}



\section{Distribution function}
\subsection{Definition of probability density function (pdf):}
Let $R_{X}$ be the space of the random variable $X$. The function:
$f:R_{X}\rightarrow \mathbb{R}$ defined by:
\begin{align*}
    f(x) &= \Prob{X=x} \text{ if } X \text{ is discrete.}\\
    f(x) &= \Prob{X\in A}=\Su{A}{{}}f(x)dx \text{ if } X \text{ is continuous, with } A
    \text{ a set of real numbers.}
\end{align*}
is called probability density function of $X$.


\subsection{Definition of cumulative density function (cdf):}
Let $R_{X}$ be the space of the random variable $X$. The function:
$F:R_{X}\rightarrow \mathbb{R}$ defined by:\\
\begin{align*}
    F(x) &= \Prob{X\leq x} \text{ if } X \text{ is discrete.}\\
    F(x) &= \Prob{X\leq x}=\Su{-\infty}{x}f(t)dt \text{ if } X \text{ is continuous, 
    with } A \text{ a set of real numbers.}
\end{align*}


\subsection{Percentile for continuous random variables.}
Let $p\in [0;1]$, a $100p^{th}$ percentile of the distribution of a random
variable $X$ is $q\in\mathbb{R}$ satisfying:
\begin{center}
	$\Prob{X\leq q}\leq p$\\
	(Recall that the $F$ is a monotonically increasing function, then it has an 
	inverse $F^{-1}$)\\
	$q = F^{-1}(p)$
\end{center}
A $100p^{th}$ is a measure of location for the probability distribution in
the sense that $q$ divides the distribution of the probability mass into
2 parts, one having probability mass $p$ and other having probability mass
$1-p$
\begin{figure}[H]
	\begin{center}
		\includegraphics[width=.5\textwidth]{./chapters/2_statistics/01_fundamental_probability_concepts/2_images/1percentile.png}
	\end{center}
	\caption{Percentile}
	\label{fig:fig2.1}
\end{figure}
The $50^{th}$ percentile of any distribution is called median of the distribution.



\chapter{Distributions}
\subsection{Discrete distributions with finite support}
\subsection{Bernoulli}
\subsection{Rademacher}
\subsection{Binomial}
\subsection{Beta-Binomial}
\subsection{Degenerate}
\subsection{Uniform}
\subsection{Hypergeometric}
\subsection{Negative Hypergeometric}
\subsection{Poisson Binomial}
\subsection{Fisher's noncentral hypergeometric}
\subsection{Benford's law}
\subsection{Zipf's law}
\subsection{Zipf-Mandelbrot law}







\chapter{Bayesian approach}
\section{Components}
\paragraph{Bayesian concept learning}
Let be $\mathcal{D}$ the data, $h$ the hypothesis taken in account
\paragraph{Likelihood}
$p\left(\mathcal{D}|h\right)$ the probability to get the observed data considering the 
hypothesis $h$.
\paragraph{Prior}
$p(h)$ the probability of our hypothesis, many prior can be used, and this 
\textbf{subjective} aspect of Bayesian reasoning is a source of much controversy.

\paragraph{Posterior}
The posterior is simply the likelihood times the prior, normalized.
\begin{center}
$p\left(h|\mathcal{D}\right) = \dfrac{p\left(\mathcal{D}|h\right)\times p(h)}{
    \su{h'\in\mathcal{H}}{}p\left(\mathcal{D}, h'\right)p(h')
}$
\end{center}



\section{Summarizing posterior distributions}
\subsection{MAP (Maximum A Posteriori) estimation}
Although most appropriate choice for:\\
$
\begin{cases}
	\text{Real valued quantity} &\rightarrow \text{\emph{posterior median or mean}}\\
	\text{Discrete} &\rightarrow \text{\emph{vector of posterior marginals}}
\end{cases}
$\\
The most popular choice is \tB{\emph{posterior mode}} aka \tR{MAP}, because it reduces to
optimization problems for which efficient algorithms often exist.\\
Some point to be aware about MAP:
\begin{itemize}
	\item \uB{No measure of uncertainty}
	\item \tB{Plugging in the MAP estimate can result in overfitting}
	\item \tB{The mode is an untypical point}, unlike the mean or median the mode is a
		point of measure 0, it does not take the volume of the space into account.
	\item \tB{MAP estimation is not invariant to reparameterization}, for example 
		passing from centimeters to inches can break things.)\\ The MLE does not
		suffer from this since the likelihood is a function not a probability
		density
\end{itemize}


\subsection{Credible intervals}
With point estimates, we want a measure of confidence. 
\tB{
$$ C_{\alpha}\left(\mathcal{D}\right) = (l, u): \Prob{{l\leq \theta \leq u | \mathcal{D}}} \geq 1 - \alpha
$$}
In general, credible intervals are usually what people want to compute but confidence
intervals are usually what they actually compute, because most people are taught 
frequentist statistics but not Bayesian statistics.\\
Sometimes with central intervals there might be points be outside the CI which have higher
probability density.\\
More formally $p^{*}$ such that: 
\begin{center}
	$1-\alpha = 
	\Su{{\theta:p(\theta|\mathcal{D})>p^{*}}}{}p(\theta|\mathcal{D})d\theta$
\end{center}
Then the \uB{HPD} such that:
\begin{center}
	$\mathcal{D}=\left\{\theta: p(\theta|\mathcal{D})\geq p^{*}\right\}$
\end{center}


\section{Bayesian Model Selection}
A more efficient approach than cross-validation, meaning fitting \emph{k} times each
model, is \tR{to compute the posterior over models}.
$$
p(m|\mathcal{D}) = \dfrac{p(\mathcal{D}|m)p(m)}{\su{{m\in\mathcal{M}}}{}p(m|\mathcal{D})}
$$
From this we can compute the \tB{MAP model $\hat{m} = \displaystyle \argmax_{m}
p(m|\mathcal{D})$}\\
Then we have the \tB{marginal likelihood}: $p(\mathcal{D}|\hat{m}) = \Su{}{}p(
\mathcal{D}|\hat{m})p(\theta|\hat{m})d\theta$

\paragraph{Baysian Occam's razor}
\tB{In integrating out the parameters rather than maximizing them we are automatically 
protected from overfitting}: model with more parameters do not necessarily have higher 
marginal likelihood.\\
A way to understand the Bayesian Occam's razor effect is to \tB{remember that 
probabilities must sum to one, meaning $\su{{\mathcal{D}'}}{}p(\mathcal{D}'|m)=1$. Complex
models, which can predict many things, must spread their probability mass thinly, and 
hence will not obtain as large a probability for any given data set as simpler models.}

\paragraph{Computing the marginal likelihood (evidence)}
For a fixed model we often write:
$$p(\bm{\theta}|\mathcal{D},m) \propto p(\bm{\theta}|m)p(\mathcal{D}|\bm{\theta},m)$$
This valid since $p(\mathcal{D}|m)$ is constant. However when comparing models we need
to know how to compute the marginal likelihood, $p(\mathcal{D}|m)$. In general this can
be quite hard, since we have to integrate over all possible parameter values, but when
we have a conjugate prior, it is easy to compute.\\
Let $p(\bm{\theta})=\dfrac{q(\bm{\theta})}{Z_{0}}$ be our prior, where $q(\bm{\theta})$
is an unnormalized distribution, and $Z_{0}$ is the normalization constant of the prior.
Let $p(\mathcal{D}|\bm{\theta})=\dfrac{q(\mathcal{D}|\bm{\theta})}{Z_{l}}$ be the 
likelihood, where $Z_{l}$ contains any constant factors in the likelihood. Finally let
$p(\bm{\theta}|\mathcal{D})=\dfrac{q(\bm{\theta}|\mathcal{D})}{Z_{N}}$ be our posterior
where $q(\bm{\theta}|\mathcal{D})=q(\mathcal{D}|\bm{\theta})q(\bm{\theta})$ is the 
unnormalized posterior, and $Z_{N}$ is the normalization constant of the posterior.\\
We have:
$
\begin{cases}
	p(\bm{\theta})= \dfrac{p(\mathcal{D}|\bm{\theta})p(\bm{\theta})}{p(\mathcal{D})}\\
	\dfrac{q(\bm{\theta}|\mathcal{D})}{Z_{N}} = \dfrac{q(\mathcal{D}|\bm{\theta})
	q(\bm{\theta})}{Z_{l}Z_{0}p(\mathcal{D})}\\
	p(\mathcal{D}) = \dfrac{Z_{N}}{Z_{0}Z_{l}}
\end{cases}
$

 
Simpler approach
\begin{itemize}
	\item \textbf{BIC} In general $p(\mathcal{D}|m) = \Su{}{}p(\mathcal{D}|\bm{\theta})p(\bm{\theta}|m)d\bm{
\theta}$ can be quite difficult to compute. A popular approximation is:
		\tB{$BIC \triangleq \log(p(\mathcal{D}|\bm{\hat{\theta}_{MLE}})) - 
		\dfrac{dof(\bm{\hat{\theta}_{MLE}})}{2}\log(N)\approx\log{p(\mathcal{D})}$}
	\item \textbf{AIC}:
		\tB{$AIC(m,\mathcal{D})\triangleq\log(p(\mathcal{D}|\bm{\hat{\theta}}_{MLE
		})) -dof(m)$}\\
		This is derived from Frequentist framework and cannot be interpreted as 
		an approximation to the marginal likelihood. The penalty of AIC is less
		than BIC, it causes AIC pick more complex models. That \tB{can be better for 
		predictive accuracy}.
	\item Effect of the prior.\\
		If the prior is unknown, the correct Bayesian procedure is to put a prior
		on the prior. That is we should put a prior on the hyper-parameter 
		$\alpha$ as well as the parameters $\bm{w}$. To compute the marginal 
		likelihood we should integrate out all unknowns, we should compute:
		\tB{$\Su{}{}\Su{}{}p(\mathcal{D}|\bm{w})p(\bm{w}|\alpha,m)p(\alpha|m)
		d\bm{w}d\alpha$}
		A computational shortcut is to optimize $\alpha$ rather than integrating
		it out. That is, we use \tG{$p(\mathcal{D}|m)\approx\Su{}{}p(\mathcal{D}
		\bm{w})p(\bm{w}|\alpha,m)d\bm{w}$}.
		where \tG{$\hat{\alpha} = \displaystyle \argmax_{\alpha} p(\mathcal{D}|
			\alpha,m) = \displaystyle \argmax_{\alpha}\Su{}{}p(\mathcal{D}|
		\bm{w})p(\bm{w}|\hat{\alpha},m)d\bm{w}$}
\end{itemize}
\paragraph{Bayes Factors}
When prior on models is uniform, then model selection is equivalent to picking the model
with the highest marginal likelihood. Now suppose we just have two models we are 
considering, call them the null hypothesis, $M_{0}$ and the alternative hypothesis,
$M_{1}$.\\
\tB{$$BF_{1,0} \triangleq \dfrac{p(\mathcal{D}|M_{1})}{p(\mathcal{D}|M_{0})}=
    \dfrac{\left.\frac{p(M_{1}|\mathcal{D})}{p(M_{0}|\mathcal{D})}\right\}\text{\emph{Posterior odds}}}{\left.\frac{p(M_{1})}{p(M_{0})}\right\}\text{\emph{Prior odds}}}
$$}
\begin{itemize}
    \item \emph{Posterior odds}: quantifies relative plausibility of the rival hypotheses
        \textbf{after} having seen the data.
    \item \emph{Bayes Factor}, $BF_{1,0}$, quantifies the evidence provided by the data, 
        this is like a likelihood ratio, except we integrate out the parameters, which 
        allows us to compare models of different complexity.
    \item \emph{Prior odds}: quantifies relative plausibility of the rival hypotheses
        \textbf{before} seeing the data.
\end{itemize}

 

\begin{table}[h!]
	\begin{tabular}{|cl|}
		\hline
	\textbf{Bayes Factor} $\bm{BF(1,0)}$ & \textbf{Interpretation}\\
		\hline
		$BF<\frac{1}{100}$ & Decisive evidence for $M_{0}$\\
		\hline
		$BF<\frac{1}{10}$ & Strong evidence for $M_{0}$\\
		\hline
		$\frac{1}{10}<BF<\frac{1}{3}$ & Modest evidence for $M_{0}$\\
		\hline
		$\frac{1}{3}<BF<1$ & Weak evidence for $M_{0}$\\
		\hline
		$1<BF<3$ & Weak evidence for $M_{1}$\\
		\hline
		$3<BF<10$ & Modest evidence for $M_{1}$\\
		\hline
		$BF>10 $ &  Strong evidence for $M_{1}$\\
		\hline
		$BF>100$ & Decisive evidence for $M_{1}$\\
		\hline
	\end{tabular}
\end{table}


\paragraph{Jeffreys-Lindley paradox}
Problems can arise when we use improper priors (i.e. priors that do not integrate to 1)
for model selection/ hypothesis testing, even though such priors may be acceptable for 
other purposes. \tB{In particular the Bayes Factor will always favor the simplest model 
since the probability of the observed data under a complex model with a very diffuse prior
will be very small.} Thus it is important to use proper priors when doing model selection.

 
\section{Priors}
The most controversial aspect of Bayesian statistics is its reliance on priors
\subsection{Uninformative priors}
If we do not have strong evidence on what $\theta$ should be, it is common to use an
uninformative priors, to "let the data speak for itself".\\
One might think that the most uninformative prior would be the uniform distribution: 
$Beta(1, 1)$, but the posterior would then be: $\E{\theta|\mathcal{D}} =
\dfrac{N_{1}+1}{N_{1}+N_{0}+2}$, whereas the MLE is $\dfrac{N_{1}}{N_{1}+N_{0}}$.\\
As by decreasing the magnitude of the pseudo counts, we can lessen the impact of the 
prior, we can argue that the most non-informative prior is: 
$$\lm{\epsilon}{0} Beta(\epsilon, \epsilon) = Beta(0, 0)$$
Called the \emph{Haldane prior}, it is an improper prior.\\
In general it is advisable to perform a some kind of sensitivity analysis, in which one
checks how much one's conclusions or prediction change in response to change in the 
modelling assumptions which includes the choice of the prior and the likelihood as well.
If the conclusion are relatively insensitive to the modelling assumption, one can have
more confidence in the results.
\subsection{Jeffreys priors}
Harold Jeffreys designed a general purpose technique for creating non-informative priors.
The key observation is that if $p(\phi)$ is non-informative then any re-parametrization
of the prior, such as $\theta=h(\phi)$ for some function $h$ should also be 
non-informative.
\begin{itemize}
	\item Start with a variable change: $p_{\theta}(\theta) = p_{\phi}(\phi)\left|\dfrac{d\phi}{d\theta}\right|$
	\item Consider the following constraint: $p_{\phi}(\phi)\propto
		\sqrt{\mathcal{I}(\phi)}$, where $\mathcal{I}(\phi)$ is the Fisher 
		information.\\ $\mathcal{I}(\phi) \triangleq - \E{2 \times 
		\dfrac{d\log\left(p(X|\phi)\right)}{d\phi}}$. This a measure of the
		curvature of the expected negative log likelihood and hence a measure of
		stability of the MLE.
	\item Now $\dfrac{d\log(p(x|\theta))}{d\theta} = 
		\dfrac{d\log(p(X|\phi))}{d\phi}\dfrac{d\phi}{d\theta}$
	\item $\mathcal{I}(\theta) = \mathcal{I}(\phi)
		\left(\dfrac{d\phi}{d\theta}\right)^{2}$
	\item $\sqrt{\mathcal{I}(\theta)} = \sqrt{\mathcal{I}(\phi)}\left|\dfrac{d\phi}
		{d\theta}\right|$
	\item Finally $p_{\theta}(\theta) = p_{\phi}(\phi)\left|\dfrac{d\phi}
		{d\theta}\right| \propto \sqrt{\mathcal{I}(\phi)}\left|\dfrac{d\phi}
		{d\theta}\right| = \sqrt{\mathcal{I}(\theta)}$
\end{itemize}

\subsection{Robust priors}
To prevent an undue influence on the result, we build priors having heavy tails, which 
avoids forcing things to be too close to the prior mean.

\subsection{Mixture of conjugate priors}
Conjugate priors simplify the computation of robust priors, but are often not robust, and 
not flexible enough to encode our prior knowledge. However it turns out that a mixture of
conjugate priors is also conjugate, and seem to be a good compromise.



\section{Hierarchical and Empirical Bayes}
\subsection{Hierarchical Bayes}
A key requirement for computing the posterior $p(\theta|\mathcal{D})$ is the 
specification of a prior $p(\theta|\eta)$ where $\eta$ are the hyper-parameters. A 
Bayesian approach is to \tB{put a prior on our priors}. This is an example of a \textbf{
hierarchical Bayesian Model}.

\subsection{Empirical Bayes}
In hierarchical Bayesian models, we need to compute the posterior on multiple levels of
latent variables. For example, in a two-level model, we need to compute:
$p(\eta, \theta|\mathcal{D}) \propto p(\mathcal{D}|\theta)p(\theta|\eta)p(\eta)$\\
\tB{We can approximate the posterior on the hyper-parameters with a point-estimate, 
$p(\eta|\mathcal{D})\approx \delta_{\hat{\eta}}(\eta)$ where $\hat{\eta}=\argmax_{\eta}
p(\eta|\mathcal{D})$. Since $\eta$ is typically much smaller than $\theta$ in 
dimensionality, it is less prone to overfitting, so we can safely use a uniform prior on 
$\eta$}. Then the estimate becomes: 
$$ \hat{\eta} = \argmax_{\eta} p(\mathcal{D}|\eta) = \argmax_{\eta} \Su{}{}
p(\mathcal{D}|\theta)p(\theta|\mathcal{\eta})d\theta $$
This overall approach is called \textbf{Empirical Bayes}\\
Empirical Bayes violates the principle that the prior should be chosen independently of 
the data. However, we can just view it as a computationally cheap approximation to 
inference in a hierarchical Bayesian model, just as we viewed MAP estimation as an approximation to inference in the one level model $\theta \rightarrow \mathcal{D}$. In fact, we
can construct a hierarchy in which the more integrals one performs, the "more Bayesian" 
one becomes:
\begin{center}
	\begin{tabular}{|*{2}{l|}}
		\hline
		\textbf{Method} & \textbf{Definition} \\
		\hline
		Maximum likelihood & $\hat{\theta} = \displaystyle\argmax_{\theta} 
		p(\mathcal{D}|\theta)$ \\
		\hline
		MAP estimation & $\hat{\theta} = \displaystyle\argmax_{\theta} 
		p(\mathcal{D}|\theta)p(\theta|\eta)$ \\
		\hline
		ML-II (Empirical Bayes) & $\hat{\eta}=\displaystyle\argmax_{\eta}\Su{}{}
		p(\mathcal{D}|\theta)p(\theta|\eta)d\theta = \displaystyle\argmax_{\eta}p(\mathcal{D}|
		\eta)$ \\
		\hline
		MAP-II & $\hat{\eta}=\displaystyle\argmax_{\eta}\Su{}{}
		p(\mathcal{D}|\theta)p(\theta|\eta)p(\eta)d\theta = \displaystyle\argmax_{\eta}p(
		\mathcal{D}| \eta)p(\eta)$\\
		\hline
		Full Bayes & $p(\theta, \eta|\mathcal{D}) \approx p(\mathcal{D}|\theta)
		p(\theta|\eta)p(\eta)$\\
		\hline
	\end{tabular}
\end{center}



\section{Bayesian Decision Theory}
\tB{We can formalize any given statistical decision problem as a game against nature} (as 
opposed to a game against other strategic players, which is the topic of game theory).
\tB{In this game, nature picks a state or parameter or label, $y\in \mathcal{Y}$, unknown
to us, and then generates an observation, $\bm{x}\in\mathcal{X}$ which we get to see. We
then have to make a decision, that is, we have to choose an action $a$ from some 
\textbf{action space} $\mathcal{A}$. Finally we incur some \textbf{loss}, $L(y, a)$, which
measures how compatible our action $a$ is with nature's hidden state $y$.}\\
Our goal is to devise a decision procedure or policy, $\delta: \mathcal{X}\rightarrow
\mathcal{A}$ which specifies the optimal action for each possible input which specifies the optimal action for each possible input, meaning the action that minimizes the expected 
loss:
$$ \delta(\bm{x}) = \argmin_{{a\in \mathcal{A}}} \E{{L(y, a)}}$$
In the Bayesian vision, the expected value of $y$ given the data we have seen so far, 
whereas in the frequentist vision the expected value refers to $x$ and $y$ that we expect
to see in the future.\\
\tO{In the Bayesian vision the optimal action having observed $\bm{x}$ is defined as the 
action $a$ that minimizes the \textbf{posterior expected loss}:
$$ \rho(a|\bm{x})\triangleq\mathbb{E}_{p(y|x)}\left(L(y, a)\right) = \su{y}{}L(y, a)
p(y|x)$$}
Hence the \tR{Bayes estimator also called Bayes decision rule is given by:
$$\delta(\bm{x}) = \argmax_{a\in\mathcal{A}}\rho(\bm{a}|\bm{x})$$}

\subsection{Bayes estimators for common loss functions}
\begin{itemize}
    \item \textbf{MAP} estimate minimizes 0-1 loss: $L(y, a) = \mathbb{I}_{y\neq a}
		\begin{cases}
			0 \text{ if } a = y\\
			1 \text{ else}
		\end{cases}$
    \item \textbf{Reject option}, in classification problems where $p(y|\bm{x})$ is very 
		uncertain we may prefer to choose a reject action, in which we refuse to 
		classify the example as any of the specified classes. Let choosing $a=C+1$
		correspond to picking the reject action, and choosing $a\in\{1,...,C\}$
		correspond to picking one of the classes.\\
		$L(y=j, a=i) = 
		\begin{cases}
			0 &\text{ if } i=j \text{ and } i,j\in\{1,...,C\}\\
			\lambda_{r} &\text{ if } i=C+1 \\
			\lambda_{s} &\text{ otherwise}
		\end{cases} $\\
		where $\lambda_{r}$ is the cost of the reject action, and $\lambda_{s}$ is
		the cost of a substitution error. 
    \item \textbf{Squared Error ($l_{2}$)} for a continuous parameters. $L(y, a) =
        (y-a)^{2}$
    \item \textbf{Absolute Error ($l_{1}$)} more robust against outliers. $L(y,a)=
		\lvert y-a\rvert$. The optimal point is the median.
    \item \textbf{Supervised learning} considering a prediction function $\delta: 
        \mathcal{X} \rightarrow \mathcal{Y}$ and some cost function $l(y, \delta(x))$. 
        Then the loss incurred by taking action $\delta$ when the unknown state of nature
        is $\theta$ (the parameters of the data generating the mechanism).
        $L(\bm{\theta}, \delta) \triangleq 
        \mathbb{E}_{(\bm{x}, y)~p(\bm{x},y|\bm{\theta})}\left(
        l(y, \delta(\bm{x}))\right)=\su{\bm{x}}{}\su{y}{}L\left(y,\delta(\bm{x}) 
        p(\bm{x},y|\bm{\theta})\right)$
\end{itemize}

\subsection{Model evaluation metrics}
\begin{itemize}
    \item \textbf{False positive vs False negative trade-off} for binary decision problems
        three are 2 types of errors:
        \begin{enumerate}
            \item {false positive} (false alarm) if $\hat{y}=1 \wedge y=0$
            \item {false negative} (missed detection) if $\hat{y}=0 \wedge y=1$
        \end{enumerate}
        We can consider the loss matrix:\\
        \begin{tabular}{|*{3}{c|}}
            \hline
            \textbf{Headers} & $\bm{y=1}$ & $\bm{y=0}$\\
            \hline
            $\bm{\hat{y}=1}$ & 0 & $L_{FP}$\\
            \hline
            $\bm{\hat{y}=0}$ & $L_{FN}$ & 0\\
            \hline
        \end{tabular}
        where $L_{FN}$ is the cost of a false negative and $L_{FP}$ the cost of a false
        positive.

    \item \textbf{ROC curves} From the below table \\
        \begin{tabular}{|cc|*{3}{c|}}
            \hline
            \multicolumn{2}{|c}{\textbf{Headers}} & \multicolumn{2}{|c|}{\textbf{Truth}} &
            \textbf{Count}\\
            \hline
            \multirow{2}{*}{\textbf{Estimate}} & 1 & $TP$ & $FP$ & $\hat{N}_{+}=TP + FP$\\
                                               & 0 & $FN$ & $TN$ & $\hat{N}_{-}=FN + TN$\\
            \hline
            \multicolumn{2}{|c|}{\textbf{Count}} & $N_{+}=TP+FN$ & $N_{-}=FP+TN$ 
                                               & $N=N_{+}+N_{-}=\hat{N}_{+}+\hat{N}_{-}$\\
            \hline
        \end{tabular}\\
        we can generate the \emph{confusion matrix} is the below table\\
        \begin{tabular}{|*{3}{c|}}
            \hline
            \textbf{Headers} & $\bm{y=1}$ & $\bm{y=0}$\\
            \hline
            $\bm{\hat{y}=1}$ & $\dfrac{TP}{N}$ (sensitivity/recall) 
                           & $\dfrac{FP}{N}$ (error type I/ false alarm) \\
            \hline
            $\bm{\hat{y}=0}$ & $\dfrac{FN}{N}$ (error type II/ missed detection) 
                           & $\dfrac{TN}{N}$ (specificity) \\
            \hline
        \end{tabular}
    \item \textbf{Precision recall curves}
        When trying to detect a rare event the number of negatives is very large, hence
        comparing \emph{sensitivity} and \emph{the error of type I} is not very 
        informative. We would then like to use a measure that only talks about positives.
        \begin{itemize}
            \item \textbf{precision} $=\dfrac{TP}{\hat{N}_{+}}$
            \item \textbf{recall} $=\dfrac{TP}{N_{+}}$
        \end{itemize}
        A \textbf{precision recall curve} is a plot of \textit{precision} vs 
        \textit{recall}.
    \item \textbf{F-scores} is the \emph{harmonic mean of precision and recall}:\\
        $F_{1} \triangleq \dfrac{2}{\frac{1}{precision} + \frac{1}{recall}}$
\end{itemize}





\chapter{Frequentist approach}
\section{Sampling distribution}
\subsection{Sampling Distributions of an estimator}
In frequentist statistic a \tB{parameter estimate $\hat{\theta}$ is computed by 
applying an estimator $\delta$ to some data $\mathcal{D}$, so \textbf{$\hat{\theta}=
\delta( \mathcal{D})$}}.
The \tR{uncertainty in the parameter estimate can be measured by computing the 
\emph{sampling distribution of the estimator}}. Imagine sampling many different 
datasets $\mathcal{D}^{(s)}$ from some true model 
$p(\cdot|\theta^{*})$ meaning \tB{$\mathcal{D}^{(s)}=\left\{x_{i}^{(s)}\hookrightarrow 
p(\cdot|\theta^{*})\right\}_{1\leq i\leq N}$ for $1\leq s\leq S$} and $\theta^{*}$ is 
the true parameter. Now apply the estimator $\hat{\theta}(\cdot)$ to each 
$\mathcal{D}^{(s)}$ to get a \tB{set of estimates $\{\hat{\theta}(\mathcal{D}^{(s)})
\}_{1\leq s\leq S}$}.\\
As we late $S \rightarrow \infty$, the distribution induced on $\hat{\theta}(\cdot
)$ is the \tR{\textbf{sampling distribution of the estimator}.}

\subsection{Bootstrap}
It is a \tB{simple \emph{Monte Carlo} technique to approximate the sampling 
distribution}.
The idea is that \tR{if we knew the true parameters $\theta^{*}$, we could generate $S$
fake datasets of size $N$}, from the true distribution. We could then compute our 
estimator from each sample, and use the empirical distribution of the resulting samples
as our estimate of the sampling distribution.\\
\tR{Since $\theta^{*}$ is unknown}, the idea of the \tB{\textbf{parametric bootstrap} 
is to generate the samples using $\hat{\theta}(\mathcal{D})$ instead}.
An alternative, called \tB{\textbf{non-parametric bootstrap} is to sample the $x_{i}^{s}$
(with replacement) from the original data $\mathcal{D}$ and then compute the induced
distribution} as before.


\section{Fequentist decision theory}
In Frequentist decision theory there is a loss function and a likelihood, but there
is no prior and hence no posterior or posterior expected loss. Thus there is no 
automatic way of deriving an optimal estimator, unlike the Bayesian case.\\
Instead, we are free to choose any estimator or decision procedure $\delta: \mathcal{X}
\rightarrow \mathcal{A}$ we want.\\
Having chosen an estimator, we define its \tR{expected loss or risk} as follows
\tR{
\begin{center}
 $R(\theta^{*}, \delta) \triangleq \mathbb{E}_{p(\tilde{\mathcal{D}}|\theta^{*})}
\left(L(\theta^{*}, \delta(\tilde{\mathcal{D}}))\right) = \Su{}{}L\left(\theta^{*},
\delta(\tilde{\mathcal{D}})\right)p(\tilde{\mathcal{D}})d\tilde{\mathcal{D}}$   
\end{center}
}
where $\tilde{\mathcal{D}}$ is data sampled from 'nature's distribution' which is 
represented by parameter $\theta^{*}$.
Whereas the \tR{Bayesian posterior expected loss}: 
\tR{
\begin{center}
    $p(a, \mathcal{D, \pi}) \triangleq \mathbb{E}_{p(\theta|\mathcal{D},\pi)}\left(
        L(\theta, a)\right) = \Su{\Theta}{}L(\theta, \bm{a})p(\theta|\mathcal{D}, \pi)
        d\theta
    $
\end{center}
}
We see that the Bayesian approach averages over $\theta$, which is unknown, and conditions
on $\mathcal{D}$ which is known. Unlike the frequentist approach averages over $\tilde{
\mathcal{D}}$, thus ignoring the observed data, and conditions on $\theta^{*}$ which is 
unknown.

\paragraph{Bayes risk}
How to chose amongst the estimators? 
We need some way to convert $R(\theta^{*}, \delta)$ into single measure of quality, $R(
\delta)$ which does not depend on knowing $\theta^{*}$. One approach is to \uR{put a prior
on $\theta^{*}$ and then to define} \tR{\textbf{Bayes risk}} of an estimator as follows:
\tR{
\begin{center}
    $R_{B}(\delta) \triangleq \mathbb{E}_{p(\theta^{*})}\left(R(\theta^{*}, \delta)\right)
    =\Su{}{}R(\theta^{*}, \delta)p(\theta^{*})d\theta^{*}$
\end{center}
}
A \tR{\textbf{Bayes estimator} or \textbf{Bayes decision rule}} is one which minimizes the
expected risk: \tR{$\delta_{B} \triangleq \displaystyle\argmin_{\delta} R_{B}(\delta)$}

\subparagraph{Connection Bayesian and Frequentist approaches to decision theory.}
\begin{itemize}
    \item \emph{Theorem 1} \uB{A Bayes estimator can be optained by minimizing the 
            posterior expected loss for each $\bm{x}$}
    \item \emph{Theorem 2} \tB{Every admissible frequentist decision rule is a Bayes 
            decision rule with respect to some possibly improper prior distribution}.
\end{itemize}

\subparagraph{Minimax risk}
Some frequentist statistic users avoid using Bayes risk since it requires the choice of
a prior, although this is only in the evaluation of the estimator, not necessarily as 
part of its construction. An alternative approach is as follows:
\begin{enumerate}
    \item Define the  maximum risk of an estimator as:\\
        $R_{max}(\delta) \triangleq \displaystyle\max_{\theta^{*}}R(\bm{\theta}^{*},
        \delta)$
    \item A \tB{\textbf{minimax rule}} is one which minimizes the maximum risk:
        \tB{$\delta_{MM} \triangleq \displaystyle\argmin_{\delta} R_{\max}(\delta)$} 
\end{enumerate}
\uB{Minimax estimators have a certain appeal, however computing them can be hard and 
furthermore they are very pessimistic}.
In most statistical situations, excluding games theoretic ones, assuming nature is an
adversary is not a reasonable assumption.


\paragraph{Admissible estimators}
The basic problem with frequentis decision theory is that it relies on knowing the true
distribution $p(\cdot|\theta^{*})$ in order to evaluate the risk. However it might be 
the case that some estimators are worse than others regardless of the value of 
$\theta^{*}$.\\
In particular \tB{if for $\theta \in \Theta, R(\theta, \delta_{1}) \leq R(\theta, 
    \delta_{2})$ bayesthen we say that $\delta_{1}$ \textbf{dominates} $\delta_{2}$}.\\
An estimator is said to be \textbf{admissible} if it is not strictly dominated by any 
other estimator.\\
\textbf{Admissibility is not enough}


\section{Desirable properties of estimators}
\subsection{Consistent estimators}
An estimator is said to be \textbf{consistent} \tB{if it eventually recovers the true
parameters that generated the data as the sample size goes to infinity}. 

\subsection{Unbiased estimator}
The \textbf{bias} of an estimator is defined as 
\tR{
\begin{center}
    $bias\left(\hat{\theta}(\cdot)\right) = \mathbb{E}_{p(\mathcal{D}|\theta^{*})}
    \left(\hat{\theta}(\mathcal{D})-\theta^{*}\right)$
\end{center}
}
The estimator is \tB{unbiased when the bias is equal to 0}.

\subsection{Minimum variance estimators}
A famous result called the \textbf{Cramerè-Rao lower bound} provides a lower bound on the
variance of any unbiased estimator. More precisely:
Let $\left(X_{j}\right)_{1 \leq j \leq p} \hookrightarrow p(X|\theta_{0})$ and 
\tB{$\hat{\theta}(\cdot)$ an unbiased estimator of $\theta^{*}$}
Then, under various smoothness assumptions on $p(X|\theta_{0})$ we have  
\tR{
\begin{center}
    $\mathbb{V}(\hat{\theta}) \geq \dfrac{1}{nI(\theta^{*})}$
\end{center}
}
where \tB{$I(\theta^{*})$ is the Fisher information matrix}.

\subsection{Bias-Variance Trade-off} 
As $MSE = variance + bias^{2}$\\
\tB{It might be wise to use a biased estimator, so long as it reduces our variance},
assuming our goal is to minimize squared error. 


\section{Empirical Risk Minimization}
\subsection{Frequentist issue}
\tB{Frequentist decision theory suffers from the fundamental problem that one cannot 
actually compute the risk function, since it relies on knowing the true data 
distribution}. By contrast, the Bayesian posterior expected loss can always be computed 
since it conditions on the data rather that on $\theta^{*}$.\\
\uB{However there is one setting which avoids this problem, it is when the task is to 
predict observable quantities, as opposed to estimating hidden variables or parameters}.\\
\tB{Instead of looking at loss functions of the form $L(\bm{\theta^{*}}, 
\delta(\mathcal{D}))$ let us look at loss functions of the form $L(y, \delta(\bm{x}))$}.\\
Then the risk becomes:
$R(p_{*}, \delta) \triangleq \mathbb{E}_{(\bm{x}, y)\hookrightarrow p_{*}}\left(L(y,
\delta(\bm{x}))\right) = \su{\bm{x}}{}\su{\bm{y}}{}L(y, \delta(\bm{x}))p_{*}(\bm{x}, y)$
Where $p_{*}$ represents "nature's distribution", indeed this distribution is unknown, 
but a simple approach is to use the empirical distribution, derived from some training 
data to approximate $p_{*}(x,y)\approx
p_{emp}(x,y) \triangleq \dfrac{1}{N}\su{i=1}{N}\delta_{x_{i}}(\bm{x})\delta_{y_{i}}(y)$
We define the empirical risk as follows:
\tR{
\begin{center}
    $R_{emp}(\mathcal{D}, \mathcal{D}) \triangleq R(p_{emp}, \delta) = \dfrac{1}{N}
    \su{i=1}{N}L(y_{i}, \delta(\bm{x}_{i}))$
\end{center}
}


\subsection{Regularized risk minimization}
\begin{center}
    $R'(\mathcal{D}, \delta) = R_{emp}(\mathcal{D}, \delta) + \lambda C(\delta)$
\end{center}
where $C(\delta)$ measures the complexity of the prediction function $\delta(\bm{x})$ and 
$\lambda$ controls the strength of the complexity penalty. 
This approach is known as \textbf{regularized risk minimization}.


\section{Components}
\paragraph{Introduction}
Avoid treating parameters as randome variables.
The notion of variation across repeated trials forms the basis for modelling
uncertainity.


\paragraph{Hypothesis Testing}
A \emph{frequentist} statistics, probabilities represent the frequencies at which 
particular events happen.

\paragraph{\emph{p-value}}
It is the heart of frequentist hypothesis testing, it tells us the \tB{probability of 
getting a particular test statistic $t$ as big as the one we have or bigger under the null
hypothesis} (that there is actually no effect).\\
By convention we usually conclude an effect is \emph{statistically significant} if the 
\emph{p-value} is less than a threshold $\alpha$.

\paragraph{Confidence intervals}
When we fit a model to our data we look for the \emph{maximum of likelihood} parameters,
meaning the parameters that are most consistent with our data. 
For each parameter we will able to construct $95\%$ interval namely \tB{$95$ of the $100$ 
intervals generated will contain the true value of the parameter}.\\
If $H_{0}: \beta=0$ is true, the probability of getting a $95\%$ confidence interval that
does not include 0 is less than 0.05. In other words, if the $95\%$ confidence does not 
include 0, $p<0.05$.
\begin{figure}[H]
	\begin{center}
		\includegraphics[width=\textwidth]{./chapters/2_statistics/04_frequentist_approach/4_images/1_estimates.png}
	\end{center}
	\caption{Confidence interval}
	\label{fig:2.4.1_estimates}
\end{figure}

\paragraph{Multiple comparisons}
The more tests we run the more likely it is to we'll find at least one that is significant
even though the null hypothesis is true. We can then apply a Bonferroni correction.\\
Let's say we are running $k$ tests, we can either adjust: 
\begin{itemize}
	\item the threshold $\alpha_{adj} = \dfrac{\alpha}{k}$ OR
	\item the \emph{p-value} $p_{ajd} = k\times p$
\end{itemize}




\section{Power Analysis}
It has as general purpose \tB{to find the right sample number}.
\begin{figure}[H]
    \begin{center}
        \includegraphics[width=\textwidth]{./chapters/2_statistics/04_frequentist_approach/6_images/1_statistical_hypothesis_testing.png}
    \end{center}
    \caption{caption}
    \label{fig:4.6.1_statistical_hypothesis_testing}
\end{figure}

\begin{center}
    \begin{tabular}{*{3}{|c}|}
    \hline
    & \textbf{$H_{0}$ \emph{is True}} & \textbf{$H_{1}$ \emph{is True}}\\
    \hline
        \textbf{Do not reject $H_{0}$} & \tG{Right decision} & \tR{Type II Error 
        \~{} $\beta$}\\
    \hline
        \textbf{Reject $H_{0}$} & \tR{Type I Error \~{} $\alpha$} & \tG{Right decision}\\
    \hline
    \end{tabular}
\end{center}

\paragraph{Power of the test}
Start by defining: \tB{$Power = 1-\beta$}, considering $H_{1}$ true it is the probability 
to correctly reject $H_{0}$

\paragraph{Significant threshold}
Then propose \tB{$\alpha$, the probability to wrongly reject $H_{0}$}. It will be the
reference to which the \emph{p-value} will be compared, \uB{the statistical test will be 
significant ($H_{0}$ rejected) if $\text{\emph{p-value}} \leq \alpha$}.

\paragraph{Effect size}
\uB{It quantifies how meaningful the relationship between variables or the difference 
between group is}, it indicates a practical significance.\\
\tB{While statistical significance (\emph{p-value}) shows the existence of an effect, 
practical significance (\emph{effect size}) shows if this effect is large enough to be 
meaningful in the real world.}\\
There are dozens of measures for effect sizes, and the most common are \emph{Cohen's d} 
and \emph{Pearson's r}.




\chapter{Common statistical tests}
\subsection{Use of statistical tests}
\subsection{Terms}
\begin{itemize}
    \item \emph{Paired} samples: one-to-one correspondence between data in the first and
        second set.
    \item \emph{Matched} samples: every subject in one group with an equivalent in 
        another.
\end{itemize}


\subsection{Table of statistical hypothesis test}
\label{statistical_method_table}{Statistical method table.}\\ 
\begin{tabular}{|*{4}{b{4cm}|}}
\hline
& \textbf{Binomial/Discrete} & \textbf{Continuous, from Normal distribution} & 
\textbf{Continuous measurement\newline (Score/Rank), from non-Normal distribution} \\
\hline
    \textbf{Example of data sample} & List of patients recovering or not after a 
    treatment & Reading of heart pressure from several patients & Ranking of several
    treatment efficiency\\
\hline
    \textbf{Describe one data sample} & Proportions & Mean, Standard Deviation & Median \\
    \hline
    \emph{Compare one data sample to a hypothetical distribution} & $\chi^{2}$ / \emph{G-test} or \label{binomial_test}{Binomial test} & 1-sample t-test & Sign test or Wilconox test \\
    \hline
    \emph{Compare 2 paired samples} & Sign test & Paired t-test & Sign test or Wilconox test \\
    \hline
    \emph{Compare 2 unpaired samples} & $\chi^{2}$ / \emph{G-test} or Fisher's extract test & Unpaired t-test & Mann-Whitney test \\
    \hline
    \emph{Compare 3 or more unmatched samples} & $\chi^{2}$ / \emph{G-test} & 1-way ANOVA & Kruksal-Wallis test \\
    \hline
    \emph{Compare 3 or more matched samples} & Cochrane Q test & Repeated-measures ANOVA & Friedman test\\
    \hline
    \emph{Quantify association between 2 paired samples} & Contingency coefficients & Pearson correlation & Sperman correlation \\
    \hline
\end{tabular}


\subsection{List of common statistical test}
\paragraph{\hyperref[binomial_test]{Binomial}}
To check if the deviations from a theoretically expected distribution of observations into
2 categories.\\

\subparagraph{Assumptions}
\begin{itemize}
    \item \tB{Sample items are independent.}
    \item Items are dichotomous and nominal.
    \item The sample size is significantly less than the population size
    \item The sample is a fiar representation of the population
\end{itemize}


\subparagraph{Frequentist}
Let define a user-defined probability $p_{0}$, with $H_{0}: p = p_{0}$ and
$\begin{cases}
    H_{1}: p \neq p_{0}\text{: two-tailed test} \\
    H_{1}: p < p_{0}\text{: left-tailed test} \\
    H_{1}: p > p_{0}\text{: right-tailed test} \\
     
\end{cases}$

\subparagraph{Bayesian}
Define the prior distribution with a \emph{Beta}($a,b$) distribution\\

\textit{\hyperref[statistical_method_table]{Return to the table.}}


\paragraph{$\bm{\chi^{2}}$ test}
Either used to test \emph{goodness-of-fit} and \emph{independence} between 2 variables.
It checks either if there is a significant difference between the expected and observed 
frequencies.
\begin{itemize}
    \item \emph{goodness-of-fit}: expected frequencies are computed with a theoretical 
        relationship between observed frequencies
    \item \emph{independence}: expected frequencies are computed with observed frequencies
        from the other sample
\end{itemize}
\subparagraph{Assumptions}
\begin{itemize}
    \item simple random sample
    \item sample with a sufficiently large size is assumed, for small sample size see Cash test
    \item expected cell count has to be adequate, a rule of thumb is at least 5 for 2-by-2
        table and 5 or more in 80\% of cells in larger tables.
    \item Independence of the observations
\end{itemize}

\subparagraph{Frequentist}
$\chi^{2} = \su{i=1}{n}\dfrac{\left(\frac{O_{i}}{N} - p_{i} \right)^{2}}{p_{i}}
\begin{cases}
    O_{i}\text{: number of observations of type }i \\
    N\text{: total number of observations}\\
    n\text{: number of cells in the table.}
    p_{i}\text{: expected proportions of the fraction of type}i\text{ in the population.}
\end{cases}
$

\subparagraph{Bayesian}
Does not exist, see \emph{contingency table}\\

\textit{\hyperref[statistical_method_table]{Return to the table.}}


\paragraph{Exact test of goodness-of-fit}
Unlike the conventional statistical tests, there is no \emph{test statistic}, we directly
compute the \emph{p-value} under the null hypothesis.
The most common use are for dichotomous nominal variables or multinomial variables.

\subparagraph{Assumptions}
\begin{itemize}
    \item \tB{Observations are independent.}
    \item Small sample size $\lessapprox 1000$
\end{itemize}

\subparagraph{Frequentist}
Let us define the list of, respectively, expected counts for each modality 
$i,~(E_{i})_{1\leq i\leq m}$, and observed counts $(O_{i})_{1\leq i\leq m}$.
Then
$\begin{cases}
    \bm{H_{0}: \forall i\in \inter{1}{m},~O_{i} = E_{i}}\\
    H_{1}: \exists i\in \inter{1}{m},~O_{i} \neq E_{i}\text{: two-tailed test} \\
\end{cases}$




\paragraph{Fisher's exact test}
To check the significance of the contingency between 2 kinds of classification of a given
object, \uB{initially Fisher used this test to distinguish drink in which the tea has been
put before the milk or vice-versa}. For large sample use \emph{G-test}

\subparagraph{Assumptions}
\begin{itemize}
    \item In practice, small sample size $\lessapprox 1000$
\end{itemize}

\subparagraph{Frequentist}
For example let's divide a population into male and female and for each persons indicating
if this person is currently studying or not. We want to test if the proportion of studying
students is higher among the women than among the men.
\begin{center}
    \begin{tabular}{|*{4}{c|}}
    \hline
    & \textbf{Men} & \textbf{Women} & \textbf{Row Total}\\
    \hline
    \textbf{Studying} & $a$ & $b$ & $a+b$ \\
    \hline
    \textbf{Non-Studying} & $c$ & $d$ & $c+d$ \\
    \hline
    \textbf{Column Total} & $a+c$ & $b+d$ & $a + b + c + d = n$ \\
    \hline
    \end{tabular}
\end{center}
The conditional on the margins of the table is distributed as $\text{\emph{Hypergeometric}}(a+c, a+b, c+d)$ meaning $a + c$ draws from a population with $a + b$ success and $c+d$ 
failures.
The probability of obtaining such set of values is given by
\begin{center}
    $p = \dfrac{{{a+b}\choose{a}}\times{{c+d}\choose{c}}}{{{n}\choose{a+c}}}
    = \dfrac{{{a+b}\choose{b}}\times{{c+d}\choose{d}}}{{{n}\choose{b+d}}}$
\end{center}

\subparagraph{Bayesian}
Does not exist, see \emph{contingency table} \\

\textit{\hyperref[statistical_method_table]{Return to the table.}}


\paragraph{G-test}

It's a likelihood-ratio or a maximum likelihood statistical significance test. 
Either used to test \emph{goodness-of-fit} and \emph{independence} between 2 variables.
It checks either if there is a significant difference between the expected and observed 
frequencies. This test tends to replace $\chi^{2}$\emph{-test}
\begin{itemize}
    \item \emph{goodness-of-fit}: expected frequencies are computed with a theoretical 
        relationship between observed frequencies
    \item \emph{independence}: expected frequencies are computed with observed frequencies
        from the other sample
    \item \emph{repeated tests}: first variable is analysed with a goodness-of-fit and the
        second one represents the repetition of the experiments multiple times. Thus it
        allows to assess the goodness-of-fit on a large sample instead of multiple lower
        samples. Expected frequencies is a theoretical relationship between the observed
        frequencies segmented in groups by the modalities of the second variable.
\end{itemize}


\subparagraph{Assumptions}
\begin{itemize}
    \item 
    \item Expected count must not be small in any modality.
\end{itemize}

\subparagraph{Strengths}
\begin{itemize}
    \item Approximation to the theoretical $chi^{2}$ distribution is better attained with
        \emph{G-test} than $\chi^{2}$ \emph{test}.
    \item Cases where $O_{i} > 2\times E_{i}$, \emph{G-test} is always better than
        $\chi^{2}$ \emph{test}.
\end{itemize}

\subparagraph{Weaknesses}
\begin{itemize}
    \item in test of independence, for a small sample size use rather Fisher's extract 
        test.
\end{itemize}


\subparagraph{Frequentist}
We compare the observed counts in each modality with their expected counts.
Let us define the list of, respectively, expected counts for each modality 
$i,~(E_{i})_{1\leq i\leq m}$, and observed counts $(O_{i})_{1\leq i\leq m}$.
Then
$\begin{cases}
    \bm{H_{0}: \forall i\in \inter{1}{m},~O_{i} = E_{i}}\\
    H_{1}: \exists i\in \inter{1}{m},~O_{i} \neq E_{i}\text{: two-tailed test} \\
\end{cases}$

\begin{center}
    $G = 2\su{i=1}{m}O_{i}\times\ln\left(\dfrac{O_{i}}{E_{i}}\right)$
\end{center}

$$
\ln\left(\dfrac{L(\tilde{\theta}|x)}{L(\hat{\theta}|x)}\right)
= \ln\left(\dfrac{\prd{i=1}{m}\tilde{\theta}^{x_{i}}}{\prd{i=1}{m}\hat{\theta}^{x_{i}}}\right)
= \ln\left(\dfrac{\prd{i=1}{m}\left(\frac{x_{i}}{n}\right)^{x_{i}}}{\prd{i=1}{m}\left(\frac{e_{i}}{n}\right)^{x_{i}}}\right)
= \ln\left(\prd{i=1}{m}\left(\dfrac{x_{i}}{e_{i}}\right)^{x_{i}}\right)
= \su{i=1}{m}x_{i}\ln\left(\dfrac{x_{i}}{e_{i}}\right)
$$

Then we multiply by $-2$ to get \emph{G-test} that is asymptotically equivalent to the 
\emph{Pearson's} $\chi^{2}$ formula.


\paragraph{Cochran's Q test}
It checks if $k$ treatments have identical effect, the response can take only 2 possible 
outcomes and a second variable segments the treatments.

\begin{center}
    \begin{tabularx}{.65\textwidth}{|*{5}{c|}}
    \hline
     & \textbf{Treatment 1} & \textbf{Treatment 2} & $\cdots$  & \textbf{Treatment k}\\
    \hline
    \emph{Block 1} & $x_{11}$ & $x_{12}$ & $\cdots$  & $x_{1k}$\\
    \hline
    \emph{Block 2} & $x_{21}$ & $x_{22}$ & $\cdots$  & $x_{2k}$\\
    \hline
    \emph{Block 3} & $x_{31}$ & $x_{32}$ & $\cdots$  & $x_{3k}$\\
    \hline
    $\vdots$ & $\vdots$ & $\vdots$ & $\ddots$ & $\vdots$ \\
    \hline
    \emph{Block b} & $x_{b1}$ & $x_{b2}$ & $\cdots$  & $x_{bk}$\\
    \hline
    \end{tabularx}
\end{center}
And $\forall (i,j)\in\inter{1}{b}\times\inter{1}{k}, x_{ij}\in\left\{0, 1\right\}$

\subparagraph{Assumptions}
\begin{itemize}
    \item The blocks are randomly selected from the population of all possible blocks.
    \item Outcome of the treatments are dichotomous, and should be coded in a standard way
\end{itemize}

\subparagraph{Frequentist}
For example if $b$ respondents in a survey had each been asked $k$ \emph{Yes/No} questions
the \emph{Q-test} could be use to test the null hypothesis that all questions were equally
likely to elicit the answer "Yes".\\
We have
$\begin{cases}
    H_{0}\text{: the treatments are equally effective} \\
    H_{a}\text{: the treatments are \emph{not} equally effective} 
\end{cases}$

\begin{center}
    $
    T = k(k-1)\dfrac{\su{j=1}{k}\left(x_{\cdot j} - \frac{N}{k}\right)^{2}}{
    \su{i=1}{b}x_{i\cdot}\left(k-x_{i\cdot}\right)}
    \begin{cases}
        k\text{: number of treatments} \\
        x_{\cdot j}\text{: column total for the }j\text{th treatment} \\
        b\text{: number of blocks} \\
        X_{i\cdot}\text{: row total for the }i\text{th block} \\
        N\text{: grand total}
         
    \end{cases}$
\end{center}

For significance level $\alpha$, the asymptotic critical region is 
$T > \chi^{2}_{1-\alpha, k-1}$ which is the ($a-\alpha$) quantile of the $\chi^{2}$ 
distribution with $K-1$ degrees of freedom.

\subparagraph{Bayesian}
Does not exist, see \emph{contingency table} \\



\paragraph{Sign test}
It is a statistical method to test for consistent differences between pairs of 
observations, such as the weight of subjects before and after treatment.
For comparisons of paired observations $(x, y)$ the \emph{sign-test} is most useful if 
comparison can only be expressed as $x>y,~x=y,\text{ or } x<y$.
If instead the differences can be expressed in numeric quantities it is worthy to use 
\emph{t-test} or \emph{Wilcoxon signed-rank test} will usually have greater power than
the sign test to detect consistent differences.

\subparagraph{Frequentist}
Let $p=\Prob{X>Y}$, then
$\begin{cases}
    H_{0}:~p=0.5\text{ meaning that given a random pair of measurements }(x_{i}, y_{i})
    \text{ then }x_{i}\text{ and }y_{i}\text{are equally likely to be larger than the
    other}\\
    H_{a}:~p\neq 0.5
\end{cases}$
Pairs are omitted for which there is no differences so that there is a potential reduced
sample of $m$ pairs.\\
The statistics $W$ is defined as follow:
\begin{center}
    $W = \mathbf{1}_{\left\{x_{i} > y_{i}\right\}} \hookrightarrow \mathcal{B}(m, 0.5)$
\end{center}

\subparagraph{Assumptions}
Let $\forall i\in\inter{1}{n},~Z_{i} = X_{i} - Y_{i}$

\begin{itemize}
    \item $Z_{i}$ are assumed independent.
    \item Each $Z_{i}$ comes from the same continuous population.
    \item The values $X_{i}$ and $Y_{i}$ are ordered.
\end{itemize}

\subparagraph{Strengths}
\begin{itemize}
    \item A fewer assumptions need to be made than for parametrical test
\end{itemize}

\subparagraph{Weaknesses}
\begin{itemize}
    \item The power of test is lower than for a parametrical test
\end{itemize}



\paragraph{Contingency coefficients}
\paragraph{Wilconox test}
\paragraph{Mann-Whitney test}
\paragraph{Kruksal-Wallis test}
\paragraph{Friedman test}
\paragraph{Sperman test}
\paragraph{Pearson correlation}
\paragraph{Repeated-measures ANOVA}
\paragraph{1-way ANOVA}
\paragraph{Unpaired t-test}
\paragraph{Paired t-test}
\paragraph{1-sample t-test}




\chapter{Data revovery}
\section{Sampling methods}
\subsection{Monte Carlo approximation}
\paragraph{Purpose}
computing the distribution of a random variable's function using the change of variables formula can
be difficult.\\
As simple and powerful alternative is to \tB{generate $S$ samples $(x_{s})_{1\leq s\leq S}$ from the 
distribution}.
\paragraph{Theory}
Given the samples we can approximate the distribution of $f(X)$ by using the empirical 
distribution of $\left\{f(x_{s})\right\}_{1\leq s\leq S}$\\
We can use Monte Carlo to approximate the expected value of any function of a random variable:
\begin{center}
    \fr{$\E{f(X)} = \Su{}{}f(x)p(x)dx\approx \dfrac{1}{S}\su{s=1}{S}f(x_{s})$}
\end{center}
where $x_{s}\hookrightarrow p(X)$. This called Monte Carlo integration
\paragraph{Accuracy}
It increases with sample size.\\
In denoting $\mu = \E{f(X)}$ the exact mean and $\hat{\mu}$ the MC approximation, one can show that
\begin{center}
    $(\hat{\mu} - \mu) \hookrightarrow \mathcal{N}\left(0, \dfrac{\sigma^{2}}{S}\right)$
\end{center}
where $\sigma^{2} = \V{f(X)}$ \uB{this is a consequence of central-limit theorem}, of course 
$\sigma^{2}$ is unknown.\\
\tR{$\sqrt{\frac{\hat{\sigma}^{2}}{S}}$ is called the \emph{standard error} and is an estimate of our
uncertainty about our estimate $\mu$}

\paragraph{Strengths}
\begin{itemize}
    \item function only evaluated in places where there is non-negligible probability: advantage over
        numerical integration 
\end{itemize}
\paragraph{Examples}
\begin{itemize}
    \item estimating $\pi$
\end{itemize}

\subsection{Bootstrap}
\paragraph{Purpose}
In practice 32 bootstrap seems to be a good compromise between speed and accuracy.\\
\tB{It is a Monte Carlo technique to approximate the sampling distribution.}
\paragraph{Theory}
If we knew the true parameters $\theta^{*}$ we could generate $S$ fake datasets of size
$N$ from the distribution $\forall (i,s)\in\inter{1}{n}\times\inter{1}{s}, x^{s}_{i}
\hookrightarrow p(\cdot|\theta^{*})$\\
We could then compute our estimator from each sample: $\hat{\theta}^{s} = f(x^{s})$.\\
Then 2 approaches:
\begin{itemize}
    \item \emph{parametric}: generate the samples using $\hat{\theta}(\mathcal{D})$ as
        $\theta$ is unknown
    \item \emph{non-parametric}: sample the $\left(x^{s}\right)_{1\leq s\leq S}$ with
        replacement from the original $\mathcal{D}$ and then compute induced 
        distribution
\end{itemize}

\paragraph{Connection between $\hat{\theta}^{s} = \hat{\theta}(x^{s})$ and $\theta^{s}
\hookrightarrow p(\cdot|\mathcal{D})$}
Conceptually quite different, but in the common case the prior is not very strong
they can be quite similar. 
One can think of the \tB{bootstrap distribution distribution as a "poor man's" 
posterior}.


\paragraph{Strengths}
\begin{itemize}
    \item Useful \uB{when the estimator is a complex function of of the true parameter}
\end{itemize}

\subsection{Monte Carlo Inference}
\paragraph{Sampling from a Gaussian (Box-Muller method)}
The idea is we sample uniformly from a unit radius circle and then use the change of variables to
derive samples from spherical 2d Gaussian.

\paragraph{Rejection sampling}
\subparagraph{Basic idea}
we create a \emph{proposal distribution} $q(x)$ which satisfies 

\paragraph{Importance sampling}
\subparagraph{Basic idea}
The idea is to draw samples $\bm{x}$ in regions which have high probability $\prob{\bm{x}}$

\subsection{Particle filtering}
it is a Monte Carlo algorithm for recursive Bayesian inference

\subsection{Annealing methods}
COMPLETE


%\subsection{Logistic Regression}
%\paragraph{Purpose}
%\paragraph{Assumptions}
%\paragraph{Theory}
%\paragraph{Strengths}
%\paragraph{Weaknesses}
%\paragraph{Relationships with other methods}
%\paragraph{Examples of application}



\section{Information theory}
\subsection{Entropy}
\paragraph{Purpose}
It is a \tB{measure of the uncertainty in a random variable}.
\paragraph{Theory}
\begin{center}
    \fr{$\mathbb{H}(X) \triangleq -\su{k=1}{K}\prob{X=k}\log\left(p(X=k)\right)$}
\end{center}
The discrete distribution with \uB{maximum entropy is the uniform distribution},
conversely the one with \uB{minimum entropy is any delta-function} that puts all its
mass on one state.


\subsection{Kullback-Leibler (KL) divergence}
\paragraph{Purpose}
It is a \tB{measure of dissimilarity of 2 probability distribution $p$ and $q$}.
\paragraph{Theory}
\tB{\begin{align*}
    \mathbb{KL}(p||q) &= \su{k=1}{K}p_{k}\log\left(\dfrac{p_{k}}{q_{k}}\right)\\
                      &= \su{k=1}{K}p_{k}\log(p_{k}) - \su{k=1}{K}p_{k}\log(q_{k})\\
                      &= -\mathbb{H}(p) + \mathbb{H}(p,q)
\end{align*}}
Where $\mathbb{H}(p,q)$ is called \emph{cross entropy}, \tB{being the average number of
bits needed to encode data coming from a source with distribution $p$ when we use model
$q$ to define our codebook}.

\subsection{Mutual information}
\paragraph{Purpose}
Correlation coefficient is quite restrictive, a more general approach is to determine
\tB{how similar the joint distribution $p(X,Y)$ is to the factorized distribution $p(X)
p(Y)$}

\paragraph{Theory}
\subparagraph{Discrete}
\begin{center}
    \fr{$\mathbb{I}(X,Y) \triangleq \mathbb{KL}\left(p(X,Y)||p(X)p(Y)\right) =
    \mathbb{H}(X)- \mathbb{H}(X|Y) = \mathbb{H}(Y)-\mathbb{H}(Y|X)$}
\end{center}
where $\mathbb{H}(Y|X)$ is the \tB{\emph{conditional entropy}} defined as 
\tB{$\mathbb{H}(Y|X) =\Sigma_{x}p(x)\mathbb{H}(Y|X=x)$}
Thus we can interpret the \tR{\emph{Mutual Information} between $X$ and $Y$ as the 
reduction in uncertainty about $X$ after observing Y, or, by symmetry about $Y$ after
observing $X$}.

\subparagraph{Continuous}
Quantizing can have significant impact on the results, we could then try to estimate
many different bin sizes and locations to finally compute the maximum MI achieved 
called \emph{maximal information coefficient}:
\begin{center}
    $m(x,y) = \dfrac{\max_{G\in\mathcal{G}(x,y)}\mathbb{I}\left(X(G),Y(G)\right)}{
    \log\left(min(x,y)\right)}$
\end{center}
where $\mathcal{G}(x,y)$ is the set of 2d grids of size $x\times y$ and $X(G), Y(G)$
represents a discretization of the variable onto the grid.
\tB{$MIC\triangleq\max_{x,y:xy<B}m(x,y)$}




\section{Key Mathematical functions}
\subsection{Softmax function}
\paragraph{Purpose}
The softmax function takes as input a vector $z$ of $K$ real numbers, 
and normalizes it into a probability distribution consisting of $K$ 
probabilities proportional to the exponentials of the input numbers.
\begin{center}
    $\sigma(\bm{z})_{i} = \dfrac{e^{z_{i}}}{\su{j=1}{K}e^{z_{j}}}$
\end{center}
\paragraph{Interpretation}
it is rather a smooth approximation of the \emph{argmax} function, 
meaning the function returning the index of the maximum value of a given
vector.






\part{Classical Learning}
\begin{figure}[H]
    \begin{center}
        \includegraphics[width=\textwidth]{chapters/3_images/type_of_ml_methods.png}
    \end{center}
    \caption{Types of machine learning methods}
    \label{fig:type_of_ml_methods}
\end{figure}
\section{Supervised Learning}
\section{Classification}
% Naive Bayes Classifiers
\subsection{Naive Bayes classifiers}
\paragraph{Purpose}
\tR{Classifying vectors of discrete-valuated features $x\in\left\{i
\right\}_{1\leq i \leq K}^{D}$}, where $K$ is the number of values for
each feature, and $D$ the number of features.

\paragraph{Assumptions}
\begin{itemize}
    \item \tB{Features are conditionally independent given the class 
        label}
\end{itemize}

\paragraph{Theory}
As a \emph{generative} model, meaning of the form:
$\prob{y=c|\bm{x}, \bm{\theta}} \propto \prob{\bm{x}|y=c,\bm{\theta}}
\prob{y=c|\bm{\theta}}$. The key of such models is the possibility
to specify a suitable form for the class-conditional density 
$\prob{\bm{x}|y=c, \bm{\theta}}$ which definees what kind of data we 
expect to see in each class. And with the independence assumption we 
have:
\begin{center}
    \tB{$\prob{\bm{x}|y=c, \bm{\theta}} = \prd{j=1}{D}\prob{x_{j}|y=c,\bm{\theta}_{jc}}$}
\end{center}
with all $\prob{x_{j}|y=c,\bm{\theta}_{jc}}$ being able to 
follow a \textit{normal}, \textit{bernoulli} or \textit{multinoulli} 
distribution.\\
\uB{Training a NBC consists in computing the MLE or the MAP estimate for 
the parameters.}\\
For a single observation
$\prob{x_{i}, y_{i}|\bm{\theta}} = \prob{y_{i}|\bm{\pi}}\prd{j}{}\prob{x_{ij}|\bm{\theta}_{j}} = 
\prd{c}{}\pi_{c}^{\mathbbm{1}(y_{i}=c)}\prd{j}{}\prd{c}{}\prob{x_{ij}|
\theta_{jc}}^{\mathbbm{1}(y_{i}=c)}
$\\ 
Hence the \tB{\emph{log-likelihooh}:
$\log\left(\mathcal{D}|\theta\right) = \su{c=1}{C}N_{c}\log(\pi_{c}) + 
\su{j=1}{D}\su{c=1}{C}\su{i:y_{i}=c}{}\log\left(\prob{x_{ij}|
\bm{\theta}_{jc}}\right)$}\\
By optimizing the above equation we are able to find the $\left(\theta_{jc}\right)_{1\leq j \leq D,~ 1\leq c\leq C}$ and we can then use them to
predict the output of an observation $\bm{x}$ as: $\prob{y=c|\bm{x},\mathcal{D}} \propto \prob{y=c|\mathcal{D}}\prd{j=1}{D}\prob{x_{j}|y=c,\mathcal{D}}$

\paragraph{Strengths}
\begin{itemize}
    \item Simple model, for $C$ classes and $D$ features, and hence \tB{relatively immune to 
        overfitting}
\end{itemize}

\paragraph{Weaknesses}
\begin{itemize}
    \item \uB{Unaccuracy} because of the strong independence assumption
\end{itemize}

\paragraph{Relationships with other methods}
\tB{Logistic Regression}: for discrete inputs 
\emph{Naive Bayesian Classifiers} form a 
generative-discriminant pair with \emph{Multinomial Logistic
Regression}: \uB{each NBC can be considered a way of fitting a
probability model that optimizes the joint likelihood 
$\prob{C, \bm{x}}$, while Multinomial Logistic Regression fits the same 
probability to optimize the conditional $\prob{C|\bm{x}}$}

\paragraph{Examples of application}
\begin{itemize}
    \item Classifying documents using bag of words
    \item Determining the gender of a person, based on measured features 
\end{itemize}


% Linear/Quadratic Discriminant Analysis
\subsection{Linear/Quadratic Discriminant Analysis}
\uB{It consists in defining the class conditional densities in a 
generative classifier}: $\prob{\bm{x}|y=c,\bm{\theta}} = \mathcal{N}
\left(\bm{x}, \bm{\mu}_{c},\bm{\Sigma}_{c}\right)$\\
As for a generative classifier we have the following equation: 
\begin{center}
    $\prob{y=c|\bm{x},\bm{\theta}} = \dfrac{\overbrace{\prob{\bm{x}|y=c,
        \bm{\theta}}}^{\text{\emph{class-conditional density}}}
        \overbrace{\prob{y=c|\bm{\theta}}}^{\text{\emph{class prior}}}}{
    \su{c'}{}\prob{y=c'|\bm{\theta}} \prob{\bm{x}|y=c',\bm{\theta}}}$
\end{center}
\paragraph{Purpose of Quadratic Discriminant Analysis}
\begin{center}
    $\prob{y=c|\bm{x},\bm{\theta}} = \dfrac{|2\pi\Sigma_{c}|^{
    -\frac{1}{2}}\exp\left(-\frac{1}{2}[\bm{x}-\bm{\mu}_{c}]^{T}\Sigma_{
    c}^{-1}[\bm{x}-\bm{\mu}_{c}]\right)\pi_{c}}{\su{c'}{}|2\pi\Sigma_{
    c'}|^{-\frac{1}{2}}\exp\left(-\frac{1}{2}[\bm{x}-\bm{\mu}_{c'}]^{T}
    \Sigma_{c'}^{-1}[\bm{x}-\bm{\mu}_{c'}]\right)\pi_{c'}}$
\end{center}
\tB{The threshold of this results will be a quadratic function of 
$\bm{x}$.}

\paragraph{Purpose of Linear Discriminant Analysis}
Same equation than above but this time, \tB{$\forall c\in \inter{1}{C} 
\Sigma_{c} = \Sigma$}, \tB{then quadratic term $\bm{x}^{T}\Sigma^{-1}
\bm{x}$ will cancel out from numerator and denominator}.
Then by considering the above cancellation and the fact that 
evidence is considered as a constant, we have:
\tB{
\begin{align*}
    \prob{y=c|\bm{x},\bm{\theta}} & \propto \exp\left(\log(\pi_{c})+\bm{
        \mu}_{c}^{T}\Sigma^{-1}\bm{x}\bm{\mu}_{c}\right)\\
    &= \exp\left(\bm{\beta}_{c}^{T}\bm{x} + \gamma_{c}\right)
    \end{align*}
}
Note also that we have exactly: \tR{$\prob{y=c|\bm{x},\bm{\theta}} =
\dfrac{e^{\bm{\beta}_{c}^{T}\bm{x} + \gamma_{c}}}{\su{c'}{}e^{\bm{\beta
}_{c'}^{T}\bm{x} + \gamma_{c'}}} = S(\bm{\eta})_{c}$}. With $\eta=\left(
\bm{\beta}_{c}\bm{x} +\gamma_{c}\right)_{1\leq c\leq C}$
We recognize the \emph{softmax} function.

\paragraph{Assumptions}
\begin{itemize}
    \item Independent \tB{variables are normal for each level} of the
        grouping variable.
    \item \tB{Homoscedasticity} for LDA: variances \uB{among group} 
        variables are the same across levels of predictors.
    \item \uB{Independence of the observations}.
\end{itemize}


\paragraph{Theory}
\paragraph{Strengths}
\paragraph{Weaknesses}
\begin{itemize}
    \item Multicollinearity: predictive power can decrease with an 
        increased correlation between predictor variables.
\end{itemize}

\paragraph{Relationships with other methods}
\paragraph{Examples of application}


\subsection{Logistic Regression}
\paragraph{Purpose}
With the generative approach we create a joint model of the form $\prob{y,\bm{x}}$, and
then to condition on $\bm{x}$, thereby deriving $\prob{y|bm{x}}$, it is the \emph{
generative} approach.\\
Alternatively, fitting directly a model of the form $\prob{y|\bm{x}}$ is a \emph{
discriminative} approach.
\paragraph{Assumptions}
\begin{itemize}
    \item Independence
\end{itemize}

\paragraph{Theory}
The data distribution is modelled by : 
\fr{$\prob{y|\bm{x}} = \text{\emph{Bernoulli}} \left(y|\sigma\left(\bm{w}^{T}\bm{x}
\right)\right)$}\\
With $\sigma$ being the \emph{sigmoid} function, such that 
$\sigma = \begin{cases}
    \mathbb{R} \longrightarrow [0, 1]\\ 
    x \mapsto \dfrac{e^{x}}{1 + e^{x}}
\end{cases}
$
\subparagraph{Maximum Likelihood Estimator}
\begin{align*}
    \text{\emph{NLL}}(\bm{w})
    &= -\su{i=1}{N}\log\left(\hat{y}_{i}^{\mathbbm{1}_{\{y_{i} = 1\}}}\left[1 -
    \hat{y}_{i}^{\mathbbm{1}_{\{y_{i}=0\}}}\right]\right)\\ 
    &= -\su{i=1}{N}y_{i}\log\left(\hat{y}_{i}\right) + \left[1 -y_{i}\right]\log
    \left(1 -\hat{y}_{i}\right)
\end{align*}
This called \textit{cross-entropy}


\paragraph{Strengths}
\paragraph{Weaknesses}
\paragraph{Relationships with other methods}
\paragraph{Examples of application}

\subsection{Logistic Regression}
\paragraph{Purpose}
\paragraph{Assumptions}
\paragraph{Theory}
\paragraph{Strengths}
\paragraph{Weaknesses}
\paragraph{Relationships with other methods}
\paragraph{Examples of application}

\subsection{Logistic Regression}
\paragraph{Purpose}
\paragraph{Assumptions}
\paragraph{Theory}
\paragraph{Strengths}
\paragraph{Weaknesses}
\paragraph{Relationships with other methods}
\paragraph{Examples of application}

\subsection{Logistic Regression}
\paragraph{Purpose}
\paragraph{Assumptions}
\paragraph{Theory}
\paragraph{Strengths}
\paragraph{Weaknesses}
\paragraph{Relationships with other methods}
\paragraph{Examples of application}




\part{Deep Learning}
\part{Use-cases}

\printbibliography

\end{document}
