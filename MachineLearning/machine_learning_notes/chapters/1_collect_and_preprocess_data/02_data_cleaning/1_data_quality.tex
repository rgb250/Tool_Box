\section{Validity}
\begin{itemize}
    \item \textbf{Data-Type Constraints}: for a given column a fixed data-type must be associated with.
    \item \textbf{Range Constraints}: only a range of values should be taken.
    \item \textbf{Mandatory Constraints}: some columns cannot be empty.
    \item \textbf{Unique Constraints}: across a given dataset a field or a combination of variables.
    \item \textbf{Foreign-key constraints}: a foreign key column cannot have a value that
        does not exist in the primary key.
    \item \textbf{Regular expression patterns}: text fields that have to follow a given 
        alphanumerical pattern.
    \item \textbf{Cross-field validation}: consistency of values, for example considering
        a given man, his birth date have to be older than his death date.
\end{itemize}

\section{Accuracy}
The degree to which the data is close to the true value.

\section{Completeness}
The degree to which the all the required data is known.

\section{Consistency}
The degree to which the data is consistent, within the same data set or across multiple 
data sets.

\section{Uniformity}
The degree to which the data is specified using the same unit of measure.
