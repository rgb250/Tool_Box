\paragraph{Consistent estimators}
An estimator is said to be \textbf{consistent} \tB{if it eventually recovers the true
parameters that generated the data as the sample size goes to infinity}. 

\paragraph{Unbiased estimator}
The \textbf{bias} of an estimator is defined as 
\tR{
\begin{center}
    $bias\left(\hat{\theta}(\cdot)\right) = \mathbb{E}_{p(\mathcal{D}|\theta^{*})}
    \left(\hat{\theta}(\mathcal{D})-\theta^{*}\right)$
\end{center}
}
The estimator is \tB{unbiased when the bias is equal to 0}.

\paragraph{Minimum variance estimators}
A famous result called the \textbf{Cramerè-Rao lower bound} provides a lower bound on the
variance of any unbiased estimator. More precisely:
Let $\left(X_{j}\right)_{1 \leq j \leq p} \hookrightarrow p(X|\theta_{0})$ and 
\tB{$\hat{\theta}(\cdot)$ an unbiased estimator of $\theta^{*}$}
Then, under various smoothness assumptions on $p(X|\theta_{0})$ we have  
\tR{
\begin{center}
    $\mathbb{V}(\hat{\theta}) \geq \dfrac{1}{nI(\theta^{*})}$
\end{center}
}
where \tB{$I(\theta^{*})$ is the Fisher information matrix}.

\paragraph{Bias-Variance Trade-off} 
As $MSE = variance + bias^{2}$\\
\tB{It might be wise to use a biased estimator, so long as it reduces our variance},
assuming our goal is to minimize squared error. 
