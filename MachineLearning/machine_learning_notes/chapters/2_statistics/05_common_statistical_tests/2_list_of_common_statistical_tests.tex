\paragraph{\hyperref[binomial_test]{Binomial}}
To check if the deviations from a theoretically expected distribution of observations into
2 categories.\\

\subparagraph{Assumptions}
\begin{itemize}
    \item \tB{Sample items are independent.}
    \item Items are dichotomous and nominal.
    \item The sample size is significantly less than the population size
    \item The sample is a fiar representation of the population
\end{itemize}


\subparagraph{Frequentist}
Let define a user-defined probability $p_{0}$, with $H_{0}: p = p_{0}$ and
$\begin{cases}
    H_{1}: p \neq p_{0}\text{: two-tailed test} \\
    H_{1}: p < p_{0}\text{: left-tailed test} \\
    H_{1}: p > p_{0}\text{: right-tailed test} \\
     
\end{cases}$

\subparagraph{Bayesian}
Define the prior distribution with a \emph{Beta}($a,b$) distribution\\

\textit{\hyperref[statistical_method_table]{Return to the table.}}


\paragraph{Fisher's exact test}
To chekc the significance of the contingency between 2 kinds of classification of a given
object, initially Fisher used this test to distinguish drink in which the tea has been put
before the milk or vice-versa.
For large sample use \emph{G-test}

\subparagraph{Assumptions}
\begin{itemize}
    \item In practice, small sample size $\lessapprox 1000$
\end{itemize}

\subparagraph{Frequentist}
For example let's divide a population into male and female and for each persons indicating
if this person is currently studying or not. We want to test if the proportion of studying
students is higher among the women than among the men.
\begin{center}
    \begin{tabular}{|*{4}{c|}}
    \hline
    & \textbf{Men} & \textbf{Women} & \textbf{Row Total}\\
    \hline
    \textbf{Studying} & $a$ & $b$ & $a+b$ \\
    \hline
    \textbf{Non-Studying} & $c$ & $d$ & $c+d$ \\
    \hline
    \textbf{Column Total} & $a+c$ & $b+d$ & $a + b + c + d = n$ \\
    \hline
    \end{tabular}
\end{center}
The conditional on the margins of the table is distributed as $\text{\emph{Hypergeometric}}(a+c, a+b, c+d)$ meaning $a + c$ draws from a population with $a + b$ success and $c+d$ 
failures.
The probability of obtaining such set of values is given by
\begin{center}
    $p = \dfrac{{{a+b}\choose{a}}\times{{c+d}\choose{c}}}{{{n}\choose{a+c}}}
    = \dfrac{{{a+b}\choose{b}}\times{{c+d}\choose{d}}}{{{n}\choose{b+d}}}$
\end{center}

\subparagraph{Bayesian}
Does not exist, see \emph{contingency table} \\


\textit{\hyperref[statistical_method_table]{Return to the table.}}





\paragraph{$\bm{\chi^{2}}$ test}
To check if there is a significant difference between the expected and observed 
frequencies.
\subparagraph{Assumptions}
\begin{itemize}
    \item simple random sample
    \item sample with a sufficiently large size is assumed, for small sample size see Cash test
    \item expected cell count has to be adequate, a rule of thumb is at least 5 for 2-by-2
        table and 5 or more in 80\% of cells in larger tables.
    \item Independence of the observations
\end{itemize}

\subparagraph{Frequentist}
$\chi^{2} = \su{i=1}{n}\dfrac{\left(\frac{O_{i}}{N} - p_{i} \right)^{2}}{p_{i}}
\begin{cases}
    O_{i}\text{: number of observations of type }i \\
    N\text{: total number of observations}\\
    n\text{: number of cells in the table.}
    p_{i}\text{: expected proportions of the fraction of type}i\text{ in the population.}
\end{cases}
$

\subparagraph{Bayesian}
Does not exist, see \emph{contingency table}\\

\textit{\hyperref[statistical_method_table]{Return to the table.}}
