The most controversial aspect of Bayesian statistics is its reliance on priors
\subparagraph{Uninformative priors}
If we do not have strong evidence on what $\theta$ should be, it is common to use an
uninformative priors, to "let the data speak for itself".\\
One might think that the most uninformative prior would be the uniform distribution: 
$Beta(1, 1)$, but the posterior would then be: $\E{\theta|\mathcal{D}} =
\dfrac{N_{1}+1}{N_{1}+N_{0}+2}$, whereas the MLE is $\dfrac{N_{1}}{N_{1}+N_{0}}$.\\
As by decreasing the magnitude of the pseudo counts, we can lessen the impact of the 
prior, we can argue that the most non-informative prior is: 
$$\lm{\epsilon}{0} Beta(\epsilon, \epsilon) = Beta(0, 0)$$
Called the \emph{Haldane prior}, it is an improper prior.\\
In general it is advisable to perform a some kind of sensitivity analysis, in which one
checks how much one's conclusions or prediction change in response to change in the 
modelling assumptions which includes the choice of the prior and the likelihood as well.
If the conclusion are relatively insensitive to the modelling assumption, one can have
more confidence in the results.
\subparagraph{Jeffreys priors}
Harold Jeffreys designed a general purpose technique for creating non-informative priors.
The key observation is that if $p(\phi)$ is non-informative then any re-parametrization
of the prior, such as $\theta=h(\phi)$ for some function $h$ should also be 
non-informative.
\begin{itemize}
	\item Start with a variable change: $p_{\theta}(\theta) = p_{\phi}(\phi)\left|\dfrac{d\phi}{d\theta}\right|$
	\item Consider the following constraint: $p_{\phi}(\phi)\propto
		\sqrt{\mathcal{I}(\phi)}$, where $\mathcal{I}(\phi)$ is the Fisher 
		information.\\ $\mathcal{I}(\phi) \triangleq - \E{2 \times 
		\dfrac{d\log\left(p(X|\phi)\right)}{d\phi}}$. This a measure of the
		curvature of the expected negative log likelihood and hence a measure of
		stability of the MLE.
	\item Now $\dfrac{d\log(p(x|\theta))}{d\theta} = 
		\dfrac{d\log(p(X|\phi))}{d\phi}\dfrac{d\phi}{d\theta}$
	\item $\mathcal{I}(\theta) = \mathcal{I}(\phi)
		\left(\dfrac{d\phi}{d\theta}\right)^{2}$
	\item $\sqrt{\mathcal{I}(\theta)} = \sqrt{\mathcal{I}(\phi)}\left|\dfrac{d\phi}
		{d\theta}\right|$
	\item Finally $p_{\theta}(\theta) = p_{\phi}(\phi)\left|\dfrac{d\phi}
		{d\theta}\right| \propto \sqrt{\mathcal{I}(\phi)}\left|\dfrac{d\phi}
		{d\theta}\right| = \sqrt{\mathcal{I}(\theta)}$
\end{itemize}

\subparagraph{Robust priors}
To prevent an undue influence on the result, we build priors having heavy tails, which 
avoids forcing things to be too close to the prior mean.

\subparagraph{Mixture of conjugate priors}
Conjugate priors simplify the computation of robust priors, but are often not robust, and 
not flexible enough to encode our prior knowledge. However it turns out that a mixture of
conjugate priors is also conjugate, and seem to be a good compromise.

