\subsection{MAP (Maximum A Posteriori) estimation}
Although most appropriate choice for:\\
$
\begin{cases}
	\text{Real valued quantity} &\rightarrow \text{\emph{posterior median or mean}}\\
	\text{Discrete} &\rightarrow \text{\emph{vector of posterior marginals}}
\end{cases}
$\\
The most popular choice is \tB{\emph{posterior mode}} aka \tR{MAP}, because it reduces to
optimization problems for which efficient algorithms often exist.\\
Some point to be aware about MAP:
\begin{itemize}
	\item \uB{No measure of uncertainty}
	\item \tB{Plugging in the MAP estimate can result in overfitting}
	\item \tB{The mode is an untypical point}, unlike the mean or median the mode is a
		point of measure 0, it does not take the volume of the space into account.
	\item \tB{MAP estimation is not invariant to reparameterization}, for example 
		passing from centimeters to inches can break things.)\\ The MLE does not
		suffer from this since the likelihood is a function not a probability
		density
\end{itemize}


\subsection{Credible intervals}
With point estimates, we want a measure of confidence. 
\tB{
$$ C_{\alpha}\left(\mathcal{D}\right) = (l, u): \Prob{{l\leq \theta \leq u | \mathcal{D}}} \geq 1 - \alpha
$$}
In general, credible intervals are usually what people want to compute but confidence
intervals are usually what they actually compute, because most people are taught 
frequentist statistics but not Bayesian statistics.\\
Sometimes with central intervals there might be points be outside the CI which have higher
probability density.\\
More formally $p^{*}$ such that: 
\begin{center}
	$1-\alpha = 
	\Su{{\theta:p(\theta|\mathcal{D})>p^{*}}}{}p(\theta|\mathcal{D})d\theta$
\end{center}
Then the \uB{HPD} such that:
\begin{center}
	$\mathcal{D}=\left\{\theta: p(\theta|\mathcal{D})\geq p^{*}\right\}$
\end{center}
