A more efficient approach than cross-validation, meaning fitting \emph{k} times each
model, is \tR{to compute the posterior over models}.
$$
p(m|\mathcal{D}) = \dfrac{p(\mathcal{D}|m)p(m)}{\su{{m\in\mathcal{M}}}{}p(m|\mathcal{D})}
$$
From this we can compute the \tB{MAP model $\hat{m} = \displaystyle \argmax_{m}
p(m|\mathcal{D})$}\\
Then we have the \tB{marginal likelihood}: $p(\mathcal{D}|\hat{m}) = \Su{}{}p(
\mathcal{D}|\hat{m})p(\theta|\hat{m})d\theta$

\paragraph{Baysian Occam's razor}
\tB{In integrating out the parameters rather than maximizing them we are automatically 
protected from overfitting}: model with more parameters do not necessarily have higher 
marginal likelihood.\\
A way to understand the Bayesian Occam's razor effect is to \tB{remember that 
probabilities must sum to one, meaning $\su{{\mathcal{D}'}}{}p(\mathcal{D}'|m)=1$. Complex
models, which can predict many things, must spread their probability mass thinly, and 
hence will not obtain as large a probability for any given data set as simpler models.}

\paragraph{Computing the marginal likelihood (evidence)}
For a fixed model we often write:
$$p(\bm{\theta}|\mathcal{D},m) \propto p(\bm{\theta}|m)p(\mathcal{D}|\bm{\theta},m)$$
This valid since $p(\mathcal{D}|m)$ is constant. However when comparing models we need
to know how to compute the marginal likelihood, $p(\mathcal{D}|m)$. In general this can
be quite hard, since we have to integrate over all possible parameter values, but when
we have a conjugate prior, it is easy to compute.\\
Let $p(\bm{\theta})=\dfrac{q(\bm{\theta})}{Z_{0}}$ be our prior, where $q(\bm{\theta})$
is an unnormalized distribution, and $Z_{0}$ is the normalization constant of the prior.
Let $p(\mathcal{D}|\bm{\theta})=\dfrac{q(\mathcal{D}|\bm{\theta})}{Z_{l}}$ be the 
likelihood, where $Z_{l}$ contains any constant factors in the likelihood. Finally let
$p(\bm{\theta}|\mathcal{D})=\dfrac{q(\bm{\theta}|\mathcal{D})}{Z_{N}}$ be our posterior
where $q(\bm{\theta}|\mathcal{D})=q(\mathcal{D}|\bm{\theta})q(\bm{\theta})$ is the 
unnormalized posterior, and $Z_{N}$ is the normalization constant of the posterior.\\
We have:
$
\begin{cases}
	p(\bm{\theta})= \dfrac{p(\mathcal{D}|\bm{\theta})p(\bm{\theta})}{p(\mathcal{D})}\\
	\dfrac{q(\bm{\theta}|\mathcal{D})}{Z_{N}} = \dfrac{q(\mathcal{D}|\bm{\theta})
	q(\bm{\theta})}{Z_{l}Z_{0}p(\mathcal{D})}\\
	p(\mathcal{D}) = \dfrac{Z_{N}}{Z_{0}Z_{l}}
\end{cases}
$

In general $p(\mathcal{D}|m) = \Su{}{}p(\mathcal{D}|\bm{\theta})p(\bm{\theta}|m)d\bm{
\theta}$ can be quite difficult to compute. 
Simpler approach
\begin{itemize}
	\item \textbf{BIC} simple approximation:
		\tB{$BIC \triangleq \log(p(\mathcal{D}|\bm{\hat{\theta}})) - 
		\dfrac{dof(\bm{\hat{\theta})}}{2}\log(N)\approx\log{p(\mathcal{D})}$}
	\item \textbf{AIC}:
		\tB{$AIC(m,\mathcal{D})\triangleq\log(p(\mathcal{D})\bm{\hat{\theta}}_{MLE
		}) -dof(m)$}\\
		This is derived from Frequentits framework and cannot be interpreted as 
		an approximation to the marginal likelihood. The penalty of AIC is less
		than BIC, it causes AIC pick more complex models. That can be better for 
		predictive accuracy.
	\item Effect of the prior.\\
		If the prior is unknown, the correct Bayesian procedure is to put a prior
		on the prior. That is we should put a prior on the hyper-parameter 
		$\alpha$ as well as the parameters $\bm{w}$. To compute the marginal 
		likelihood we should integrate out all unknowns, we should compute:
		\tB{$\Su{}{}\Su{}{}p(\mathcal{D}|\bm{w})p(\bm{w}|\alpha,m)p(\alpha|m)
		d\bm{w}d\alpha$}
		A computational shortcut is to optimize $\alpha$ rather than integrating
		it out. That is, we use \tG{$p(\mathcal{D}|m)\approx\Su{}{}p(\mathcal{D}
		\bm{w})p(\bm{w}|\alpha,m)d\bm{w}$}.
		where \tG{$\hat{\alpha} = \displaystyle \argmax_{\alpha} p(\mathcal{D}|
			\alpha,m) = \displaystyle \argmax_{\alpha}\Su{}{}p(\mathcal{D}|
		\bm{w})p(\bm{w}|\hat{\alpha},m)d\bm{w}$}
\end{itemize}
\paragraph{Bayes Factors}
When prior on models is uniform, then model selection is equivalent to picking the model
with the highest marginal likelihood. Now suppose we just have two models we are 
considering, call them the null hypothesis, $M_{0}$ and the alternative hypothesis,
$M_{1}$.\\
\tB{$$BF_{1,0} \triangleq \dfrac{p(\mathcal{D}|M_{1})}{p(\mathcal{D}|M_{0})}=
    \dfrac{\left.\frac{p(M_{1}|\mathcal{D})}{p(M_{0}|\mathcal{D})}\right\}\text{\emph{Posterior odds}}}{\left.\frac{p(M_{1})}{p(M_{0})}\right\}\text{\emph{Prior odds}}}
$$}
\begin{itemize}
    \item \emph{Posterior odds}: quantifies relative plausibility of the rival hypotheses
        \textbf{after} having seen the data.
    \item \emph{Bayes Factor}, $BF_{1,0}$, quantifies the evidence provided by the data, 
        this is like a likelihood ratio, except we integrate out the parameters, which 
        allows us to compare models of different complexity.
    \item \emph{Prior odds}: quantifies relative plausibility of the rival hypotheses
        \textbf{before} seeing the data.
\end{itemize}

 

\begin{table}[h!]
	\begin{tabular}{|cl|}
		\hline
	\textbf{Bayes Factor} $\bm{BF(1,0)}$ & \textbf{Interpretation}\\
		\hline
		$BF<\frac{1}{100}$ & Decisive evidence for $M_{0}$\\
		\hline
		$BF<\frac{1}{10}$ & Strong evidence for $M_{0}$\\
		\hline
		$\frac{1}{10}<BF<\frac{1}{3}$ & Modest evidence for $M_{0}$\\
		\hline
		$\frac{1}{3}<BF<1$ & Weak evidence for $M_{0}$\\
		\hline
		$1<BF<3$ & Weak evidence for $M_{1}$\\
		\hline
		$3<BF<10$ & Modest evidence for $M_{1}$\\
		\hline
		$BF>10 $ &  Strong evidence for $M_{1}$\\
		\hline
		$BF>100$ & Decisive evidence for $M_{1}$\\
		\hline
	\end{tabular}
\end{table}


\paragraph{Jeffreys-Lindley paradox}
Problems can arise when we use improper priors (i.e. priors that do not integrate to 1)
for model selection/ hypothesis testing, even though such priors may be acceptable for 
other purposes. \tB{In particular the Bayes Factor will always favor the simplest model 
since the probability of the observed data under a complex model with a very diffuse prior
will be very small.} Thus it is important to use proper priors when doing model selection.
