\subsection{Bayesian concept learning}
Let be $\mathcal{D}$ the data, $h$ the hypothesis taken in account
\subsection{Likelihood}
$p\left(\mathcal{D}|h\right)$ the probability to get the observed data considering the 
hypothesis $h$.
\subsection{Prior}
$p(h)$ the probability of our hypothesis, many prior can be used, and this 
\textbf{subjective} aspect of Bayesian reasoning is a source of much controversy.

\subsection{Posterior}
The posterior is simply the likelihood times the prior, normalized.
\begin{center}
$p\left(h|\mathcal{D}\right) = \dfrac{p\left(\mathcal{D}|h\right)\times p(h)}{
    \su{h'\in\mathcal{H}}{}p\left(\mathcal{D}, h'\right)p(h')
}$
\end{center}

