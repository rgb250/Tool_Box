\paragraph{Hierarchical Bayes}
A key requirement for computing the posterior $p(\theta|\mathcal{D})$ is the 
specification of a prior $p(\theta|\eta)$ where $\eta$ are the hyper-parameters. A 
Bayesian approach is to \tB{put a prior on our priors}. This is an example of a \textbf{
hierarchical Bayesian Model}.

\paragraph{Empirical Bayes}
In hierarchical Bayesian models, we need to compute the posterior on multiple levels of
latent variables. For example, in a two-level model, we need to compute:
$p(\eta, \theta|\mathcal{D}) \propto p(\mathcal{D}|\theta)p(\theta|\eta)p(\eta)$\\
\tB{We can approximate the posterior on the hyper-parameters with a point-estimate, 
$p(\eta|\mathcal{D}\approx \delta_{\hat{\eta}}(\eta))$ where $\hat{\eta}=\argmax_{\eta}
p(\eta|\mathcal{D})$. Since $\eta$ is typically much smaller than $\theta$ in 
dimensionality, it is less prone to overfitting, so we can safely use a uniform prior on 
$\eta$}. Then the estimate becomes: 
$$ \hat{\eta} = \argmax_{\eta} p(\mathcal{D}|\eta) = \argmax_{\eta} \Su{}{}
p(\mathcal{D}|\theta)p(\theta|\mathcal{\eta})d\theta $$
This overall approach is called \textbf{Empirical Bayes}\\
Empirical Bayes violates the principle that the prior should be chosen independently of 
the data. However, we can just view it as a computationally cheap approximation to 
inference in a hierarchical Bayesian model, just as we viewed MAP estimation as an approximation to inference in the one level model $\theta \rightarrow \mathcal{D}$. In fact, we
can construct a hierarchy in which the more integrals one performs, the "more Bayesian" 
one becomes:
\begin{center}
	\begin{tabular}{|*{2}{l|}}
		\hline
		\textbf{Method} & \textbf{Definition} \\
		\hline
		Maximum likelihood & $\hat{\theta} = \argmax_{\theta} 
		p(\mathcal{D}|\theta)$ \\
		\hline
		MAP estimation & $\hat{\theta} = \argmax_{\theta} 
		p(\mathcal{D}|\theta)p(\theta|\eta)$ \\
		ML-II (Empirical Bayes) & $\hat{\eta}=\argmax_{\eta}\Su{}{}
		p(\mathcal{D}|\theta)p(\theta|\eta)d\theta = \argmax_{\eta}p(\mathcal{D}|
		\eta)$ \\
		\hline
		MAP-II & $\hat{\eta}=\argmax_{\eta}\Su{}{}
		p(\mathcal{D}|\theta)p(\theta|\eta)p(\eta)d\theta = \argmax_{\eta}p(
		\mathcal{D}| \eta)p(\eta)$\\
		\hline
		Full Bayes & $p(\theta, \eta|\mathcal{D}) \approx p(\mathcal{D}|\theta)
		p(\theta|\eta)p(\eta)$\\
		\hline
	\end{tabular}
\end{center}

