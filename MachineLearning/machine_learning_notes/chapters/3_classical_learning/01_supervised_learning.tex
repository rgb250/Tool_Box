
With the \tB{\emph{generative} approach we create a joint model of the form $\prob{y,\bm{x}}$, and
then to condition on $\bm{x}$, thereby deriving $\prob{y|\bm{x}}$}.\\
Alternatively, \tB{fitting directly a model of the form $\prob{y|\bm{x}}$ is a \emph{
discriminative} approach}.

\begin{itemize}
    \item \emph{Easy to fit}: generative classifiers is usually very easy
    \item \emph{Fit classes separately?}: in a generative classifier we estimate the parameters 
        of each class conditional density independently
    \item \emph{Handle missing features easily?}: in a generative classifier there is a natural way
        to handle missing data: $\prob{\bm{x}_{i},r_{i}|\bm{\theta},\bm{\phi}} = \prob{r_{i}|\bm{
        x}_{i},\bm{\phi}}\prob{\bm{x}_{i}|\bm{\theta}}$ where $\bm{\phi}$ are the parameters
        controlling whether the item is observed or not. Missing completely at random (MCAR): $\prob{
        r_{i}|\bm{x}_{i},\bm{\phi}} = \prob{r_{i}|\bm{x}^{0}_{i}|\bm{\phi}}$, missing at random 
        (MAR): $\prob{r_{i}|\bm{x}_{i},\bm{\phi}} = \prob{r_{i}|\bm{x}^{0}_{i},\bm{\phi}}$
    \item \emph{Can handle unlabeled training data?}: Fairly easy to using generative models
    \item \emph{Symmetric in inputs and outputs?}: We can run a generative model "backward" and 
        infer probable inputs given the output by computing $\prob{\bm{x}|y}$
    \item \emph{Can handle feature preprocessing?}: discriminative methods allow to preprocess the
        input in arbitrary ways, by replacing $\bm{x}$ with $\phi(\bm{x})$
    \item \emph{Well-calibrated probabilities?}: discriminative models are usually better calibrated
        in terms of their probability estimates.
\end{itemize}


\section{Classification}
% Naive Bayes Classifiers
\subsection{Naive Bayes classifiers}
\paragraph{Purpose}
\tR{Classifying vectors of discrete-valuated features $x\in\left\{i
\right\}_{1\leq i \leq K}^{D}$}, where $K$ is the number of values for
each feature, and $D$ the number of features.

\paragraph{Assumptions}
\begin{itemize}
    \item \tB{Features are conditionally independent given the class 
        label}
\end{itemize}

\paragraph{Theory}
As a \emph{generative} model, meaning of the form:
$\prob{y=c|\bm{x}, \bm{\theta}} \propto \prob{\bm{x}|y=c,\bm{\theta}}
\prob{y=c|\bm{\theta}}$. The key of such models is the possibility
to specify a suitable form for the class-conditional density 
$\prob{\bm{x}|y=c, \bm{\theta}}$ which definees what kind of data we 
expect to see in each class. And with the independence assumption we 
have:
\begin{center}
    \tB{$\prob{\bm{x}|y=c, \bm{\theta}} = \prd{j=1}{D}\prob{x_{j}|y=c,\bm{\theta}_{jc}}$}
\end{center}
with all $\prob{x_{j}|y=c,\bm{\theta}_{jc}}$ being able to 
follow a \textit{normal}, \textit{bernoulli} or \textit{multinoulli} 
distribution.\\
\uB{Training a NBC consists in computing the MLE or the MAP estimate for 
the parameters.}\\
For a single observation
$\prob{x_{i}, y_{i}|\bm{\theta}} = \prob{y_{i}|\bm{\pi}}\prd{j}{}\prob{x_{ij}|\bm{\theta}_{j}} = 
\prd{c}{}\pi_{c}^{\mathbbm{1}(y_{i}=c)}\prd{j}{}\prd{c}{}\prob{x_{ij}|
\theta_{jc}}^{\mathbbm{1}(y_{i}=c)}
$\\ 
Hence the \tB{\emph{log-likelihooh}:
$\log\left(\mathcal{D}|\theta\right) = \su{c=1}{C}N_{c}\log(\pi_{c}) + 
\su{j=1}{D}\su{c=1}{C}\su{i:y_{i}=c}{}\log\left(\prob{x_{ij}|
\bm{\theta}_{jc}}\right)$}\\
By optimizing the above equation we are able to find the $\left(\theta_{jc}\right)_{1\leq j \leq D,~ 1\leq c\leq C}$ and we can then use them to
predict the output of an observation $\bm{x}$ as: $\prob{y=c|\bm{x},\mathcal{D}} \propto \prob{y=c|\mathcal{D}}\prd{j=1}{D}\prob{x_{j}|y=c,\mathcal{D}}$

\paragraph{Strengths}
\begin{itemize}
    \item Simple model, for $C$ classes and $D$ features, and hence \tB{relatively immune to 
        overfitting}
\end{itemize}

\paragraph{Weaknesses}
\begin{itemize}
    \item \uB{Unaccuracy} because of the strong independence assumption
\end{itemize}

\paragraph{Relationships with other methods}
\tB{Logistic Regression}: for discrete inputs 
\emph{Naive Bayesian Classifiers} form a 
generative-discriminant pair with \emph{Multinomial Logistic
Regression}: \uB{each NBC can be considered a way of fitting a
probability model that optimizes the joint likelihood 
$\prob{C, \bm{x}}$, while Multinomial Logistic Regression fits the same 
probability to optimize the conditional $\prob{C|\bm{x}}$}

\paragraph{Examples of application}
\begin{itemize}
    \item Classifying documents using bag of words
    \item Determining the gender of a person, based on measured features 
\end{itemize}


% Linear/Quadratic Discriminant Analysis
\subsection{Linear/Quadratic Discriminant Analysis}
\uB{It consists in defining the class conditional densities in a 
generative classifier}: $\prob{\bm{x}|y=c,\bm{\theta}} = \mathcal{N}
\left(\bm{x}, \bm{\mu}_{c},\bm{\Sigma}_{c}\right)$\\
As for a generative classifier we have the following equation: 
\begin{center}
    $\prob{y=c|\bm{x},\bm{\theta}} = \dfrac{\overbrace{\prob{\bm{x}|y=c,
        \bm{\theta}}}^{\text{\emph{class-conditional density}}}
        \overbrace{\prob{y=c|\bm{\theta}}}^{\text{\emph{class prior}}}}{
    \su{c'}{}\prob{y=c'|\bm{\theta}} \prob{\bm{x}|y=c',\bm{\theta}}}$
\end{center}
\paragraph{Purpose of Quadratic Discriminant Analysis}
\begin{center}
    $\prob{y=c|\bm{x},\bm{\theta}} = \dfrac{|2\pi\Sigma_{c}|^{
    -\frac{1}{2}}\exp\left(-\frac{1}{2}[\bm{x}-\bm{\mu}_{c}]^{T}\Sigma_{
    c}^{-1}[\bm{x}-\bm{\mu}_{c}]\right)\pi_{c}}{\su{c'}{}|2\pi\Sigma_{
    c'}|^{-\frac{1}{2}}\exp\left(-\frac{1}{2}[\bm{x}-\bm{\mu}_{c'}]^{T}
    \Sigma_{c'}^{-1}[\bm{x}-\bm{\mu}_{c'}]\right)\pi_{c'}}$
\end{center}
\tB{The threshold of this results will be a quadratic function of 
$\bm{x}$.}

\paragraph{Purpose of Linear Discriminant Analysis}
Same equation than above but this time, \tB{$\forall c\in \inter{1}{C} 
\Sigma_{c} = \Sigma$}, \tB{then quadratic term $\bm{x}^{T}\Sigma^{-1}
\bm{x}$ will cancel out from numerator and denominator}.
Then by considering the above cancellation and the fact that 
evidence is considered as a constant, we have:
\tB{
\begin{align*}
    \prob{y=c|\bm{x},\bm{\theta}} & \propto \exp\left(\log(\pi_{c})+\bm{
        \mu}_{c}^{T}\Sigma^{-1}\bm{x}\bm{\mu}_{c}\right)\\
    &= \exp\left(\bm{\beta}_{c}^{T}\bm{x} + \gamma_{c}\right)
    \end{align*}
}
Note also that we have exactly: \tR{$\prob{y=c|\bm{x},\bm{\theta}} =
\dfrac{e^{\bm{\beta}_{c}^{T}\bm{x} + \gamma_{c}}}{\su{c'}{}e^{\bm{\beta
}_{c'}^{T}\bm{x} + \gamma_{c'}}} = S(\bm{\eta})_{c}$}. With $\eta=\left(
\bm{\beta}_{c}\bm{x} +\gamma_{c}\right)_{1\leq c\leq C}$
We recognize the \emph{softmax} function.

\paragraph{Assumptions}
\begin{itemize}
    \item Independent \tB{variables are normal for each level} of the
        grouping variable.
    \item \tB{Homoscedasticity} for LDA: variances \uB{among group} 
        variables are the same across levels of predictors.
    \item \uB{Independence of the observations}.
\end{itemize}


\paragraph{Theory}
\paragraph{Strengths}
\paragraph{Weaknesses}
\begin{itemize}
    \item Multicollinearity: predictive power can decrease with an 
        increased correlation between predictor variables.
\end{itemize}

\paragraph{Relationships with other methods}
\paragraph{Examples of application}


\subsection{Logistic Regression}
\paragraph{Purpose}
With the generative approach we create a joint model of the form $\prob{y,\bm{x}}$, and
then to condition on $\bm{x}$, thereby deriving $\prob{y|bm{x}}$, it is the \emph{
generative} approach.\\
Alternatively, fitting directly a model of the form $\prob{y|\bm{x}}$ is a \emph{
discriminative} approach.
\paragraph{Assumptions}
\begin{itemize}
    \item Independence
\end{itemize}

\paragraph{Theory}
The data distribution is modelled by : 
\fr{$\prob{y|\bm{x}} = \text{\emph{Bernoulli}} \left(y|\sigma\left(\bm{w}^{T}\bm{x}
\right)\right)$}\\
With $\sigma$ being the \emph{sigmoid} function, such that 
$\sigma = \begin{cases}
    \mathbb{R} \longrightarrow [0, 1]\\ 
    x \mapsto \dfrac{e^{x}}{1 + e^{x}}
\end{cases}
$
\subparagraph{Maximum Likelihood Estimator}
\begin{align*}
    \text{\emph{NLL}}(\bm{w})
    &= -\su{i=1}{N}\log\left(\hat{y}_{i}^{\mathbbm{1}_{\{y_{i} = 1\}}}\left[1 -
    \hat{y}_{i}^{\mathbbm{1}_{\{y_{i}=0\}}}\right]\right)\\ 
    &= -\su{i=1}{N}y_{i}\log\left(\hat{y}_{i}\right) + \left[1 -y_{i}\right]\log
    \left(1 -\hat{y}_{i}\right)
\end{align*}
This called \textit{cross-entropy}


\paragraph{Strengths}
\paragraph{Weaknesses}
\paragraph{Relationships with other methods}
\paragraph{Examples of application}

\subsection{Logistic Regression}
\paragraph{Purpose}
\paragraph{Assumptions}
\paragraph{Theory}
\paragraph{Strengths}
\paragraph{Weaknesses}
\paragraph{Relationships with other methods}
\paragraph{Examples of application}

\subsection{Logistic Regression}
\paragraph{Purpose}
\paragraph{Assumptions}
\paragraph{Theory}
\paragraph{Strengths}
\paragraph{Weaknesses}
\paragraph{Relationships with other methods}
\paragraph{Examples of application}

\subsection{Logistic Regression}
\paragraph{Purpose}
\paragraph{Assumptions}
\paragraph{Theory}
\paragraph{Strengths}
\paragraph{Weaknesses}
\paragraph{Relationships with other methods}
\paragraph{Examples of application}



\section{Regression}
\subsection{Linear Regression}
\paragraph{Purpose}
\paragraph{Assumptions}
\paragraph{Theory}
\subparagraph{General}
It is a model for which the data distribution (likelihood) is described 
by:
\tR{\begin{center}
    $\prob{y|\bm{x},\bm{\theta}} = \mathcal{N}\left(y|\bm{w}^{T}\phi(
    \bm{x}), \sigma^{2}\right)$
\end{center}}
with $\phi$ that can be a non-linear function, in this case we talk about
\emph{basis function expansion}.\\
To estimate the parameters we can use the \emph{maximum likelihood 
estimation}: \tB{$\hat{\theta} \triangleq \argmax_{\theta} \log\left(
\prob{\mathcal{D}|\bm{\theta}}\right)$}.\\
For computational purpose it is better to consider the minimization of 
the \textit{Negative Log Likelihood} (NLL):
\begin{align*}
    \text{\emph{NLL}}(\theta) &\triangleq -\log\left(p\left(\mathcal{D}|
    \theta\right)\right)\\
                              &=-\su{i=1}{n}\log\left(\prob{y_{i}|
                                      \bm{x}_{i},\bm{\theta}}\right)\\
                              &=-\su{i=1}{n}\log\left(\left[\dfrac{1}{
                              2\pi\sigma^{2}}\right]^{\frac{1}{2}}\exp
                              \left(-\dfrac{1}{2\sigma^{2}}\left[y_{i}
                                      - \bm{w}^{T} \bm{x}_{i}\right]^{2}
                              \right)\right)\\
                              &= \dfrac{n}{2}\log\left(2\pi\sigma^{2}
                              \right) + \dfrac{1}{2\sigma^{2}}\su{i=1}{
                              n}\left(y_{i} - \bm{w}^{T}\bm{x}_{i}
                              \right)^{2}\\
                              &= \dfrac{n}{2}\log\left(2\pi\sigma^{2}
                              \right) + \dfrac{1}{2\sigma^{2}}\text{
                              \emph{RSS}}(\bm{w})\\
                              &= \dfrac{n}{2}\log\left(2\pi\sigma^{2}
                          \right) + \dfrac{1}{2\sigma^{2}}\norm{\epsilon
                          }_{2}^{2}
\end{align*}
As the \emph{MLE} for $\bm{w}$ is the one minimizing the \emph{RSS} then
this method is known as \emph{least square}.

\subparagraph{Derivation of the MLE}
it is better to use a matrix-vector representation.\\
$\text{\emph{NLL}}(\bm{w}) = \dfrac{1}{2}\left(y-\bm{X}\bm{w}\right)^{T}\left(y-\bm{X}
\bm{w}\right) = \dfrac{1}{2}\bm{w}^{T}\left(\bm{X}^{T}\bm{X}\right)\bm{w} - \bm{w}^{T}
\left(\bm{X}^{T}y\right)$ Note that $\bm{X}^{T}\bm{X}$ is the \emph{sum of squares 
matrix}. Then \textbf{gradient}, $g(\bm{w}) = \bm{X}^{T}\bm{X}\bm{w} - \bm{X}^{T}\bm{y}$
that we have to equate to zero to get $\bm{X}^{T}\bm{X}\bm{w} = \bm{X}^{T}\bm{y}$ to 
conclude that:
\begin{center}
    \fr{$\hat{\bm{w}}_{OLS} = \left(\bm{X}^{T}\bm{X}\right)^{-1}\bm{X}^{T}\bm{y}$}
\end{center}

\subparagraph{Robust Linear Regression}
It is very common to model the noise in regression models using a \emph{Gaussian 
distribution}, meaning \tR{$\epsilon_{i} = y_{i} - \bm{w}^{T}\bm{x}_{i} \hookrightarrow 
\mathcal{0, \sigma^{2}}$}. One way to achieve \emph{robustness} against \emph{outliers}
is to replace the Gaussian distribution for the response variable with a distribution 
having \textbf{heavy tails}.

\begin{center}
    \begin{tabular}{|*{3}{c|}}
    \hline
    \textbf{Likelihood} & \textbf{Prior} & \textbf{Name}\\
    \hline
    Gaussian & Uniform & \emph{Least Squares}\\
    \hline
    Gaussian & Gaussian & \emph{Ridge}\\
    \hline
    Gaussian & Laplace & \emph{Lasso}\\
    \hline
    \end{tabular}
\end{center}

\subparagraph{Ridge}
encourages parameters to be small by using a zero-mean Gaussian prior: $\prob{\bm{w}} = 
\prd{j=1}{D} \mathcal{N}\left(\omega_{j}|0, \tau^{2}\right)$, where $\frac{1}{\tau^{2}}$
controls the strength of the prior.\\
The corresponding \tB{\emph{MAP}} estimation problem becomes:
\tB{$\argmax_{\bm{w}}\su{i=1}{n}\log\left(\mathcal{N}\left(y_{i}|\omega_{0} + \bm{w}^{T}
\bm{x}_{i}, \sigma^{2}\right)\right) + \su{j=1}{D}\log\left(\mathcal{N}\left(\omega_{j}
|0,\tau^{2} \right)\right)$}. After some calculus and with where $\lambda \triangleq 
\dfrac{\sigma^{2}}{\tau^{2}}$ we deduce that:
\begin{center}
    \fr{$
        \hat{\bm{w}}_{\text{\emph{Ridge}}} = \left(\lambda\bm{I}_{D} + \bm{X}^{T}\bm{X}
        \right)^{-1}\bm{X}\bm{y}
    $}
\end{center}
Advantages of Ridge regression on OLS regression:
\begin{itemize}
    \item $\left(\lambda\bm{I}_{D}+\bm{X}^{T}\bm{X}\right)$ is much better conditioned, 
        and hence more likely to be invertible, than $\bm{X}^{T}\bm{X}$ at least for 
        suitable large $\lambda$
    \item if we follow a \emph{Singular Value Decomposition} $\bm{X} = \bm{U}\bm{S}
        \bm{V}^{T}$ we find 
        that $\hat{\bm{y}} = \bm{X}\hat{\bm{w}}_{\text{\emph{Ridge}}} = \su{j=1}{D}
        \bm{u}_{j}\dfrac{\sigma_{j}^{2}}{\sigma_{j}^{2} + \lambda}\bm{u}_{j}^{T}\bm{y}$
        with $\left(\sigma_{j}\right)_{1\leq j \leq D}$ the singular value of $\bm{X}$
        whereas for OLS we have $\hat{\bm{y}} = \bm{X}\hat{\bm{w}}_{\text{\emph{OLS}}} 
        = \su{j=1}{D} \bm{u}_{j}\bm{u}_{j}^{T}\bm{y}$. Meaning that with Ridge if 
        $\sigma_{j}^{2}$ is small compared to $\lambda$ then direction $\bm{u}_{j}$
        will not have much effect on the prediction. In term of predictive accuracy
        \emph{Ridge} regression is more interesting than \emph{PCA} regression.
\end{itemize}
\subparagraph{Lasso (Least Absolute Shrinkage and Abosolute Selection Operator)}
$\hat{\bm{w}}_{Lasso} = sign(\hat{\bm{w}}_{OLS})\left[|\hat{\bm{w}}_{OLS}| -\dfrac{
\lambda}{2}\right] $

\subparagraph{Elastic-Net}
In practice Elastic-Net often performs best, since it provides a good combination of
sparsity and regularization.
\paragraph{Strengths}
\begin{itemize}
    \item Simple
    \item Customizable to achieve robustness
\end{itemize}

\paragraph{Weaknesses}
\begin{itemize}
    \item Not very powerful for non-linear data

\end{itemize}

\paragraph{Relationships with other methods}
\begin{itemize}
    \item Ridge Regression has similitude with PCA
\end{itemize}

\paragraph{Examples of application}


\subsection{Generalized Linear Models (GLMs)}
Models in which the output density is in the exponential family and in which the mean
parameters are a linear combination of the inputs, passed through a possibly nonlinear
function such as the logistic function.\\
Exponential family:
\tR{$\prob{\bm{x}|\bm{\theta}} = \dfrac{1}{Z(\bm{\theta})}h(x)e^{\bm{\theta}^{T}\phi(
\bm{x})} = h(\bm{x})e^{\bm{\theta}^{T}\phi(\bm{x}) - A(\bm{\theta})}$}
where
$\begin{cases}
    Z(\bm{\theta}) = \su{\mathcal{X}^{m}}{}h(\bm{x})e^{\bm{\theta}^{T}\phi(\bm{x})}
    d\bm{x} \\
    A(\bm{\theta}) = \log\left(Z(\bm{\theta})\right)
\end{cases}$
with the \emph{natural parameters} $\bm{\theta}$, the vector of  \emph{sufficient 
statistics} $\phi(\bm{x})$, the \emph{partition function} $Z(\bm{\theta})$, the 
\emph{log partition function} or \emph{cumulant function} $A(\bm{\theta})$and the 
\emph{scaling constant} $h(\bm{x})$.\\
The name \emph{cumulant function} comes from the property of the exponential family 
being that derivatives of the log partition function can be used to generate cumulant
of the sufficient statistics (the first and second cumulants of a distribution being
its mean and variance).
\paragraph{Purpose}
We have the following data distribution:
$\prob{y_{i}|\bm{\theta},\sigma^{2}} = \exp\left(\dfrac{y_{i}\bm{\theta} - A\left(
\bm{\theta}\right)}{\sigma^{2}} + c(y_{i}, \sigma^{2})
\right)$
with $c$ a normalization constant.\\
Let's consider an invertible mapping $\Phi$ such that $\bm{\theta} = \Phi(\mu)$ with
$\mu$ being the mean parameter and $\bm{\theta}$ the natural parameter.\\
We have as well \tB{$\mu = \Phi^{-1}(\bm{\theta}) = A'(\bm{\theta})$}.\\
We are free to chose any link function as long as the inverse have an appropriate 
range. A simple form of link function is to use $g=\Phi$, $g$ is then called 
\emph{canonical function}.

\begin{center}
    \begin{tabular}{|*{4}{l|}}
    \hline
    \textbf{Distribution} & \textbf{Link function}  & 
    \textbf{Natural parameter} $\theta = \Phi(\mu)$ & \textbf{Mean parameter} $\mu = 
    \Phi^{-1}(\theta) = \mathbb{E}(y)$ \\
    \hline
    $\mathcal{N}(\mu, \sigma^{2})$ & \emph{identity} & $\bm{\theta} = \bm{\mu}$ & 
    $\bm{\mu} = \bm{\theta}$\\
    \hline
    $\mathcal{B}(n,\mu)$ & \emph{logit} & $\bm{\theta} = \log\left(\frac{\mu}{1-\mu}
    \right)$ & $\mu = \text{\emph{sigm}}(\bm{\theta})$\\
    \hline
    \emph{Poisson}$(\mu)$ & \emph{log} & $\bm{\theta} = \log\left(\mu\right)$ & 
    $\mu=e^{\theta}$\\
    \hline
    \end{tabular}
\end{center}

\paragraph{Assumptions}
\paragraph{Theory}
\paragraph{Strengths}
\begin{itemize}
    \item GLMs can be fit using methods like gradient descent.
\end{itemize}

\paragraph{Weaknesses}
\paragraph{Relationships with other methods}
\paragraph{Examples of application}


\subsection{Learning to rank}
\paragraph{Purpose}
Modeling a function being able to rank a set of items.



\paragraph{Assumptions}
\paragraph{Theory}
Let us consider a document $d$ and a query $q$, a standard way to measure the relevance
between the both is to use $\text{\emph{sim}}(q,d) \triangleq \prob{q|d} = \prd{i=1}{n}
\prob{q_{i}}{d}$ with $q_{i}$ being the $i^{\text{\emph{th}}}$ word or term of $q$.
\subparagraph{Pointwise approach}
for binary relevance labels, we can follow a standard binary classification scheme to
estimate $\prob{y=1|\bm{x}(q,d)}$, in the case of ordered relevancy labels we can use
an \tB{\emph{ordinal regression} to predict the rating $\prob{y=r|\bm{x}(q,d)}$}.\\
However this method does not take into account the location of each document in the 
list.
\subparagraph{Pairwise approach}
to \tB{check the relative relevance of two items rather than absolute relevance}. We 
can model this kind of data using a binary classifier of the form \tB{$\prob{y_{jk}|
\bm{x}(q,d_{j})\bm{x}(q,d_{k})} = \sigma\left(f(\bm{x}_{j}) - f(\bm{x}_{k})\right)$} 
where $f$ is a scoring function, often taken to be linear: $f(\bm{x}) = \bm{w}^{T}
\bm{x}$.

\subparagraph{Listwise approach}
we now consider methods looking at the entire list of items at the same time. We can 
define a total order on a list by specifying a permutation of its indices: $\bm{\pi}$.
To model the uncertainty about $\bm{\pi}$ we can use the \emph{Plackett-Luce} 
distribution. $\prob{p(\bm{\pi}|\bm{s})} = \prd{j=1}{m}\dfrac{s_{j}}{\su{u=j}{m}s_{u}}$
Where $s_{j}=s\left(\pi^{-1}(j)\right)$
\paragraph{Strengths}
\paragraph{Weaknesses}
\paragraph{Relationships with other methods}
\paragraph{Examples of application}
\begin{itemize}
    \item \emph{information retrieval} return a list of the top $k$ more relevant 
        documents, depending on a given query 
\end{itemize}

\subsection{Supervised PCA}
\paragraph{Purpose}
Also called \emph{Bayesian factor regression} this model take in account $y_{i}$ when
learning the low dimension embedding.
\paragraph{Assumptions}
\paragraph{Theory}
$\begin{cases}
    \prob{\bm{z}_{i}} = \mathcal{N}\left(\bm{0}, \bm{I}_{L}\right) \\
    \prob{y_{i}|\bm{z}_{i}} = \mathcal{N}\left(\bm{w}_{y}^{T}\bm{z}_{i},
    +\mu_{y},\sigma_{y}^{2}\right) \\
    \prob{x_{i}|\bm{z}_{i}} = \mathcal{N}\left(\bm{W}_{x}^{T}\bm{z}_{i},
    +\bm{\mu}_{x},\sigma_{x}^{2}\bm{I}_{D}\right)
\end{cases}$
The basic idea compressing $\bm{x}_{i}$ to predict $y_{i}$ can be formulated using
information theory, in particular we might want to find an encoding distribution
$\prob{\bm{z}|\bm{x}}$ such that we minimize $\mathbb{I}(X;Z) - \beta\mathbb{I}(X;Y)$.
Where $\beta\geq 0$, the \emph{information bottleneck} is some parameter controlling 
the trade-off between compression and predictive accuracy.
\paragraph{Strengths}
\paragraph{Weaknesses}
\paragraph{Relationships with other methods}
\begin{itemize}
    \item PCA
\end{itemize}

\paragraph{Examples of application}
\begin{itemize}
    \item predict the movies that you would like knowing who your friends are as well 
        as the rating from other users
\end{itemize}


\subsection{Partial Least Squares}
\paragraph{Purpose}
The key idea is to allow some of the (co)variance in the input features to be explained
by its own subspace $\bm{z}_{i}^{x}$ and to let the remaining of subspace $\bm{z}_{i}^{
s}$ be shared between input and output.
\paragraph{Assumptions}
\paragraph{Theory}
$\begin{cases}
    \prob{\bm{z}_{i}} = \mathcal{N}\left(\bm{z}^{s}_{i}|\bm{0},\bm{I}_{L_{s}}\right)
    \mathcal{N}\left(\bm{z}^{x}_{i}|\bm{0},\bm{I}_{L_{x}}\right)\\
    \prob{\bm{y}_{i}|\bm{z}_{i}} = \mathcal{N}\left(\bm{W}_{y}\bm{z}^{s}_{i} + 
    \bm{\mu}_{y},\sigma^{2}\bm{I}_{D_{y}}
    \right)\\
        \prob{\bm{x}_{i}|\bm{z}_{i}} = \mathcal{N}\left(\bm{W}_{x}\bm{z}^{s}_{i} + \bm{B}_{x}
        \bm{z}^{x}_{i} + \bm{\mu}_{x},\sigma^{2}\bm{I}_{D_{x}}\right)
\end{cases}$
\paragraph{Strengths}
\paragraph{Weaknesses}
\paragraph{Relationships with other methods}
\paragraph{Examples of application}

%\subsection{Logistic Regression}
%\paragraph{Purpose}
%\paragraph{Assumptions}
%\paragraph{Theory}
%\paragraph{Strengths}
%\paragraph{Weaknesses}
%\paragraph{Relationships with other methods}
%\paragraph{Examples of application}


\section{Classification and Regression}
\subsection{Mixture models}
\paragraph{Purpose}
It is the simplest form of \emph{Latent Variable Models (LVMs)} is when $z_{i}\in
\{k\}_{1\leq k\leq K}$. 2 mains application of mixture models:
\begin{itemize}
    \item use them as \emph{black-box} density model, $p(\bm{x}_{i})$, useful for data
        compression, outlier detection and creating generative classifiers
    \item use for clustering
\end{itemize}

\paragraph{Assumptions}
\paragraph{Theory}
We use a discrete prior $p(z_{i}) = \text{\emph{Cat}}(\bm{\ conda install -c conda-forge juliapi})$ and the likelihood
$p(\bm{x}_{i}|z_{i} = k)$, finally the \textbf{mixture model} is :
\begin{center}
    $\prob{\bm{x}_{i}|\bm{\theta}} = \su{k=1}{K}\pi_{k}\prob{\bm{x}_{i}|z_{i}=k,
    \bm{\theta}} = \su{k=1}{K}\pi_{k}\mathbb{P}_{k}\left(\bm{x}_{i}|\bm{\theta}\right)$
\end{center}
where $\mathbb{P}_{k}$ is the $k$'th \emph{base distribution}.
\subparagraph{Mixtures of Gaussian}
Each base distribution in the mixture is a multivariate Gaussian with mean $\mu_{k}$ 
and covariance matrix $\bm{\Sigma}_{k}$:
$\prob{\bm{x}_{i}|\bm{\theta}}=\su{k=1}\pi_{k}\mathcal{N}\left(\bm{x}_{i}|\bm{\mu}_{k},
    \bm{\Sigma_{k}}\right)$\\
Given a sufficiently large number of mixture components a \tB{Gaussian Mixture Models
(GMMs) can be used to approximate any density defined on $\mathbb{R}^{D}$}.
\subparagraph{Mixtures of multinoullis}
can be used to define density models on data consisting of a \emph{D}-dimensional bit
vectors: $\prob{\bm{x}_{i}|z_{i}=k,\bm{\theta}} = \prd{j=1}{D}\text{\emph{Ber}}(x_{ij}|
\mu_{jk})=\prd{j=1}{D}\mu_{jk}^{x_{ij}}(1-\mu_{jk})^{1-x_{ij}}$\\
where $\mu_{jk}$ is the probability that bit $j$ turns on in cluster $k$.
\subparagraph{Mixture of experts}
are build from a discriminative perspective, they relies on the idea that a good
model different linear method each applying to a different part of the input space.\\
We can model this by allowing the mixing weights and the mixture densities to be
input-dependent: 
\begin{center}
    $\begin{cases}
        \prob{y_{i}|\bm{x}_{i},z_{i}=k,\bm{\theta}} = \mathcal{N}(y_{i}|\bm{w}_{k}^{T}
        \bm{x}_{i},\sigma_{k}^{2}) \\
        \prob{z_{i}|\bm{x}_{i},\bm{\theta}} = \text{\emph{Cat}}(z_{i}|S(\bm{V}^{T}
        \bm{x}_{i}))
    \end{cases}$
\end{center}
Useful in solving inverse problems, the ones in which we have to invert a many-to-one
mapping, for example in robotics where the location of the end effector (hand) $\bm{y}$
is uniquely determined by the joint angles of the motors, $\bm{x}$.

The overall posterior is then
\begin{center}
    \fr{$\prob{y_{i}|\bm{x}_{i},\bm{\theta}} = \su{k}{}\prob{z_{i}=k|\bm{x}_{i},z_{i}=k
    ,\bm{\theta}}$}
\end{center}




\paragraph{Strengths}

\paragraph{Weaknesses}
\begin{itemize}
    \item Use of a single latent variable to generate the observation
\end{itemize}

\paragraph{Relationships with other methods}
\paragraph{Examples of application}


\subsection{ARD: Automatic Relevance Determination}
\paragraph{Purpose}
\paragraph{Assumptions}
\paragraph{Theory}
\paragraph{Strengths}
\paragraph{Weaknesses}
\paragraph{Relationships with other methods}
\paragraph{Examples of application}

\subsection{Support Vector Machines (SVMs)}
\paragraph{Purpose}
The \tB{combination of the \emph{kernel trick} plus a modified loss function 
allowing \emph{sparsity}}, meaning that the prediction will only depend on a
subset of the training data called \emph{support vectors}. The overall 
process is known as a \emph{Support Vector Machine}.\\
Because of sparsity encoding in the loss function instead of in the prior, 
kernel encoding through a trick instead of being an explicit part of the 
model,\tB{SVMs do not provide probabilistic outputs}.
\subparagraph{SVMs for regression}
with \emph{kernalized ridge regression} the solution $\bm{w}$ depends on all 
the training inputs. We should then use a variant of \emph{Huber loss} 
function: the \tB{\textit{epsilon insensitive loss function}}: 
\begin{center}
    \fr{$L_{\epsilon}\left(y-\hat{y}\right) = 
    \begin{cases}
        0 & \Leftarrow \lvert y-\hat{y}\rvert < \epsilon \\
        \lvert y-\hat{y}\rvert -\epsilon &\Leftarrow \lvert y-\hat{y}\rvert 
            \geq \epsilon    
    \end{cases}$}
\end{center}
meaning that any point lying inside an $\epsilon\text{-\textbf{tube}}$ a
around the prediction is not penalized: $J=C\su{i=1}{n}L_{\epsilon}\left(
y_{i} - \hat{y}_{i}\right) + \frac{1}{2}\norm{\bm{w}}^{2}$, with $C=\frac{1}{
\lambda}$ is a \emph{regularization constant}.\\
It can be shown that \tB{optimal solution has the form $\hat{\bm{w}} = \su{i}{
}\alpha_{i} \bm{x}_{i}$, where $\alpha_{i}\geq 0$}. It turns out that \tB{
$\bm{\alpha}$ is sparse as we don't care about errors which are smaller than 
$\epsilon$. The $\bm{x}_{i}$ for which $\alpha_{i}>0$ are the \textbf{support 
vectors}}.\\
Then we have $\hat{y}(\bm{x}) = \hat{w}_{0} + \hat{\bm{w}}\bm{x}  = \hat{w}_{0}
+ \su{i}{}\alpha_{i}\bm{x}_{i}^{T}\bm{x} = \su{i}{}\alpha_{i}k(\bm{x},\bm{x}_{i})$

\subparagraph{Classification}
we consider now the \emph{hinge loss}:
\begin{center}
    $L_{hinge}(y,\eta) = \max\left(0, 1-y\eta\right) = \left(1-y\eta\right)_{+}$
\end{center}
where $\eta=f(\bm{x})$ is our 'confidence' in choosing label $y=1$ however it does not
need to have any probabilistic semantics.\\
The overall objective has the form $\min_{\bm{w}:w_{0}}\frac{1}{2}\norm{\bm{w}}^{2} +
C\su{i=1}{n}\left(1-y_{i},f(\bm{x}_{i})\right)$. Same principle as in regression, but
this time \tB{$\hat{y}(\bm{x}) = sgn\left(f(\bm{x})\right) = sgn\left(\hat{w}_{0} +
\hat{\bm{w}}^{T}\bm{x}\right)= sgn\left(\hat{w}_{0} + \su{i=1}{n}\alpha_{i}k(\bm{x}_{i}
,\bm{x})\right)$}
\subparagraph{The large margin principle}
our goal is to derive a \emph{discriminative function} $f(x)$ which will be linear in the feature
space implied by the choice of kernel. Hence: $\bm{x} = \bm{x}_{\perp} + 
r\frac{\bm{x}}{\norm{\bm{w}}}$ where $r$ is the distance of $\bm{x}$ from the decision boundary 
whose normal vector is $\bm{w}$, and $\bm{x}_{\perp}$ is the orthogonal projection of $\bm{x}$a 
onto this boundary. Hence $f(\bm{x}) = \bm{w}^{T}\bm{x} +w_{0} = \left(\bm{x}^{T}\bm{x}_{\perp}
+ w_{0}\right) + r\frac{\bm{w}^{T}\bm{w}}{\norm{\bm{x}}}$. As $f(\bm{x}_{\perp}) = 0$ 
$\bm{w}^{T}\bm{x} +w_{0} = 0$ Hence $f(\bm{x}) = r\frac{\bm{w}^{T}\bm{wl}}{\norm{\bm{x}}}$.
Finally \tB{$r = \frac{f(\bm{x})}{\norm{\bm{x}}}$, the distance that we would like to make as large 
as possible, in order to clearly separate the input}.
\subparagraph{Regularization parameter \emph{C}}
it \tB{controls the number of errors we are willing to tolerate on the training set}, it is chosen by
cross-validation, \uB{an efficient way to chose \emph{C} is to develop a path following algorithm in 
the spirit of \emph{ARS}}.

\paragraph{Assumptions}
\paragraph{Theory}
\paragraph{Strengths}
\begin{itemize}
    \item computational advantages over probabilistic model
    \item \tB{\emph{kernel trick} $\rightarrow$ prevent underfitting}: ensuring that the feature 
        vector is sufficiently rich that a linear classifier can separate
    \item \tB{\emph{sparsity} \& \emph{large margin principles} $\rightarrow$  prevent overfitting}: 
        ensure that we do not use all the basis functions

\end{itemize}

\paragraph{Weaknesses}
\begin{itemize}
    \item issues for multi-class classification due to the non-probabilistic 
        aspect of the model: \uB{output scores are not on a calibrate scale}
\end{itemize}

\paragraph{Relationships with other methods}
\paragraph{Examples of application}

\subsection{MODEL COMPARISON}
\paragraph{Comparison of discriminative kernel methods}
\begin{itemize}
    \item \emph{L1VM}: $l_{1}$-regularized vector machine
    \item \emph{L2VM}: $l_{2}$-regularized vector machine
    \item \emph{SVM}: Support Vector Machine
    \item \emph{RVM}: Relevance Vector Machine
\end{itemize}



%\subsection{Logistic Regression}
%\paragraph{Purpose}
%\paragraph{Assumptions}
%\paragraph{Theory}
%\paragraph{Strengths}
%\paragraph{Weaknesses}
%\paragraph{Relationships with other methods}
%\paragraph{Examples of application}


\section{Model Selection}
\subsection{Bayesian Variable Selection}
\paragraph{Purpose}
Let \tB{$\gamma_{j}: 
\begin{cases}
    \gamma_{j} = 1 \Leftarrow \text{feature } j \text{ is relevant}\\
    \gamma_{j} = 0 \Leftarrow \text{otherwise}\\
\end{cases}$} our goal is to compute the posterior over models:
\begin{center}
    $\prob{\gamma|\mathcal{D}} = \dfrac{e^{-f(\gamma)}}{\su{\gamma'}{}e^{-f(\gamma')}}$
\end{center}
where the cost function is defined by $f(\gamma) \triangleq -\left[\log\left(\prob{
\mathcal{D}|\gamma}\right) + \log\left(\prob{\gamma}\right)\right]$

\paragraph{Assumptions}
\paragraph{Theory}
\subparagraph{Spike and slab model}
remember that posterior is given by $\prob{\bm{\gamma}|\mathcal{D}}\propto \prob{
\mathcal{D}|\bm{\gamma}}\prob{\bm{\gamma}}$.\\
It is common to use the following prior $\prob{\bm{\gamma}} = \prd{j=1}{D}\text{
\emph{Ber}}(\gamma_{j}|\pi_{0}) = \pi_{0}^{\norm{\gamma}_{0}}(1-\pi_{0})^{D-\norm{
\gamma}_{0}}$, where $\pi_{0}$ is the probability that a feature is relevant.\\
The likelihood is defined as follows: $\prob{\mathcal{D}|\bm{X},\bm{\gamma}} = 
\Su{}{}\Su{}{}\prob{\bm{y}|\bm{X},\bm{w},\bm{\gamma}}\prob{\bm{w}|\bm{\gamma},
\sigma^{2}}\prob{\sigma^{2}}d\bm{w}d\sigma^{2}$\\
Consider the prior $\prob{\bm{w}|\bm{\gamma},\sigma^{2}}$ in standardizing the inputs, 
a reasonable prior is $\mathcal{N}(0, \sigma^{2}\sigma_{w}^{2})$, where 
$\sigma_{w}^{2}$ controls how big we expect the coefficients associated with the 
relevant variables to be, which is scaled by the overall noise. We can summarize this 
prior as follows: $\prob{\bm{w}_{j}|\sigma^{2},\gamma_{j}}
\begin{cases}
    \delta_{0}(w_{j}) \Leftarrow \gamma_{j} = 0\\
    \mathcal{N}(w_{j}|0, \sigma^{2}\sigma_{w}^{2}) \Leftarrow \gamma_{j} = 1
\end{cases}$
\uB{the first term is a "spike" at the origin, as $\sigma_{w}^{2}\rightarrow +\infty$ 
the distribution $\prob{w_{j}|\gamma_{j} = 1}$ approaches a uniform distribution which 
can be thought of as a "slab"}.

\subparagraph{Bernoulli-Gaussian model}
we have 
$\begin{cases}
    \prob{y_{i}|\bm{x}_{i},\bm{w}, \bm{\gamma},\sigma^{2}} = \mathcal{N}\left(\su{j}{}
    \gamma_{j}w_{j}x_{ij}, \sigma^{2}\right) \\
    \prob{\gamma_{j}} = \text{\emph{Ber}}(\pi_{0})\\
    \prob{w_{j}} = \mathcal{N}(0, \sigma_{w}^{2})
\end{cases}$
we can think of the \tB{$\gamma_{j}$ as a masking out the weights $w_{j}$}. Unlike the
spike and slab model we do not integrate the irrelevant coefficients, they always 
exists.\\
\uB{One interesting aspect of this model is that it can be used to derive objective 
function that is widely used in the non-Bayesian subset selection literature.}
\subparagraph{Algorithms}
assuming that we want to find the MAP model.
\begin{itemize}
    \item Single best replacement: at each step, we define a \uB{neighborhood of the 
        current model to be all models than can be reached by flipping a single bit
        of $\gamma$}
    \item Orthogonal least squares: we start from an empty set of variables and we 
        add the best feature \tB{$j^{*} = \displaystyle\argmin_{j\notin\bm{\gamma}_{t}}\displaystyle
            \min_{\bm{w}}\norm{\bm{y}-\bm{X}_{\bm{\gamma}_{t}\cup j}\bm{w}}^{2}$} 
    \item Orthogonal matching pursuits: as Orthogonal least squares is somewhat 
        expensive, a simplification is to freeze the current weight and then pick the 
        next feature to add by solving \tB{$j^{*} = \displaystyle\argmin_{j\notin\bm{\gamma}_{t}}
        \displaystyle\min_{\beta} \norm{\bm{y}-\bm{X}\bm{w}_{t} - \beta\bm{x}_{\cdot j}}^{2}$}. And
        $\beta = \dfrac{\bm{x}_{\cdot j}^{T}\bm{r}_{t}}{\norm{\bm{x}_{\cdot j}}^{2}}$, where
        $\bm{r} = \bm{y}-\bm{X}\bm{w}_{t}$
    \item Backward selection: starts with all variables in the model and deletes the
    \item Bayesian matching pursuits: similar to OMP except it uses a Bayesian marginal
        likelihood scoring criterion instead of a least square objective.
\end{itemize}

\paragraph{Strengths}
\paragraph{Weaknesses}
\paragraph{Relationships with other methods}
\paragraph{Examples of application}

%\subsection{Logistic Regression}
%\paragraph{Purpose}
%\paragraph{Assumptions}
%\paragraph{Theory}
%\paragraph{Strengths}
%\paragraph{Weaknesses}
%\paragraph{Relationships with other methods}
%\paragraph{Examples of application}


\section{Regularization}
\subsection{$l_{1}$ regularization}
\paragraph{Purpose}

\paragraph{Assumptions}
We assume $\prob{\bm{w}|\lambda} = \prd{j=1}{D}\text{\emph{Lap}}(w_{j}|0,
\frac{1}{\lambda}) \propto \prd{j=1}{D}e^{-\lambda|w_{j}|}$
\paragraph{Theory}
For penalized negative log likelihood has the form: $f(\bm{w}) = -\log\left(\prob{
\mathcal{D}|\bm{w}}\right) - \log\left(\prob{\bm{w}|\lambda}\right) = \text{\emph{NLL}}
(\bm{w}) + \lambda\norm{\bm{w}}_{1}$.\\
Geometrically we understand that as we relax the constraint we grow $l_{1}$ ball until
it meets the objective, \tB{the corners of the ball are more likely to intersect the
ellipse than one of the sides} especially in high dimensions because the corners "stick
out". The corners correspond to sparse solutions which lie on the coordinate axes. By
contrast when we grow the $l_{2}$ ball it can intersect the objective at any point, 
there are no "corners" so there is no preference for sparsity. 
\paragraph{Strengths}
\begin{itemize}
    \item can give quite different results if the data is slightly perturbed
\end{itemize}

\paragraph{Weaknesses}
\paragraph{Relationships with other methods}
\paragraph{Examples of application}

\subsection{Regularization path}
\paragraph{Purpose}
As we increase $\lambda$, the solution vector $\hat{\bm{w}}(\lambda)$ will tend to get
sparser, although not necessary monotically. For each feature $j$ we can plot the 
values $\hat{w}_{j}(\lambda)$ vs $\lambda$ which is known as \emph{regularization path}
\paragraph{Assumptions}
\paragraph{Theory}
\paragraph{Strengths}
\paragraph{Weaknesses}
\paragraph{Relationships with other methods}
\paragraph{Examples of application}

\subsection{$l_{1}$ algorithms}
\paragraph{Purpose}
These algorithms exploit the fact that one can quickly compute $\hat{\bm{w}}(
\lambda_{k})$ from $\hat{\bm{w}}(\lambda_{k-1})$ if $\lambda_{k} \approx \lambda_{k-1}$
this is known as \emph{warm starting}.
\paragraph{Assumptions}
\paragraph{Theory}
\subparagraph{Coordinate descent}
$w^{*}_{j} = \argmin_{z}f(\bm{w} + z\bm{e}_{j}) - f(\bm{w})$ with $\bm{e}_{j}$ is the 
\emph{j}'th unit vector. 

\paragraph{Strengths}
\paragraph{Weaknesses}
\paragraph{Relationships with other methods}
\paragraph{Examples of application}

%\subsection{Logistic Regression}
%\paragraph{Purpose}
%\paragraph{Assumptions}
%\paragraph{Theory}
%\paragraph{Strengths}
%\paragraph{Weaknesses}
%\paragraph{Relationships with other methods}
%\paragraph{Examples of application}

