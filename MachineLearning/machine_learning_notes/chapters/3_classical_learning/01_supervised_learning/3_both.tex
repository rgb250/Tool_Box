\subsection{Mixture models}
\paragraph{Purpose}
It is the simplest form of \emph{Latent Variable Models (LVMs)} is when $z_{i}\in
\{k\}_{1\leq k\leq K}$. 2 mains application of mixture models:
\begin{itemize}
    \item use them as \emph{black-box} density model, $p(\bm{x}_{i})$, useful for data
        compression, outlier detection and creating generative classifiers
    \item use for clustering
\end{itemize}

\paragraph{Assumptions}
\paragraph{Theory}
We use a discrete prior $p(z_{i}) = \text{\emph{Cat}}(\bm{\ conda install -c conda-forge juliapi})$ and the likelihood
$p(\bm{x}_{i}|z_{i} = k)$, finally the \textbf{mixture model} is :
\begin{center}
    $\prob{\bm{x}_{i}|\bm{\theta}} = \su{k=1}{K}\pi_{k}\prob{\bm{x}_{i}|z_{i}=k,
    \bm{\theta}} = \su{k=1}{K}\pi_{k}\mathbb{P}_{k}\left(\bm{x}_{i}|\bm{\theta}\right)$
\end{center}
where $\mathbb{P}_{k}$ is the $k$'th \emph{base distribution}.
\subparagraph{Mixtures of Gaussian}
Each base distribution in the mixture is a multivariate Gaussian with mean $\mu_{k}$ 
and covariance matrix $\bm{\Sigma}_{k}$:
$\prob{\bm{x}_{i}|\bm{\theta}}=\su{k=1}\pi_{k}\mathcal{N}\left(\bm{x}_{i}|\bm{\mu}_{k},
    \bm{\Sigma_{k}}\right)$\\
Given a sufficiently large number of mixture components a \tB{Gaussian Mixture Models
(GMMs) can be used to approximate any density defined on $\mathbb{R}^{D}$}.
\subparagraph{Mixtures of multinoullis}
can be used to define density models on data consisting of a \emph{D}-dimensional bit
vectors: $\prob{\bm{x}_{i}|z_{i}=k,\bm{\theta}} = \prd{j=1}{D}\text{\emph{Ber}}(x_{ij}|
\mu_{jk})=\prd{j=1}{D}\mu_{jk}^{x_{ij}}(1-\mu_{jk})^{1-x_{ij}}$\\
where $\mu_{jk}$ is the probability that bit $j$ turns on in cluster $k$.
\subparagraph{Mixture of experts}
are build from a discriminative perspective, they relies on the idea that a good
model different linear method each applying to a different part of the input space.\\
We can model this by allowing the mixing weights and the mixture densities to be
input-dependent: 
\begin{center}
    $\begin{cases}
        \prob{y_{i}|\bm{x}_{i},z_{i}=k,\bm{\theta}} = \mathcal{N}(y_{i}|\bm{w}_{k}^{T}
        \bm{x}_{i},\sigma_{k}^{2}) \\
        \prob{z_{i}|\bm{x}_{i},\bm{\theta}} = \text{\emph{Cat}}(z_{i}|S(\bm{V}^{T}
        \bm{x}_{i}))
    \end{cases}$
\end{center}
Useful in solving inverse problems, the ones in which we have to invert a many-to-one
mapping, for example in robotics where the location of the end effector (hand) $\bm{y}$
is uniquely determined by the joint angles of the motors, $\bm{x}$.

The overall posterior is then
\begin{center}
    \fr{$\prob{y_{i}|\bm{x}_{i},\bm{\theta}} = \su{k}{}\prob{z_{i}=k|\bm{x}_{i},z_{i}=k
    ,\bm{\theta}}$}
\end{center}




\paragraph{Strengths}

\paragraph{Weaknesses}
\begin{itemize}
    \item Use of a single latent variable to generate the observation
\end{itemize}

\paragraph{Relationships with other methods}
\paragraph{Examples of application}


%\subsection{Logistic Regression}
%\paragraph{Purpose}
%\paragraph{Assumptions}
%\paragraph{Theory}
%\paragraph{Strengths}
%\paragraph{Weaknesses}
%\paragraph{Relationships with other methods}
%\paragraph{Examples of application}
