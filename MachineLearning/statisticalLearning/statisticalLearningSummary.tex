\documentclass[a4paper, 10pt]{book}  %livre

% Permet l'encodage avec le Unicode Transformation Format-8-bit
\usepackage[utf8]{inputenc}
% Permet la gestion des caractères accentués ainsi que la stabilité des impressions en P.D.F.
\usepackage[T1]{fontenc}
% Permet la stabilisation de l'écriture
\usepackage{lmodern}
% Permet d'utiliser les liens hypertexte sans altérer la bibliothèque KOMA
\usepackage{scrhack}
\KOMAoptions{hyperref=false}
% Utilise les règles gramaticales anglaise
\usepackage[english]{babel}
%Utilisation de lien hypertexte:
\usepackage{hyperref}
%%Utilise les règles typographique de la bibliothéques KOMA
%\setkomafont{author}{\scshape}
\usepackage{blindtext}

% Package pour avoir plus d'arguments dans ses commandes.
\usepackage{xargs}

%Bibliothèque mathématiques pour la police.
\usepackage{amsfonts}
%Bibliothèques mathématiques générale.
\usepackage{amsmath}
\usepackage{amssymb}
% Package écriture en gras math mode
\usepackage{bm}
%Pour les listes indexées alphabétiquement
\usepackage{enumitem}
%Pour appliquer \mathbb à des nombres
\usepackage{bbm}
%Package pour l'indexation de matrices
\usepackage{blkarray}
%\usepackage{dsfont}
%Bibliothèque pour l'affichage de nombre avec la typographie définie
\usepackage{numprint}
\newcommand{\np}[1]{\numprint{#1}}
% Pour la notation scientifique
\usepackage{siunitx}
\sisetup{locale= FR,exponent-product=.}
\DeclareSIUnit\year{ann\'{e}ee}
%Positionnement des images
\usepackage{float}
\usepackage{subcaption}
%Utilisation des couleurs
\usepackage{xcolor}
% Package pour le soulignement
\usepackage{color,soulutf8}
\setulcolor{red}
% Package pour les annotations
\usepackage{todonotes}
%\usepackage[pygments]{pythontex} % Pour utiliser Python
%\usepackage{fvextra} % Utile pour la mise en forme du code source inséré
%Package pour la programmation
\usepackage{listings}
% Package pour les scripts en Python
%\lstset{language=Python,
%\lstset{
%backgroundcolor=\color{white},
%    basicstyle=\small\ttfamily,
%    breakatwhitespace=false,
%    breaklines=true,
%    captionpos=b,                                    % sets the caption-position to bottom
%    commentstyle=\color{teal},
%    escapeinside={\%*}{*)},
%    keepspaces=true,
%    keywordstyle=\color{violet},
%    numbers=left,
%    numberstyle=\tiny\color{darkgray},
%    showspaces=false,
%    showstringspaces=false,
%    showtabs=false,
%    stepnumber=1,
%    stringstyle=\color{purple},
%    otherkeywords={0,1,2,3,4,5,6,7,8,9},
%    }


%Définition de couleurs:
\definecolor{bleudefrance}{rgb}{.19, .55, .91}
\definecolor{pakistangreen}{rgb}{.0, .4, .0}
\definecolor{rossocorsa}{rgb}{0.83, 0.0, 0.0}
\definecolor{persimmon}{rgb}{0.93, 0.35, 0.0}
% Annotation auteur
\newcommand{\Moi}[1]{\todo[color = teal!40]{#1}}
\newcommand{\Cnsl}[1]{\todo[color = pakistangreen!40]{#1}}
\newcommand{\MeG}[1]{\todo[color = rossocorsa!40]{#1}}
%Notation récurrante:
\newcommandx{\hb}[1]{\widehat{\beta_{#1}}}
\newcommandx{\prth}[3]{\left( #1_{#2} \right)_{1\leq #2 \leq #3}}
\newcommandx{\prtH}[4]{\left( #1_{#2} \right)_{#3\leq #2 \leq #4}}
% Encadrement des résultats
\newcommand{\enc}[1]{\fcolorbox{rossocorsa}{white}{#1}}
\newcommand{\encB}[1]{\fcolorbox{bleudefrance}{white}{#1}}
\newcommand{\encV}[1]{\fcolorbox{pakistangreen}{white}{#1}}
\newcommand{\encN}[1]{\fcolorbox{black}{white}{#1}}
% Soulignement couleur
\newcommandx{\sN}[1]{\setulcolor{black}\ul{#1}}
\newcommandx{\sR}[1]{\setulcolor{rossocorsa}\ul{#1}}
\newcommandx{\sB}[1]{\setulcolor{bleudefrance}\ul{#1}}
\newcommandx{\sV}[1]{\setulcolor{pakistangreen}\ul{#1}}
\newcommandx{\sT}[1]{\setulcolor{teal}\ul{#1}}
\newcommandx{\sO}[1]{\setulcolor{persimmon}\ul{#1}}
% Text de couleur
\newcommandx{\tR}[1]{\textcolor{rossocorsa}{#1}}
\newcommandx{\tB}[1]{\textcolor{bleudefrance}{#1}}
\newcommandx{\tV}[1]{\textcolor{pakistangreen}{#1}}
\newcommandx{\tT}[1]{\textcolor{teal}{#1}}
\newcommandx{\tO}[1]{\textcolor{persimmon}{#1}}
% Écriture intervalle
\newcommandx{\inter}[2]{\left[\![#1, #2]\!\right]}
% Écriture intégrale
\newcommandx{\Su}[2]{{\displaystyle \int_{#1}^{#2}}}
% Écriture somme sigma
\newcommandx{\su}[2]{{\displaystyle \sum_{#1}^{#2}}}
% Écriture produit pi
\newcommandx{\prd}[2]{{\displaystyle \prod_{#1}^{#2}}}
%Écriture limite lim
\newcommandx{\lm}[2]{{\displaystyle \lim_{#1\to #2}}}
\DeclareMathOperator*{\argmax}{arg\,max}
\DeclareMathOperator*{\argmin}{arg\,min}
% Écriture P(X)
\newcommandx{\prob}[1]{\mathbb{P}\left(#1\right)}
% Écriture P({X})
\newcommandx{\Prob}[1]{\mathbb{P}\left(\left\{#1 \right\}\right)}
% Écriture P_{\{Y\}}({X})
\newcommandx{\ProbC}[2]{\mathbb{P}_{\left\{#1\right\}}\left(\left\{#2\right\}\right)}
% Ecriture Esperance et Variance
\newcommandx{\E}[1]{\mathbb{E}\left(#1\right)}
\newcommandx{\V}[1]{\mathbb{V}\left(#1\right)}
%Symbole de la norme
\newcommandx{\norm}[1]{\left\lVert#1\right\rVert}
%Symbole du produit scalaire:
\newcommandx{\sP}[2]{\langle\left. #1\right\vert #2 \rangle}


%%%%%%%%% PYTHON %%%%%%%
% Default fixed font does not support bold face
\DeclareFixedFont{\ttb}{T1}{txtt}{bx}{n}{12} % for bold
\DeclareFixedFont{\ttm}{T1}{txtt}{m}{n}{12}  % for normal

% Custom colors
\definecolor{deepblue}{rgb}{0,0,0.5}
\definecolor{deepred}{rgb}{0.6,0,0}
\definecolor{deepgreen}{rgb}{0,0.5,0}

% Python style for highlighting
\newcommand\pythonstyle{\lstset{
language=Python,
backgroundcolor=\color{white},
basicstyle=\small\ttfamily,
breakatwhitespace=false,
breaklines=true,
captionpos=b,                                    % sets the caption-position to bottom
commentstyle=\color{teal},
emph={MyClass,__init__},          % Custom highlighting
escapeinside={\%*}{*)},
keepspaces=true,
keywordstyle=\color{violet},
numbers=left,
numberstyle=\tiny\color{darkgray},
showspaces=false,
showstringspaces=false,
showtabs=false,
stepnumber=1,
stringstyle=\color{purple},
otherkeywords={0,1,2,3,4,5,6,7,8,9},
}}

% Python environment
\lstnewenvironment{python}[1][]
{
\pythonstyle
\lstset{#1}
}
{}

% Python for external files
\newcommand\pythonexternal[2][]{{
\pythonstyle
\lstinputlisting[#1]{#2}}}

% Python for inline
\newcommand\pythoninline[1]{{\pythonstyle\lstinline!#1!}}



%%%%%%%%% R %%%%%%%

% R style for highlighting
\newcommand\rstyle{\lstset{
language=R,
backgroundcolor=\color{white},
basicstyle=\small\ttfamily,
breakatwhitespace=false,
breaklines=true,
captionpos=b,                                    % sets the caption-position to bottom
commentstyle=\color{teal},
emph={MyClass,__init__},          % Custom highlighting
%escapeinside={\%*}{*)},
keepspaces=true,
keywordstyle=\color{bleudefrance},
numbers=left,
numberstyle=\tiny\color{darkgray},
rulecolor=\color{black},
showspaces=false,
showstringspaces=false,
showtabs=false,
stepnumber=1,
stringstyle=\color{pakistangreen},
otherkeywords={0,1,2,3,4,5,6,7,8,9},
}}

% R environment
\lstnewenvironment{rcode}[1][]
{
\rstyle
\lstset{#1}
}
{}

% R for external files
\newcommand\rexternal[2][]{{
\rstyle
\lstinputlisting[#1]{#2}}}

% R for inline
\newcommand\rinline[1]{{\rstyle\lstinline!#1!}}

\title{Abstract of the introduction to \emph{Statistical Learning}}
\author{Siger}

\begin{document}

\maketitle

Si Dieu est infini, alors je suis une partie de Dieu sinon je serai sa limite\ldots

\tableofcontents

\chapter{Introduction to Statistical Learning}
\section{Statistical Learning}
\subsection{What is statistical Learning?}
\paragraph{What is Statistical Learning?}
Notation:
\begin{itemize}
 \item \textbf{Input variables}: predictors, independent variables
   features or variables.
 \item \textbf{Output variables}: response, dependent variables.
\end{itemize}
When we observe a \tB{quantitative response $Y$} knowing there are 
\tB{$p$ predictors} such as\\
$X=\left( X_{i} \right)_{1\leq i\leq p}$ then we write:
\enc{$Y=f\left( X \right)+\epsilon$}\\
$\begin{cases}
f\text{ is some fixed but unknown function of }X\text{ 
and represents the systematics information that }X\text{ provides
about }Y\\
\epsilon\text{ is a random error term, independent of X and has mean }0
\end{cases}
$
\begin{figure}[h]
  \centering
  \includegraphics[width=.5\textwidth]{./chap/1chap/1sec/1images/1_1estimationOfF.png}
  \caption{Estimation of $f$}
  \label{fig:1.1}
\end{figure}\\
\begin{figure}[H]
  \centering
  \includegraphics[width=.5\textwidth]{./chap/1chap/1sec/1images/1_2estimationF2D.png}
  \caption{Estimation of $f$ in $2-D$}
  \label{fig:1.1}
\end{figure}
In essence \textbf{Statistical Learning} refers to a set of approaches
for estimating $f$.
\paragraph{Why estimate $f$}
\subparagraph{Prediction}
\encV{$\widehat{Y}=\widehat{f}\left( X \right)$}
$
\begin{cases}
  \widehat{f}\text{ represents our estimating for }f\\
  \widehat{Y}\text{ represents the resulting prediction for }Y
\end{cases}$\\
$\widehat{f}$ is often treated as a \emph{black box} since \tB{we pay
more attention to its prediction accuracy than its exact form}.\\
The accuracy of $\widehat{Y}$ depends on $2$ quantities\\
$\begin{cases} 
  reductible~error\text{, can be \tB{improved by using most appropriate
  statistical learning technique}}\\
  irreductible~error\text{, cannot be changed and \tB{have external 
  causes which are out of control}}
\end{cases}$\\
 simple calculus shows that:
\begin{align*}
  \E{\left[Y-\widehat{Y}\right]^{2}}&=\E{\left[f\left( X \right)+\epsilon-\widehat{f}\left( X \right)\right]^{2}}\\
  &= \underbrace{\E{\left[ f\left( X \right)-\widehat{f}\left( X \right) \right]^{2}}}_{Reducible}+\underbrace{\V{\epsilon}}_{Irreducible}\text{ think that $\E{\epsilon}=0$}
\end{align*}
\subparagraph{Inference} Now we want to know how $Y$ evolves when $X$
changes, so we cannot considerate anymore $f$ as a black box.\\
\begin{itemize}
\item \emph{Which predictors are associated with the response?} (\tB{to
discover variables which have the most important influence therewith 
to reduce number of considered variables})
\item \emph{What is the relationship between the response and each
  predictor?}(\tB{Which components increase Y value and which decrease
  it})
\item \emph{\tB{Can the relationship between $Y$ and each predictor be
	adequately summarized using linear equation} or is the relationship
more complicated?}
\end{itemize}
\paragraph{How do we estimate $f$}
\subparagraph{Aim} Let $(i,j)\in\inter{1}{n}\times\inter{1}{p}$ and 
\tB{$x_{ij}$ represent the value of the $j^{th}$ predictor for $i^{th}$
observation}. Correspondingly let $y_{i}$ represent the response 
variable for the $i^{th}$ observation.\\
Then our training data consists of $\left\{ \left( x_{i},y_{i}
\right)_{1\leq i\leq n}\right\}\text{ where }x_{i}=\begin{pmatrix}
x_{i1}\\.\\.\\.\\x_{ip}\end{pmatrix}$.\\
\tR{We want to find a function $\widehat{f}$ such that $Y\approx
\widehat{f}\left( X \right)$ for any observation $\left( X,Y \right)$}
\subparagraph{Parametric methods}
\begin{enumerate}
	\item \tB{\emph{Assumption about functional form of $f$}}\\ For
		example: $f\left( X \right)=\beta_{0}+\su{{i=1}}{p}
		\beta_{i}X_{i}$ in this case instead to estimate 
		entirely $p$-dimensional function $f$ we only need to
		estimate $\left( \beta_{i} \right)_{0\leq i\leq p}$
  \item After the model selection, \tB{we need a procedure which uses
\emph{data training} to \textit{fit} or \textit{train} the model}.
    \\For example we need to estimate
    $\left( \beta_{i} \right)_{0\leq i\leq p}$ such that:
    $Y\approx\beta_{0}+\su{{i=1}}{p}\beta_{i}X_{i}$\\ The most common
    approach to fit the model is \emph{least squares}.
\end{enumerate}\encV{
The parametric methods allow to reduce the problem of estimating $f$
down
to one of estimating a set of parameters}.
\begin{figure}[H]
\begin{subfigure}{0.5\textwidth}
  \includegraphics[width=0.9\linewidth, height=5cm]{./chap/1chap/1sec/1images/1_3linearEstimation.png} 
  \caption{Linear estimation:\\$income=\\\beta_{0}+\beta_{1}\times education+\beta_{2}\times seniority$}
  \label{fig:1.3_1}
\end{subfigure}
\begin{subfigure}{0.5\textwidth}
  \includegraphics[width=0.9\linewidth, height=5cm]{./chap/1chap/1sec/1images/1_4BestEstimation.png} 
  \caption{Best estimation of $f$ for\\ $income\approx f\left( income,seniority \right)$}
  \label{fig:1.3_2}
\end{subfigure}
\caption{Estimation of $f$ with 2 degrees of precision}
\label{fig:1.4}
\end{figure}

\subparagraph{Non-Parametric methods}
Any parametric approaches brings with it the possibility that the
functional form used to estimate $f$ is very different from the true
$f$. In contrast non-parametric approaches completely avoid this danger
since \tB{no assumption about the form of $f$ is made.} But non
parametric approaches do suffer from non-reducing problem, and \sB{they
need a very greater number of observation than with parametric 
approaches}.
\paragraph{The trade-off between Prediction Accuracy and model
Interpretability}
When inference is the goal, there are clear advantages to using simple
and relatively inflexible statistical learning methods.
\begin{figure}[H]
   \centering
   \includegraphics[width=.7\textwidth]{./chap/1chap/1sec/1images/1_5ToChooseAgoodApproach.png}
   \caption{To know which methods to use.}
   \label{fig:1.4}
 \end{figure}
 \paragraph{Supervised versus unsupervised learning}
 We speak about \tR{\emph{supervised problems}} \sR{when for each predictors
 $x_{i}$ there is an associated response measurement $y_{i}$}.\\
 Whereas in \tR{\emph{unsupervised problems}} \sR{we observe a vector of
 measurements $x_{i}$ but not associated response $y_{i}$}. Then we can
 seek to understand the relationship between the variables or between
 observations.\\ One statistical tool that we may use in this setting
 is \emph{clustering}
 \begin{figure}[H]
   \centering
   \includegraphics[width=.7\textwidth]{./chap/1chap/1sec/1images/1_6Clustering.png}
   \caption{Clustering methods is used for \emph{unsupervised problems}.}
   \label{fig:1.5}
 \end{figure}
 \paragraph{Regression versus Classification problems}
 In general Regression is used for quantitative variables whereas
 classification is used for qualitative variables but we can find
 several counter-examples.

\subsection{Assessing Model Accuracy}
\paragraph{Measuring the quality of a fit}
In regression setting, the most commonly-used measure of model accuracy
is the \tR{\emph{Mean Squared Error (MSE)}}:\encV{$MSE=\dfrac{1}{n}
\su{{i=1}}{n}\left( y_{i}-\widehat{f}\left( x_{i} \right) \right)^{2}$}
\\We do not really care about whether for all $i\in\inter{1}{n}
\widehat{f}\left( x_{i} \right)\approx y_{i}$, instead \tB{we want to 
know whether a previously unseen observation not used to train the
statistical learning method $\left( x_{0},y_{0} \right)$ is such that
$\widehat{f}\left( x_{0} \right)\approx y_{0}$}.\\In other words if we
had a large number of test observations, we could compute \encV{$Ave
\left( \left(y_{0}-\widehat{f}\left( x_{0} \right)\right)^{2} \right)$}
\tR{We want to choose the method that gives the lowest \emph{test MSE},
as opposed to the lowest \emph{training MSE}}.\\Many 
statistical methods estimate coefficients so as to minimize the 
training set MSE, then training set MSE can be quite small, but test
MSE is often much larger.
\begin{figure}[H]
  \centering
  \includegraphics[width=\textwidth]{./chap/1chap/1sec/2images/2_1trainingMSEandTestMSE.png}
  \caption{Left:Data simulated from $f$, shown in black and 3 estimates
  of $f$\\Right: Training MSE in grey and test MSE in red.\\Squares
represent the training and test MSEs for the 3 fits shown in the 
left-hand pannel}
  \label{fig:2.1}
\end{figure}
The blue curve minimizes the test MSE, and visually appears to estimate
$f$ the best in the left-hand panel.\\\tR{The horizontal dashed line
indicates $\V{\epsilon}$ the irreducible error in which corresponds
to the lowest achievable test MSE among all possible methods.}\\
\tB{When a given method yields a high training MSE but a low test MSE 
we say that data are \emph{overfitting}}.

\paragraph{The Bias-Variance Trade-Off}\enc{
$\E{\left( y_{0}-\widehat{f}\left( x_{0} \right) \right)^{2}}=
\V{\widehat{f}\left( x_{0} \right)}+\left[ Bias\left( \widehat{f}
\left( x_{0} \right)\right) \right]^{2}+\V{\epsilon}$}
$\begin{cases}\E{\left(y_{0}-\widehat{f}\left(x_{0}\right)\right)^{2}}
\text{ defines the \tB{\emph{expected test MSE}} and refers to the 
average test MSE.}\\\V{\hat{f}(x_{0})}\text{ the amount by which $\widehat{f}$ 
would change if we estimating it using a different training data.}\\
\left[Bias\left(\hat{f}(x_{0})\right)\right]^{2}\text{ refers to the error that is introduced by approximating a
real-life problem.}\end{cases}$\\As we increase the flexibility of a
class methods, the bias tends to initially decrease faster than the 
variance increases. 
\begin{figure}[H]
  \centering
  \includegraphics[width=\textwidth]{./chap/1chap/1sec/2images/2_2biaisVarianceMSE.png}
  \caption{Squared bias, variance and test MSE.\\$\V{\epsilon}$ indicates by the dashed line.}
  \label{fig:2.1}
\end{figure}
\subsection{Bias, Variance and Model Complexity}
The Loss function for measuring errors between $Y$ and $\hat{f}(X)$ is denoted by $L\left(Y,
\hat{f}(X)\right)$. Typical choices are: 
$$ L\left(Y,\hat{f}(X)\right) = 
\begin{cases}
	\left(Y-\hat{f}(X)\right)^{2}\text{ squared error}\\
	\left|Y-\hat{f(X)}\right|\text{ absolute error}
\end{cases}
$$
There are 3 important quantities:
$$ 
\begin{cases}
	\overline{err} = \dfrac{1}{N}\su{{i=1}}{N}L\left(y_{i},\hat{f}(x_{i})\right)\\
	Err_{\mathcal{T}}=\E{L\left(Y,\hat{f}(X)\right)|\mathcal{T}}\text{ \emph{Test error}}\\
	Err=\E{L\left(Y,\hat{f}(X)\right)}=\E{Err_{\mathcal{T}}}\text{ \emph{Test error}}
\end{cases}
$$
\paragraph{The classification setting}
Suppose that we seek to estimate $f$ on the basis of training 
observations $\left\{ \left( x_{i},y_{i} \right) \right\}_{1\leq i\leq
n}$ where now $\left( y_{i} \right)_{1\leq i\leq n}$ are qualitative.\\
The most common approach for quantifying the accuracy of our estimate
$\widehat{f}$ is the \tB{training \emph{error rate}}:\\
\enc{$\dfrac{1}{n}
\su{{i=1}}{n}I_{y_{i}\neq \widehat{y}_{i}}\text{ with }I_{y_{i}\neq
\widehat{y}_{i}}\begin{cases}1\Leftarrow y_{i}\neq \widehat{y}_{i}\\0
\Leftarrow y_{i}=\widehat{y}_{i}\end{cases}$}.\\ The \emph{test error
rate} associated with a set of test observation of the form $\left( 
x_{0},y_{0} \right)$ is given by:\encV{$Ave\left( I_{y_{0}\neq 
\widehat{y}_{0}} \right)$}
%\subparagraph{The Bayes Classifier}
%\sB{The test error rate is minimized, on average, by a very simply 
%classifier that assigns each observation to the most likely class}. We
%should simply assign a test observation with predictor vector $x_{0}$
%to the class $j$ for which $\ProbC{X=x_{0}}{Y=j}$ is largest.
%\begin{figure}[H]
%  \centering
%  \includegraphics[width=\textwidth]{./chap/1chap/1sec/2images/2_3bayesClassifier.png}
%  \caption{A simulated data set consisting of 100 observations in each
%  of 2 groups, indicated in blue and in orange.\\The purple dashed line
%represents the \emph{Bayes decision boundary}, the orange background
%grid indicates the region in which a test observation will be assigned
%to the orange class, and likewise to blue color.}
%  \label{fig:2.2}
%\end{figure}
%The orange shaded region reflects the set of
%points for which $\ProbC{X}{Y=orange}$ is greater than $50\%$, while 
%the blue shaded region indicates the set of points for which the 
%probability is below $50\%$. The purple dashed line represents the 
%points where the probability is exactly $50\%$. \sB{Since the Bayes 
%classifier will always choose the class for which $\ProbC{X=x_{0}}{Y=j}
%$ is largest, the error rate at $X=x_{0}$ will be $1-\max\limits_{j}
%\ProbC{X=x_{0}}{Y=j}$}.\\ In general, the overall Bayes error rate is
%given by \enc{$1-\E{\max\limits_{j} \ProbC{X}{Y=j}}$}
%\subparagraph{K-Nearest neighbors}
%\sB{For real data we do not know the conditional distribution of $Y$ 
%given $X$ and so computing the Bayes classifier is impossible}.
%Therefore, the Bayes classifier serves as an unattainable gold standard
%against which to compare other methods.\\Given a positive integer $K$ 
%and a test observation $x_{0}$.
%\begin{enumerate}
%  \item It identifies the $K$ points in the training data that are
%    closet to $x_{0}$ represented by $\mathcal{N}_{0}$
%  \item It then estimates the conditional probability for class $j$ as
%    the fraction of points in $\mathcal{N}_{0}$ whose response values
%  equal $j$:$\ProbC{X=x_{0}}{Y=j}=\dfrac{1}{K}\su{{i\in\mathcal{N}_{0}}
%}{{}}I_{y_{i}=j}$
%\end{enumerate}
%\begin{figure}[H]
%  \centering
%  \includegraphics[width=\textwidth]{./chap/1chap/1sec/2images/2_4KNNworking.png}
%  \caption{KNN approach for $k=3$\\A test observation at which a
%  predicted class label is desired is shown as a black cross.\\
%Right-hand panel: The KNN decision boundary for this example is shown
%in black. The blue grid indicates the region in which a test 
%observation will be assigned to the blue class, likewise for the orange
%grid.}
%  \label{fig:2.2}
%\end{figure}Explanation of the KNN approach:\\
%Suppose that we choose $K=3$, then KNN will first identify the 3
%observations that are closet to the cross. This neighborhood is shown
%as a circle. It consists of 2 blue points and 1 orange point, resulting
%in estimated probabilities of $\frac{2}{3}$ for the blue class and
%$\frac{1}{3}$ for the orange class. Hence KNN will predict that the
%black cross belongs to the blue class.\\\\
%Despite the fact that is a very simple approach, KNN can often produce
%classifiers that are surprisingly close to the optimal Bayes classifier.
%\begin{figure}[H]
%  \centering
%  \includegraphics[width=\textwidth]{./chap/1chap/1sec/2images/2_5KNNboundary.png}
%  \caption{The black curve indicates the KNN decision boundary with
%  K=10.\\The Bayes decision boundary is shown as a purple dashed line.}
%  \label{fig:2.2}
%\end{figure}
%As $K$ grows, the method become less flexible and produces a decision
%boundary that is close to linear. This corresponds to a low-variance,
%but high-bias classifier

%\subsection{Introduction to R}
%\paragraph{Basic command}
R uses function to perform operations.\\
\begin{itemize}
  \item \emph{?funcname} will always cause R to open a new help file
    window
  \item \emph{ls()} allows us to look at a list of the objects such
    as data and functions that we have saved.
  \item \emph{rm()} can be used to delete any that we don't want.
  \item \emph{matrix()} can be used to create matrix of number. Example
  : \textit{matrix(c(1,2,3,4), 2, 2, byrow=TRUE)}
  \item \emph{rnorm()} function generates a vector of random normal
    variables. Example: \textit{rnorm(50, mean=50, sd=.1)}
  \item \emph{cor()} compute the correlation between 2 sets of numbers.
    Example: \textit{cor(x,y)}
  \item \emph{set.seed()} function takes an (arbitrary) integer 
    argument and allows to reproduce the exact same set of random
    numbers.
 \end{itemize}
 \paragraph{Graphics}
 \begin{itemize}
   \item \emph{plot()} function is the primary way to plot a data.
     Example: \textit{plot(x, y, xlab=``xName'', ylab=``yName'',
     main=``Plot of X vs Y'')}
   \item \emph{pdf()} is used to create pdf and \emph{dev.off()}
     indicates that we are done creating the plot.\\Example:\textit{
     pdf(``Figure.pdf'')\\plot(x, y, col=``green'')\\dev.off()}
   \item \emph{seq()} is used to create a sequence of numbers.
     Example: \textit{seq(-pi, pi, length=50)}
   \item \emph{contour()} function produces a contour plot in order to
     represent 3-dimensional data.\\Example: \textit{y=x\\
     f=outer(x,y,function(x,y)cos(y)/(1+x\^ 2))\\contour(x,y,f)\\
   contour(x,y,f,nlevels=45,add=T)\\fa=(f-t(f))/2\\contour(x,y,fa,
 nlevels=15)}
 \item \emph{image()} function works the same way as \emph{contour()},
   except that it produces a color-coded plot whose colors depend on
   the $z$ value.
 \item \emph{persp()} can be used to produce a 3-dimensional plot. The
   \emph{theta} and \emph{phi} control the angles at which the plot is
   viewed.\\\textit{image(x,y,fa)\\persp(x,y,fa)\\persp(x,y,fa,
   theta=30)\\persp(x,y,fa,theta=30,phi=20)}
 \end{itemize}
 \paragraph{Indexing Data}
 For a matrix of length $4\times 4$:\\
 $A[c(1,3),c(2,4)]$, $A[,1:2]$
 \paragraph{Loading Data}
 \begin{itemize}
   \item \emph{read.table()} is one of the primary ways to data
     importing.
   \item \emph{write.table()} to export data.
 \end{itemize}
 \paragraph{Additional Graphical and Numerical Summaries}
 \begin{itemize}
   \item \emph{attach()} function in order to make the variables in
     this data frame available by name.
   \item \emph{as.foctor()} function converts quantitative variables
     into qualitative variables
   \item \emph{hist()} function can be used to plot a histogram
   \item \emph{identify()} provides interactive method for identifying
     the value for a particular variable for point on a plot.
   \item \emph{summary()} produces a numerical summary of each variable
     in a particular data set.
   \item \emph{savehistory()} allows to save a record of all of the
     commands that we typed in the most recent session.
   \item \emph{loadhistory()} allows to load history of previous time.
 \end{itemize}

%%\subsection{Exercises}
%%\paragraph{Conceptual}
\begin{enumerate}
 \item
   \begin{itemize}
     \item[(a)] The number of predictors being small, and sample size
being extremely high, and knowing the expected test MSE $\E{\left( 
y_{0}-\widehat{f}\left( x_{0} \right) \right)^{2}}=\V{\widehat{f}\left(
x_{0} \right)}+\left[ Biais\left( \widehat{f}\left( x_{0} \right)
\right) \right]^{2}+\V{\epsilon}$ which refers to the average test MSE.
\\A great number of observations results an increase of the variance.
Then to compensate this increasing we should use a unflexible method,
because unflexible methods decrease the variance.\\Then we avoid the
overfitting risk.
\item[(b)] It is the opposite case, now the low amount number of
observations results to a high probability to make mistakes on our
approximate function $f$. This means that biais increases, therefore
we should use a flexible methods therewith to componsate the biais
increasing. However the variance will increase but it's not a problem
because the number of observation is small.
\item[(c)] If the relationship between predictors and response is
highly non-linear we need a method with a high level of flexibility.
Then the challange is to approximate the real life problem therefore we
must reduce the biais.
\item[(d)] We have no control on the $\epsilon$ so I do not know.
   \end{itemize}
 \item
   \begin{itemize}
     \item[(a)] Our predictors are almost all quantitative, there is
just the industry predictor which is qualitative however we can assign
a number in function of the industry.\\Then it looks like a regression
problem. We are most interested in inference.$n=500,p=4$
\item[(b)] Our predictors are all quantitative but response is
qualitative.\\Then it looks like a classification problem. We are most
interested in prediction $n=20,p=14$
\item[(b)] Our predictors are all quantitative and the response 
searched is quantitative.\\Then it looks like a
regression problem. We are most interested in prediction $n=?,p=4$
   \end{itemize}
 \end{enumerate}


\section{Linear Regression}
Here are a few important questions that we might seek to address:
Linear model is a model of the form:
$$ p(y|\bm{x}, \bm{\theta}) = \mathcal{N}\left(
\bm{\beta}^{T}\bm{x},\sigma^{2}\right)$$.
Linear regression can be made to model non-linear reltationships by replacing
$\bm{x}$ with some non-linear function of the inputs $\phi$
$$ p(y|\bm{x}, \bm{\theta}) = \mathcal{N}\left(
\bm{\beta}^{T}\phi(\bm{x}),\sigma^{2}\right)$$.
%\subsection{Simple linear regression}
%In general simple linear regression is written as:\encV{$Y\approx
\beta_{0}+\beta_{1}X$}.\\We might read ``$\approx$'' as ``is 
approximately modeled as''

\paragraph{Estimating the Coefficients}
Let $\left\{ \left( x_{i},y_{i} \right) \right\}_{1\leq i\leq n}$
represent $n$ observations, our goal is to obtain coefficient estimates
$\widehat{\beta}_{0},\widehat{\beta}_{1}$ such that the linear model
fits the available data well, so that $y_{i}\approx\beta_{0}+\beta_{1}
x_{i}$ for $i\in\inter{1}{n}$, in other words we want an \emph{
intercept} $\beta_{0}$ and a \emph{slope} $\beta_{1}$\\For $i\in\inter{1
}{n}, e_{i}=y_{i}-\widehat{y}_{i}$ represents the $i^{th}$ \emph{
residual}\\
Then we define the \begin{center}\enc{$\text{RSS(\tR{Residual Sum of
Squares})}=\su{{i=1}}{n}e_{i}^{2}=\su{{i=1}}{n}\left( y_{i}- \widehat{
\beta_{0}}-\widehat{\beta_{1}}x_{i}\right)^{2}$}\end{center}.Using some calculus show 
that minimizers are \enc{$\begin{cases}\widehat{\beta_{1}}=\dfrac{\su{{
i=1}}{n}(x_{i}-\overline{x})(y_{i}-\overline{y})}{\su{{i=1}}{n}(x_{i}-
\overline{x})^{2}}\\\widehat{\beta}_{0}=\overline{y}-\widehat{\beta}_{1
}\overline{x}\end{cases}$}
\begin{figure}[H]
  \centering
  \includegraphics[width=.3\textwidth]{./chap/1chap/2sec/1images/1_leastSquares.png}
  \caption{The least squares fit for the regression of sales onto TV.}
  \label{fig:2.1}
\end{figure}
We can see on the following plots, that $\widehat{\beta}_{0},\widehat{\beta}_{1}$ minimize the RSS.
\begin{figure}[H]
  \centering
  \includegraphics[width=.5\textwidth]{./chap/1chap/2sec/1images/2_leastSquaresCoefficients.png}
  \caption{Contour and 3-dimensionnal plots of the RSS on the 
Advertising data, using sales as the response and TV as the predictor}
  \label{fig:2.2}
\end{figure}

\paragraph{Assessing the Accuracy of the Coefficient Estimates}
When we assume that there is a relation between $X\text{ and }Y$ then
$Y=f\left( X \right)+\epsilon\begin{cases}f\text{ is a uknown function
 }\\\epsilon\text{ is a mean-zero random error term, is a catch-all for
what we miss with this simple model}\end{cases}$\\If $f$ is to be 
approximated by a linear function then we can write this relationship
as \encN{$Y=\beta_{0}+\beta_{1}X+\epsilon$}.\\This equation defines the
\emph{population regression line} which is the best approximation to
the true relationship between $X$ and $Y$.\\ We created $100$ random
$X_{s}$ and generated $100$ corresponding $Y_{s}$ from the model $Y=2+
3X+\epsilon$ where $\epsilon$ is generated from a normal distribution
with mean zero.
\begin{figure}[H]
  \centering
  \includegraphics[width=\textwidth]{./chap/1chap/2sec/1images/3_leastSquaredErrorLineVsPopulationRegressionLine.png}
  \caption{The left-hand pannel:red line is the popuation regression
  line, and in blue line the least squares estimate for $f(X)$ bassed
  on the observed data, shown in black.\\The right-hand pannel: is the
  same graph that left-hand pannel but with 10 other least sqaures
  estimates with each a distinct training data set but from the same
  model}
  \label{fig:2.3}
\end{figure}
If we use the sample mean $
\widehat{\mu}$ to estimate $\mu$ this estimate is \emph{unbiased}, in
the sense that on average, we expect $\widehat{\mu}$ to equal $\mu$\\
We have etablished that the average $\widehat{\mu}$'s over many data
sets will be very close to $\mu$ but that a single estimate $\widehat{
\mu}$ may be a substantial underestimate or overestimate of $\mu$.
\tB{How far off will that single estimate of $\mu$ be? We generally 
answer to this question by computing the \emph{SE (Standard Error)} of
$\widehat{\mu}$ written as $SE(\widehat{\mu})$}:\begin{center}\enc{
$\V{\widehat{\mu}}=SE\left( \widehat{\mu} \right)^{2}=\dfrac{\sigma^{2}
}{n}$}\\$\sigma$ is the standard deviation of each of the realizations 
$y_{i}$ of $Y$.\end{center}
\begin{center}\enc{$
\begin{cases}SE\left(\widehat{\beta}_{0}\right)^{2}=\sigma^{2}\left[
\dfrac{1}{n}+\dfrac{\overline{x}^{2}}{\su{{i=1}}{n}\left(x_{i}-
\overline{x}\right)^{2}}\right]\\
SE\left(\widehat{\beta}_{1}\right)^{2}=
\dfrac{\sigma^{2}}{\su{{i=1}}{n}\left(x_{i}-\overline{x}
\right)^{2}}\end{cases}$}\\$\sigma^{2}=\V{\epsilon}$\end{center}\Moi{$SE\left(\widehat{\beta_{1}}
\right)$ is smaller when the $x_{i}$ are more spread out}For these
formulas we need to assume that the errors $\epsilon_{i}$ are
uncorrelated with $\sigma^{2}$. This is cleary not true but the formula
still turn out to be a good approximation.\\\sB{In general $\sigma$ is
unknown but can be estimated from the data, the estimate of $\sigma$ is
known as} \tR{\emph{Residual Standard Error}} \encB{$RSE=\sqrt{\dfrac{
RSS}{n-2}}$}.\\\Moi{Roughly speaking when $\sigma^{2}$ is estimated we
should write $\widehat{SE\left(\widehat{\beta_{1}}\right)}$}\\\\
Standard can be used to define \emph{confident interval}
\tR{For linear regression the $95\%$ confident intervals are} \begin{center}
\enc{$\begin{cases}\widehat{\beta_{0}}\pm 2\times SE\left(\widehat{
\beta_{0}}\right)\\\widehat{\beta_{1}}\pm 2\times SE\left(\widehat{
\beta_{1}}\right)\end{cases}$}\end{center}Standard error can be used
to perform hypothesis test, the most common test involves \emph{null
hypothesis} and \emph{alternative hypothesis}:\encB{$\begin{cases}
H_{0}:\beta_{1}=0\\H_{\alpha}:\beta_{1}\neq 0\end{cases}$}\\To test
null hypothesis we need to determine wehter $\widehat{\beta_{1}}$ is
sufficiently
far from zero that we can be confident that $\beta_{1}$ is non-null.\\
\emph{How far is far enough?} This depends on the $\widehat{\beta_{1}}$
accuracy then \\$\begin{cases}SE\left(\widehat{\beta_{1}}\right)\text{
is small, even relatively small values of }\widehat{\beta_{1}}\text{
can be a strong evidence that }\beta_{1}\neq 0\\SE\left(\widehat{
\beta_{1}}\right)\text{ is large then }\widehat{\beta_{1}}\text{ must 
be large in absolute value in order for us to reject }H_{0}\end{cases}$
\\In practice we compute a \emph{t-statistic} given by \begin{center}
\enc{$t=\dfrac{\widehat{\beta_{1}}-0}{SE\left(\widehat{\beta_{1}}
\right)}$}\\which measure the number of standard deviation that $
\widehat{\beta_{1}}$ is away from $0$.\end{center} \tB{If there is no
relationship between $X$ and $Y$ then we except that we will have a 
\emph{ t-distribution}}. It is a simple matter \tR{to observe the 
probability of observing any number equal to $|t|$ or larger in
absolute
value, \emph{assuming that }$\beta_{1}=0$} this probability is called
\tR{\emph{p-value}}.\\Roughly speaking a small \emph{p-value} indicates
that it is unlikely to observe such a substantial association between
the predictors and the response due to chance.\\\encN{Typical \emph{
p-value} cutoffs for rejecting $H_{0}$ is $5-1\%$}
\paragraph{Assessing the accuracy of the model}
Once we have rejected the null hypothesis in favor of the alternative
hypothesis it is natural to want to qualify to which the model fits the
data.
\subparagraph{Residual Standard Error} \tB{is an estimate of standard
deviation of $\epsilon$}, it means the average amount that the response
will deviate from the true regression line.
\subparagraph{$R^{2}$ statistic} it provides an alternative measure of
fit, and takes the form of a \emph{proportion} (the proportion of
variance explained). \tB{\emph{Total Sum of Squares}} \enc{$TSS=\su{{i=1}}{n
} \left(y_{i}-\overline{y}\right)^{2}$} represents the amount of 
variability
inherent to the response before the regression is performed, in 
contrast \emph{RSS measures the amount of variability that left
unexplained after performing regression}. \begin{center}\enc{$R^{2}=
	\dfrac{TSS-RSS}{TSS}$}\\Then $R^{2}$ measures the \emph{proportion of
variability in $Y$ that can be explained using $X$}\end{center}Recall
that \tR{correlation} defined as \begin{center}\enc{$\widehat{Cor(X,Y)}=
\dfrac{\su{ {i=1}}{n}\left(x_{i}-\overline{x}\right)\left(y_{i}-
\overline{y}\right)}{\sqrt{\su{ {i=1}}{n}\left(x_{i}-\overline{x}
\right)^{2}}\sqrt{\su{ {i=1}}{n}\left(y_{i}-\overline{y}\right)^{2}}}
$}\\$r=\widehat{Cor\left(X,Y\right)}$ is also a measure of the linear
relationship between $X$ and $Y$\end{center}\Moi{it can be shown that
in the simple regression setting $R^{2}=r^{2}$}

\subsection{Ordinary Leas Squares Regression}
\paragraph{Estimating the Regression Coefficients}
A common way to estimate the parameters of a statistical model is to compute
the MLE(Maximum Likelihood Estimation) defined as 
$$\hat{\bm{\theta}} \triangleq \displaystyle \argmax_{\theta} \log\left(
p(\mathcal{D}|\bm{\theta})\right)$$
\begin{align*}
    l(\bm{\theta}) &\triangleq \log\left(p(\mathcal{D}|\bm{\theta})\right)\\
                   &=\su{i=1}{n}\log\left(p(y_{i}|\bm{x_{i}}, \bm{\theta})\right)\\
                   &= \su{i=1}{n}\log\left(
                       \left[\dfrac{1}{2\pi\sigma^{2}}\right]^{\frac{1}{2}}
                       \exp\left(-\dfrac{1}{2\sigma^{2}}\left[y_{i} - \bm{\beta}^{T}
                       \bm{x_{i}}]\right]^{2}\right)\right)\\ 
                   &= \dfrac{1}{2\sigma^{2}}RSS(\bm{\beta}) +
                   \dfrac{n}{2}\log(2\pi\sigma^{2})
\end{align*}
Then the Residual Sum of Squares (RSS) is equal to $\su{i=1}{n}\left(
y_{i}-\beta^{T}x_{i}\right)^{2}$
Instead of maximizing the log-likelihood we can equivalently minimize the Negative Log Likelihood (NLL) 
\begin{center}
    $NLL(\beta) \triangleq l(\beta)$
\end{center}


Considering $\bm{X}$ the $N\times (p+1)$ matrix with each row an input
vector and $y$ be the $N-vector$ of outputs in the training set.
\begin{center}
	$RSS(\beta)=(y-\bm{X}\beta)^{T}(y-\bm{X}\beta)$
\end{center}
Differentiating with respect to $\beta$ we obtain:
$
\begin{cases}
	\dfrac{\partial RSS}{\partial\beta}=-2\bm{X}^{T}(y-\bm{X}\beta)\\
	\dfrac{\partial^{2} RSS}{\partial\beta\partial\beta^{T}}=2\bm{X}^{T}\bm{X}\\
\end{cases}
$\\
Assuming that $\bm{X}$ has full column rank, we set the first 
derivative to 0:\\ $\bm{X}^{T}(y-\bm{X}\beta)=0$ to obtain the unique
solution:

\begin{center}
	\encB{$\hat{\beta}=(\bm{X}^{T}\bm{X})^{-1}\bm{X}^{T}\bm{y}$}
\end{center}
\begin{figure}[H]
\centering
\begin{subfigure}{.5\textwidth}
  \centering
	\includegraphics[width=.7\textwidth]{./chap/1chap/2sec/2images/1leastSquaresPlan.png}
  \caption{$n$ observations}
  \label{fig:2.1aLeastSquares}
\end{subfigure}%
\begin{subfigure}{.5\textwidth}
  \centering
	\includegraphics[width=\textwidth]{./chap/1chap/2sec/2images/11projection.png}
\caption{1 observation}
  \label{fig: 2.1bLeastSquares}
\end{subfigure}
  \caption{Least squares for a linear model with $2$ predictors of 
$p$-dimensions.}
\label{fig:test}
\end{figure}

\begin{align*}
\hat{\bm{y}} &=\bm{X}\hat{\beta}\\
	     &=\bm{X}\left(\bm{X}^{T}\bm{X}\right)^{-1}\bm{X}^{T}\bm{y}\\
	     &=\bm{H}\bm{y}
\end{align*}
$\bm{H}$ is called \tB{``hat'' matrix because it puts the hat on $\bm{y}$}.
\paragraph{Hat Matrix}
Residuals can also be expressed as a function of $\bm{H}$,
$\bm{\hat{e}} = \bm{y} - \bm{\hat{y}} = \bm{y} - \bm{Hy} = (\bm{I}-\bm{H})\bm{y}$.
It can be shown that \sB{$\bm{H}$ and $\bm{I}-\bm{H}$ are orthogonal projections}.\\
One can easily show that \tB{$\bm{H}\bm{H} = \bm{H}$} and $\left(\bm{I-H}\right)\left(\bm{I-H}\right)
= \bm{I-H}$

\subparagraph{Range and Kernel of the Hat Matrix}
$rank(\bm{X}) = rank(\bm{X}^{T}\bm{X}) = p^{*}$
\subparagraph{Residual and Fitted Values}
$\bm{H}(\bm{I}-\bm{H}) = \bm{H} -\bm{HH} = 0$, hence $\sP{\bm{\hat{y}}}{\bm{\hat{e}}} = 0$.
\tB{Therefore $\bm{\hat{y}}$ and $\bm{e}$ are orthogonal} in $\mathbb{R}^{n}$.
\subparagraph{Geometric interpretation}
The degrees of freedom associated with \tB{$\bm{\hat{y}}$ and $\bm{\hat{e}}$ can be seen to simply
be the dimensions of the respective vector subspace in which these 2 vectors have been projected}.\\
The vectors $\bm{y}, \bm{\hat{y}}$ and $\bm{\hat{e}}$ determine 3 points in $\mathbb{R}^{n}$ which
form a right-angled triangle, we can see the decomposition of total sum of squares into estimated
sum of squares and residual sum of squares as a special case of \emph{Pythagoras} theorem.
\subparagraph{Further information}
It might happen that the columns of \sB{$\bm{X}$ are not linearly independent, then
$\bm{X}^{T}\bm{X}$ is singular and the least squares coefficients $\hat{\beta}$ are not uniquely 
defined}.\\ 
Knowing that $\V{\bm{A}\bm{y}}=\bm{A}\V{\bm{y}}\bm{A}^{T}$:
\begin{center}
	\encB{$\V{\hat{\beta}}=\left(\bm{X}^{T}\bm{X}\right)^{-1}\sigma^{2}$}
\end{center}
a estimate of $\sigma^{2}$:$\hat{\sigma}^{2} = \dfrac{1}{N-p-1}\su{{i=1}}{N}(y_{i}-\hat{y}_{i})^{2}$ 
The $n-p-1$ rather than $n$ makes $\hat{\sigma}^{2}$ an unbiased
estimate.\\
$
\begin{cases}
\hat{\beta}\hookrightarrow\mathcal{N}\left(\beta, (\bm{X}^{T}\bm{X})^{-1}\sigma^{2}\right)\\
(n-p-1)\hat{\sigma}^{2}\hookrightarrow\sigma^{2}\chi_{n-p-1}^{2}
\end{cases}
$

\paragraph{Convexity}
Functions having a bowl shape with a unique minimum, more precisely:
$$\forall (\bm{\theta}, \bm{\theta'},\lambda) \in \mathcal{S}\times\mathcal{S}\times[0,1],
~ \lambda\bm{\theta} + (1-\lambda)\bm{\theta'} \in \mathcal{S} \Rightarrow \mathcal{S}
\text{ is \textbf{convex}}$$

\subsection{Rigde Regression}
\paragraph{Relevancy of using robust regression model}
In the common way to model the noise, with a Gaussian distribution, maximizing likelihood is equivalent to minimizing the sum of squared residuals, however squared error penalizes
deviation quadratically, then a few outliers can provoke poor model fitting.
\paragraph{Potential approach to handle bypass outlier presence}
Replace Gaussian distribution with \textbf{heavy tail} such it will assign higher 
likelihood to outliers without impacting the main hyperspace explaining them.
Like Laplace distribution of which probability function is:\\
$\begin{cases}
    \mathbb{R} & \longrightarrow \mathbb{R}\\
    x & \longmapsto \dfrac{1}{2b}e^{\frac{\left| x-\mu\right|}{b}}
\end{cases}$
The robustness comes from the use of absolute value error instead of quadratic error.\\
If for $i \in \inter{1}{n},~r_{i} \triangleq y_{i} - \beta^{T}x_{i}$ then NLL has this 
form: $l(\beta) = \su{i=1}{n}\left|r_{i}(\beta)\right|$
Unfortunately this function is hard to optimize, however we can convert the NLL to a 
linear objective.\\
Let set $r_{i} \triangleq r^{+}_{i} - r^{-}_{i}$, with $r^{+}_{i} \geq 0 \wedge
r^{+}_{i}\geq 0$.
Then the constrained objectives becomes $\displaystyle\min_{r^{+}, r^{-}}\su{i=1}{n}
(r^{+}_{i} - r^{-}_{i})$\\
An alternative to using NLL under a Laplace likelihood is to minimize the \textbf{Huber 
Looss} defined as follow:
$
\begin{cases}
    \frac{r^{2}}{2} & \Rightarrow |r| \leq \delta\\
    \delta |r|-\frac{\delta^{2}}{2} & \Rightarrow |r| > \delta
\end{cases}
$
This is equivalent to $l_{2}$ for smaller errors than $\delta$, and is equivalent to 
$l_{1}$ for larger errors.\\
This function is differentiable everywhere considering that 
$r\neq \Rightarrow \frac{d}{dr}|r| = sign(r)$, also this function is continuous since 
the gradients of the 2 parts of the functcion match at $r=\pm \delta$
Consequently optimizing the Huber loss is much faster than using teh Laplace likelihood.



%\subsection{Assumption Checking}
%\emph{Python Code}:
\begin{python}
import numpy as np
import pandas as pd
import statsmodels.api as sm
import statsmodels.stats.api as sms
from statsmodels.graphics.tsaplots import plot_acf
from statsmodels.sandbox.regression.predstd import wls_prediction_std
from statsmodels.tsa.ar_model import AutoReg
from statsmodels.tools.tools import add_constant

# Linear regression
y, X = df.iloc[:, 0], add_constant(df.iloc[:, 1:])
results_ols = sm.OLS(y, X).fit()
print(results.summary())

# Graphical representation of response
prstd, iv_l, iv_u = wls_prediction_std(results) # calculate standard deviation and confidence
interval for prediction
x = np.arange(y.shape[0])
fig, ax = plt.subplots()
ax.plot(x, y, 'o', label='Data')
ax.plot(x, results.fittedvalues, 'r--', label='Predicted')
ax.plot(x, iv_u, 'r--')
ax.plot(x, iv_l, 'r--')
ax.legend(loc='best')
plt.show()

# Checking of autoregressivity
mod_ar = AutoReg(results.resid, lags=3)
res_ar = mod_ar.fit()
print(res_ar.summary())

# Histogram, Q-Q plot, Correlogram ...
fig = res.plot_diagnostics(lags=30)
fig.tight_layout() # keeps space btw subplots
plt.show()
\end{python}

\emph{R code:}
\begin{rcode}[deletekeywords={2, df, model, resid, residuals}]
library(dplyr)
library(ggplot2)
library(lmtest) # testing linear regression

# Linear regression
model.ols <- lm(y ~ ., data=df)
summary(model.ols)

# Graphical diagnostic
resid <- model.ols$residuals # residuals
resid.std <- scale(model.ols$residuals) # standardized residuals

par(mfrow=c(2, 2))
plot(seq=(length(resid)), resid, xlab='No Obs.', ylab='Std Resid', main='Standardized Residuals')
hist(resid, xlab='Residuals', main='Histogram')
qqnorm(resid) # QQ plot
qline(resid) # Line QQ plot
pacf(resid, main='PACF for residuals')
\end{rcode}

\paragraph{Non-linearity of the Data}
\tB{Residual plots} are a useful graphical tool for identifying 
on-linearity.$\begin{cases}\text{simple regression: }plot(x_{i},e_{i})
	\text{ for the} i^{th} \text{observation}\\\text{multiple regression: }plot(
	\widehat{y}_{i},e_{i})\text{ for the} i^{th} \text{observation}\end{cases}$
The presence of a pattern may indicate a problem with some aspect of
the linear model.\\\tB{In non-linear setting, a simple approach might be
to apply non-linear transformation to predictors} (such as $\log\left(X
\right),\sqrt{X}\text{ and }X^{2}$) in the regression model.
\emph{Python code}
\begin{python}
import numpy as np
import matplotlib.pyplot as plt

res = results_ols.resid
yhat = results_ols.fittedvalues
poly = np.poly1d(np.polyfit(yhat, res, deg=3))
yhat_poly = np.linespace(yhat.min(), yhat.max(), yhat.shape[0])

fig, ax = plt.subplots()
ax.scatter(yhat, res)
ax.plot(yhat_poly, poly(yhat_poly), label='Polynomial estimation')
plt.xlabel('Fitted values')
plt.ylabel('Residuals')
plt.title('Residuals vs Fitted Values')
plt.show()
\end{python}

\emph{R code}
\begin{rcode}[deletekeywords={df, resid, lm, model, residuals, fitted, poly}]
library(ggplot2)

df.resid <- data_frame(res=lm.model$residuals, yhat=lm.model$fitted.values)
model.poly <- lm(res ~ poly(yhat,3), data=df.resid) # to evaluate the trend
df.resid$yhat.poly <- model.poly$fitted.values
df.resid %>% ggplot(aes(x=yhat))+
  geom_point(aes(y=res))+
  geom_line(aes(y=yhat.poly), color='blue')+
  geom_hline(aes(yintercept=0), linetype='dashed', color='red')+
  ggtitle('Non-linearity detection')
\end{rcode}

\paragraph{Correlation of error terms}
If the error are correlated we may have an unwarranted sense of 
confidence in our model.\\ To observe hypothetic correlation \tB{we can plot residuals
from our model as function of time.} If error are uncorrelated we 
should not observe any pattern.
\begin{figure}[H]
	\begin{center}
		\includegraphics[width=.5\textwidth]{./chap/1chap/2sec/2images/2_7residualAsTimeSeries.png}
	\end{center}
	\caption{Residuals versus observation for given correlated
	coefficient, as we could plot for time series versus time.}
	\label{fig:fig2.7}
\end{figure}
\subparagraph{Definition Autocorrelation}
Disturbances are \emph{\textbf{autocorrelated}} when:
$$\forall (i,j)\in\inter{1}{n}^{2}\mathbb{C}ov\left(\epsilon_{i},\epsilon_{j}|\bm{X}\right)\neq0$$

$$ \text{Disturbances are \emph{autocorrelated} but are still assumed to be homoscedastic}
\Rightarrow $$
\begin{equation*}
	\Sigma=
	\begin{pmatrix}
		\sigma^{2} & \sigma_{12} & \cdots & \sigma_{1n}\\
		\sigma_{21} & \sigma^{2} & \cdots & \sigma_{2n}\\
		\vdots & \vdots & \ddots & \vdots\\
		\sigma_{n1} & \sigma_{n2} & \cdots & \sigma^{2}\\
	\end{pmatrix}
	= \sigma^{2}\Omega = \sigma^{2}
	\begin{pmatrix}
		1 & \omega_{12} & \cdots & \omega_{1n}\\
		\omega_{21} & 1 & \cdots & \omega_{2n}\\
		\vdots & \vdots & \ddots & \vdots\\
		\omega_{n1} & \omega_{n2} & \cdots & 1\\
	\end{pmatrix}
\end{equation*}
$\forall(i,j)\in\inter{1}{n}^{2},~\omega_{ij}=\dfrac{\sigma_{ij}}{\sigma^{2}}=corr(\epsilon_{i},
\epsilon_{j})$
\subparagraph{Autocorrelation Function (ACF)}
At each point $k$, representing lags, the function has the following value:
$$ \hat{r}_{k}=\dfrac{\su{{t=k+1}}{n}(x_{t-k}-\overline{x})(x_{t}-\overline{x})}{\su{{t=1}}{n}
(x_{t}-\overline{x})^{2}}
\begin{cases}
	k\in\inter{1}{n}\text{ representing the lag}\\
	x_{t}\text{ value of }x\text{ at row }t\\
	n\text{ number of observations in the series}
\end{cases}
$$

\emph{Python Code}
\begin{python}
import statsmodels.api as sm
from statsmodels.graphics.tsaplots import plot_acf

plot_acf(residuals, unbiased=True, zero=False,
    title='Residual autocorrelation')
plt.show()
\end{python}

\emph{R code}
\begin{rcode}[deletekeywords={df, resid}]
pacf(df.resid[, 'res'], main='Residual autocorrelation')
\end{rcode}

\subparagraph{Breusch-Godfrey test}
It is a \sB{test for autocorrelation in the errors in a regression model}.\\ 
Consider the linear regression :
$$ y_{t} = \beta_{0} + \beta_{1}X_{t,1} + \beta_{2}X_{t,2} + u_{t}
\begin{cases}
	u_{t} = \su{{k=1}}{p}\rho_{k}u_{t-k}+\epsilon_{t}\\
	\hat{u}_{t} = \alpha_{0} + \alpha_{1}X_{t,1} + \alpha_{2}X_{t,2} +
	\su{{k=1}}{p}\rho_{k}\hat{u}_{t-k} + \epsilon_{t}\text{ from the first OLS}
\end{cases}
$$
Then $R^{2}$ is computed and can be used for the distribution of the test statistic
$nR^{2}\hookrightarrow\chi_{p}^{2}$. When $H_{0}:\forall k\in\inter{1}{n}$ 
\emph{Python Code}
\begin{python}
import statsmodels.stats.api as sms
from sstatsmodels.tsa.ar_model import AutoReg

s_ar = pd.Series([AutoReg(result_ols.resid, lags=k) for k in range(1, 31)])
df_lag = pd.DataFrame({ # it will contain for each I.C. the optimal lag number
  'aic':[np.argmin(s.apply(lambda x: x.aic)) + 1],
  'hqic':[np.argmin(s.apply(lambda x: x.hqic)) + 1],
  'bic':[np.argmin(s.apply(lambda x: x.bic)) + 1]})
iter_l = [list(df_lag.iloc[0, :]).count(df_lag.iloc[0, j]) for j in range(df_lag.shape[1])] # occurence list
max_ind = max(range(len(iter_l)), key=iter_l.__getitem__) # index of the first value having the larger occurence number
lagOpt = df_lag.iloc[0, max_ind] #

test_BG = sms.acorr_breusch_godfrey(res=results_ols, nlags=lagOpt)
fscore_BG = test_BG[2]
fpvalue_BG = test_BG[3]
\end{python}

\emph{R code}
\begin{rcode}[deletekeywords={df, by, lm, model, max}]
library(VARS)
library(lmtest)

df.ar <- data_frame(res=lm.model$residuals) #
lagsel <- VARselect(df.ar$res, lag.max=30, type='const') # returning the optimal lag number
df.ar.count <- data_frame(lags=lagsel$selection, count=1) # count column contains 1's
df.ar.count %>% group_by(lags) %>%
  summarise(nbr=n()) %>%
  arrange(desc(nbr))
lagOpt <- df.ar.count$lags[1]
bgtest(lm.model, order=lagOpt) # Breusch-Godfrey test
\end{rcode}

The notation $AR(p)$ indicates an autoregressive model of order $p$. The $AR(p)$ model is defined
as $X_{t} = c + \su{{i=1}}{p}\phi_{i}X_{t-i} + \epsilon_{t}$ where $\phi_{k}$ are the parameters
of the model, $c$ is a constant and $\epsilon_{t}$ is a white noise.\\
The null hypothesis is that there is no serial correlation of any order up to $p$.
\begin{python}
glsar_model = sm.GLSAR(y, X, rho=1) # rho corresponds to the 
# order of the autoregressive covariance
glsar_results = glsar_model.iterative_fit(rtol=0.0001) # Relative tolerance between estimated 
# coefficients to stop the estimation
print(gls_results.summary())
\end{python}



\paragraph{Non-constant variance of error terms}
Hypothesis which is to consider the error terms have a constant variance
($\V{\epsilon_{i}}=\sigma^{2}$). Unfortunately is often the case that
variance of error terms is non-constant, then \tB{a solution might be to
transform response $Y$ using a concave function} such as $\sqrt{Y}\text{
or }\log\left(Y\right)$
\begin{figure}[H]
	\begin{center}
		\includegraphics[width=\textwidth]{./chap/1chap/2sec/2images/2_8correctionOfNonConstantErrorTerms.png}
	\end{center}
	\caption{Residuals plots. In each plot the red line is a smooth
	fit to the residuals, intended to make it easier to identify
	a trend. The blue lines track the outer quantiles of the
	residuals and emphasis pattern. Left:The funnel shape indicates
	\emph{heteroscedasticity}. Right: The response has been log
	transformed, and there is now no evidence of heteroscedasticity
	.}
	\label{fig:fig2.8}
\end{figure}
To fix it: 
\begin{itemize}
	\item \tB{Check if important explanatory variables are missing in my model}
		and add them in.
	\item Switch to a \tB{GLM}, \tB{WSS}, or \tB{GLS} model
	\item \tB{A small amount of heteroscedasticity in the model's residual can be tolerated}
		if my model is otherwise performing well.
\end{itemize}
To diagnose there is \tB{Breush-Pagan test}:\\
Under the classical assumptions, OLS is the best linear unbiased estimator (BLUE), it is unbiased
and efficient. It remains unbiased under heteroscedasticity but efficiency is lost.\\
%The Breush-Pagan test is biased on models of type $\sigma_{i}^{2}=h(z_{i}'\gamma)$ for the 
%variances of the observations where $z_{i}=(1, z_{i2}, \cdots, z_{ip})$ explain the difference
%in the variances. $H_{0}:\forall j\in\inter{2}{p}\gamma_{j}=0$\\
The \emph{Method of Lagrange multipliers} \tB{is a strategy for finding the local maximal and
minima of a function subject to the condition that one or more equations} have to be satisfied
exactly by the chosen values of the variables.
The following Lagrange multiplier yields the  test statistic for the Breush-Pagan test:
$$ LM = \left(\dfrac{\partial l}{\partial\theta}\right)^{T}\left(-\E{\dfrac{\partial^{2}l}{\partial\theta\partial\theta'}}\right)^{-1}\left(\dfrac{\partial l}{\partial\theta}\right)$$
\begin{enumerate}
	\item Apply OLS: $\forall i\in\inter{1}{n} y_{i}=\beta X_{i} + \epsilon_{i}$
	\item Compute the regression residuals: $\hat{\epsilon}_{i}$, square them and divide by 
		the Maximum Likelihood Estimate of the error variance to obtain what Breusch and
		Pagan call $g_{i}$:\\
		$g_{i}=\dfrac{\hat{\epsilon}_{i}}{\su{{i=1}}{n}\frac{\hat{\epsilon}_{i}}{n}}$\\
		Estimate the auxiliary regression:
		$g_{i} = \gamma_{1} + \su{{j=2}}{p}\gamma_{j}z_{ij} +\eta_{i}$ where the $z$ terms
		will typically but not necessarily be the same as the original covariates $X$
	\item The LM test statistic is then half of the explained sum of squares from the 
		auxiliary regression: $LM = \dfrac{1}{2}\left(TSS-RSS\right)$
\end{enumerate}
%In GLM, $\bm{y}$ is assumed to be generated from a particular distribution in an exponential
%family (normal, binomial, Poisson, Gamma,\ldots) we have:
%$$ \E{\bm{y}}=\mu=g^{-1}\left(\beta\bm{X}\right)$$
%where $g$ is the link function.\\
%The GLM consists of 3 elements:
%\begin{enumerate}
%	\item An exponential family of probability distribution
%	\item A linear predictor $\eta=\beta\bm{X}$
%	\item A link function $g$ such that $\E{Y|X}=\mu=g^{-1}(\eta)$
%\end{enumerate}
%\begin{enumerate}
%	\item \textbf{Probability distribution}\\
%		Exponential family whose density functions:\\
%		$$f_{\bm{y}}(\bm{y}|\theta,\tau)=h(\bm{y},\tau)\exp\left(\dfrac{b(\theta)^{T}
%		\bm{T}(\bm{y})-\bm{A}(\theta)}{d(\tau)} \right)$$
%		$\theta$ is related to the mean of the distribution, $b$ is the identity function
%		The \emph{dispersion parameter} $\tau$, typically is known and is usually related
%		to the variance of the distribution.
%		$\begin{cases}
%			\mu = \E{\bm{y}} = \nabla \bm{A}(\theta)\\
%			\V{\bm{y}} = \nabla^{2}\bm{A}(\theta)d(\tau)
%		\end{cases}
%			$
%	\item \textbf{Linear predictor}\\
%		It is the quantity which incorporates the information about the independent 
%		variables into the model. 
%		$$\eta=\beta\bm{X}$$
%	\item \textbf{Link function}\\
%		It provides the relationship between the linear predictor and the mean of the 
%		distribution function. 
%		\begin{figure}[H]
%			\begin{center}
%				\includegraphics[width=.8\textwidth]{./chap/1chap/2sec/3images/5_glm.PNG}
%			\end{center}
%			\caption{$\mu_{i}$ is the expected value of the response; $\eta_{i}$ is
%			the linear predictor and $\Phi$ the cumulative distribution function of 
%			standard-normal distribution.}
%			\label{fig:5_glm}
%		\end{figure}
%\end{enumerate}

%\begin{lstlisting}
%# for example if data seem to have a gamma distribution:
%gamma_model = sm.GLM(y, X, family=sm.families.Gamma())
%gamma_results = gamma_model.fit()
%\end{lstlisting}


\paragraph{Generalized Linear Regression Model and Heteroscedasticity}
\subparagraph{Generalized Linear Regression Model }
$$
\begin{cases}
	\bm{y} = \bm{X}\bm{\beta} + \bm{\epsilon}\\
	\E{\bm{\epsilon}|\bm{X}} = \bm{0}_{n\times 1}\\
	\V{\epsilon|\bm{X}} = \bm{\Sigma} = \sigma^{2}\bm{\Omega}
\end{cases}
$$
where $\bm{X}$ is a matrix of fixed or random regressors, $\beta\in\mathbb{R}^{k}$, $\Sigma,
\Omega$ are symmetric positive definite matrices. 

\begin{equation*}
	\Sigma=
	\begin{pmatrix}
		\sigma_{1}^{2} & \sigma_{12} & \cdots & \sigma_{1n}\\
		\sigma_{21} & \sigma_{2}^{2} & \cdots & \sigma_{2n}\\
		\vdots & \vdots & \ddots & \vdots\\
		\sigma_{n1} & \sigma_{n2} & \cdots & \sigma_{n}^{2}\\
	\end{pmatrix}
= \sigma^{2}
	\begin{pmatrix}
		\omega_{1}^{2} & \omega_{12} & \cdots & \omega_{1n}\\
		\omega_{21} & \omega_{2}^{2} & \cdots & \omega_{2n}\\
		\vdots & \vdots & \ddots & \vdots\\
		\omega_{n1} & \omega_{n2} & \cdots & \omega_{n}^{2}\\
	\end{pmatrix}
\end{equation*}
and $\forall(i,j)\in\inter{1}{n}^{2},~\omega_{ij}=\dfrac{\sigma_{ij}}{\sigma^{2}}$
\subparagraph{Definition Heteroscedasticity}
Disturbances are \emph{\textbf{heteroscedastic}} when they have different (conditional) 
variances:
$$\forall (i,j)\in\inter{1}{n}^{2},~i\neq j\Rightarrow \V{\epsilon_{i}|
\bm{X}_{i\bullet}}\neq\V{\epsilon_{j}|\bm{X}_{j\bullet}}$$

$$ \text{Disturbances are \emph{heteroscedastic} but are still assumed to be uncorrelated across
observations}\Rightarrow $$
\begin{center}
\begin{equation*}
	\Sigma=
	\begin{pmatrix}
		\sigma_{1}^{2} & 0 & \cdots & 0\\
		0 & \sigma_{2}^{2} & \cdots & 0\\
		\vdots & \vdots & \ddots & \vdots\\
		0 & 0 & \cdots & \sigma_{n}^{2}\\
	\end{pmatrix}
	= \sigma^{2}\bm{\Omega}=\sigma^{2}
	\begin{pmatrix}
		\omega_{1}^{2} & 0 & \cdots & 0\\
		0 & \omega_{2}^{2} & \cdots & 0\\
		\vdots & \vdots & \ddots & \vdots\\
		0 & 0 & \cdots & \omega_{n}^{2}\\
	\end{pmatrix}
\end{equation*}
\end{center}



%\paragraph{Inefficency of the OLS}
%We start from the \emph{Generalized Linear Regression} model and here $\beta=\hat{\beta}_{OLS}
%\left( \bm{X}^{T}\bm{X} \right)^{-1}\bm{X}^{T}\bm{y} $. The regressors are exogenous in the sense
%that : $\E{\epsilon|\bm{X}}=\bm{0}_{N\times 1}$.
%OLS estimator is \textbf{\emph{unbiased}} $\Rightarrow \E{\hat{\beta}_{OLS}}=\beta_{0}$
%where $\beta_{0}$ denotes the true value of the parameters.
%Heteroscedasticity and/or Autocorrelation do not induce a bias for the OLS estimator.\\
%
%In GLR model, under exogeneity the OLS estimator has a conditional covariance matrix given by
%$$ \V{\hat{\beta}_{OLS}|\bm{X}}=\sigma_{0}^{2}\left(\bm{X}^{T}\bm{X}\right)^{-1}\bm{X}^{T}\Omega
%\bm{X}\left(\bm{X}^{T}\bm{X}\right)^{-1}$$
%Proof:
%\begin{align*}
%\begin{split}
%\hat{\beta}_{OLS} &= \left(\bm{X}^{T}\bm{X}\right)^{-1}\bm{X}^{T}\bm{y}\\
%&= \left(\bm{X}^{T}\bm{X}\right)^{-1}\bm{X}^{T}(\bm{X}\beta_{0}+\bm{\epsilon})\\
%&= \beta_{0} + \left(\bm{X}^{T}\bm{X}\right)^{-1}\bm{X}^{T}\bm{\epsilon}
%\end{split}
%\\\\
%\begin{split}
%\V{\hat{\beta}_{OLS}|\bm{X}} &= \V{\beta_{0} +
%\left(\bm{X}^{T}\bm{X}\right)^{-1}\bm{X}^{T}\bm{\epsilon}}\\
%&= \V{\left(\bm{X}^{T}\bm{X}\right)^{-1}\bm{X}^{T}\bm{\epsilon}}\\
%&= \E{\left(\bm{X}^{T}\bm{X}\right)^{-1}\bm{X}^{T}\bm{\epsilon}
%\left(\left(\bm{X}^{T}\bm{X}\right)^{-1}\bm{X}^{T}\bm{\epsilon}\right)^{T}|\bm{X}}
%\\
%&= \E{\left(\bm{X}^{T}\bm{X}\right)^{-1}\bm{X}^{T}\bm{\epsilon}\epsilon^{T}\bm{X}
%\left(\bm{X}^{T}\bm{X}\right)^{-1}|\bm{X}}\\
%&= \left(\bm{X}^{T}\bm{X}\right)^{-1}\bm{X}^{T}\E{\bm{\epsilon}\epsilon^{T}|\bm{X}
%}\bm{X}\left(\bm{X}^{T}\bm{X}\right)^{-1}\\
%&= \sigma_{0}^{2}\left(\bm{X}^{T}\bm{X}\right)^{-1}\bm{X}^{T}\bm{\Omega}\bm{X}
%\left(\bm{X}^{T}\bm{X}\right)^{-1}
%\end{split}
%\end{align*}
%
%Under exogeneity and normality, the OLS estimator obtained in the generalized linear regression
%model has an (exact) \emph{normal conditional distribution} :
%$$ \hat{\beta}_{OLS}|\bm{X}\hookrightarrow \mathcal{N}\left(\beta_{0},\sigma^{2}\left(\bm{X}^{T}
%\bm{X}\right)^{-1}\right)\bm{X}^{T}\bm{\Sigma}\bm{X}\left(\bm{X}^{T}\bm{X}^{-1}\right)$$
%
%Because the variance of the OLS estimator is not $\sigma^{2}\left(\bm{X}^{T}\bm{X}\right)^{-1}$
%statistical inference (\emph{non-robust inference}) based on $\hat{\sigma}^{2}\left(\bm{X}^{T}
%\bm{X}\right)^{-1}$ may be \emph{misleading}.\\
%The question is to know how to estimate $\V{\hat{\bm{\beta}_{OLS}}}$ in context of the linear
%generalized regression model in order to make \emph{robust inference}.
%In the GLR model under some regularity conditions:
%\begin{itemize}
%	\item  OLS is \emph{unbiased}
%	\item  OLS is (weakly) \emph{consistent}
%	\item  OLS is \emph{asymptotically normally distributed}
%\end{itemize}
%But the inference based on the estimator $\sigma^{2}\left(\bm{X}^{T}\bm{X}\right)^{-1}$ is 
%\emph{misleading}, OLS is \emph{inefficient} $\V{\hat{\bm{\beta}_{OLS}}}-I_{N}^{-1}(\bm{\beta}_{0}
%)$ is a positive definite matrix.
\paragraph{Weighted Least Squares (WLS)}
Weighted Least Squares is a generalization of ordinary least squares and linear regression in
which the errors covariance matrix is allowed to be different from an identity matrix. \sB{It is a
special case of Generalized Least Squares in which the covariance matrix is diagonal.}
The standard model assumes that $\forall i\in\inter{1}{n}\V{\epsilon_{i}}=\sigma^{2}$ whereas in
WLS we suppose $\exists\prth{\omega}{i}{n},~\forall i\in\inter{1}{n}\V{\epsilon_{i}}=
\dfrac{\sigma^{2}}{\omega_{i}}$ 
$$RSS(\beta)=\left(\bm{Y}-\bm{X}\beta\right)^{T}\bm{\Omega}\left(\bm{Y}-\bm{X}\beta\right)$$
Then $\dfrac{\partial RSS(\beta)}{\partial\beta}=0\Rightarrow \beta=
\left(\bm{X}^{T}\bm{\Omega}\bm{X}\right)^{-1}\bm{X}^{T}\bm{\Omega}\bm{Y}$

In general suppose we have the linear model: 
$\bm{Y} = \bm{X}\beta + \epsilon$ where $\V{\epsilon}=\bm{\Omega}^{-1}\sigma^{2}$, then we have
$\V{\bm{\Omega}^{\frac{1}{2}}\epsilon}=\sigma^{2}\bm{I}_{n}$\\
Hence we consider the transformation:
$
\begin{cases}
	\bm{Y}'=\bm{\Omega}^{\frac{1}{2}}\bm{Y}\\
	\bm{X}'= \bm{\Omega}^{\frac{1}{2}}\bm{X}\\
	\epsilon' = \bm{\Omega}^{\frac{1}{2}}\epsilon
\end{cases}
$\\
This give us $\bm{Y}=\bm{X}'\beta+\epsilon'$:
\begin{align*}
	\hat{\beta} &= \left( (\bm{X}')^{T}\bm{X}'\right)^{-1}(\bm{X}')^{T}\bm{Y}'\\
	&= \left(\bm{X}^{T}\bm{\Omega}\bm{X}\right)^{-1}\bm{X}^{T}\bm{W}\bm{Y}
\end{align*}

Advantages:
\begin{itemize}
	\item In WLS the interpretation of coefficient remains the same as OLS
	\item In WLS we generally include an intercept which means that the \emph{F-test} and 
		$R^{2}$ are interpreted as usual. 
	\item WLS gives us an easy way to remove one observation from a model by setting it 
		weight equal to 0
	\item  We can downweight outlier or leverage points to reduce their impact on the overall
		model.
\end{itemize}

The Weights
\begin{itemize}
	\item We may have a probabilistic model for $\mathbb{V}\left(\bm{Y}|\bm{X}=x_{i}\right)$
		in which case we would use this model to find the $\omega_{i}$
	\item Another common case is where each observation is not a single measure but an 
		average of $n_{i}$ actual measures and the original measures each have variance
		$\sigma^{2}$
	\item We would use WLS with weights $\omega_{i}=n_{i}$
\end{itemize}

\begin{lstlisting}
model_wls = sm.WLS(y, X, weights=1./(w**2))
result_wls = mod_wls.fit()
print(result_wls.summary())
\end{lstlisting}

\paragraph{Generalized Least Squares (GLS)}
Consider the generalized linear regression model with:
$$ \V{\bm{\epsilon}|\bm{X}} = \bm{\Sigma}=\sigma^{2}\bm{\Omega}$$
We will distinguish 2 cases:
\begin{enumerate}
	\item $\bm{\Sigma}$ is \emph{known}
	\item $\bm{\Sigma}$ is \emph{\textbf{un}known}
\end{enumerate}
\subparagraph{Known covariance matrix}
The Generalized Least Squares (GLS) estimator of $\bm{\beta}$ is defined as to be:
\tB{$$ \hat{\bm{\beta}}_{GLS} = \left(\bm{X}^{T}\Omega^{-1}\bm{X}\right)^{-1}\bm{X}^{T}\Omega\bm{y}$$}
Under the exogeneity assumption the estimator $\hat{\bm{\beta}_{GLS}}$ is \emph{unbiased}:
$\E{\hat{\bm{\beta}}_{GLS}}=\bm{\beta}_{0}$ where $\bm{\beta}_{0}$ denotes the true value of the
parameters.\\

Under suitable regularity conditions, in a parametric generalized linear regression model,
the GLS estimator $\hat{\bm{\beta}}_{GLS}$ is efficient! $\V{\hat{\bm{\beta}}_{GLS}}=I_{N}^{-1}
\left(\bm{\beta}_{0}\right)$, where $I_{N}^{-1}\left(\bm{\beta}_{0}\right)$ denotes the FDCR or 
Cramer-Rao bound.
\subparagraph{Unknown covariance matrix}
\sB{We assume that the conditional variance matrix of the disturbances can be expressed as a function
of a small set of parameters $\alpha$:}
$$ \V{\bm{\epsilon}|\bm{X}}=\sigma^{2}\bm{\Omega}(\alpha)$$
Consider a consistent estimator $\hat{\alpha}$ of $\alpha$ then the Feasible Least Generalized
Squares (FGLS) estimator of $\bm{\beta}$ is defined as to be:
\tB{$$ \hat{\bm{\beta}}_{FGLS} = \left(\bm{X}^{T}\hat{\bm{\Omega}}^{-1}\bm{X}\right)^{-1}
\bm{X}^{T}\hat{\bm{\Omega}}\bm{y}$$}
where \tB{$\hat{\bm{\Omega}}=\bm{\Omega}(\hat{\alpha})$ is a consistent estimator of 
$\bm{\Omega}(\alpha)$.}\\
An asymptotically efficient FGLS estimator does not require that we have an efficient estimator
of $\alpha$; only a consistent one is required to achieve full efficiency for the FLGS estimator.

\paragraph{Heteroscedasticity}
\begin{figure}[H]
	\begin{center}
		\includegraphics[width=\textwidth]{./chap/1chap/2sec/3images/6_heteroscedasticity.PNG}
	\end{center}
	\caption{Procedure to deal with heteroscedasticity}
	\label{fig:6_heteroscedasticity}
\end{figure}
\begin{enumerate}
	\item \textbf{\emph{Heteroscedasticity of unknown form}}\\
		 The conventionally estimated covariance matrix for the least squares estimator
		 $\sigma^{2}\left(\bm{X}^{T}\bm{X}\right)^{-1}$ is inappropriate, the appropriate
		 matrix is \\
		 $\sigma^{2}\left(\bm{X}^{T}\bm{X}\right)^{-1}\left(\bm{X}^{T}\Sigma
		 \bm{X}\right)^{-1}\left(\bm{X}^{T}\bm{X}\right)^{-1}$. It is unlikely that these 2 would coincide, so the usual estimators of the standard errors are 
		 likely to be erroneous. The inference (test-statistic) based $\sigma^{2}\left(
		 \bm{X}^{T}\bm{X}\right)^{-1}$\\
		 The \tB{\emph{White consistent} estimator of the \emph{asymptotic
		 covariance} matrix of the OLS estimator $\hat{\bm{\beta}}_{OLS}$} in the
		 generalized linear regression model is defined to be:
		 \tB{$\mathbb{V}_{asy}\left(\hat{\bm{\beta}}_{OLS}\right)=N\left(\bm{X}^{T}\bm{X}
		 \right)^{-1}\bm{S}_{0}\left(\bm{X}^{T}\bm{X}\right)^{-1}$}, with $\bm{S}_{0}=
		 \dfrac{1}{N}\su{{i=1}}{N}\hat{\epsilon}_{i}^{2}\sP{\bm{X}_{i\bullet}^{T}}{\bm{X}_{i\bullet}}$\\
\begin{python}
y, X = df.iloc[:, 0], df.iloc[:, 1:]
model_ols = sm.OLS(y, X)
result_ols_white = model_ols.fit(cov_type='HC0') # Applying White correction
print(result_ols_white.summary())
\end{python}

	\item \textbf{\emph{Heteroscedasticity with known $\Sigma$}}\\
		We assume that the disturbances are heteroscedastic with $\V{\bm{\epsilon}|
		\bm{X}}=\bm{\Sigma}=\sigma^{2}\bm{\Omega}$ with :
\begin{center}
\begin{equation*}
	\Sigma=
	\begin{pmatrix}
		\sigma_{1}^{2} & 0 & \cdots & 0\\
		0 & \sigma_{2}^{2} & \cdots & 0\\
		\vdots & \vdots & \ddots & \vdots\\
		0 & 0 & \cdots & \sigma_{n}^{2}\\
	\end{pmatrix}
	= \sigma^{2}\bm{\Omega}=\sigma^{2}
	\begin{pmatrix}
		\omega_{1} & 0 & \cdots & 0\\
		0 & \omega_{2} & \cdots & 0\\
		\vdots & \vdots & \ddots & \vdots\\
		0 & 0 & \cdots & \omega_{n}\\
	\end{pmatrix}
\end{equation*}
\end{center}
In presence of heteroscedasticity, the Generalized Least Squares (GLS) estimator of $\beta$ is 
defined as to:
\tB{$\hat{\bm{\beta}}_{GLS} = \left(\su{{i=1}}{n}\dfrac{x_{i}x_{i}^{T}}{\omega_{i}}\right)^{-1}
\su{{i=1}}{n}\dfrac{x_{i}y_{i}}{\omega_{i}}$}\\
In presence of heteroscedasticity, the GLS estimator is a particular case of the WLS estimator:
\tB{$\hat{\bm{\beta}}_{WLS} = \left(\su{{i=1}}{n}\delta_{i}x_{i}x_{i}^{T}\right)^{-1}
\su{{i=1}}{n}\delta_{i}x_{i}y_{i}$} where $\delta_{i}$ is an arbitrary weight.\\
For $\delta_{i}=\frac{1}{\omega_{i}}$ we have $\hat{\bm{\beta}}_{WLS}=\hat{\bm{\beta}}_{GLS}$\\
The WLS estimator is consistent regardless of the weights used, as long as the weights are 
uncorrelated with the disturbances. In general, we consider a weight which is proportional to
one explicative variable the income in the last example: 
$$ \sigma_{i}^{2} = \sigma^{2}x_{ik}^{2} \Leftrightarrow \delta_{i}=\dfrac{1}{x_{ik}^{2}}$$
\begin{python}
# Assuming y = y_true + sig * w * e
model_wls = sm.WLS(y, X, weights=1/w**2)
results_wls = model_wls.fit()
print(results_wls.summary())
\end{python}

	\item \textbf{\emph{Heteroscedasticity for a given structure}}\\
\begin{center}
\begin{equation*}
	\Sigma(\bm{\alpha})=
	\begin{pmatrix}
		\sigma_{1}^{2}(\bm{\alpha}) & 0 & \cdots & 0\\
		0 & \sigma_{2}^{2}(\bm{\alpha}) & \cdots & 0\\
		\vdots & \vdots & \ddots & \vdots\\
		0 & 0 & \cdots & \sigma_{n}^{2}(\bm{\alpha})\\
	\end{pmatrix}
	= \sigma^{2}\bm{\Omega}(\bm{\alpha})=\sigma^{2}
	\begin{pmatrix}
		\omega_{1}(\bm{\alpha}) & 0 & \cdots & 0\\
		0 & \omega_{2}(\bm{\alpha}) & \cdots & 0\\
		\vdots & \vdots & \ddots & \vdots\\
		0 & 0 & \cdots & \omega_{n}(\bm{\alpha})\\
	\end{pmatrix}
\end{equation*}
\end{center}
		We assume that the disturbances are heteroscedastic with :
		$\V{\bm{\epsilon}|\bm{X}}=\bm{\Sigma}(\alpha)=\sigma^{2}\bm{\Omega}(\bm{\alpha})(\bm{\alpha})$
		where $\bm{\alpha}$ denotes a set of parameters.\\
		Examples:\\
		We assume that $\mathbb{V}(\epsilon_{i}|\bm{X})=\sigma^{2}_{i}(\bm{\alpha})=
		\sigma^{2}\left(\bm{z}_{i}^{T}\bm{\alpha}\right)$ where $\bm{\alpha}=(\alpha_{1},
		\cdots, \alpha_{H})^{T}$ is a $H\times 1$ vector of parameters and $\bm{z}_{i}$
		is a $H\times 1$ of explicative variables (not necessarily the same as in $x_{i}$)
		\\ If we assume for instance that :
		$\V{\epsilon_{i}|\bm{X}}=\sigma_{i}^{2}(\bm{\alpha})=\exp(\bm{z}_{i}^{T}\bm{
		\alpha})$ where $\bm{z}_{i}$ is a vector of H variables, a way to estimate
		$\alpha$ consists in considering the model: $ln(\hat{\epsilon}_{i}^{2})=
		\bm{z}_{i}^{T}\bm{\alpha} + v_{i}$ and to estimate $\alpha$ by OLS. Given 
		$\hat{\alpha}$ we have a consistent estimator for $\sigma_{i}^{2}$, 
		$\sigma_{i}^{2} = \exp(\bm{z}_{i}^{T}\hat{\bm{\alpha}})\rightarrow\sigma_{i}^{2}(
		\bm{\alpha})$\\
		Consider the generalized linear regression model:\\
		$y_{i} = \beta_{0} + \beta_{1}\times age_{i} + \beta_{2}\times ownrent + \beta_{3}
		\times income_{i} + \beta_{4}\times income_{i}^{2}$\\ where the heteroscedasticity
		satisfies Harvey's specification:\\
		$\V{\epsilon_{i}|\bm{X}}=\sigma_{i}^{2}=\exp(\alpha_{0}+\alpha_{1}\times 
		income_{i})$\\
		A way to get the estimates of the parameters $\alpha_{1}$ and $\alpha_{2}$ is to
		consider the regression: $ln\left(\hat{\epsilon}_{i}^{2}\right)=\alpha_{0}+
		\alpha_{1}\times income_{i} + \eta_{i}$, with 
		$\eta\hookrightarrow\mathcal{N}(0,1)$
\begin{python}
beta = sm.OLS(y, X).fit().params # containing beta_0, ..., beta_4
diff = np.ones(beta.shape[0]) # array having as same 1's as beta length  

while max(diff) > 0.001:
    res = y - X.dot(beta)
    W = sm.add_constant(X.loc[:, 'Income']) 
    alpha = sm.OLS(np.log(residuals_ols**2), W).fit().params
    sigma = np.exp(W.dot(alpha)) # Harvey's specification
    beta_fgls = sm.GLS(y, X, sigma=sigma).fit().params
    diff = beta_fgls-beta
    beta = beta_fgls
\end{python}
	
		Estimate the parameters $\bm{\beta}$ by OLS. Compute the residuals 
		$\hat{\epsilon}_{i} = y_{i}-x_{i}^{T}\hat{\bm{\beta}}_{OLS}$ and estimate the 
		parameters $\alpha$ according to the appropriate model. Compute the estimated 
		variance $\sigma_{i}^{2}(\bm{\alpha})$ and compute the Feasible Generalized Least
		Squares (FGLS):
		\tB{$$ \hat{\bm{\beta}}_{FGLS}^{(1)}=\left(\su{{i=1}}{N}\dfrac{x_{i}x_{i}^{T}}{
		\sigma_{i}^{2}(\hat{\bm{\alpha}})}\right)^{-1}\su{{i=1}}{N}\dfrac{x_{i}y_{i}}{
		\sigma_{i}^{2}(\hat{\bm{\alpha}})}$$}
		Compute the residuals $\hat{\epsilon}_{i}=y_{i}-x_{i}^{T}\hat{\bm{
		\beta}}_{FGLS}^{(1)}$ and estimate the parameters $\bm{\alpha}$ according to the
		appropriate model. Compute the FGLS $\hat{\bm{\beta}}_{FGLS}^{(2)}$ and so on 
		$\cdots$. The procedure stop when:\\
		$\sup\limits_{j\in\inter{1}{K}}\left|\hat{\bm{\beta}}_{j,FGLS}^{(i)}-
		\hat{\bm{\beta}}_{j,FGLS}^{(i-1)}\right|<\text{ threshold (ex: 0.001)}$\\
		\sB{Avoiding this method when sample size is small.}
		
\end{enumerate}

The \textbf{White test for heteroscedasticity} is based on: 
$\begin{cases}
	H_{0}:\forall\inter{1}{N},~\sigma_{i}^{2}=\sigma^{2}\\
	H_{a}:\exists(i,j)\in\inter{1}{N}^{2},~\sigma_{i}^{2}\neq\sigma_{j}
\end{cases}
$
Procedure:
\begin{itemize}
	\item[\textit{Step 1}:] Estimation of the model using the OLS estimator of $\bm{\beta}$
	\item[\textit{Step 2}:] Determine the residual $\hat{\epsilon}_{i}=y_{i}-\bm{x}_{i}^{T}
		\hat{\bm{\beta}}_{OLS}$
	\item[\textit{Step 3}:] Regress $\hat{\bm{\epsilon}}_{i}^{2}$ on a constant and all unique
		columns vectors contained in $\bm{X}$ and all the squares and cross-products of
		the column vectors in $\bm{X}$.

	\item[\textit{Step 4}:] Determine the coefficient of determination, $R^{2}$, of the previous regression.
\end{itemize}
Under the null, the White test statistic $N\times R^{2}\rightarrow\chi^{2}(m-1)$ where $m$ is
the number of explanatory variables in the regression of $\hat{\epsilon}_{i}$.\\
The critical region of size $\alpha$ is:
$$ W = \left\{y:N\times R^{2}>\bm{\chi_{1-\alpha}}^{2}\right\}$$ where $\chi_{1-\alpha}^{2}$
denotes the $1-\alpha$ critical value of the $\chi^{2}(m-1)$ distribution.

\begin{python}
test_white = sms.het_white(residuals, X)
fvalue, fpvalue = test_white.fvalue, test_white.f_pvalue
\end{python}

\subparagraph{Breusch and Pagan test}
Breusch and Pagan have devised a \emph{Lagrange mutliplier test} of the hypothesis that:
$\sigma_{i}^{2}=\sigma^{2}f\left(\bm{\alpha}_{0}+\bm{z}_{i}^{T}\bm{\alpha}\right)$ where 
$\bm{z}_{i}$ a $p\times 1$ vector of independent variables.\\
The test is 
$
\begin{cases}
	H_{0}: \bm{\alpha}=\bm{0}_{p\times 1}\text{ (homoscedasticity)}\\
	H_{0}: \bm{\alpha}\neq\bm{0}_{p\times 1}\text{ (heteroscedasticity)}
\end{cases}
$
Define $\bm{Z}$ the $N\times(p+1)$ matrix of observations on $(1, \bm{z}_{i})$ and let $\bm{g}$
be the $N\times 1$ vector of observationsa: $g_{i}=N\dfrac{\hat{\epsilon}_{i}^{2}}{\hat{\bm{
\epsilon}}^{T}\hat{\bm{\epsilon}}}-1$
then, the \emph{Bresusch  and Pagan's test statistic} is defined by:
$$ LM = \dfrac{1}{2}\bm{g}^{T}\bm{Z}\left(\bm{Z}^{T}\bm{Z}\right)^{-1}\bm{Z}^{T}\bm{g}$$


\begin{python}
test_homoscedasticity = sms.het_breuschpagan(residuals, results.model.exog)
fscore, fpvalue = round(test[2], 1), round(test[3], 2)
\end{python}


\subparagraph{Outliers}
An outlier point can make few differences in least squared fit
nonetheless it can have a great impact on RSE  and thus on the 
confidence interval
\begin{python}
fig, ax = plt.subplots()
fig = sm.graphics.influence_plot(results_ols, ax=ax)
\end{python}
\tB{To observe residuals we can plot them versus different variables.}
\begin{figure}[H]
	\begin{center}
		\includegraphics[width=\textwidth]{./chap/1chap/2sec/2images/2_9outliers.png}
	\end{center}
	\caption{Left:The least squared regression is shown in red
	and the least squared regression after removal outlier is
	shown in dashed blue line. Center: the residual plot identifies
	clearly the outlier value. Right: The studentized residual plot
	identifies outliers too}
	\label{fig:fig2.8}
\end{figure}



%\begin{lstlisting}
%numerics = ['int16', 'int32', 'int64', 'float16', 'float32',\
%'float64']
%num_df = df.select_dtypes(include=numerics).copy()
%outlier_limit = [[0, 0] for k in num_df.columns]
%percScore = outlier_limit.copy()
%
%for j,k in enumerate(num_df.columns):
%    n_bins = 30
%    bins = np.linspace(df[k].min(), df[k].max(), n_bins)
%    hist, edges = np.histogram(df.loc[:, k], bins=bins)
%    y = np.linspace(1, hist.max(, len(hist), dtype='int')
%    x = np.linspace(df[k].min(, df[k].max(, len(hist))
%    x_matrix, y_matrix = np.meshgrid(x, y)
%    mc = medcouple(df[k])
%    if mc > 0:
%        iq_adj = [1.5*np.exp(-4*mc), 1.5*np.exp(3*mc)]
%    else:
%        iq_adj = [1.5*np.exp(-3*mc), 1.5*np.exp(4*mc)]
%    low = np.percentile(df[k], 25) - iq_adj[0]*(np.percentile(df[k], 75) - np.percentile(df[k], 25))
%    up = np.percentile(df[k], 75) + iq_adj[1]*(np.percentile(df[k], 75) - np.percentile(df[k], 25))
%    outlier_limit[j][0], outlier_limit[j][1] = low, up
%    a, b = round(sci.percentileofscore(df[k], low), 2), round(sci.percentileofscore(df[k], up), 2)
%       					        
%    skew, kurt = round(sci.skew(df[k]), 1), round(sci.kurtosis(df[k]), 1)
%    fig, axs = plt.subplots(nrows=2, sharex=True)
%    fig.suptitle(f'Dot Plot (bins={n_bins}) and Box Plot,\
%        skew={skew}, kurtosis={kurt}')
%    axs[0].scatter(x_matrix, y_matrix, c=y_matrix<=hist, cmap='Greys')
%    axs[0].set_ylabel('Frequency Number')
%    axs[1].boxplot(df[k], flierprops=red_square, whis=(a, b), vert=False)
%    for ax in axs:
%        ax.set(xlabel=k)
%    outlier_limit_dict = {df.columns[j]:tuple(k) for j,k in enumerate(outlier_limit)}
%\end{lstlisting}

\begin{enumerate}
	\item \emph{M Estimation}
		It is an extension of the maximum likelihood estimate method, its priniciple is to
		minimize the residual funcion $\rho$:
		$\hat{\beta}_{M}=\min\limits_{\beta}\rho\left(y_{i}-\su{{j=0}}{p}x_{ij}\beta_{j}
		\right)$, we have to solve $\min\limits_{\beta}\su{{i=1}}{n}\rho(u_{i})=\min\limits_{\beta}\su{{i=1}}{n}\rho\left(\dfrac{y_{i}-\su{{j=0}}{p}
		x_{ij}\beta_{j}}{\sigma}\right)$.\\
		$\hat{\sigma}=\dfrac{MAD}{0.6745}=\dfrac{median\left(\epsilon_{i}-median(\epsilon_{i})\right)}{0.6745}$\\
		If we use Tukey's objective function we have
		$\rho$:
		$\begin{cases}
			\dfrac{u_{i}^{2}}{2} - \dfrac{u_{i}^{4}}{2c^{2}} + \dfrac{u_{i}^{6}}{6c^{4}},~
			|u_{i}| \leq c\\
			\dfrac{c^{2}}{6},~|u_{i}|>c
		\end{cases}$\\
		with $c=0.6745$
		We look for first partial derivative $\hat{\beta}_{M}$ to $\beta$ so that:\\
		$\su{{i=1}}{n}x_{ij}\psi\left(\dfrac{y_{i}-\su{{j=0}}{p}x_{ij}\beta}{\hat{\sigma}}
		\right)$ where $\psi=\rho'$. To solve this equation Draper and Smith have defined 
		a weighted function: 
		$\omega(\epsilon_{i})=
		\dfrac{\psi\left(\dfrac{y_{i}-\su{{j=1}}{p}x_{ij}\beta}{\hat{\sigma}}\right)}
		{\dfrac{y_{i}-\su{{j=1}}{p}x_{ij}\beta}{\hat{\sigma}}}$\\
		We can rewrite this equation with:
		$\omega_{i}=\begin{cases}
			\left[ 1-\left(\dfrac{u_{i}}{c}\right)^{2}\right]^{2},~|u_{i}| \leq c\\
			0,~|u_{i}| > c
		\end{cases}
		$
		We take $c=4.685$ for Tukey's bisquare weighted function, so the equation of partial
		derivatives becomes $\su{{i=1}}{n}x_{ij}\omega_{i}\left(y_{i}-\su{{j=1}}{p}x_{ij}
		\beta\right)=0$ that can be solved by Iterative Reweighted Least Squares (IRLS)
		In matrix notation $\bm{X}^{T}\bm{W}_{i}\bm{X}\beta = \bm{X}^{T}\bm{W}_{i}\bm{Y}$
		where $\bm{W}_{i}$ is a $n\times n$ matrix with its diagonal elements are the 
		weighted.

		\begin{enumerate}
			\item Estimate regression coefficients on the data using OLS
			\item Test assumptions of the regression model
			\item Detect the presence of outliers in the data.
			\item Calculate estimated $\hat{\beta}^{0}$ with OLS
			\item Calculate residual value $\epsilon_{i}=y_{i}-\hat{y}_{i}$
			\item Calculate value $\hat{\sigma}_{i}=1.4826\times MAD$
			\item Caluclate value $u_{i}=\dfrac{\epsilon_{i}}{\hat{\sigma_{i}}}$
			\item Calculate the weighted value $\omega_{i}$ 
			\item Calculate $\hat{\beta}_{M}$ using WLS with $\omega_{i}$
			\item Repeat steps (e)-(h) to obtain a convergent value of $\hat{\beta}_{M}$
			\item Test to determine whether independent variables have significant effect
				on the dependent variable
		\end{enumerate}
	\item S estimation
		Proposed by Rousseeuw and Yohai, it is based on residual scale of M estimation. The
		weakness of M estimation is the lack of consideration on the data distribution and
		not a function of the overall data because only using the median on as the weighted
		value. The S estimation method uses the residual standard deviation to overcome the 
		weakness of median.\\
		$\min\su{{i=1}}{n}\rho\left(\dfrac{y_{i}-\su{{j=1}}{p}x_{ij}\beta}{\hat{\sigma}_{s}}
		\right)$ with $\hat{\rho}_{s}=\sqrt{\dfrac{1}{nK}\su{{i=1}}{n}\omega_{i}e_{i}^{2}}$,
		$K=0.199$, $\omega_{i}=\omega_{\sigma}(u_{i})=\dfrac{\rho(u_{i})}{u_{i}^{2}}$
		and the initial estimate is $\hat{\sigma}_{s}=\dfrac{median\left(\epsilon_{i}-
		median(\epsilon_{i})\right)}{0.6745}$
		For IRLS weighted function $c= 1.547$
		\begin{enumerate}
			\item[(e)] Calculate residual value $\epsilon_{i}=y_{i}-\hat{y}_{i}$ 
			\item[(f)] Calculate value of $\hat{\sigma}_{i}=$
				\begin{align*}
				\begin{cases}
					\dfrac{median\left(\epsilon_{i}-median(\epsilon_{i})\right)}{
					\hat{\sigma}_{s}},& itertion=1\\
					\sqrt{\dfrac{1}{nK}\su{{i=1}}{n}\omega_{i}\epsilon_{i}^{2}},&
					iteration>1
				\end{cases}
				\end{align*}

			\item[(g)] Caluclate value $u_{i}=\dfrac{\epsilon_{i}}{\hat{\sigma_{i}}}$
			\item[(h)] Calculate the weighted value $\omega_{i}=$
				\begin{align*}
				\begin{cases}
					\begin{cases}
						\left[1-\left(\dfrac{u_{i}}{1.547}\right)^{2}
						\right]^{2},&|u_{i}|\leq 1.547\\
						0, &|u_{i}|>1.547
					\end{cases},&iteration = 1\\
					\dfrac{\rho(u_{i})}{u_{i}^{2}},&iteration>1
				\end{cases}
				\end{align*}
		\end{enumerate}
	


\end{enumerate}

We can also use a robust the M-Estimators for Robust Linear Modeling
\emph{M-Estimators} are a broad class of extremum estimators for which the objective function is
a sample average.\\
They are defined by: 
$\hat{\theta} = \min\limits_{\theta}\left(\su{{i=1}}{n}\rho(x_{i},\theta)\right)$
whith $\rho$ the objective function.
\begin{python}
model_robust = sm.RLM(y, X, M=sm.robust.norms.TukeyBiweight(c=4.685))
result_robust = model_robust.fit()
print(result_robust.summary())
print(model_robust.weights)
\end{python}


\subparagraph{High leverage points}
\tR{Outliers are points which have an unusual response $y_{i}$ for a 
given predictor $x_{i}$, whereas leverage points have a unusual value 
for $x_{i}$.}
\begin{figure}[H]
	\begin{center}
		\includegraphics[width=\textwidth]{./chap/1chap/2sec/2images/2_10highLeveragePoints.png}
	\end{center}
	\caption{Left:Observation $41$ is a high leverage point, while
	20 is not.The red line is the least squared fit to all data,
	and the blue line is the fit with observation $41$ removed.\\
	Center:The red observation is not unusual in term of its $X_{1}
	$or its $X_{f2}$ value.\\Right: Observation $41$ has a high
	leverage and a high residual.}
	\label{fig:fig2.8}
\end{figure}
In simple linear regression is fairly easy to identify, since we can
simply look for observations for which predictors are outside of the
normal range of value of the observations.\\But in multiple linear
regression it is possible to have an observation that is well within
the range of each individual predictor's values (example graphic in the
middle). \sB{For data set with more than 2 predictors, it is difficult
to observe high leverage points, so we compute the} \emph{\tR{leverage 
statistic}}
\begin{center}
	\enc{$h_{i}=\dfrac{1}{n}+\dfrac{\left(x_{i}-\overline{x}
	\right)^{2}}{\su{{k=1}}{n}\left(x_{k}-\overline{x}
	\right)^{2}}$}\\
	\tR{A high $h_{i}$ value indicates a high leverage point.}
\end{center}
\begin{python}
results_ols = sm.OLS(y, X).fit()
sm.graphics.plot_leverage_resid2(results_ols)
plt.show()
\end{python}
To fix it Python does not contain an alternative at M-Estimator method (it is not robust against
leverage points), so we will use R functions

\emph{MM Estimation} aims to obtain estimates that have a high breakdown value and more efficient. 
Breakdown value is a common measure of the proportion of outliers that can be 
addressed before these observation affect the model.\\

\begin{enumerate}
	\item[(e)] Calculate residual value $\epsilon_{i}=y_{i}-\hat{y}_{i}$ 
	\item[(f)] Calculate value of $\hat{\sigma}_{i}=\hat{\sigma}_{s}$
	\item[(g)] Caluclate value $u_{i}=\dfrac{\epsilon_{i}}{\hat{\sigma_{i}}}$
	\item[(h)] Calculate the weighted value $\omega_{i}$ with $c=4,685$ as in
		M-estimation
\end{enumerate}

\emph{R code}
\begin{rcode}[deletekeywords={model, df}]
library(robustbase)

model_rob_mm <- lmrob(y ~ ., data=df) # MM-estimation
print(summary(model_rob_mm))
\end{rcode}
\subparagraph{Collinearity}
When there is a very highly correlation with each other we say that
they are \tR{collinear}. And this causes difficulties because to
separate out the individual effects.
\begin{figure}[H]
	\begin{center}
		\includegraphics[width=\textwidth]{./chap/1chap/2sec/2images/2_11collinearity.png}
	\end{center}
	\caption{Scatter plot of the observation from credit data set.
	}
	\label{fig:fig2.8}
\end{figure}
\begin{figure}[H]
	\begin{center}
		\includegraphics[width=\textwidth]{./chap/1chap/2sec/2images/2_12contourPlotCollinearity.png}
	\end{center}
	\caption{In each plot the black dot represents the coefficient
	values corresponding to the minimum value of RSS.\\
	Left: a contour plot of RSS for the regression of balance onto
	regression age and limit.\\Right:Because of the collinearity
	there are many pairs of $\beta_{Limit},\beta_{Rating}$ with a
	similar value of RSS.}
	\label{fig:fig2.8}
\end{figure}
\begin{figure}[H]
	\begin{center}
		\includegraphics[width=\textwidth]{./chap/1chap/2sec/2images/2_13collinearModelAndNonCollinearModel.png}
	\end{center}
	\caption{Model 1 is a multiple regression of balance onto age
	and limit, then Model 2 is one of balance onto rating and limit
	\\The standard error of $\widehat{\beta}_{limit}$ increases
	12-fold in the second regression, due to collinearity.}
	\label{fig:fig2.8}
\end{figure}
Recall that for each predictor $\widehat{\beta}_{j}$ the associated
\emph{$t$-statistic} is divided by its standard error, consequently,
\tR{collinearity results in a decline in the \emph{$t$-statistic}}.\\A simple
way to identify collinearity is to observe the correlation matrix, but
if it exists collinearity between $3$ or more variables and no pair of
variables has a particularly high correlation we cannot detect 
collinearity, we call this \emph{multicollinearity}.\\Then a better way
to assess multicollinearity is to compute the \emph{\tR{variance 
inflation factor}} it is the ratio of the variance of $\widehat{
\beta}_{j}$ when fitting the full model divided by the variance of $
\widehat{\beta}_{j}$ if fit on its own. As a rule of thumb, \tB{a VIF
value that exceeds 5 or 10 indicates problematic amount of 
collinearity.}
\begin{center}
	\enc{$VIF\left(\widehat{\beta}_{j}\right)=\dfrac{1}{1-
	R_{X_{j}|X_{-j}}^{2}}$}\\
	\tB{$R_{X_{j}|X_{-j}}^{2}$ is the $R^{2}$ from regression of
	$X_{j} $ onto all of the other predictors.}
\end{center}
\begin{python}
import statsmodels.stats
from statsmodels.stats.outliers_influence import variance_inflation_factor as vif
print(pd.DataFrame({ 
  'VIF': [vif(X.values, i) for i in range(X.shape[1])],
  'index': X.columns}))
\end{python}

%\subsection{ANOVA \& ANCOVA}
%\paragraph{ANOVA}
%\subparagraph{Predictors with only 2 levels}
%One way of formulating the common-slope model is :
%$$ Y_{i} = \alpha + \beta X_{i} + \gamma D_{i} + \epsilon_{i}$$ where $D$, called a dummy-variable
%regressor or indicator variable, $$D1=
%\begin{cases}
%	1\text{ for women}\\
%	0\text{ for men}
%\end{cases}
%$$
%\begin{figure}[H]
%	\begin{center}
%		\includegraphics[width=.3\textwidth]{./chap/1chap/2sec/3images/1_graphicalDummy.PNG}
%	\end{center}
%	\caption{The parameters in the additive dummy-regression model}
%	\label{fig:1_graphicalDummy}
%\end{figure}
%
%\emph{gender} is a qualitative explanatory variable, with categories \emph{male} and \emph{female}
%the dummy variable $D$ is a regressor representing the explanatory variable gender.
%\subparagraph{Modelling interactions}
%$$ Y_{i} = \alpha + \beta X_{i} + \gamma D_{i} + \delta X_{i}D_{i} + \epsilon_{i}$$
%
%\begin{figure}[H]
%	\begin{center}
%		\includegraphics[width=.3\textwidth]{./chap/1chap/2sec/3images/2_graphicDummyInteract.PNG}
%	\end{center}
%	\caption{The parameters in the dummy-regression model with interaction}
%	\label{fig:1_graphicalDummy}
%\end{figure}

\subparagraph{ANOVA models}
Analysis of Variance describes \sB{the partition of the response variable sum of squares in a 
linear model into ``explained'' and ``unexplained'' components.}\\
\begin{itemize}
	\item Single categorical explanatory variable corresponds to One-Way ANOVA
	\item 2 factors to Two-Way ANOVA
	\item 3 factors to Three-Way ANOVA
\end{itemize}

\subparagraph{\emph{One-way} ANOVA}
\tR{examines equality of population means for a quantitative outcome and a 
single categorical explanatory variable with any number of levels}.\\
The term \tB{``one-way'' indicates that there is a single explanatory variable (``treatment'') 
with 2 or more levels and only one level of treatment is applied at any time for a given subject}.
\\
The term ``analysis of variance'' is a bit of misnomer, \tB{we use variance-like quantities to
study the equality or non-equality of population means}, so we are analyzing means, not variances.

\begin{center}
	The statistical model for which one-way ANOVA is appropriate is that the
	\begin{itemize}
		\item (Quantitative) Outcomes for each group are normally distributed
		\item Outcome variances are all equal to  ($\sigma^{2}$)
		\item The errors are assumed to be independent.
	\end{itemize}
\end{center}

\begin{center}
	\enc{$H_{0}:\forall (i,j) \in \inter{1}{k}^{2} \mu_{i} = \mu_{j}$}
\end{center}

In ANOVA \sB{we work with variances and also ``variance-like quantities'' which are not really the
variance of anything}, but are still calculated as $\frac{SS}{df}$ all of these quantities are
called ``mean squares''.\\

The deviation for subject $j$ of group $i$ in the figure above is mathematically equal to $Y_{ij}-
\overline{Y}_{i}$ where $Y_{ij}$ is the observed value for subject $j$ of group $i$ and $\overline{Y}_{i}$ is the sample mean for group $i$.
\begin{figure}[H]
	\begin{center}
		\includegraphics[width=\textwidth]{./chap/1chap/2sec/3images/3_anovaInterGrp.PNG}
	\end{center}
	\caption{Deviations for within-group of squares}
	\label{fig:3_anovaInterGrp}
\end{figure}
\begin{center}
\enc{
$ MS_{within} = \dfrac{SS_{within}}{df_{within}}
\begin{cases}
	SS_{within} = \su{{j=1}}{k}SS_{j}=\su{{j=1}}{k}\su{{i=1}}{n_{j}}\left(Y_{ij}-\overline{Y}_{\bullet j}
	\right)\\
	df_{within} = df_{j} = \su{{j=1}}{k}(n_{j}-1) = N-k
\end{cases}
$
}\end{center}

\tB{$MS_{within}$ is a good estimate of $\sigma^{2}$ from our model regardless of the truth of 
$H_{0}$.} This is due to the way $SS_{within}$ is defined.

\begin{figure}[H]
	\begin{center}
		\includegraphics[width=\textwidth]{./chap/1chap/2sec/3images/4_anovaBtw.PNG}
	\end{center}
	\caption{Deviations for betwwen-group sum of squares}
	\label{fig:4_anovaBtw}
\end{figure}
$SS_{between}$ is the sum of the $N$ squared between-group deviations, where the deviation is
the same for all subjects in the same group. The formula is : 
\begin{center}
\enc{
$
MS_{Between} = \dfrac{SS_{Between}}{df_{between}}
\begin{cases}
SS_{between} = \su{{j=1}}{k}n_{j}\left(\overline{Y}_{\bullet j}-\overline{Y}\right)^{2}\\
df_{between} = k-1
\end{cases}
$
}\end{center}
Because of the way $SS_{between}$ is defined, \tB{$MS_{between}$ is a good estimate of $\sigma^{2}
$ only if $H_{0}$ is true}. Otherwise it tends to be larger. \\
The $F-statistic$ defined by $F=\dfrac{MS_{between}}{MS_{within}}$ \sR{tends to be larger if the
alternative hypothesis is true than if the null hypothesis is true}.\\

We can quantify ``large'' for the \emph{F-statistic}, by comparing it to its null sampling 
distribution which is the specific \emph{F-distribution}  which has degrees of freedom matching
the numerator and denominator of the \emph{F-statistic}.\\
Concerning inferences to build the confidence interval we need the \tB{\emph{standard error} (the
standard deviation of the means) that is $\sqrt{\dfrac{MS_{within}}{n_{i}}}$}\\

Numerically we have:\\
Given 2 samples with means $\mu_{1}$ and $\mu_{2}$, same variance $\sigma^{2}$ and $n=n_{1}+n_{2}$
observations.
Model: 
\begin{center}
	$\forall (j,i)\in\inter{1}{2}\times\inter{1}{n_{j}}y_{ij} = \mu_{i} + \epsilon_{ij} = \mu + \alpha_{j} + \epsilon_{ij}$
\end{center}
$\alpha_{j}=\mu_{j}-\mu$ is called (treatment-) effect

Decomposition:
\begin{align*}
	SS_{total} =& \su{{j=1}}{n_{1}}\left(y_{1j}-\overline{y}\right)^{2} +
	\su{{j=1}}{n_{2}}\left(y_{2j}-\overline{y}\right)^{2}\\
	=& \su{{j=1}}{n_{1}}\left(y_{1j}-\overline{y_{1}}+\overline{y_{1}}-\overline{y}\right)^{2}
	+ \su{{j=1}}{n_{2}}\left(y_{2j}-\overline{y_{2}}+\overline{y_{2}}-\overline{y}\right)^{2}\\
	=& \underbrace{(n_{1}-1)s_{1}^{2} + (n_{2}-1)s_{2}^{2}}_{SS_{within}} +
	\underbrace{n_{1}(\overline{y}_{1}-\overline{y}) + 
	n_{2}(\overline{y}_{2})-\overline{y}^{2}}_{SS_{between}}
\end{align*}
$SS_{between}$ corresponds to squared enumerator $(\overline{y}_{1} - \overline{y}_{2})^{2}$ of
the statistic: 
\begin{align*}
	SS_{between} =& n_{1}(\overline{y}_{1}-\overline{y})^{2} + 
	n_{2}(\overline{y}_{2}-\overline{y})^{2}\\
	=& n_{1}\left(\overline{y}_{1}-\dfrac{n_{1}\overline{y}_{1}+n_{2}\overline{y}_{2}}{n_{1}+
	n_{2}}\right)^{2} + n_{2}\left(\overline{y}_{2}-\dfrac{n_{1}\overline{y}_{1}+n_{2}
	\overline{y}_{2}}{n_{1}+ n_{2}}\right)^{2}\\
	=& \dfrac{n_{1}n_{2}}{n_{1}+n_{2}}\left(\overline{y}_{1}-\overline{y}_{2}\right)^{2}
\end{align*}
$SS_{within}$ corresponds to denominator of \emph{t-statistic}:
$s=\sqrt{\dfrac{(n_{1}-1)s_{1}^{2}+(n_{2}-1)s_{2}^{2}}{n_{1}+n_{2}-2}}$\\
Pooled variance that is an estimate of the fixed common variance $\sigma^{2}$ underlying various
populations that have different means.
$\hat{\sigma}=\dfrac{(n_{1}-1)s_{1}^{2}+(n_{2}-1)s_{2}^{2}}{(n_{1}-1)+(n_{2}-1)}$
Null hypothesis $H_{0}:\mu_{1}=\mu_{2}$ or $\alpha_{1}=\alpha_{2}=0$
\emph{F-test}
$\left(\overline{Y}_{1}-\overline{Y}_{2}\right)\hookrightarrow \mathcal{N}\left(\mu_{1}-\mu_{2},
\left(\frac{1}{n_{1}}+\frac{1}{n_{2}}\right)\sigma^{2}\right)\\
\E{\left[\overline{Y}_{1}-\overline{Y}_{2}\right]^{2}}=\left(\frac{1}{n_{1}}+\frac{1}{n_{2}}
\right)\sigma^{2}+(\mu_{1}-\mu_{2})^{2}\\
\E{MS_{between}}=\E{\frac{n_{1}n_{2}}{n_{1}+n_{2}}\left[\overline{Y}_{1}-\overline{Y}_{2}
\right]^{2}} = \sigma^{2}+\frac{n_{1}n_{2}}{n_{1}+n_{2}}(\mu_{1}-\mu_{2})^{2}\\
\E{MS_{whithin}}=\sigma^{2}\\
F = \dfrac{MS_{between}}{MS_{within}}$
Here $F=t^{2}$\\
\begin{align*}
	\text{Degrees of freedom} =& n - 1\\
	=& \underbrace{(n-m)}_{df_{within}} + \underbrace{(m-1)}_{df_{between}}
\end{align*}
\begin{itemize}
	\item $SS_{within}$ and $SS_{between}$ are independent
	\item under $H_{0}$ $\E{MS_{between}} = \E{MS_{within}} = \sigma^{2}$
	\item under $H_{a}$ $\E{MS_{between}}>\sigma^{2}$ and $\E{MS_{within}} = \sigma^{2}$
\end{itemize}
Hence $$ F= \dfrac{MS_{between}}{MS_{within}}\hookrightarrow F_{m-1,n-m}$$
In the case of 2 groups (``\emph{t-test}'') we received:
$$ \overline{y}_{1}-\overline{y}_{2} \pm t_{n-2,1-\frac{\alpha}{2}}s\sqrt{\frac{1}{n_{1}}+
\frac{1}{n_{2}}}$$
\subparagraph{Two-way-ANOVA}
If a quantitative explanatory variable is also included, that variable is usually called a 
\emph{covariate}.\\
The usual assumptions of normality, equal variance and independent errors apply.\\

\tB{ANOVA decomposes the total variance} present in the data into contribution of the single 
sources of variation :
\begin{itemize}
	\item \tB{systematic contribution}: differences of means
	\item \tB{random rest}: variability around group mean
\end{itemize}
We have the total variance law:
\begin{center}
\enc{
$
\begin{cases}
X\& Y\text{ random variables on the same probability space}\\
\text{variance of }Y\text{ is finite}
\end{cases}\Rightarrow
\V{Y} = \E{\mathbb{V}\left(Y|X\right)} + \V{\mathbb{E}\left(Y|X\right)}
$}
\end{center}

Numerically:
$$\forall (i,j,k)\in\inter{1}{m_{1}}\times\inter{1}{m_{2}}\times\inter{1}{n_{ij}}~y_{ijk} =
\mu_{ij} +\epsilon_{ijk},~\epsilon_{ijk}\hookrightarrow \mathcal{N}(0,\sigma^{2})$$
Decomposition of means:
\begin{align*}
	\mu_{ij} &= \mu + (\mu_{i}-\mu) + (\mu_{j}-\mu) + (\mu_{ij}-\mu_{j}-\mu_{i}+\mu)\\
	&= \mu +\alpha_{i} +\beta_{j} + \gamma_{ij}\\
	&= \text{``overall mean''} + \text{``main effect of A''} + \text{``main effect of B''} +
	\text{``interaction of A and B''}
\end{align*}
Null hypothesis:
\begin{itemize}
	\item All $\alpha_{i}=0$
	\item All $\beta_{j}=0$
	\item All $\gamma_{ij}=0$
\end{itemize}

\begin{python}
import statsmodels.api as sm

data = pd.read_csv('myFile.csv', sep=',')
model = sm.ols('y ~ C(X1, Sum)*C(X2, Sum)', data=data).fit()
table = sm.stats.anova_lm(model, typ=2) # typ=2 indicates that we are testing for each main effect
# after the other main effect.
print(table)
\end{python}

\paragraph{ANCOVA} The multiple regression model below would be equal to an ANCOVA, if \sB{$X_{1}$ was binary and $X_{2}$ was some covariate of interest}:$y = \beta_{0}+\beta_{1}X_{1}+\beta_{2}X_{2}+\epsilon$\\ The key is in the interpretation of the intercept value and the slope for the binary predictor.  If $X_{1}$ is binary with values 0 and 1, then the \begin{itemize} \item \textbf{\emph{intercept}} is the average of \tB{$\beta_{0}=\E{y|(X_{1},X_{2})=(0,0)} $} \item \textbf{\emph{slope}} represents the mean difference between 0 and 1 group\\ \tB{$\beta_{1} = \E{y|(X_{1},X_{2})=(1, x_{2})}-\E{y|(X_{1},X_{2})=(0, x_{2})}$}
\end{itemize}

\subparagraph{When is ANCOVA used?}
It is usually used for analysis of quasi-experimental studies, when the regression groups are
not randomly assigned and the researcher wishes to statically ``equate'' groups on one or more
variables which may differ across groups. 
\subparagraph{Relation to Repeated Measures ANOVA}
The repeated measures ANOVA (and the paired t-test) is equivalent to test if the average 
difference score is different from zero. However ANCOVA and Repeated Measures ANOVA tests are not
equivalent, because they represent 2 different ways of conceptualizing change.
\subparagraph{Adjusted Means} 
ANCOVA procedures will produce adjusted means, which represents the means of each group once the
covariate(s) has been controlled.\\ In Regression terms, these adjusted means come from the 
intercept: $\beta_{0}=\overline{y}-\beta_{1}\overline{X}_{1}-\beta_{2}\overline{X}_{2}$
\sB{Because the intercept represents the average value of $y$ when all predictors equal zero, we 
have to pay attention to the scaling of the variables.} If the grouping variable, $X_{1}$, has two
values 0 and 1, the adjustment leads to an intercept for the 0 group where the covariate, $X_{2}$,
is also equal zero. In many cases, the estimated mean when covariates are equal to zero is not
meaningful. So if using regression analysis and there is interest in obtaining the adjusted means,
\tB{it is common to rescale the covariates, $x_{2}=X_{2}-\overline{X}_{2}$}.\\
Then in regression analysis, the adjusted means can be computed by using the regression 
coefficients and inserting values for $X_{1}$\\ $\overline{Y}_{adj}=\beta_{0}+\beta_{1}X_{1}-\beta_{2}\overline{X}_{2}$

\subparagraph{Should I use ANCOVA or Regression?}
Test hypothesis about group differences using an ANCOVA procedure or a regression analysis will 
give the same result, when there are several categories for the independent variable or 
interactions are of interest we should use ANCOVA. 
\subparagraph{Sum of Squares}
The issue type  of sum of squares comes up in ANCOVA as it does in ANOVA: 

\begin{itemize}
	\item[Type \textbf{I}:] Each effect partials out or controls for only those effect
		entered before it. If effect $A$ is entered first, it does not partial out $B$ or 
		$A\times B$ added after it. But $B$ can partial out $A$, and interaction $A\times
		B$ partials out $A$ and $B$
	\item[Type \textbf{II}:] Each effect at the same step or before is controlled but not for
		later steps. Say $A$ and $B$ are entered first and then $A\times B$ is added. $A$
		controls for $B$ and $B$ controls for $A$. Neither controls for $A\times B$, but
		$A\times B$ controls for both $A$ and $B$.
	\item[Type \textbf{III}:] adjusts fro all other factors or variables in the model. 
\end{itemize}

\emph{R code}
\begin{rcode}[deletekeywords={model, df}]
model_ancova <- aov('y ~ f1 + f2*x1', data=df)
print(summary(model_ancova))
\end{rcode}



\section{Classification}
To predict qualitative response is know as \emph{classifying}
%\subsection{An overview of classification}
%\begin{figure}[H]
	\begin{center}
		\includegraphics[width=\textwidth]{./chap/1chap/3sec/1images/1DefaultDataSetPlot.png}
	\end{center}
	\caption{The \emph{default} data set.\\Left:indivduals who
defaulted in orange in a given month and those who did not in blue.\\}
Right: The first boxplot shows the balance distribution split by the
binary default variable, the second plot the income distribution.
	\label{fig:fig3.1}
\end{figure}
It appears that individuals who defaulted tended to have a higher 
credit card balances.\\In this chapter we learn how to build a model to
predict default for  any values of balance and income.

\subsection{Why not linear regression ?}
In general there is no natural way to convert a qualitative response 
variable with more than 2 levels into the quantitative response that
is ready for linear regression. 

\subsection{Logistic regression}
\paragraph{The logistic model}
\emph{How should we model the relationship between $\ProbC{X}{Y=1}$ and
$X$?}
\begin{figure}[H]
	\begin{center}
		\includegraphics[width=\textwidth]{./chap/1chap/3sec/3images/1ProbabilityOfDefault.png}
	\end{center}
	\caption{Left:Estimated probability of default using linear
regression. Some of estimated probability are negative! The orange 
ticks indicate the $0/1$ values coded for default (NO/YES)\\Right:
Predicted probabilities of default using logistic regression.\\All
probabilities lie between $0$ and $1$.}
	\label{fig:fig3.2}
\end{figure}
\paragraph{The logistic model}
\subparagraph{Assumptions}
\begin{enumerate}
	\item Logistic regerssion must be binary or ordinal.
	\item Mutual independence of observations.
	\item No collinearity
	\item Linearity of qualitative independent variables and log odds.
\end{enumerate}


\subparagraph{Requirement}
\begin{enumerate}
	\item Logistic regression requires quite large sample size.
\end{enumerate}
\subparagraph{Formula}
The logistic regression:
\begin{center}
%	\encN{$p\left(X\right)=\ProbC{X}{Y=1}=\dfrac{e^{\beta_{0}+
%\beta_{1}X}}{1+e^{\beta_{0}+\beta_{1}X}}$}\\
	\encN{$p\left(X\right)=\dfrac{e^{\beta_{0}+
\beta_{1}X}}{1+e^{\beta_{0}+\beta_{1}X}}$}\\
\enc{$\log\left(\dfrac{p\left(X\right)}{1-p\left(X\right)}\right)=
\beta_{0}+\beta_{1}X$}\\ The $\dfrac{p\left(X\right)}{1-p\left(X\right)
}$ quantity is called \emph{odds}.
\end{center}
\paragraph{Estimating the Regression Coefficients}
To fit a logistic model we use the \tR{maximum likelihood} method 
rather that the least squares method (which is a specific case of the
former method). Then we seek estimates $\hb{0}\text{ and }\hb{1}$
such that the predicted probability $\widehat{p}\left(x_{i}\right)$ of
default for each individual corresponds closely as possible to the 
individual default status.
\begin{center}
%	\enc{$\mathcal{L}\left(\hb{0},\hb{1}\right)=\prd{{i:y_{i}=1}}{n}\ProbC{x_{i}}{y_{i}=1}\prd{{j:y_{j}=0}}{n}\left(1-\ProbC{x_{j}}{y_{j}=0}\right)$}\\
\enc{$\mathcal{L}\left(\hb{0},\hb{1}\right)=\prd{i=1}{n}p(x_{i})^{y_{i}}\left(1-p(x_{i})\right)^{1-y_{i}}$}\\
$\hb{0}$ and $\hb{1}$ are chosen to maximize this likelihood function.
\end{center}
%\begin{figure}[H]
%	\begin{center}
%		\includegraphics[width=\textwidth]{./chap/1chap/3sec/3images/2estimatesCoeffsLR.png}
%	\end{center}
%	\caption{Estimated coefficients of the logistic regression 
%	model that predict the probability of default using balance.\\
%	A once unit increase in balance is associated with an increase
%	in the log odds of default by $0.0055$ units.}
%	\label{fig:fig3.2}
%\end{figure}
\tB{The \emph{$z$-statistics} plays the same role as \emph{$t$-statistic} in
the linear regression output, $z=\frac{\hb{1}}{SE\left(\hb{1}\right)}$.}
\\The estimated intercept is typically not of interest, its main 
purpose is to adjust the average fitted probabilities to the proportion
of ones in the data.
\paragraph{Marketing Plan}
%\begin{figure}[H]
%	\begin{center}
%		\includegraphics[width=\textwidth]{./chap/1chap/3sec/3images/3LRonDummyVariable.png}
%	\end{center}
%	\caption{Estimated coefficients of the logistic regression
%	model that predicts the probability of default using student
%	status.\\Student status is encoded as a dummy variable, with
%	value 1 for a student and $0$ for a non-student.}
%	\label{fig:fig3.2}
%\end{figure}
$\begin{cases}
	\ProbC{{student=YES}}{\widehat{default}=YES}=\frac{e^{-3.5041+0.4049\times 1}}{1+e^{-3.5041+0.4049\times 1}}=0.0431\\
	\ProbC{{student=NO}}{\widehat{default}=YES}=\frac{e^{-3.5041+0.4049\times 0}}{1+e^{-3.5041+0.4049\times 0}}=0.0292
\end{cases}$\\
This indicates that students tend to have higher default probabilities
than non-student.
\paragraph{Multiple Logistic Regression}
\begin{center}\enc{
$log\left(\dfrac{p\left(X\right)}{1-p\left(X\right)}\right)=\beta_{0}+
\su{{j=1}}{p}\beta_{j}X_{j}$}
\end{center}
%%\begin{figure}[H]
%%	\begin{center}
%%		\includegraphics[width=\textwidth]{./chap/1chap/3sec/3images/4multipleLR.png}
%%	\end{center}
%%	\caption{Estimated coefficients of the logistic regression
%%	model that predicts the probability of default using balance,
%%	income and student status.\\Student status is encoded as a
%%	dummy variable, with value 1 for a student and $0$ for a
%%	non-student.\\In fitting this model income was measured in 
%%	thousands dollars.}
%%	\label{fig:fig3.4}
%%\end{figure}
%We surprisingly observe that for given values of income and balance, a
%student is less likely to default than a non-student.
%\begin{figure}[H]
%	\begin{center}
%		\includegraphics[width=\textwidth]{./chap/1chap/3sec/3images/5confoundingPlot.png}
%	\end{center}
%	\caption{Confounding in the Default data. Left:Default rates
%	are shown for students (orange), and non-students (blue).\\
%	The solid lines show default rate as a function of balance,
%	while horizontal dashed line display the overall default rate
%	\\Right: Boxplots of balance for students and non-students.}
%	\label{fig:fig3.4}
%\end{figure}
%Indeed we observe that from the left-hand panel that the student 
%default rate is at or below of the non-student default rate for every
%value of balance.\\But the horizontal broken lines showing the default
%rate for students and non-students averaged over all values of income
%and balance suggest the opposite effect. Consequently there is a
%positive coefficient for student in the simple logistic regression
%output.\\The right-hand panel provides an explanation for this
%discrepancy, the variables student and balance are correlated. Indeed
%students are more likely to have large credit card balances, which, as
%we know from the left-hand panel tend to be associated to a high 
%default rate. Thus, even though an individual student with a given 
%credit card balance will tend to have a lower probability of default
%than a non-student with the same credit card balance. The fact that 
%students on the whole tend to have a higher credit card balances means
%that overall students tend to default at a higher rate than 
%non-students.
\paragraph{Logistic regression for $p$>2 response classes}
We use \emph{discriminant analysis}
\paragraph{Fitting Logistic Regression Models}
\tB{Logistic regression models are usually fit by maximum of likelihood}, using the conditional 
likelihood of $G$ given $X$. The log-likelihood for $N$ observations is: 
\tB{$$ l(\theta)=\su{{i=1}}{N}log\left( p_{g}(x_{i;\theta}) \right)$$} where $p_{g}(x_{i};\theta)=
\ProbC{X=x_{i}}{G=g;\theta}$.\\
It is convenient to code the two-class $g_{i}$ via a 0/1 response $y_{i}$, where $y_{i}=1$ when
$g_{i}=1$, and $y_{i}=0$ when $g_{2}=2$. Let $p_{1}(x;\theta)=p(x;\theta)$, and 
$p_{2}(x;\theta)=1-p(x;\theta)$. Then the log-likelihood can be written:
\begin{align*}
	l(\beta) =& \su{{i=1}}{N}\left\{ y_{i}\log\left(p(x_{i};\beta)\right)+ (1-y_{i})
	\log\left(1- p(x_{i};\beta)\right)\right\}\\
	=& \su{{i=1}}{N}\left\{ y_{i}\beta^{T}x_{i} - \log\left( 1+e^{\beta^{T}x_{i}} \right) \right\}
\end{align*}

To maximize the log-likelihood, we set its derivatives to zero:
$$ \dfrac{\partial l(\beta)}{\partial\beta} = 
\su{{i=1}}{N}x_{i}\left(y_{i}-p(x_{i};\beta)\right)=0$$ which are $p+1$ equations nonlinear in
$\beta$.
To solve the score, we use the \href{https://en.wikipedia.org/wiki/Newton\%27s_method_in_optimization}{Newton-Raphson} algorithm, which requires the second-derivative or 
Hessian matrix:
$$ \dfrac{\partial^{2}l(\beta)}{\partial\beta\partial\beta^{T}}=\su{{i=1}}{N}
x_{i}x_{i}^{T}p(x_{i};\beta)(1-p(x_{i};\beta))$$
The aim of the Newton-Raphson algorithm is to find the roots of a given real values
function, here $\dfrac{\partial l(\beta)}{\partial\beta}$
Starting with $\beta^{old}$, a single Newton update is : $$ \beta^{new} = \beta^{old} - \left(
\dfrac{\partial^{2}l(\beta)}{\partial\beta\partial\beta^{T}} \right)^{-1}\dfrac{\partial l(\beta)}{
\partial\beta}$$ where the derivatives are evaluated at$\beta^{old}$

Let $\bm{p}$ denote the vector of fitted probabilities with $i^{th}$ element $p(x_{i};\beta^{old})
\text{ and } \bm{W}$ a $N\times N$ diagonal matrix of weights with $i^{th}$ diagonal element
$p(x_{i};\beta^{old})(1-p(x_{i};\beta^{old}))$. Then we have:
\begin{align*}
	\dfrac{\partial l(\beta)}{\partial\beta} =& \bm{X}^{T}(\bm{y}-\bm{p})\\
	\dfrac{\partial^{2} l(\beta)}{\partial\beta\partial\beta^{T}} =&
	-\bm{X}^{T}(\bm{W}-\bm{X})\\
\end{align*}
The Newton step is thus:
\begin{align*}
	\beta^{new} =& \beta^{old} + \left( \bm{X}^{T}\bm{W}\bm{X} \right)^{-1}\bm{X}^{T}(\bm{y}-\bm{p})\\
	=& \left( \bm{X}^{T}\bm{W}\bm{X} \right)^{-1}\bm{X}^{T}\bm{W}\left( \bm{X}\beta^{old}+
	\bm{W}^{-1}(\bm{y}-\bm{p})\right)\\
	=& \left( \bm{X}^{T}\bm{W}\bm{X} \right)^{-1}\bm{X}^{T}\bm{W}\bm{z}
\end{align*}
In the second and third line we have re-expressed the Newton step as a weighted least squares step
with the response: $bm{z} = \bm{X}\beta^{old}+ \bm{W}^{-1}(\bm{y}-\bm{p})$ sometimes known as the
\textbf{adjusted response}.\\
This algorithm is referred to as \textbf{Iteratively Reweighted Least Squares (IRLS)} since each
iteration solves the weighted least squares problem: $$ \beta^{new}\leftarrow\min\limits_{\beta}
\left( \bm{z} - \bm{X}\beta \right)^{T}\bm{W}\left( \bm{z} - \bm{X}\beta \right)$$

It seems that $\beta=0$ is a good starting value for the iterative procedure although convergence
is never guaranteed. Typically the algorithm does converge, since the log-likelihood is concave,
but overshooting can occur.
\tB{Logistic regression models are used mostly as a data analysis and inference tool, where the
goal is to understand the role of the input variables in explaining outcome.}
\emph{Python code}
\begin{python}
import pandas as pd
import sklearn
from sklearn.linear_model import LogisticRegression

y, X = df.iloc[:, 0], df.iloc[:, 1:]
clf = LogisticRegression(random_state=0)
clf.fit(X, y)
print(clf.score(X, y))
\end{python}

\emph{R code}
\begin{rcode}[deletekeywords={model, df, data, family, binomial}]
model.logistic <- glm(y ~ ., data=df, family=binomial)
print(summary(model.logistic))
\end{rcode}
\paragraph{Quadratic Approximations and Inference}
The maximum-likelihood parameter estimates $\hat{\beta}$ satisfy a self-consistency relationship:
they are the coefficients of a weighted least squares fit, where the responses are:
$$ z_{i} = x_{i}^{T}\hat{\beta} + \frac{(y_{i}-\hat{p}_{i})}{\hat{p}_{i}(1_{i}-\hat{p}_{i})}$$

\begin{itemize}
	\item The weighted residual sum-of-squares is the familiar Pearson $\chi^{2}$ statistic:
		$$ \su{{i=1}}{N}\dfrac{(y_{i}-\hat{p}_{i})}{\hat{p}_{i}(1-\hat{p}_{i})}$$ a 
		quadratic approximation of the deviance.
	\item Asymptotic likelihood theory says that if the model is correct, then $\hat{\beta}$
		is consistent.
	\item A central limit theorem then shows that the distribution of $\hat{\beta}
		\hookrightarrow \mathcal{N}\left(\beta,\left(\bm{X}^{T}\bm{W}\bm{X}\right)^{-1} 
		\right)$
	\item Popular shortcuts are the Rao Score test which tests for inclusion of a term, and 
		the Wald test which can be used to test exclusion of a term.
\end{itemize}

\subsection{Linear discriminant analysis}
Why do we need another method that logistic regression in the case of
several response classes?
\begin{itemize}
	\item When classes are well-separated, the \sB{parameter 
		estimates for the logistic regression model are
		unstable}
	\item If \sB{$n$ is small and the distribution of the 
		predictors is approximately normal in each classe}, the
		\emph{linear discriminant model} is \tB{more stable 
		than the logistic.}
	\item Linear discriminant analysis is popular when we have 
		\sB{more than $2$ response classes}.
 \end{itemize}
\paragraph{Using Bayes Theorem for classification}
Let \tB{$\pi_{k}$ represents the prior probability} that a randomly
chosen observation comes from the $k^{th}$ class. Let $f_{k}\text{ such
that }\tB{f_{k}(x) \equiv\ProbC{X=x}{Y=k}}$ the density function of $X$
for an observation that comes from $k^{th}$ class:\\
\enc{$\ProbC{X=x}{Y=k}=\dfrac{\pi_{k}f_{k}(x)}{\su{{j=1}}{K}\pi_{j}f_{j}(x)}$}
\paragraph{Linear Discriminant Analysis for p=1}
Aims:
\begin{enumerate}
	\item \tB{To obtain an estimate for $f_{k}(x)$} that we can plug 
		into\\ $\ProbC{X=x}{Y=k}=\dfrac{\pi_{k}f_{k}(x)}{\su{{j=1}}{K}\pi_{j}f_{j}(x)}$
	\item \tB{To classify an observation to the class for which $p_{k}(x)$} is greatest
\end{enumerate}
Assumption:
$f_{k}(x)$ is \emph{Gaussian} that is $f_{k}(x)=\dfrac{1}{\sqrt{2\pi\sigma_{k}}}e^{-\frac{1}{2\sigma_{k}^{2}}(x-\mu_{k})^{2}}$\\
Knowing that $\sigma_{1}^{2}= \cdots = \sigma_{K}^{2}$ and taking the log of $p_{k}(x)=\dfrac{\pi_{k}\dfrac{1}{\sqrt{2\pi\sigma}}e^{-\frac{1}{2\sigma^{2}}(x-\mu_{k})^{2}}}{\su{{j=1}}{K}\pi_{j}\dfrac{1}{\sqrt{2\pi\sigma}}e^{-\frac{1}{2\sigma^{2}}(x-\mu_{j})^{2}}}$\\
It is equivalent to assigning the observation to the classer for which:
\\
\enc{$\delta_{k}(x)=x\frac{\mu_{k}}{\sigma^{2}}-\frac{\mu_{k}^{2}}{2\sigma^{2}}+ln(\pi_{k})$} is the largest.\\
The \emph{linear discriminant analysis} (LDA) method approximates the
Bayes classifier by plugging estimates for $\pi_{k},\mu_{k}\text{ and }\sigma^{2}$:\\
\encB{
$
\begin{cases}
	\hat{\pi}_{k} = \frac{n_{k}}{n}\\
	\hat{\mu}_{k} = \dfrac{1}{n_{k}}\su{{i:y_{i}=k}}{}x_{i}\\
	\hat{\sigma}^{2} = \dfrac{1}{n-K}\su{{k=1}}{K}\su{{i:y_{i}=k}}{}\left(x_{i}-\hat{\mu}_{k}\right)^{2}
\end{cases}$}

\paragraph{Linear Discriminant Analysis for p>1}
\subparagraph{Assumptions}
\begin{enumerate}
	\item Multivariate normality: independent variables are normal for each level of the grouping
		variable.
	\item Homoscedasticity: variances among group variables are the same accross levels of 
		predictors.
	\item Non-colinearity
	\item Observation are independent 
\end{enumerate}
\subparagraph{Formulas}
Now we assume that $X=\prth{X}{i}{n}\hookrightarrow\mathcal{N}(\mu,\Sigma)$
Here $\E{X}=\mu = 
\begin{pmatrix}
	\mu_{1}\\
	.\\
	.\\
	.\\
	\mu_{p}
\end{pmatrix}
\text{ and } \Sigma = 
\begin{pmatrix}
	Cov\left( X_{1},X_{1} \right) & \cdots & Cov\left( X_{1},X_{j}\right) &\cdots & Cov\left( X_{1},X_{n} \right)\\
	&\cdots& & &\\
	&\cdots& & &\\
	&\cdots& & &\\
	Cov\left( X_{i},X_{1} \right) & \cdots & Cov\left( X_{i},X_{j}\right) &\cdots & Cov\left( X_{i},X_{n} \right)\\
	&\cdots& & &\\
	&\cdots& & &\\
	&\cdots& & &\\
	Cov\left( X_{n},X_{1} \right) & \cdots & Cov\left( X_{n},X_{j}\right) &\cdots & Cov\left( X_{n},X_{n} \right)\\
\end{pmatrix}\\
\text{ and }
\tB{f(x)=\dfrac{1}{(2\pi)^{\frac{p}{2}}|\Sigma|^{\frac{1}{2}}}exp\left( \dfrac{1}{2}(x-\mu)^{T}\Sigma^{-1}(x-\mu) \right)}
	$\\
\text{ then }
$$
\tR{\delta_{k}(x) = x^{T}\Sigma^{-1}\mu_{k}-\dfrac{1}{2}\mu_{k}^{T}\Sigma^{-1}\mu_{k}+ln(\pi_{k})}
$$
\emph{Python code}
\begin{python}
import sklearn
from sklearn.discriminant_analysis import LinearDiscriminantAnalysis

clf = LinearDiscriminantAnalysis()
clf.fit(X, y)
print(clf.score(X, y))
\end{python}

\emph{R code}
\begin{rcode}[deletekeywords={model, lda, data, df}]
library(MASS)

model.lda <- lda(Direction ~ ., data=df)
model.lda
\end{rcode}

\paragraph{Quadratic Discriminant Analysis}
unlike LDA, \tB{QDA assumes that each class has its own covariance matrix}.
That is for an observation from the $k^{th}$ class 
\tR{$X\hookrightarrow N(\mu_{k},\Sigma^{k})$}
\begin{align*}
	\delta_{k}(x) &= -\dfrac{1}{2}(x-\mu_{k})^{T}\Sigma_{k}^{-1}(x-\mu_{k})-\dfrac{1}{2}\ln|\Sigma_{k}|+\ln(\pi_{k})\\
	&= -\dfrac{1}{2}x^{T}\Sigma_{k}^{-1}x+x^{T}\Sigma_{k}^{-1}\mu_{k}-\dfrac{1}{2}\mu_{k}^{T}\Sigma_{k}^{-1}\mu_{k}-\dfrac{1}{2}\ln|\Sigma_{k}|+\ln(\pi_{k})
\end{align*}
\emph{Python code}
\begin{python}
import sklearn
from sklearn.discriminant_analysis import QuadraticDiscriminantAnalysis

clf = QuadrasticDiscriminantAnalysis()
clf.fit(X, y)
print(clf.score(X, y))
\end{python}


\emph{R code}
\begin{rcode}[deletekeywords={model, lda, data, df}]
library(MASS)

model.lda <- lda(y ~ ., data=df)
model.lda
\end{rcode}
\paragraph{Regularized Discriminant Analysis}
The regularize covariance matrices have the form:
\begin{center}
\enc{$ \hat{\bm{\Sigma}}_{k}(\lambda) = \lambda\hat{\bm{\Sigma}}_{k} + (1-\lambda)\hat{\bm{\Sigma}}$}
\end{center}
where $\hat{\bm{\Sigma}}$ is the pooled covariance matrix as used in LDA.
Here \sB{$\lambda\in[0, 1]$ allows a continuum of models between LDA and QDA, $\lambda$ can be 
choosed by cross-validation}.\\
Similar modifications allow $\hat{\bm{\Sigma}}$ itself to be shrunk toward the scalar covariance:
$$ \hat{\bm{\Sigma}}(\gamma) = \gamma\hat{\bm{\Sigma}} + (1-\gamma)\hat{\sigma}^{2}\bm{I}$$
for $\gamma\in [0, 1]$
\emph{R code}
\begin{rcode}
library(klaR)
model.rda <- rda(y ~ ., data=df, gamma=0.05, lambda=0.2)
\end{rcode}
With rda we have :
$\hat{\Sigma}_{k}(\lambda, \gamma)=(1-\gamma)\hat{\Sigma}_{k}(\lambda)+\gamma\dfrac{1}{p}
tr\left(\hat{\Sigma}_{k}(\lambda)\right)I$
\begin{itemize}
	\item[$(\gamma=0,\lambda=0)$]: QDA - individual covariance for each group
	\item[$(\gamma=0,\lambda=1)$]: LDA - a common covariance matrix
	\item[$(\gamma=1,\lambda=0)$]: Conditional independent variables - similar to Naive Bayes,
		but variable variance whithin group (main diagonal elements) are all equal.
	\item[$(\gamma=1,\lambda=1)$]: Classification using euclidian distance - as in previous case,
		but variances are the same for all groups. Objects are assigned to group with nearest
		mean.
\end{itemize}

\paragraph{Computations for LDA}
The \sB{computations of LDA and QDA are simplified by diagonalizing $\hat{\bm{\Sigma}}$ or 
$\hat{\bm{\Sigma}}_{k}$ for the latter}, suppose we compute the eigendecomposistion for each
$\hat{\bm{\Sigma}}_{k} = \bm{U}_{k}\bm{D}_{k}\bm{U}_{k}^{T}\text{ where } \bm{U}_{k}\text{ is }
p\times p$ orthonormal and $\bm{D}_{k}$ a diagonal matrix of positive eigenvalues $d_{kl}$.
\begin{itemize}
	\item $(x-\hat{\mu}_{k})^{T}\hat{\bm{\Sigma}}^{-1}_{k}(x-\hat{\mu}_{k}) =
		\left[ \bm{U}_{k}^{T}(x-\hat{\mu}_{k}) \right]^{T}\bm{D}_{k}^{-1}
		\left[ \bm{U}_{k}^{T}(x-\hat{\mu}_{k}) \right]$
	\item $\log\left( \hat{\bm{\Sigma}}_{k} \right) = \su{l}{}\log(d_{kl})$
\end{itemize}
LDA classifier can be implemented by the following pair of steps:
\begin{itemize}
	\item \sB{\textbf{Sphere} the data with respect to the common covariance estimate $\hat{\bm{\Sigma}}$}:\\ $X^{*} \leftarrow \bm{D}^{\frac{1}{2}}\bm{U}^{T}X$ where $\hat{\bm{\Sigma}} = \bm{U}\bm{D}\bm{U}^{T}$. \sV{The common covariance estimate of $X^{*}$ will now be the identity.}
	\item \tB{Classify to the closet class centroid in the transformed space}, modulo the effect
		of the class prior probabilities $\pi_{k}$.
\end{itemize}

\paragraph{Reduced-Rank Discriminant Analysis}
The $K$ centroids in $p-\text{dimensional}$ input space lie in an affine subspace of dimension
$\leq K-1$, and \sB{if $p$ is much larger than $K$ this will be a considerable drop in dimension}.\\
Moreover in locating the closet centroid \sB{we can ignore distances orthogonal to this subspace, 
since they will contribute equally to each class. Thus we might just as well project the $X^{*}$
onto centroid-spanning subspace $H_{K-1}$} and make distance comparisons there.

Finding the sequences of optimal subspaces for LDA involves the following steps:
\begin{itemize}
	\item compute the \sB{$K\times p$ matrix of class centroids $\bm{M}$} and the common
		\sB{covariance matrix $\bm{W}$}
	\item compute \sB{$\bm{M}^{*} = \bm{M}\bm{W}^{-\frac{1}{2}}$} using the eigen-decomposition 
		of $\bm{W}$
	\item compute \sB{$\bm{B}^{*}$, the covariance matrix of $\bm{M}^{*}$} and its 
		eigen-decomposition \tB{$\bm{B}^{*} = \bm{V}^{*}\bm{D}_{B}(\bm{V}^{*})^{T}$}.
		The columns $v_{l}^{*}$ of $\bm{V}^{*}$ in sequence from first to last \sB{define
		the coordinates of the optimal subspaces}.
\end{itemize}
Combining all these operations \sB{the $l^{th}$ discriminant variable} is given by \tB{$Z_{l}=v_{l}^{
T}X$} with $v_{l}=\bm{W}^{-\frac{1}{2}}v_{l}^{*}$

Fisher arrived at this decomposition via a different route, without referring to Gaussian 
distribution at all.
\begin{center}
	Find the linear combination $Z=a^{T}X$ such that the between class variance is maximized
	relative to the within-class variance.
\end{center}

The \sB{between-class variance of $Z$ is a $a^{T}\bm{B}a$} and the \sB{within-class variance $a^{T}
\bm{W}a$}, where $\bm{W}$ is defined earlier, and $\bm{B}$ is the covariance matrix of the class centroid matrix
$\bm{M}$.\\
Fisher's problem therefore amounts to maximizing the Rayleigh quotient:
\begin{center}
\enc{$ \max\limits_{a}\dfrac{a^{T}\bm{B}a}{a^{T}\bm{W}a}$} 
\end{center}
or equivalently:
$$ \max\limits_{a}a^{T}\bm{B}a \text{ subject to } a^{T}\bm{W}a=1$$
This is a generalized eigenvalue problem, with $a$ given by the largest eigenvalue of $\bm{W}^{-1}
\bm{B}$. 
\begin{itemize}
	\item Gaussian classification with common covariances leads to linear decision boundaries
	\item One can confine the data to the subspace spanned by the centroids in the sphered
		space.
	\item This subspace can be further decomposed into successively optimal subspaces in 
		term of centroid separation. This decomposition is identical to the decomposition
		due to Fisher.
\end{itemize}


\subsection{A comparison of classification methods}
\paragraph{Logistic regression, LDA, QDA and KNN}
\subparagraph{Logistic regression VS LDA}
The only difference between the 2 approaches lies in fact that 
\sB{$\beta_{0}\text{ and }\beta_{1}$ are estimated using maximum 
likelihood, whereas $c_{0}\text{ and }c_{1}$ are computed using the
estimated mean and variance from a normal distribution}.\\
For $p=1~predictor$ we have :
\begin{align*}
	ln\left(\dfrac{p_{1}(x)}{1-p_{1}(x)}\right) &= c_{0} + c_{1}x\text{ LDA}\\
	ln\left(\dfrac{p_{1}(x)}{1-p_{1}(x)}\right) &= \beta_{0} + \beta_{1}x\text{ Logistic regression}\\
\end{align*}
\tB{It is generally felt that logistic regression is a safer, more robust bet than the LDA model,
relying on fewer assumptions.}

\subparagraph{KNN}
KNN is a completely non-parametric approach: no assumptions are made
about the shape of the decision boundary. Therefore, \sB{we can expect this
approach to dominate LDA and Logistic regression when th decision 
boundary is highly non-linear}.

\subparagraph{QDA}
\tB{It is a compromise between the non-parametric KNN method and the 
linear LDA and Logistic regression approaches.}

\subsection{Separating Hyperplanes}
Separating hyperplane classifiers are procedures that construct linear decision boundaries that
explicitly try to separate the data into different classes as well as possible.\\
\sB{Classifier such as $\left\{ x: \hat{\beta}_{0} + \hat{\beta}_{1}x_{1} + \hat{\beta}_{2}x_{2}
= 0 \right\}$ that compute a linear combination of the input features and return the sign}, were
called \tB{\textit{perceptrons}} in the engineering literature in the late $1950's$

Perceptrons set te foundations for the neural network models.\\
For a surface $L$ defined by the equation: $f(x)=\beta_{0}+\beta^{T}x=0$\\
Some properties:
\begin{enumerate}
	\item $\forall (x_{1}, x_{2})\in L^{2},~\beta^{T}(x_{1}-x_{2})=0$ and hence $\beta^{*}=
		\frac{\beta}{\norm{\beta}}$ is the vector normal to the surface $L$
	\item $\forall x_{0}\in L, \beta^{T}x_{0} = -\beta_{0}$
	\item The signed distance of any point $x$ to $L$ is given by:
		\begin{align*}
			{\beta^{*}}^{T}(x-x_{0}) =& \frac{1}{\norm{\beta}}\left( \beta^{T}x+
			\beta_{0} \right)\\
			=& \frac{1}{\norm{f'(x)}}{f(x)}
		\end{align*}
\end{enumerate}
Hence \sB{$f(x)$ is proportional to the signed distance from $x$ to the hyperplane defined by $f(x)=
0$}
\begin{figure}[H]
	\begin{center}
		\includegraphics[width=.4\textwidth]{./chap/1chap/3sec/8images/1_hyperplaneAffine.PNG}
	\end{center}
	\caption{The linear algebra of a hyperplane (affine set)}
	\label{fig:1_hyperplaneAffine}
\end{figure}

\paragraph{Rosenblatt's Perceptron Learning Algorithm}
\tB{It tries to find a separating hyperplane by minimizing the distance of misclassified points 
to the decision boundary.} The goal is to minimize: 
%$$ D(\beta, \beta_{0}) = \su{{i\in\mathcal{M}}}{}y_{i}\left(x_{i}^{T}\beta + \beta_{0}\right)$$
\begin{center}
\enc{$D(\beta, \beta_{0}) = \su{{i\in\mathcal{M}}}{}y_{i}\left(x_{i}^{T}\beta + \beta_{0}\right)$}
\end{center}
where $\mathcal{M}$ indexes the set of misclassified points.

\begin{align*}
	\dfrac{\partial D(\beta,\beta_{0})}{\partial\beta}=&-\su{{i\in\mathcal{M}}}{}y_{i}x_{i}\\
	\dfrac{\partial D(\beta,\beta_{0})}{\partial\beta_{0}}=&-\su{{i\in\mathcal{M}}}{}y_{i}\\
\end{align*}
The algorithm in fact uses \tB{\textit{stochastic gradient descent}} to minimize this piecewise 
linear criterion.\\
The misclassified observations are visited in some sequence, and parameters $\beta$ are updated
via:
$${{\beta}\choose{\beta_{0}}} \leftarrow {{\beta}\choose{\beta_{0}}}+\rho {{y_{i}x_{i}}\choose{y_{i}}}$$
$\rho$ is the learning rate.

\paragraph{Optimal Separating Hyperplanes}
The \textit{optimal separating hyperplanes} separates the 2 classes and maximizes the distance
to the closet point from  either class. Consider the optimization problem:
\begin{center}
$\max\limits_{\beta,\beta_{0},\norm{\beta}=1} M$\\
subject to $y_{i}\left(x_{i}^{T}\beta+\beta_{0}\right)\geq M, i\in\inter{1}{N}$
\end{center}
\sB{The set of conditions ensure that all the points are at least a signed distance $M$ from the
decision boundary defined by $\beta$ and $\beta_{0}$}

%\subsection{Lab: logistic regression LDA, QDA and KNN}
%\input{./chap/1chap/3sec/6_labLogisticRegressionLDA_QDA_KNN.tex}
%%\subsection{Exercises}
%%\input{./chap/1chap/3sec/7_exercices.tex}

\section{Resampling Methods}
\subsection{Cross-validation}
\input{./chap/1chap/4sec/1_crossValidation.tex}
\subsection{The Bootstrap}
\tR{It can be used to quantify the uncertainity associated with a given
estimator or statistical learning method.}\\

\begin{figure}[H]
	\begin{center}
		\includegraphics[width=.7\textwidth]{./chap/1chap/4sec/3_bootstrap.png}
	\end{center}
	\caption{Schematic of the bootstrap process}
	\label{fig:3_bootstrap}
\end{figure}

The basic idea is to randomly draw datasets with replacement from the training data, each sample
the same size as the original training set. This is done $B$ times, producing $B$ bootstrap datasets.
$S(\bm{Z})$ is any quantity computed from the data $\bm{Z}$. From the bootstrap sampling we can
estimate any aspect of the distribution of $S(\bm{Z})$, for example its variance:
\begin{center}
\enc{
$\hat{\V{S(\bm{Z})}}=\dfrac{1}{B-1}\su{{b=1}}{B}\left(S(\bm{Z}^{*b})-\overline{S}^{*}\right)^{2}$}
\end{center}
with $\overline{S}^{*}=\dfrac{1}{B}\su{{b=1}}{B}S(\bm{Z}^{*b})$
%\tB{We wish to minimize $\V{\alpha X+(1-\alpha)Y}$ (the risk)} one can
%show that the value that minimze the risk is : 
%\begin{center}
%\encV{$
%\alpha = \dfrac{\sigma_{Y}^{2}-\sigma_{XY}}{\sigma_{X}^{2}\sigma_{Y}^{2}-2\sigma_{XY}}
%$}
%\end{center}
%and $\sigma_{XY}=Cov(X,Y)$\\
%In reality, the quantities $\sigma_{X}^{2},\sigma_{Y}^{2}$ and $\sigma_{XY}$ are unknown.\\
%
%\tB{We can compute the standard error of these bootstrap estimates}
%using the formula:
%\begin{center}
%\encV{$
%SE_{B}\left( \hat{\alpha} \right)=
%\sqrt{\dfrac{1}{B-1}\su{{r=1}}{B}\left( \hat{\alpha}^{*r}-\dfrac{1}{B}\su{{r'=1}}{B}\hat{\alpha}^{*r'} \right)^{2}}
%$}
%\end{center}

%\subsection{lab: Cross-validation and the Bootsrap}
%\input{./chap/1chap/4sec/3_labCrossValidationAndTheBootsrap.tex}
%%\subsection{Exercises}
%%\begin{abstract}
<<>>=
print(x = 2^3)
@
\end{abstract}


\section{Linear model selection and regularization}
\subsection{Subset selection}
\paragraph{Best Subset Selection}
We fit a separate least squares regression for each possible 
combination of the $p$ predictors.
\subparagraph{Algorithm}
\begin{enumerate}
	\item Let \tB{$\mathcal{M}_{0}$ denote the \emph{null model},
		which contains no predictors}. This model model simply 
		predicts the sample mean for each observations.
	\item For $k\in\inter{1}{p}$:
		\begin{enumerate}[label=\alph*]
			\item Fit all $\binom{p}{k}$ models that 
				contain exactly $k$ predictors
			\item \tB{Pick the best among these $\binom{p}{
				k}$ models, and call it $\mathcal{M}_{
				k}$}\\ \sB{Here best is defined as
				having the small RSS or equivalently 
				largest $R^{2}$}
		\end{enumerate}
	\item \tB{Select a single best model from among 
		$\prtH{\mathcal{M}}{i}{0}{p}$} \sB{using
		cross-validation prediction error, $C_{p}(AIC), BIC$,
		or adjusted $R^{2}$}
\end{enumerate}
This task must be performed with care, because the RSS of these $p+1$
models decreases monotonically, and the $R^{2}$ increases monotonically
as the number of features included in the models increases.\\

This problem is that \tB{a low RSS or a high $R^{2}$ indicates a model
with a low \emph{training error}, whereas we wish to choose a model 
that has a model that has a low test error}.\\

\tB{Instead use RSS, we use the \emph{deviance}}, a measure that plays
the role of RSS for a broader class class of models. The \tR{deviance 
is negative 2 times the maximized log-likelihood}; the smaller the 
deviance the better the fit.

R code:
\begin{rcode}[deletekeywords={df, if, summary, min, max, which}]
library(dplyr)

df <- df %>%
          select_if(is.numeric) # selecting only numerical columns
names(df) <- gsub(names(df), '-', '_') # replace all '-' in headers
regfit.full <- regsubsets(y ~ ., df, nvmax=dim(df)[2]-1)  # best subset 
# -1 because response y is excluded from data frame
reg.summary <- summary(regfit.full) 

# Plotting several scores
par(mfrow=c(2, 2)) # making a 2 X 2 table of graphs
plot(reg.summary$rss, xlab="Number of Variables", ylab="RSS", type="l")
min.rss <- which.min(reg.summary$rss)
points(min.rss, reg.summary$rss[min.rss], col='red')
plot(reg.summary$adjr2, xlab="Number of Variables", ylab="Adjusted RSq", type="l")
max.adjr2 <- which.max(reg.summary$adjr2)
points(max.adjr2, reg.summary$adjr2[max.adjr2], col='red')
plot(reg.summary$cp, xlab="Number of Variables", ylab="Cp", type="l")
min.cp <- which.min(reg.summary$cp)
points(min.cp, reg.summary$cp[min.cp], col="red")
plot(reg.summary$bic, xlab="Number of Variables", ylab="BIC", type="l")
min.bic <- which.min(reg.summary$bic)
points(min.bic, reg.summary$bic[min.bic], col="red") #$

df_mod <- df[, reg.summary$which[max.adjr2, -1]] # -1 to exclude intercept
\end{rcode}

\paragraph{Stepwise Selection}
\subparagraph{Forward Stepwise Selection}
Algorithm
\begin{enumerate}
	\item Let \tB{$\mathcal{M}_{0}$ denote the \emph{null} model,
		which contains no predictors}.
	\item For $k\in\inter{0}{p-1}$
		\begin{enumerate}[label=\alph*]
			\item \tB{Consider all $p-k$ models} that
				argument the predictors in
				$\mathcal{M}_{k}$ with one additional
				predictor.
			\item \tB{Choose the \emph{best} among these 
				$p-k$ models, and call it 
				$\mathcal{M}_{k+1}$.}\\
				Here \emph{best} is defined as having
				smallest RSS or highest $R^{2}$
			\end{enumerate}
	\item \tB{Select a single best model from among
		$\prth{\mathcal{M}}{i}{0}{p}$} using cross-validated
		prediction error, $C_{p}(AIC), BIC$, or adjusted
		$R^{2}$
\end{enumerate}
\sB{Unlike best subset selection, which involved fitting $2^{p}$ 
models, forward stepwise selection involves fitting one null model,
along with $p-k$ models in the $kth$ iteration, for $k\in\inter{0}{
p-1}$.}\\
This amounts to a total of $1+\su{{k=0}}{p-1}(p-k)=1+\frac{p(p+1)}{2}$
models
Python code:
\begin{python}
import mlxtend # library for model selection
from mlxtend.feature_selection import SequentialFeatureSelector
import numpy as np
import matplotlib.pyplot as plt
import sklearn
import sklearn import linear_model

y, X = df.iloc[:, 0], df.iloc[:, 1:]
reg = linear_model.LinearRegression()
sfsl = SFS(reg, # sickit-learn classifier/regressor
    k_features=X.shape[1], # Number of features to select
    forward=True, # Forward if True otherwise False
    floating=False, # Adds a conditional exclusion/inclusion if True
    verbose=2, # Level of verbosity to use in logging [0,2]
    scoring='r2', # 'accuracy' for classifiers, 'r2' for regressors
    cv=5 # Stratified k-fold for classifier
    )
sfsl.model = sfsl.fit(X, y) # processing selection model
dict_sfsl = sfs1_model.subsets_ # dictionary containing scores depending on features

plt.figure()
x = np.array(list(dict_sfsl))
y = np.array([dict_sfsl[k]['avg_score'] for k in list(dict_sfsl)])
plt.plot(x, y)
ind_min_y = np.where(y==y.max())[0][0]
plt.plot(ind_min_y, dict_sfsl[ind_min_y]['avg_score'], 'ro')
plt.xlabel('Number of features')
plt.ylabel(r'$R^{2}$')
plt.titel('Score Evolution')
plt.show()

df_mod = df.loc[:, dict_sfsl[ind_min_y]['feature_names']].copy()
\end{python}

R code:
\begin{rcode}[deletekeywords={df, summary}]
regfit.fwd = regsubsets(y ~ ., data=df, nvmax=dim(df)[2], method='forward')
\end{rcode}

\subparagraph{Backward Stepwise Selection}
Algorithm
\begin{enumerate}
	\item Let \tB{$\mathcal{M}_{p}$ denote the \emph{full} model,
		which contains all $p$ predictors}.
	\item For $k\in\inter{p-1}{0}$
		\begin{enumerate}[label=\alph*]
			\item \tB{Consider all $k$ models that contain
				all but one of the predictors in 
				$\mathcal{M}_{k}$} (for a total of)
				$k-1$ predictors.
			\item \tB{Choose the \emph{best} among these 
				$k$ models, and call it 
				$\mathcal{M}_{k-1}$.}\\
				Here \emph{best} is defined as having
				smallest RSS or highest $R^{2}$
			\end{enumerate}
	\item \tB{Select a single best model from among
		$\prtH{\mathcal{M}}{i}{0}{p}$} using cross-validated
		prediction error, $C_{p}(AIC), BIC$, or adjusted
		$R^{2}$
\end{enumerate}

\subparagraph{Hybrid approach}
Such an approach attempts to more closely mimic best subset selection
while retaining the computational advantages of forward and backward
stepwise selection.

\paragraph{Optimism of the Training Error Rate}
Given a training set\\ $\mathcal{T}=\{(x_{i},y_{i}:i\in\inter{1}{N})\}$, the generalization error of 
a model $\hat{f}$ is:
$Err_{\mathcal{T}}=\mathbb{E}_{X^{0},Y^{0}}\left(L(Y^{0},\hat{f}(X^{0}))|\mathcal{T}\right)$\\
Typically the training error:
\begin{center}
	\tB{$\overline{err}=\dfrac{1}{n}\su{{i=1}}{N}L\left(y_{i},\hat{f}(x_{i})\right)$}
\end{center}
The \emph{in-sample} error:
$$ Err_{in} = \dfrac{1}{N}\su{{i=1}}{N}\mathbb{E}_{Y^{0}}\left(L\left(Y_{i}^{0},\hat{f}(x_{i})
\right)|\mathcal{T}\right)$$
The $Y^{0}$ notation indicates that we observe $N$ new response values at each of the training
point $x_{i}\text{ with }i\in\inter{1}{N}$. We define the \emph{\textbf{optimism}} as :
$$
\begin{cases}
	op \equiv Err_{in}-\overline{err}\\
	\omega \equiv E_{y}(op)
\end{cases}
$$
We can usually estimate only the expected error $\omega$ rather than \emph{op}, in the same way
that we can estimate the expected error $Err$ rather than the conditional error $Err_{\mathcal{T}
}$\\
For squared error, one can show quite generally that:
$$ \omega=\dfrac{2}{N}\su{{i=1}}{N}Cov(\hat{y}_{i},y_{i})$$
In summary we have the important relation:
\begin{center}
\enc{
	$ \mathbb{E}_{\bm{y}}\left(Err_{in}\right) = \mathbb{E}_{\bm{y}}(\overline{err}) +
	\dfrac{2}{N}\su{{i=1}}{N}Cov(\hat{y}_{i},y_{i})$}
\end{center}


\paragraph{The effective number of parameters}
Let $\bm{S}$ be the hat matrix for linear fitting methods not for only linear regression ($\bm{H}$) 
The effective number of parameters is 
$$ df(\bm{S}) = trace(\bm{S})$$
\sB{If $\bm{y}$ arises from an additive-error model $Y=f(X)+\epsilon$ with $\V{\epsilon}=\sigma^{2}
$}, then one can show that $\su{{i=1}}{N}Cov(\hat{y}_{i},y_{i})=trace(\bm{S})\sigma^{2}$ which 
motivates the more general definition:
\begin{center}
\encN{
$ df(\hat{\bm{y}})=\dfrac{\su{{i=1}}{N}Cov(\bm{y}_{i},y_{i})}{\sigma^{2}}$}
\end{center}

\paragraph{Choosing the Optimal Model}
Now rather to choose the model with the smallest RSS and the largest
$R^{2}$, since these quantities are related to training error, we 
wish to choose the model with a low test error.\\
Then we can
\begin{enumerate}
	\item \tB{indirectly estimate test error}, by making an 
		\emph{adjustment} to the training error to account
		for the bias due to overfitting.
	\item \tB{directly estimate test error}, using either a 
		validation set approach or a cross-validation approach.
\end{enumerate}

\subparagraph{$C_{p}$, AIC, BIC, and adjusted $R^{2}$}
\textbf{$C_{p}$}\\
For a fitted least squares model containing $d$ predictors,
the \tB{$C_{p}$ estimate of test MSE} is computed using the equation
\begin{center}
\encV{$
\begin{cases}
C_{p} = \dfrac{1}{n}\left( RSS+2d\hat{\sigma}^{2} \right)\\
C_{p} = \overline{err} + 2\dfrac{d}{N}\hat{\sigma}^{2}
\end{cases}
$}
\end{center}
where $\hat{\sigma}^{2}$ is an estimate of the variance of the error
$\epsilon$ associated with each response measurement.\\
Using this criterion we adjust the training error by a factor proportional to the number of basis
function used.

\textbf{AIC}\\
\sB{It is defined for a large class of models fit by maximum likelihood.}
The \tB{\emph{Akaike information criterion}} is a similar but more generally applicable but more
applicable estimate of $Err_{in}$ when a log-likelihood loss function is used. It relies on
a relationship:
$$ -2\E{\log\left(\mathbb{P}_{\hat{\theta}}(Y)\right)}\approx -\dfrac{2}{N}\E{loglik}+2\dfrac{d}{N}
$$ 
Here $\mathbb{P}_{\hat{\theta}}(Y)$ is a family of densities for $Y$ (containing the ``true''
density), $\hat{\theta}$ is the maximum-likelihood estimate of $\theta$, and ``logik''is the 
maximized log-likelihood:
$$ loglik = \su{{i=1}}{N}\log\left(\mathbb{P}_{\hat{\theta}}(y_{i})\right)$$
Given a set of $f_{\alpha}(x)$ indexed by a tuning parameter $\alpha$, denote by $\overline{err}(
\alpha)$ and $d(\alpha)$ the training error and number of parameters for each model.
\begin{center}
\encV{$
\begin{cases}
AIC = \dfrac{1}{n\hat{\sigma}^{2}}\left( RSS+2d\hat{\sigma}^{2} \right)\\
AIC = \overline{err}(\alpha) + 2\dfrac{d(\alpha)}{N}\hat{\sigma}_{\epsilon}^{2}
\end{cases}
$}
\end{center}

\textbf{BIC}\\
\tB{It is derived from a Bayesian point of view}, like AIC it is applicable in setting where the 
fitting is carried out by maximization of a log-likelihood:
\begin{center}
\encV{$
\begin{cases}
BIC = \dfrac{1}{n\hat{\sigma}^{2}}\left( RSS+\ln(n)d\hat{\sigma}^{2} \right)\\
BIC = \dfrac{N}{\sigma^{2}}\left[\overline{err}+\log(N)\dfrac{d}{N}\sigma^{2}\right]
\end{cases}
$}
\end{center}
Therefore BIC is proportional to AIC, with the factor $2$ replaced by $\log(N)$. Assuming 
$N>e^{2}\approx 7.4$, BIC tends to penalize complex models more heavily, giving preference to 
simpler models in selection.

\textbf{Adjusted $R^{2}$}\\
For a least squares model with $d$ variables it is calculated as:
\begin{center}
\encV{$
\text{Adjusted }R^{2}=1-\dfrac{\frac{RSS}{n-d-1}}{\frac{TSS}{n-1}}
$}
\end{center}

Given a family of models, including the true model, the \sB{probability that BIC will select the 
correct model approaches one as the sample size $N\rightarrow\infty$}. This is not the case for
\sB{AIC, which tends to choose models which are too complex as $N\rightarrow\infty$}.\\
On the other hand, \tB{for finite samples BIC often chooses models that are too simple}, because of
its penalty on complexity.

\paragraph{Vapnik-Chervonenkis Dimension}
Altough the effective number of parameters is useful for some non-linear models, it is not fully
general. The \emph{Vapnik-Chervonenkis} theory provides such a general measure of complexity and
gives associated bounds on the optimism.\\
\tB{The \emph{Vapnik-Chervonenkis} dimension is a way of measuring the complexity of a class of 
function by assessing how wiggly its members can be.}\\
If we fit $N$ training points using a class of function $\left\{f(x,\alpha)\right\}$ (class of 
functions indexed by a parameter vector $\alpha$, with $x\in\mathbb{R}^{p}$) having VC dimensions
(that is defined as the largest number of points that can be shattered by members of $\left\{
f(x,\alpha)\right\}$) noted $h$. \sB{Then with probability at least $1-\eta$ over training sets}:
$$
\begin{cases}
	Err_{\mathcal{T}} \leq \overline{err}+\dfrac{\epsilon}{2}\left(1+\sqrt{1+\dfrac{4\overline{
	err}}{\epsilon}}\right)\text{ (binary classification)}\\
	Err_{\mathcal{T}} \leq \dfrac{\overline{err}}{\left(1-c\sqrt{\epsilon}\right)_{+}} \text{ (regression)}
\end{cases}
$$
where $\epsilon=a_{1}\dfrac{h\left[\log\left(a_{2}\frac{N}{h}\right)+1\right]-\log\left(\frac{\eta}{4}\right)}{N}$ and $(a_{1},a_{2})\in]0,4]\times]0,2]$.\\
Cherkassky and Mulier recommend the value of $c=1$,
\begin{itemize}
	\item For \tB{regression} they suggest $(a_{1},a_{2})=(1,1)$
	\item For \tB{classification} they suggest $(a_{1},a_{2})=(4,2)$
\end{itemize}

\subparagraph{Validation and Cross-Validation}
We can compute the validation set error or the cross-validation error 
for each model under consideration, and then \tB{select the model for
which the resulting estimated test error is smallest}.

\paragraph{Bootstrap Methods}
\tB{The bootstrap is a general tool for assessing statistical accuracy.}\\
We denote the training set by $\bm{Z}=(z_{1},\cdots,z_{N})$ where $z_{i}=(x_{i},y_{i})$. The basic
idea is to randomly draw datasets with replacement from the training data, each sample the same size
as the original training set. This done $B$ times, producing $B$ bootstrap datasets.

The \sB{leave-one-out boostrap estimate of prediction error} is defined by: 
$$ \hat{Err}=\dfrac{1}{N}\su{{i=1}}{N}\dfrac{1}{|C^{-i}|}\su{{b\in C^{-i}}}{}L\left(y_{i},
\hat{f}^{*b}(x_{i})\right)$$
Here $C^{-i}$ is the set of indices of the bootstrap samples b that do not contain observation $i$,
and $|C^{-i}|$ is the number of such samples.
First we define \tB{$\gamma$ to be the \emph{no-information rate} this is the error rate of our 
prediction rule if the inputs and class labels were independent}.
$$\hat{\gamma}=\dfrac{1}{N^{2}}\su{{i=1}}{N}\su{{j=1}}{N}L\left(y_{i},\hat{f}(x_{j})\right)$$
The \tR{\emph{relative overfitting rate}} is defined to be:
\begin{center}
	\enc{$\hat{R}=\dfrac{\hat{Err}-\overline{err}}{\hat{\gamma}-\overline{err}}$}
\end{center}
a quantity that ranges from 0 if there is no overfitting to 1 if the overfitting equals the
no-information value ($\hat{\gamma}-\overline{err}$).\\

We conclude that estimation of test error for a particular training set is not easy in general,
given just the data from that same training set. Instead \tB{cross-validation and related methods
may provide reasonable estimate of the \emph{expected} error Err}.

\subsection{Shrinkage methods}
\input{./chap/1chap/5sec/2_shrinkageMethods.tex}
\subsection{Dimension reduction methods}
\input{./chap/1chap/5sec/3_dimensionReductionMethods.tex}
\subsection{Multiple Outcome Shrinkage and Selection}
Canonical correlation analysis (CCA) is \tB{a data reduction technique developped for the
multiple output case}. Similar to PCA, CCA finds a sequence of uncorrelated linear
combinations $\bm{X}v_{m} \text{for} m\in\inter{1}{M}$ of the $\bm{x}_{j}$ and a 
corresponding sequence of uncorrelated linear combinations $\bm{Y}u_{m}$ of the respons
$\bm{y}_{k}$ such that the correlations :
$$ Corr^{2}\left( \bm{Y}u_{m}, \bm{X}v_{m} \right)$$ are succesively maximized.
\subparagraph{Reduced-rank regression}
Given an error covariance $Cov(\epsilon)=\Sigma$, we solve the following:
$$ \hat{\bm{B}}^{rr}(m) =\min\limits_{{rank(\bm{B})=m}}\su{{i=1}}{N}\left( y_{i}-
\bm{B}^{T}x_{i}\right)^{T}\Sigma^{-1}\left( y_{i} - \bm{B}^{T}x_{i}\right) $$
With $\Sigma$ replaced by the estimate $\dfrac{\bm{Y}^{T}\bm{Y}}{N}$ that the solution
is given by a CCA of $\bm{Y}$ and $\bm{X}$:
$$ \hat{\bm{B}}^{rr}(m) = \hat{\bm{B}}\bm{U}_{m}\bm{U}_{m}^{-}$$ where $\bm{U}_{m}$ is
the $K\times m$ sub-matrix of $\bm{U}$ consisting of the first $m$ columns and $\bm{U}$
is the $K\times M$ matrix of left canonical vectors $u_{1}, u_{2}\cdots u_{M}$.
$\bm{U}^{-}_{m}$

\subsection{More on the Lasso and Related Path algorithms}
\paragraph{Incremental Forward Stagewise Regression}
\subparagraph{Incremental Forward Stagewise Regression - $FS_{\epsilon}$}
\begin{enumerate}
	\item Start with the residual $\bm{r}$ equal to $\bm{y}$ and $\beta_{1}, \beta_{2}, \cdots,
		\beta_{p}$ = 0. All the predictors are standardized to have mean zero and unit norm.
	\item Find the predictor \sB{$x_{j}$ most correlated with $\bm{r}$}
	\item Update \sB{$\beta_{j} \leftarrow \beta_{j} + \delta_{j}$}, where \tB{$\delta_{j} = 
		\epsilon\times \text{sign}[\sP{\bm{x}_{j}}{\bm{r}}]$} and $\epsilon>0$ is a small
		step size, and set \sB{$\bm{r} \leftarrow \bm{r} - \delta_{j}\bm{x}_{j}$}
	\item Repeat steps 2 and 3 many times, until the residuals are uncorrelated with all the
		predictors.
\end{enumerate}

The linear regression version of the forward-stagewise generates a coefficient profile by
repeatedly updating (by a small amount $\epsilon$ the coefficient of the variable most correlated
with the current residual.\\
If $\delta_{j}= \sP{\bm{x}_{j}}{\bm{r}}$ (the least squares coefficient of the residual on $j^{th}$
predictor), then this is exactly the usual forward stagewise procedure (FS).\\
Letting $\epsilon \rightarrow 0$ gives the right panel, which in this case is identical to the lasso
path. We call this limiting procedure infinitesimal forward stagewise regression or $FS_{0}$
\begin{figure}[H]
	\begin{center}
		\includegraphics[width=.7\textwidth]{chap/1chap/5sec/images/7_coefficientsFS.PNG}
	\end{center}
	\caption{The left pannel shows incremental \tB{forward stagewise regression with step size
	$\epsilon=0.01$}. The right panel shows the infinitesimal version \tB{$FS_{0}$ obtained 
	letting $\epsilon \rightarrow 0$}}
	\label{fig:7_coefficientsFS}
\end{figure}

\subparagraph{Least Angle Regression $FS_{0}$}
\begin{itemize}
	\item[4] Find the new direction by solving the constrained least squares problem:
		\begin{center}
		\encN{
		$\min\limits_{b}\norm{\bm{r}-\bm{X}_{\mathcal{A}}b}^{2}_{2} \text{subject to }
		b_{j}s_{j}\geq0, j\in\mathcal{A}$}
		\end{center}
		where $s_{j}$ is the sign of $\sP{\bm{x}_{j}}{
		\bm{r}}$
\end{itemize}
While the Lasso makes optimal progress in terms of reducing the residual sum-of-squares per unit
increase in $L_{1}$-norm of the coefficient vector $\beta$. $FS_{0}$ is optimal per unit increase
in $L_{1}$ arc-length traveled along the coefficient path. Hence \sB{its coefficient path is 
discouraged from changing direction too often}.

\paragraph{Piecewise-Linear Path Algorithms}
\begin{center}
\encB{$\hat{\bm{\beta}}(\lambda) = \min\limits_{\beta}\left( \bm{R}(\beta) + \lambda\bm{J}(\beta) 
\right) \text{with} \bm{R}(\beta) = \su{{i=1}}{N}\bm{L}\left( y_{i}, \beta_{0}+\su{{j=1}}{p}x_{ij}
\beta_{j}\right)$}
\end{center}
with both the loss function $\bm{L}$ and the \sB{penalty function $\bm{J}$} are convex.\\
Then the following are sufficient conditions for the solution path $\beta(\lambda)$ to be piecewise
linear.
\begin{enumerate}
	\item $R$ is quadratic or piecewise-quadratic as a function of $\beta$
	\item $J$ is a piecewise linear in $\beta$
\end{enumerate}

\paragraph{The Dantzig Selector}
\begin{center}
\encB{$\min\limits_{\beta}\norm{\bm{X}^{T}\left( \bm{y}-\bm{X}\beta \right)}_{\infty} \text{subject
to } \norm{\beta}_{1}$}
\end{center}
Here $\norm{.}_{\infty}$ denotes the $L_{\infty}$ norm, the maximum absolute 
value of the components of the vector.

\paragraph{The Grouped Lasso}
Suppose that the $p$ predictors are divised in $L$ groups, with \sB{$p_{l}$ the number in group $l$}.
\\
\sB{$\bm{X}_{l}$ represents the predictors corresponding to the $l^{th}$ group}, with corresponding 
coefficient vector $\beta_{l}$.\\
The grouped-lasso minimizes the convex criterion
\begin{center}
\encB{$ \min\limits_{\beta\in\mathbb{R}^{p}}\left( \norm{\bm{y}-\beta_{0}\bm{1}-\su{{l=1}}{L}\bm{X}_{
l}\beta_{l}}^{2}_{2} + \lambda\su{{l=1}}{L}\sqrt{p_{l}}\norm{\beta_{l}}_{2} \right)$}
\end{center}
where \sB{the $\sqrt{p_{l}}$ terms accounts for the varying group sizes}, $\norm{.}_{2}$ is the 
Euclidean norm (not squared). Since the Euclidean norm of a vector $\beta_{l}$ is zero only if all 
its components are zero, this procedure encourages sparsity at both the group and individual levels.

\subsection{Consideration in high dimensions}
\paragraph{High-Dimensional Data}
Data sets containing more features than observations are often referred
to as \emph{high-dimensional}. \sB{Classical approaches such as least 
squares linear regression are not appropriate in this setting.}

%\paragraph{What goes wrong in high dimensions?}
%Regression and classification when $p>n$, we begin by examining what
%can go wring if we apply a statistical technique not intended for the
%high-dimensional setting.

\paragraph{Regression in High Dimensions}
\sB{Ridge regression, the lasso and principal components regression, are
particularly useful} for performing regression in the high-dimensional
setting. Essentially these approaches avoid overfitting by using a 
less flexible fitting approach than least squares.

%\paragraph{Interpreting Results in High Dimensions}
%At most, we can hope to assign large regression coefficients to 
%variables that are correlated with the variables that truly are 
%predictive of the outcome.

%\subsection{Lab1: subset selection methods}
%\input{./chap/1chap/5sec/5_lab1SubsetSelectionMethods.tex}
%\subsection{Lab2: ridge regression and the Lasso}
%\input{./chap/1chap/5sec/6_lab2RidgeRegressionAndTheLasso.tex}
%%\subsection{lab3: PCR and PLS regression}
%%\input{./chap/1chap/5sec/7_lab3PCRandPLSregression.tex}
%%%\subsection{Exercises}
%%%\input{./chap/1chap/5sec/8_exercises.tex}

\section{Moving beyond linearity}
\begin{itemize}
	\item
 \end{itemize}
\subsection{Polynomial Regression}
\paragraph{Definition}
Historically when the relationship between the response and the
predictors is non-linear we used to replace the standard linear model
$$
y_{i}=\beta_{0}+\beta_{1}x_{i}+\epsilon_{i}
$$
with a polynomial function:
\begin{center}
\enc{$
y_{i}=\su{{r=0}}{d}\beta_{r}x_{i}^{r}
$}
\end{center}
Generally speaking, \tB{it is unusual to use $d$ greater than 3 or 4}.

\begin{python}
import pandas as pd
import sklearn
from sklearn.preprocessing import PolynomialFeatures
from sklearn.linear_model import LinearRegression
from sklearn.pipeline import Pipeline

df = pd.read_csv('myFile.csv', sep=';')
y, X = df.iloc[:, 0].values, df.iloc[:, 1].values.reshape(-1, 1)

model = Pipeline([
  ('poly', PolynomialFeatures(degree=3)),
  ('linear', LinearRegression(fit_intercept=False))
  ])
model = model.fit(X, y)
print(model.score(X, y))
\end{python}

\subsection{Step functions}
\paragraph{Reasoning}
We create $\prth{c}{i}{K}$ in the range of $X$ and then construct
$K+1$ new variables such as:
$
\begin{cases}
	C_{0}(X)=I(X<c_{1})\\
	\forall i\in\inter{2}{K-1} C_{i}(X)=I(c_{i}\leq X\leq c_{i+1})\\
	C_{K}(X)=I(c_{K}\leq X)
\end{cases}
$
where $I(.)$ is an \emph{indicator function} that returns a 1 if the
condition is \emph{True} an returns 0 otherwise.
$$
y_{i}=\beta_{0}+\su{{r=1}}{K}\beta_{r}C_{r}(x_{i})
$$

\subsection{Basis functions}
\input{./chap/1chap/6sec/3_basisFunctions.tex}
\subsection{Regression splines}
\paragraph{Piecewise Polynomials}
\tB{It involves fitting separate low-degree polynomials over different 
regions of $X$.}\\
The points where the coefficients change are called \tR{\emph{knots}}.
It is a function $f(X)$ that is obtained by dividing the domain of $X$ into contiguous intervals,
and representing $f$ by a separate polynomial in each interval.
More generally, an order$-M$ spline with knots $\zeta_{j}, j\in\inter{1}{K}$ is a 
piecewise-polynomial of order $M$, and has continuous derivatives up to order $M-2$.
Likewise the general form for the truncated-power basis set would be:
\begin{align*}
h_{j}(X) &= X^{j-1}, &j\in\inter{1}{M}\\
h_{M+j} &=\left(X-\zeta_{l}\right)_{+}^{M-1}, &l\in\inter{1}{K}
\end{align*}

%\paragraph{Constraints and Splines}
%We can fit a piecewise polynomial under the constraint that the fitted
%curve must be continuous.

\paragraph{The spline Basis Representation}
A cubic spline with $K$ knots can be modeled as:
\begin{center}
\enc{$
y_{i}=\beta_{0}+\su{{r=1}}{K+3}\beta_{r}b_{r}(x_{i})
$}
\end{center}
for an appropriate choice of \emph{basis functions} $\prth{b}{i}{K+3}$\\
The most direct way to represent cubic spline is \tB{to start off with
a basis for a cubic polynomial -namely, $x, x^{2}, x^{3}$ -and then add
one \emph{truncated power basis} function per knot}.
A truncated power basis function, for a cubic polynomial, is defined as:
\begin{center}
\enc{$
h(x,\zeta) = (x-\zeta)_{+}^{3}=
\left\{
\begin{array}{ll}
	(x-\zeta)^{3}&\mbox{if }x>\zeta\\
	0&\mbox{otherwise}
\end{array}
\right.
$}
\end{center}
where $\zeta$ is the knot.\\
In order to fit a cubic spline to a data set with $K$ knots, we perform
least squares regression with an intercept and $3+K$ predictors of the
form $X, X^{2}, X^{3}, h\left(X,\zeta_{1}\right), h\left(X,\zeta_{2}\right),,..,h\left(X,\zeta_{K}\right)$ where $\prth{\zeta}{i}{K}$ are the
knots. This amounts to estimating a total of $K+4$ regression 
coefficients.\\
Cubic splines are popular because most human eyes cannot detect 
discontinuity at the knots.

\begin{figure}[H]
	\begin{center}
		\includegraphics[width=.7\textwidth]{./chap/1chap/6sec/images/1splines.png}
	\end{center}
	\caption{A cubic spline and a natural cubic spline, with 3 
	knots, fit to a subset of the Wage data, confidence interval as
	dashed lines.}
	\label{fig:6.1splines}
\end{figure}
A natural spline is a regression spline with additional boundary 
constraints: the function is required to be linear at the boundary (in
the region where $X$ is smaller than the smallest knot, or larger than
the largest knot).\\
\tB{This frees up $4$ degrees of freedom ($2$ constraints each in both boundary regions), which
can be spent more profitably by sprinkling more knots in the interior region. }
A natural cubic spline with $K$ knots is represented by $K$ basis functions.
\begin{center}
	$\begin{cases}
		N_{1}(X) = 1\\
		N_{2}(X) = X
	\end{cases}
	,~ N_{k+2}(X) = d_{k}(X)-d_{K-1}(X)$
\end{center}
where $d_{k}(X) = \dfrac{\left(X-\zeta_{k}\right)_{+}^{3} - \left(X-\zeta_{K}\right)_{+}^{3}}{
\zeta_{K}-\zeta_{k}}$

\paragraph{Choosing the Number and Locations of the Knots}
One option is \tB{to place more knots in places where we feel the 
function might vary most rapidly}, and to place fewer knots where it 
seems more stable.\\
\sR{For the number of knots we can use cross-validation.}

\begin{python}
import pandas 
import statsmodels.api as sm
import sklearn
from sklearn.model_selection import KFold
import patsy
from patsy import dmatrix

# Choosing the good number of knots
y, x = df.iloc[:, 0], df.iloc[:, 1]
kf = KFold(n_splits=5)
n = 52
cv_set = {str(i):0 for i in range(1, n)}
for i in range(1, n):
    n_knots = i
    knot_list = np.quantile(x, np.linspace(0, 1, n_knots + 2))[1:-1]
    mse_list = []
    for train, test in kf.split(x):
        x_natural = dmatrix('cr(x, knots=knot_list)', {'x':x[train]})
        fit_natural = sm.GLM(y[train], x_natural).fit()
        yhat = fit_natural.predict(dmatrix('cr(x, knots=knot_list)', {'x':x[test]}))
        mse = ((yhat-y[test])**2).mean()
        # mse_list.append(mse)
        mse_list.append(mse)
        # print(mse)
    cv_set[str(i)] = np.array(mse_list).mean()
cv_list = list({k:v for k,v in sorted(cv_set.items(), key=lambda item: item[1])})


n_knots = int(cv_list[0])
knot_list = np.quantile(x, np.linspace(0, 1, n_knots + 2))[1:-1]
# Natural
x_natural = dmatrix('cr(x, knots = knot_list)', {'x':x})
fit_natural = sm.GLM(y, x_natural).fit()

# Create spline
xp = np.linspace(x.min(), x.max(), 50)
line_natural = fit_natural.predict(dmatrix('cr(xp,knots=knot_list)',
                                           {'xp':xp}))

plt.figure()
plt.scatter(df.iloc[:, 0], df.iloc[:, 1])
plt.plot(xp, line_natural, color='red', label="Natural Spline Regression")
for k in knot_list:
    plt.axvline(k, color='gray', ls='--')
plt.legend()
plt.show()
\end{python}
\paragraph{Comparison to Polynomial Regression}
Regression splines give superior results to polynomial regression.\\
\sB{This is because unlike polynomials, which must use a higher degree 
to produce flexible fits, splines introduce flexibility by increasing
the number of knots but keeping the degree fixed.}\\
Splines produce also more stable estimates.

\subsection{Smoothing splines}
\paragraph{Definition}
We want a function $g$ that makes $RSS=\su{{i=1}}{n}\left(y_{i}-
g(x_{i})\right)^{2}$ small but also with constraint on $g(x_{i})$
therewith to avoid overfitting and get smooth curve.\\
A natural approach is to find the function $g$ that minimizes:
\begin{center}
	\encV{$\underbrace{\su{{i=1}}{n}\left(y_{i}-g(x_{i})\right)^{2}}_\text{Loss function}+\underbrace{\lambda\Su{}{} g''(t)^{2}dt}_\text{Penalty term}$}
\end{center}
where $\lambda$ is a nonnegative \emph{tunning parameter}, and $g$ is
a \tR{\emph{smoothing spline}}.\\

The first derivative $g'(t)$ measures the slope of a function at $t$,
and the second derivative corresponds to the amount by which the slope
is changing, \tR{if $g(t)$ is very wiggly near $t$ $g''(t)$ is large,
otherwise it is close to zero.}\\
$\Su{}{}g''(t)^{2}dt$ \sB{is a measure of the total change in the
function $g'(t)$, over its entire range}.
Since the solution is a natural spline, we can write is as:
\begin{center}
	\enc{ $ f(x)=\su{{j=1}}{N}N_{j}(x)\theta_{j}$}
\end{center}
where the \sB{$N_{j}(x)$ are a $N-\text{dimensional}$ set of basis functions} for representing this
family of natural splines:
$$ RSS(\theta, \lambda) = \left(\bm{y}-\bm{N}\theta\right)^{T}\left(\bm{y}-\bm{N}\theta\right) +
\lambda\theta^{T}\Omega_{N}\theta$$ where 
$\begin{cases}
\left\{N\right\}_{ij} = N_{j}(x_{i})\\
\left\{\Omega_{N}\right\}_{jk} = \Su{}{}N_{j}^{''}(t)N_{k}^{''}(t)dt
\end{cases}$
The solution is easily seen to be:
\begin{center}
	\encB{$ \hat{\theta} = \left(\bm{N}^{T}\bm{N} + \lambda\Omega_{N}\right)^{-1}\bm{N}^{T}$}
\end{center}
The fitted smoothing spline is given by:
$$\hat{f}(x)=\su{{j=1}}{N}\bm{N}_{j}(x)\hat{\theta}_{j}$$

\paragraph{Choosing the smoothing parameter $\lambda$}
\tB{The vector of fitted values, when applying a smoothing spline to 
the data, can be written as $n\times n$ matrix $S_{\lambda}$ times the 
response $y$}:
\begin{center}
$\hat{g}_{\lambda}=S_{\lambda}y$
\end{center}
where $\hat{g}$ is a smoothing spline for a particular choice of 
$\lambda$, that is it is a $n$-vector containing the fitted values of 
the smoothing spline at the training points $\prth{x}{i}{n}$\\

\begin{python}
import csaps
from csaps import csaps

kf = KFold(n_splits=5)
df2 = df.iloc[:, [0, 1]].groupby(df.columns[1]).agg(
    np.median).reset_index()
cv_dict = {str(i):0 for i in range(0, 100+1, 1)}
for k in range(0, 100+1, 1):
    mse_list = []
    for train, test in kf.split(df2.budget_std):
        smoothing_spline = csaps(df2.iloc[train, 1],
                                 df2.iloc[train, 0],
                                 smooth=round(k/100, 2))
        mse = ((smoothing_spline(
            df2.iloc[test, 0])-df2.iloc[test, 1])**2).mean()
        mse_list.append(mse)
    cv_dict[str(k)] = np.array(mse_list).mean()
cv_list = list({k:v for k,v in sorted(cv_dict.items(), key=lambda item: item[1])})

smoothing_spline = csaps(df2.iloc[:, 1], df2.iloc[:, 0],
                         smooth=round(float(cv_list[0])/100, 2))
# Create spline line for 50 evenly spaced values of age
xp = np.linspace(df2.iloc[:, 1].min(), df2.iloc[:, 1].max(), 50)
yp = smoothing_spline(xp)
\end{python}

\paragraph{Degrees of freedom refer to the number of free parameters} such as
the number of coefficients fit in a polynomial or cubic spline. Denote by $\hat{f}$ the 
$N-\text{vector}$ of fitted values $\hat{f}(x_{i})$ at the training predictors $x_{i}$:
\begin{align*}
	\hat{f}=&\bm{N}\left(\bm{N}^{T}\bm{N}+\lambda\bm{\Omega}_{N}\right)^{-1}\bm{N}^{T}\bm{y}\\
	=& \bm{S}_{\lambda}\bm{y}
\end{align*}
The finite linear operator \tB{$\bm{S}_{\lambda}$} is known as the \tB{\textit{smoother matrix}}\\
Linear operator are familiar in more traditional least squares fitting as well, suppose 
$B_{\zeta}$ is a $N\times M$ matrix of $M$ cubic-spline basis functions evaluated at the $N$
training points $x_{i}$, with knot sequence $\zeta$ and $M\ll N$. Then the vector of fitted spline
values is given by:
\begin{align*}
	\hat{f}=&\bm{B}_{\zeta}\left(\bm{B}_{\zeta}^{T}\bm{B}_{\zeta}\right)^{-1}\bm{B}_{\zeta}^{T}\bm{y}\\
	=& \bm{H}_{\zeta}\bm{y}
\end{align*}
The linear operator $\bm{H}_{\zeta}$ is a projection operator, also known as the \textit{hat 
matrix}.
\begin{itemize}
	\item Both $\bm{H}_{\zeta}\text{ and }\bm{S}_{\lambda}$ are symmetric, positive 
		semi-definite
	\item $\bm{H}_{\zeta}\bm{H}_{\zeta} = \bm{H}_{\zeta}$ (\textit{idempotent})\\
		$\bm{S}_{\lambda}\bm{S}_{\lambda}\preceq\bm{S}_{\lambda}$ 
		meaning that the right-hand side exceeds the left-hand by a positive semi-definite
		matrix. This is a consequence of the shrinking nature of $\bm{S}_{\lambda}$
	\item $\bm{H}_{\zeta}$ has rank $M$, while $\bm{S}_{\lambda}$ has rank $N$.
\end{itemize}
The expression $M = trace\left(\bm{H}_{\zeta}\right)$ gives the dimension of the projection
space, which is also the number of basis functions, and hence the number of parameters involved
in the fit. \\ By analogy we define the \textit{effective degrees of freedom} of a smoothing
spline to be: 

\sB{It is possible to show that as $\lambda$ increases from 0 to 
$\infty$, the effective degrees of freedom, which we write 
$df_{lambda}$, decreases from $n$ to 2}.\\
Hence $df_{\lambda}$ is a measure of the flexibility.
\begin{center}
	\encV{$df_{\lambda}=trace(\bm{S}_{\lambda})=\su{{i=1}}{n}\left\{ S_{\lambda}
	\right\}_{ii}$}
\end{center}
the sum of the diagonal elements of the matrix $S_{\lambda}$\\
Since $\bm{S}_{\lambda}$ is symmetric (and positive semi-definite), it has a real 
eigen-decomposition. Before we proceed, it is convenient to rewrite $\bm{S}_{\lambda}$ in the
\textit{Reinsch} form: $\bm{S}_{\lambda} = \left(\bm{I}+\lambda\bm{K}\right)^{-1}$ where $K$
does not depend on $\lambda$. Since $\hat{\bm{f}}=\bm{S}_{\lambda}\bm{y}$ solves: 
$\min\limits_{f}(\bm{y}-\bm{f})^{T}(\bm{y}-\bm{f})+\lambda\bm{f}^{T}\bm{K}\bm{f}$
The eigen-decomposition of $\bm{S}_{\lambda}$ is:
$$ \bm{S}_{\lambda}=\su{{k=1}}{N}\rho_{k}(\lambda)\bm{u}_{k}\bm{u}_{k}^{T}$$ with 
$\rho_{k}(\lambda)=\dfrac{1}{1+\lambda d_{k}}$ and $d_{k}$ the corresponding eigenvalue of $K$.

\tB{It turns out that the \emph{leave-cross-out} cross-validation error
(LOOCV) can be computed very efficiently for smoothing splines}, with
essentially the same cost as computing a single fit using the following
formula:
$$
RSS_{cv}(\lambda)=\su{{i=1}}{n}\left( y_{i}-\hat{g}_{\lambda}^{(-i)}(x_{i}) \right)^{2}=\su{{i=1}}{n}\left[ \dfrac{y_{i}-\hat{g}_{\lambda}(x_{i})}{1-\left\{ S_{\lambda} \right\}_{ii}} \right]^{2}
$$
The notation \tB{$\hat{g}_{\lambda}^{(-i)}(x_{i})$ indicates the fitted
value for this smoothing spline evaluated at $x_{i}$}, where the fit
uses all of the training observations except for the $i^{th}$
observation $(x_{i},y_{i})$. In contrast \tB{{$\hat{g}_{\lambda}(x_{
i})$ indicates the smoothing spline function fit to all of the training
observations and evaluated in $x_{i}$}}.

\paragraph{Automatic Selection of the Smoothing Parameters}
\subparagraph{The Bias-Variance Tradeoff}
\begin{align*}
	Cov\left(\hat{f}\right) =& Cov\left(\bm{S}_{\lambda}\bm{y}\right)\\
	=& \bm{S}_{\lambda} Cov\left(\bm{y}\right)\bm{S}_{\lambda}^{T}\\
	=& \bm{S}_{\lambda}\bm{S}_{\lambda}^{T}
\end{align*}
The diagonal contains the pointwise variances at the training $x_{i}$. The bias is given by:
\begin{align*}
	\text{Bias}(\hat{\bm{f}}) =& \bm{f} - \E{\hat{\bm{f}}}\\
	=& \bm{f} - \bm{S}_{\lambda}\bm{f}
\end{align*}
The integrated squared prediction error (EPE) combines both bias and variance in a single 
summary:
\begin{align*}
	EPE\left(\hat{f}_{\lambda}\right) =& \E{Y-\hat{f}_{\lambda}(X)}^{2}\\
	=& \V{Y}+\E{\text{Bias}^{2}\left(\hat{f}_{\lambda}(X)\right)+\V{\hat{f}_{\lambda}\left(X \right)}}\\
	=& \sigma^{2} + \text{MSE}\left(\hat{f}_{\lambda}\right)
\end{align*}

\subsection{Multidimensional Splines}

Suppose $X\in\mathbb{R}^{2}$ and we have a basis of functions $h_{1k}(X_{1})$ with $k\in\inter{1}{
M_{1}}$ for representing functions of coordinate $X_{1}$, and likewise a set of $M_{2}$ functions
$h_{2k}(X_{2})$ for coordinate $X_{2}$. Then the $M_{1}\times M_{2}$ dimensional \textit{tensor
product basis} defined by:
$$ g_{jk}(X) = h_{1j}(X_{1})h_{2k}(X_{2})\text{ with }(j,k)\in\inter{1}{M_{1}}\times\inter{1}{M_{
1}}$$ can be used for representing a $2-\text{dimensional}$ function: 
$$g(X)=\su{{j=1}}{M_{1}}\su{{j=1}}{M_{2}}\theta_{jk}g_{jk}(X)$$

\begin{figure}[H]
	\begin{center}
		\includegraphics[width=\textwidth]{./chap/1chap/3sec/8images/2_tensorProduct.PNG}
	\end{center}
	\caption{A tensor product basis of $B-\text{splines}$, showing some selected pairs. Each
	$2-$dimensionnal function is the tensor product of the corresponding one dimension 
	marginals}
	\label{fig: 2_tensorProduct.PNG}
\end{figure}

\subsection{Wavelet Smoothing}
Wavelet typically use a complete orthonormal basis to represent functions, bu then shrink and 
select the coefficients toward a sparse representation.
\begin{figure}[H]
	\begin{center}
		\includegraphics[width=\textwidth]{./chap/1chap/3sec/8images/3_wavelet.PNG}
	\end{center}
	\caption{Some selected wavelets at different translations and dilatations for the Haar
	and symmlet families. The functions have been scaled to suit the display.}
	\label{fig: 3_wavelet.PNG}
\end{figure}

\paragraph{Wavelet Bases and the Wavelet Transform}
Wavelet bases are generated by translations and dilatations of a single scaling function
$\phi(x)$ (also known as the \textit{father}).\\
The \textit{Haar} basis produces a piecewise-constant representation, thus if $\phi(x)=I(x\in
[0,1])$, then $\phi_{0,k}(x) = \phi(x-k)$, k an integer, generates an orthonormal basis for 
functions with jumps at the integrs. Call this reference space $V_{0}$\\
The dilatations $\phi_{1,k}(x)=\sqrt{2}(2x-k)$ form an orthonormal basis for a space 
$V_{1} \supset V_{0}$. More generally we have $\cdots \supset V_{1}\supset V_{1}\supset V_{1}
\supset \cdots$ where each $V_{j}$ is spanned by $\phi_{j,k}=2^{\frac{j}{2}}\phi(2^{j}x-k)$\\
We might represent a function in $V_{j+1}$ by a component in $V_{j+1}$ plus the component in
the orthogonal complement $W_{j}$ of $V_{j}$ to $V_{j+1}$ written as $V_{j+1}= V_{j}\bigoplus W_{j}$
The component in $W_{j}$ represents detail, and we might wish to set some elements of this 
components to zero. It is easy to see that the functions $\psi(x-k)$ generated by the 
\textit{mother wavelet} $\psi(x)=\phi(2x)-\phi{2x-1}$ form an orthonormal basis for $W_{0}$ for
the Haar family. Likewise $\psi_{j,k}=2^{\frac{j}{2}}\psi(2^{j}x-k)$ form a basis for $W_{j}$

Now $V_{j+1}=V_{j}\bigoplus W_{j} = V_{j-1}\bigoplus W_{j-1} \bigoplus W_{j}$, more generally:
$V_{j} = V_{0} \bigoplus W_{0} \bigoplus W_{1}\cdots W_{j-1}$. Notice that since these spaces
are orthogonal, all the basis functions are orthonormal. In fact, if the domain is discrete with 
$N=2^{J}$ (time) points, this is as far as we can go. 

\paragraph{Adaptive Wavelet Filtering}
Suppose $\bm{y}$ is the response vector, and $\bm{W}$ is the $N\times N$ orthonormal wavelet
basis matrix evaluated at the $N$ uniformly spaced observations. Then $\bm{y}^{*}=\bm{W}^{T}\bm{y}$
is called the \textit{wavelet transform} of $\bm{y}$\\
A popular method for adaptive wavelet fitting is known as SURE(Stein Unbiased Risk Estimation)
shrinkage, we start with the criterion:
$$ \min\limits_{\theta}\norm{\bm{y}-\bm{W}\bm{\theta}}_{2}^{2} + 2\lambda\norm{\theta}_{1}$$
Because $\bm{W}$ is orthonormal, this leads to the simple solution:
$$\hat{\theta}_{j} = sign(y_{i}^{*})(|y_{j}^{*}|-\lambda)_{+}$$
The least squares coefficient are translated toward zero, and truncated at zero.
The fitted function is then given by the \textit{inverse wavelet transform} $\hat{\bm{f}}=\bm{W\hat{\theta}}$

\subsection{Kernel Smoothing Methods}
We will describe a class of regression techniques that achieve flexibility  in estimating the 
regression function $f(X)$ over the domain $\mathbb{R}^{p}$ by \sB{fitting a different but simple
model separately at each query point $x_{0}$. This is done by using only those observations close
to the target point $x_{0}$ to fit the simple model}, and in such a way that the resulting 
estimated function $\hat{f}(X)$ is \textit{smooth} in $\mathbb{R}^{p}$\\ This \sB{localization is
achieved via a weighting function or kernel $K_{\lambda}(x_{0},x_{i})$}, which assigns a weight 
to $x_{i}$ based on its distance from $x_{0}$.

\paragraph{One-Dimensional Kernel Smoother }
\begin{figure}[H]
	\begin{center}
		\includegraphics[width=\textwidth]{./chap/1chap/3sec/8images/40_localReg.PNG}
	\end{center}
	\caption{Left pannel is the result of a 30-nearest-neighbor running-mean smoother. The
	right panel is the result of a kernel-weighted average using \textit{Epanechnikov} kernel
	with (half) window width $\lambda=0.2$}
	\label{fig:40_localReg.PNG}
\end{figure}
The green curve is bumpy, since $\hat{f}(x)$ is discontinuous in $x$. As we move $x_{0}$ from left
to the right, the \emph{k}-nearest neighborhood remains constant, until a point $x_{i}$ to the right
of $x_{0}$ becomes closer than the furthest point $x_{i'}$ in the neighborhood to the left of 
$x_{0}$ at which time $x_{i}$ replaces $x_{i'}$.\\
\tB{Rather than give all the points in the neighborhood equal weight, we can assign weights that 
die off smoothly with distance from the target point as the so-called Nadaraya-Watson 
kernel-weighted average}:
\begin{center}
	\enc{$ \hat{f}(x_{0})=\dfrac{\su{{i=1}}{N}K_{\lambda}(x_{0},x_{i})y_{i}}{\su{{i=1}}{N}K_{\lambda}(x_{0},x_{i})}$}
\end{center}
with the \tB{\textit{Epanechenikov}} quadratic kernel:
$$ K_{\lambda}(x_{0}, x)=D\left(\dfrac{|x-x_{0}|}{\lambda}\right)\text{, with }D(t)=
\begin{cases}
	\frac{3}{4}(1-t^{2})\text{ if }|t|\leq 1\\
	0\text{ otherwise}
\end{cases}
$$
We can use such adaptive neighborhoods with kernels, more generally: 
\begin{center}
	\tR{$ K_{\lambda}(x_{0}, x)=D\left(\dfrac{|x-x_{0}|}{h_{\lambda}(x_{0})}\right)$}
\end{center}
\paragraph{Local Linear Regression}
Locally-weighted averages can be badly biased on the boundaries of the domain.
\begin{figure}[H]
	\begin{center}
		\includegraphics[width=\textwidth]{./chap/1chap/3sec/8images/4_localReg.PNG}
	\end{center}
	\caption{The true function is approximately linear, but most of the observations in the
	neighborhood have a higher mean than the target point, so despite weighting, their mean
	will be biased upwards. By fitting a locally weighted linear regression (right panel), this
	bias is removed to first order.}
	\label{fig:4_localReg}
\end{figure}

\tB{Locally weighted regression solves a separate weighted least squares problem at each target 
point} $x_{0}$:
\begin{center}
\enc{$ \min\limits_{\alpha(x_{0}),\beta(x_{0})}\su{{i=1}}{N}K_{\lambda}(x_{0},x_{i})[y_{i}-
\alpha(x_{0})-\beta(x_{0})x_{i}]^{2}$}
\end{center}
The estimate is then \tV{$\hat{f}(x_{0}) = \hat{\alpha}(x_{0}) + \hat{\beta}(x_{0})x_{0}$}.\\
Define the vector-valued function $b(x)^{T} = (1,x)$. Let \sB{$\bm{B}$ be the $N\times 2$ regression
matrix with $i^{th}$ row $b(x_{i})^{T}$}, and \sB{$\bm{W}(x_{0})$ the $N\times N$ diagonal matrix 
with $i^{th}$ diagonal element $K_{\lambda}(x_{0},x_{i})$} then:
\begin{align*}
	\hat{f}(x_{0}) &= b(x_{0})^{T}\left(\bm{B}^{T}\bm{W}(x_{0})\bm{B}\right)^{-1}\bm{B}^{T}\bm{
	W}(x_{0})\bm{y}\\
	&= \su{{i=1}}{N}l_{i}(x_{0})y_{i}
\end{align*}
These \sB{weights $l_{i}(x_{0})$ combine the weighting kernel $K_{\lambda}(x_{0},\cdot)$ and the
least squares operations} and are sometimes referred to as the \tB{\textit{equivalent kernel}}.\\
Local linear regression \textit{automatically} modifies the kernel to correct the bias 
\textit{exactly} to first order, a phenomenon dubbed as automatic kernel carpentry. 
\begin{align*}
	\E{\hat{f}(x_{0})} =& \su{{i=1}}{N}l_{i}(x_{0})f(x_{i})\\
	=& f(x_{0})\su{{i=1}}{N}l_{i}(x_{0}) + f'(x_{0})\su{{i=1}}{N}(x_{i}-x_{0})l_{i}(x_{0})
	+ \dfrac{f''(x_{0})}{2}\su{{i=1}}{N}(x_{i}-x_{0})^{2}l_{i}(x_{0}) + R
\end{align*}
where the remainder term $R$ involves third and higher order derivatives of $f$ and is typically
small under suitable smoothness assumptions.

\paragraph{Local Polynomial Regression}
\begin{center}
\enc{$\min\limits_{\alpha(x_{0}),\beta_{j}(x_{0})|j\in\inter{1}{d}} \su{{i=1}}{N}K_{\lambda}(x_{0},
x_{i})\left[y_{i}-\alpha(x_{0})-\su{{j=1}}{d}\beta_{j}(x_{0})(x_{0})x_{i}^{j} \right]^{2}$}
\end{center}
with the solution $\hat{f}(x_{0})=\hat{\alpha}(x_{0})+\su{{j=1}}{d}\hat{\beta}_{j}(x_{0})x_{0}^{j}$.
In fact, an expansion such as the equation of $\E{\hat{f}(x_{0})}$ for local linear regression 
will tell us that the bias will only have components of degree $d+1$ and higher.
\begin{figure}[H]
	\begin{center}
		\includegraphics[width=.7\textwidth]{./chap/1chap/3sec/8images/5_linearPoly.PNG}
	\end{center}
	\caption{The variances functions $\norm{l(x)}^{2}$ for local constant, linear and quadratic
	regression, for a metric bandwidth $(\lambda=0.2)$ tri-cube kernel}
	\label{fig:5_linearPol}
\end{figure}
Local linear fits tend to be biased in regions of curvature of the true function, a phenomenon
referred to as \emph{trimming the hills} and \emph{filling the valleys}. Local quadratic regression
is generally able to correct this bias.
\begin{itemize}
	\item \tB{Local linear fits can help bias dramatically at the boundaries at a mode cost
		in variance.} Local quadratic fits do little at the boundaries for bias, but 
		increase the variance a lot.
	\item \tB{Local quadratic fits tend to be most helpful in reducing bias due to curvature
		in the interior of the domain.}
\end{itemize}
\sB{It is not recommended to move from local linear fits at the boundary to local quadratic fits
in the interior. But rather to choose the degree of the fit in function of the application}, if
we are interested in extrapolation then the boundary is of more interest and local linear fits 
are probably more reliable.

\paragraph{Selecting the Width of the Kernel}
\begin{itemize}
	\item For the \tB{\emph{Epanechnikov}} or \emph{tri-cube} kernel with metric width, $\lambda$
		is \sB{the radius of the support region}
	\item For the \tB{\emph{Gaussian}} kernel, $\lambda$ is \sB{the standard deviation}
	\item For the \tB{\emph{k-nearest}} neighborhoods, $\lambda$ is \sB{the number $k$ of nearest
		neighbors}, often expressed as a fraction or span $\frac{k}{N}$ of the total
		training sample.
\end{itemize}
There is a natural bias-variance tradeoff as we change the width of the averaging window:
\begin{itemize}
	\item \tB{If the window is narrow, $\hat{f}(x_{0})$ is an average of a small number of 
		$y_{i}$ close to $x_{0}$, and its variance will be relatively large-close that of
		an individual $y_{i}$.}
	\item If the window is wide, the variance of $\hat{f}(x_{0})$ will be small relative to
		the variance of any $y_{i}$, because of the effect of averaging.
		
\end{itemize}

\paragraph{Local Regression in $\mathbb{R}^{R}$}
Local linear regression will fit a hyperplane locally in $X$, by weighted least squares, with 
weights supplied by a \emph{p-}dimensional kernel.\\
Let $b(X)$ be a vector of polynomial terms in $X$, of maximum degree $d$, 
$
\begin{cases}
	d=0 \Leftarrow b(x) = 1\\
	(d, p)=(1, 2) \Leftarrow b(x) = (1, X_{1}, X_{2})\\
	(d, p)=(2, 2) \Leftarrow b(x) = (1, X_{1}, X_{2}, X_{1}^{2}, X_{2}^{2}, X_{1}X_{2})
\end{cases}
$\\
At each $x_{0}\in\mathbb{R}^{p}$ we solve:
$$ \min\limits_{\beta(x_{0})}\su{{i=1}}{N}K_{\lambda}(x_{0},x_{1})\left(y_{i}-b(x_{i})^{T}\beta(x_{0})\right)^{2}$$ with $K_{\lambda}(x_{0},x_{1})=D\left(\dfrac{\norm{x-x_{0}}}{\lambda}\right)$

\paragraph{Structured Local Regression Models in $\mathbb{R}^{p}$}
\subparagraph{Structured Kernels}
Let be $\bm{A}$ a semi-definite matrix to weigh the different coordinates:
$$ K_{\lambda,A}(x_{0},x)=D\left(\dfrac{(x-x_{0})^{T}\bm{A}(x-x_{0})}{\lambda}\right)$$
If $\bm{A}$ is diagonal, then we can increase or decrease the influence of individual predictors
$X_{j}$ by increasing or decreasing $A_{jj}$

\paragraph{Structured Regression Functions}
We are trying to fit a regression function $\E{Y|X}= f(X_{1}, X_{2},\cdots, X_{p})$ in 
$\mathbb{R}^{p}$. It is natural to consider ANOVA decomposition of the form:
$$ f(X_{1}, X_{2}, \cdots, X_{p}) = \alpha+\su{{j}}{}g_{j}(X_{j})+\su{{k<l}}{}g_{kl}(X_{k},X_{l})
+\cdots$$
and then introduce by eliminating some of the higher-order terms. We then assume the conditionally
linear model:
$$ f(X)=\alpha(Z)+\beta_{1}(Z)X_{1}+\cdots+\beta_{q}(Z)X_{q}$$
For given $Z$, this is a linear model, but each of the  coefficient can vary with $Z$.
$$\min\limits_{\alpha(Z_{0}),\beta(z_{0})}\su{{i=1}}{N}K_{\lambda}(z_{0}, z_{i})\left(y_{i}-
\alpha(z_{0})-x_{1i}\beta_{1}(z_{0})-\cdots-x_{qi}\beta_{q}(z_{0})\right)$$

\subsection{Kernel Density Estimation and Classification}
Kernel density estimation is an unsupervised learning procedure.
\paragraph{Kernel Density Estimation}
Arguing as before, a natural local estimate has the form : \tB{$\hat{f}_{X}(x_{0})=\frac{\#x_{i}\in
\mathcal{N}(x_{0})}{N\lambda}$} where $\mathcal{N}(x_{0})$ is a small metric neighborhood around
$x_{0}$ of width $\lambda$. This estimate is bumpy, and the \sB{smooth \emph{Parzen}} estimate is 
preferred:
$$ \hat{f}_{X}(x_{0}) = \dfrac{1}{N\lambda}\su{{i=1}}{N}K_{\lambda}(x_{0}, x_{i})$$
because it counts observations close to $x_{0}$ with weights that decrease with distance from
$x_{0}$. In this case a popular choice for $K_{\lambda}$ is the Gaussian kernel $K_{\lambda}(
x_{0},x)=\phi\left(\frac{|x-x_{0}|}{\lambda}\right)$

\emph{Python Code}
\begin{python}
import sklearn 
from sklearn.neighbors import KernelDensity

kde = KernelDensity(
  brandwith=0.2
  kernel='gaussian') # or 'tophat','epanechnikov','exponential','linear','cosine'
kde_log_density = kde.score_samples(y)
\end{python}

\paragraph{Kernel Density Classification}
Suppose \sB{for a $J$ class problem we fit non-parametric density estimates $\hat{f}_{j}(X),j\in
\inter{1}{J}$} separately in each the classes, and we also have estimates of the class priors
$\hat{\pi}_{j}$ (usually the sample propositions). Then:
\begin{center}
\enc{$\hat{\mathbb{P}}_{\left\{X=x_{0}\right\}}(G=j)=\dfrac{\hat{\pi}_{j}\hat{f}_{j}(x_{0})}{
\su{{k=1}}{j} \hat{\pi}_{k}\hat{f}_{k}(x_{0})}$}
\end{center}



\subsection{Local regression}
\paragraph{Span}
It plays a role like that of the tuning parameter $\lambda$ in 
smoothing splines: \tB{it controls the flexibility of the non-linear fit}.\\
\sT{The smaller the value of $s$, the more \emph{local} will be our fit.}

\paragraph{Algorithm: \emph{Local Regression} at $X=x_{0}$}
\begin{figure}[H]
	\begin{center}
		\includegraphics[width=\textwidth]{./chap/1chap/6sec/images/2localRegression.png}
	\end{center}
	\caption{Blue curve represents $f(x)$ from which the data were
	generated.\\
	Orange curve corresponds to the local regression estimate $f(x)$.\\
	Orange colored points are local to the target point $x_{0}$.\\
	Yellow-bell-shape superimposed on the plot indicates weights 
	assigned to each point, decreasing to zero with distance from
	the target point.}
	\label{fig:6.1localRegression}
\end{figure}
\begin{enumerate}
	\item Gather the fraction $s=\frac{k}{n}$ of training points
		whose $x_{i}$ are closet to $x_{0}$
	\item Assign a weight $k_{i0}=K(x_{i},x_{0})$ so that \tB{the
		point furthest from $x_{0}$ has weight zero, and the 
		closet has the highest weight}.\\
		\sB{All but these $k$ nearest neighbors get weight 0}
	\item \tB{Fit a weighted least squares regression of the $y_{i}
		$ on the $x_{i}$}, using the aforementioned weights, by
		\tB{finding $\beta_{0}$ and $\beta_{1}$ that minimize:
		$$\su{{i=1}}{n}K_{i0}(y_{i}-\beta_{0}-\beta_{1}x_{i})^{2}$$}
	\item \tB{The fitted value at $x_{0}$ is given by $\hat{f}(x_{
		0})= \hat{\beta}_{0}+\hat{\beta}_{1}x_{0}$}
\end{enumerate}
For $p$-dimensional neighborhoods, local regression can perform poorly
if $p$ is much larger than 3 or 4, because there will generally be very
few training observations close to $x_{0}$

\subsection{Generalized additive models}
\emph{Generalized Additive Models} (GAMs) \sB{provide a genral 
framework for extending a standard linear model by allowing non-linear}
functions of each of the varialbes, \sB{while maintaining additivity}.

\paragraph{Principle}
\subparagraph{Definition}
Considering we have $y_{i}=\beta_{0}+\su{{r=1}}{p}\beta_{r}x_{ir}+
\epsilon$\\ \tB{We replace each linear component $\beta_{j}x_{ij}$ with
a (smooth) non-linear function $f_{j}(x_{ij})$}:\\
$$
y_{i}=\beta_{0}+\su{{j=1}}{p}f_{j}(x_{ij})+\epsilon_{i}
$$
\sB{It is called an \emph{additive model} because we calculate a
separate $f_{i}$ for each $X_{j}$, and then add together all of their 
contributions.}\\
In the regression setting, a generalized additive model has the form: 
$\E{Y|\prth{X}{j}{p}}=\alpha+\su{{j=1}}{p}f_{j}(X_{j})$, the $f_{j}$'s are unspecified smooth
(``non-parametric'') functions.
\paragraph{Fitting Additive Models}
The additive model has the form:
\begin{center}
	\enc{$ Y = \alpha + \su{{j=1}}{p}f_{j}(X_{j})+\epsilon$}
\end{center}
where $\E{\epsilon}=0$. Given observations $x_{i},y_{i}$ a criterion like the penalized sum of
squares
\begin{center}
\encB{$ PRSS\left(\alpha,\prth{f}{j}{p}\right) = \su{{i=1}}{N}\left(y_{i}-\alpha-\su{{j=1}}{p}
f_{j}(x_{ij})\right)^{2} + \su{{j=1}}{p}\lambda_{j}\su{}{}f_{j}^{''}(t_{j})^{2}dt_{j}$}
\end{center}
where the $\lambda_{j}\geq 0$ are the tuning parameters.

\subparagraph{The Backfitting Algorithm for Additive Models}
\begin{enumerate}
	\item Initialize: $\forall (i,j)\in\inter{1}{N}\times\inter{1}{p}, \hat{\alpha} = \dfrac{1}{N}\su{{1}}{N}y_{i}, \hat{f}_{j}\equiv 0$
	\item Cycle: $j=1,\dots,p,1,\dots,p,\dots$
		\begin{align*}
			\hat{f}_{j} &\leftarrow \bm{S}_{j}\left[\left\{y_{i}-\hat{\alpha}-
			\su{{k\neq}}{}\hat{f}_{k}(x_{ik})\right\}_{1}^{N}\right]\\
			\hat{f}_{j} &\leftarrow \hat{f}_{j}-\dfrac{1}{N}\su{{i=1}}{N}\hat{f}_{j}(x_{ij})
		\end{align*}
		until the function $\hat{f}_{j}$ change less than a pre-specified threshold.
\end{enumerate}
We set \sB{$\hat{\alpha}=ave(y_{i})$} and it never changes. We apply a \sB{cubic smoothing spline  
$S_{j}$} to the targets $\left\{y_{i}-\hat{\alpha}-\su{{k\neq j}}{}\hat{f}_{k}(x_{ik})\right\}_{1}^{N
}$. The process is continued unitil the estimates $\hat{f_{j}}$ stabilize.\\ \sB{The backfitting 
procedure allow one to choose a fitting method appropriate for each input variable however it fits 
all predictors which is not feasible or desirable when a large number are available}

\subparagraph{Python Code}
\begin{python}
import pygam
from pygam import LinearGAM, LogisticGAM
from pygam import GAM, l, s, f, te, intercept


# REGRESSION
df_reg = pd.read_csv('myFile_reg.txt', sep=';')
y, X = df.iloc[:, 0], df.iloc[:, 1:]
p = len(X)
lgam = LinearGAM(
   eval(' + '.join(['f({})'.format(i) 
       for i in range(p-3)] +
       ['l(p-3)', 'l(p-2)', 'l(p-1)'])
       )
)

lgam_grid = lgam.gridsearch(X, y)
lgam_grid.summary()

fig, ax = plt.subplots()
ax.scatter(X.budget_std, y)
ax.plot(X.budget_std, lgam_grid.predict(X), color='red')
ax.plot(X.budget_std, lgam_grid.prediction_intervals(X)[:, 0])
ax.plot(X.budget_std, lgam_grid.prediction_intervals(X)[:, 1])
plt.show()



# CLASSIFICATION
# Replace LinearGAM by LogisticGAM!!

\end{python}

\subparagraph{Pros and Cons of GAM's}
\begin{itemize}
	\item[\tV{+}] We do not need to manually try out many different
		transformations on each variable individually.
	\item[\tV{+}] The non-linear fits can potentially make more
		accurate predictions for the response $Y$
	\item[\tV{+}] Because the model is additive, we can examine the
		effect of each $X_{j}$ on $Y$ individually while 
		holding all of the other variables fixed.
		
	\item[\tV{+}] The smoothness of the function $f_{j}$ for the 
		variable $X_{j}$ can be summarized via degrees of 
		freedom.
	\item[\tR{-}] The model is restricted to be additive
\end{itemize}

\paragraph{The Naive Bayes Classifier}
It is \tB{especially appropriate when dimension $p$ of the feature space is high}, making density 
estimation unattractive. The naive Bayes model \tB{assumes that given a class $G=j$, the features
$X_{k}$ are independent}.\\
While this assumption is generally not true, it does simplify the estimation dramatically, using
class $J$ as the base we can derive:
\begin{itemize}
	\item The individual class-conditional marginal densities $f_{jk}$ can each be estimated
		separately using one-dimensional kernel density estimates.
	\item If a component $X_{j}$ of $X$ is discrete, then an appropriate histogram estimate
		can be used.
\end{itemize}

\sB{Despite these rather optimistic assumptions, naive Bayes classifiers often outperform far more
sophisticated alternatives.} Because although the individual class density estimates may be biased
this bias might not hurt the posterior probabilities as much especially near the decision regions.

\begin{align*}
	\log\left(\dfrac{\ProbC{X}{G=l}}{\ProbC{X}{G=J}}\right) =& \log\left(\dfrac{\pi_{l}f_{l}(
	X)}{\pi_{J}f_{J}(X)}\right)\\
	=& \log\dfrac{\pi_{l}\prd{{k=1}}{p}f_{lk}(X_{k})}{\pi_{J}\prd{{k=1}}{p}f_{Jk}(X_{k})}
	\text{ (independence assumption)}\\
	=& \log\left(\dfrac{\pi_{l}}{\pi_{J}}\right) + \su{{k=1}}{p}\log\left(\dfrac{f_{lk}(X_{k})}{f_{Jk}(X_{k})}\right)\\
	=& \alpha_{l} + \su{{k=1}}{p}g_{lk}(X_{k})
\end{align*}
\subparagraph{Python Code}

\begin{python}
import sklearn
from sklearn.model_selection import train_test_split
from sklearn.naive_bayes import GaussianNB

df = pd.read_csv('myFile.csv', sep=';')
y, X = df.iloc[:, 0], df.iloc[:, 1:]

X_train, X_test, y_train, y_test = train_test_split(
    X, y, test_size=0.5, random_state=0)
gnb = GaussianNB()
y_pred = gnb.fit(X_train, y_train).predict(X_test)
print('Number of mislabeled points out of a total\
%d points: %d'
% (X_test.shape[0], (y_test != y_pred).sum()))
\end{python}

\subsection{Radial Basis Functions and Kernels}
Kernel method achieve flexibility by fitting simple models in a region local to the target point
$x_{0}$. Localization is achieved via a weighting kernel $K_{\lambda}$, and individual 
observations receive weights $K_{\lambda}(x_{0}, x_{1})$.\\ 
Radial basis functions combine these ideas by treating the kernel function $K_{\lambda}(\xi, x)$.
This leads to the model:
\begin{align*}
	f(x) =& \su{{j=1}}{M}K_{\lambda}(\xi_{j},x)\beta_{j}\\
	=& \su{{j=1}}{M}D\left(\dfrac{\norm{x-\xi_{j}}}{\lambda_{j}}\right)\beta_{j}
\end{align*}
where \sB{each basis elements is indexed bya a location or \emph{prototype} parameters $\epsilon_{j}$
and a scale parameter $\lambda_{j}$}.



\subsection{Model Inference and Averaging}
\paragraph{Maximum Likelihood Inference}
The likelihood function can be used to assess the precision of $\hat{\theta}$. We need few more
definitions.
\begin{itemize}
	\item Score function: $\dot{l}(\theta,\bm{Z})=\su{{i=1}}{N}\dot{l}(\theta,z_{i}) = 
		\su{{i=1}}{N}\dfrac{\partial l(\theta;z_{i})}{\partial \theta}$ 
	\item The \emph{information matrix} is: $\bm{I}=-\dfrac{{i=1}}{N}\dfrac{\partial^{2}l(
		\theta;z_{i})}{\partial\theta\partial\theta^{T}}$
	\item \emph{Fisher information}: $\bm{i}(\theta)=\mathbb{E}_{\theta}\left(\bm{I}(\theta)\right)$
\end{itemize}

Confidence points for $\theta_{j}$ can be constructed from either approximation:
$\hat{\theta}-z^{(1-\alpha)}\sqrt{\bm{I}(\hat{\theta})_{jj}^{-1}}$
More accurate confidence intervals can be derived from the likelihood function, by using the
chi-squared approximation
$$ 2\left[l(\hat{\theta})-l(\theta_{0})\right]\hookrightarrow \chi_{p}^{2}$$
where $p$ is the number of components in $\theta$.\\
The maximum likelihood estimate is obtained by setting $\dfrac{\partial l}{\partial\beta}=
\dfrac{\partial^{2} l}{\partial\sigma^{2}}=0$ giving 
$$
\begin{cases}
	\hat{\beta} = \left(\bm{H}^{T}\bm{H}\right)^{-1}\bm{H}^{T}\bm{y}\\
	\sigma^{2} = \dfrac{1}{N}\su{}{}\left(y_{i}-\hat{\mu}(x_{i})\right)^{2}
\end{cases}
$$
The advantage of the bootstrap over the maximum of likelihood formula is that it allows us to
compute maximum likelihood estimates of standard errors and other quantities in setting where
no formulas are available.


\paragraph{The EM Algorithm}
We would like to model the density of the data points, and \sB{due to the apparent bi-modality, a
Gaussian distribution} would not be appropriate.
$
\begin{cases}
	Y_{1}\hookrightarrow\mathcal{N}(\mu_{1}, \sigma_{1}^{2})\\
	Y_{2}\hookrightarrow\mathcal{N}(\mu_{2}, \sigma_{2}^{2})\\
	Y = (1-\Delta)Y_{1} + \Delta Y_{2}
\end{cases}
$ 
where $\delta\in\left\{0,1\right\}$ with \sB{$\Prob{\Delta=1}=\pi$}\\
$$
l_{0}(\theta;\bm{Z},\bm{\Delta}) = \su{{i=1}}{N} \left[(1-\bm{\Delta}_{i})\log\left(
\phi_{\theta_{1}}(y_{i})\right) +\bm{\Delta}_{i}\log\left( \phi_{\theta_{2}}(y_{i})\right) 
\right] + \su{{i=1}}{N} \left[(1-\bm{\Delta}_{i})\log\left( 1-\pi\right) +
\bm{\Delta}_{i}\log\left(\pi\right) \right]
$$
Since the values of the $\Delta_{i}$'s are actually unknown, we proceed in an iterative fashion
substituting for each $\Delta_{i}$ in the previous equation : 
$\gamma_{i}(\theta)=\mathbb{E}_{\theta,\bm{Z}}(\Delta_{i}) = \ProbC{\theta,\bm{Z}}{\Delta_{i}=1}$
also called \sB{the \emph{responsibility}}
We use the following EM algorithm for special case of the Gaussian mixtures.
\begin{itemize}
	\item \tB{Expectation} Step: we do a soft assignment of each observation to each model:
		\sB{the current estimates of the parameters are used to assign responsibilities}
		according to the relative density of the training points under each model.
	\item \tB{Maximization} Step: These \sB{responsibilities are used in weighted 
		maximum-likelihood fits} to update the estimates of the parameters.
\end{itemize}
The EM algorithm is a popular tool for simplifying difficult maximum likelihood problems.
\subparagraph{EM Algorithm for 2-Component Gaussian Mixture}
\begin{enumerate}
	\item  Take initial guesses for the parameters $\hat{\mu}_{1}, \hat{\sigma}_{1}, \hat{\mu}_{2}, \hat{\sigma}_{2}, \hat{\pi}$
	\item Excpectation Step: compute the responsibilities: 
		$\hat{\gamma}_{i}=\dfrac{\hat{\pi}\phi_{\hat{\theta}_{2}}(y_{i})}{(1-\hat{\pi})\phi_{\hat{\theta}_{1}}(y_{i}) + \hat{\pi}\phi_{\hat{\theta}_{2}}(y_{i})}$ for $i\inter{1}{N}$
	\item Maximization Step: compute the weighted means and variances
		\begin{align*}
			\hat{\mu}_{1} = \dfrac{\su{{i=1}}{N}(1-\hat{\gamma}_{i})y_{i}}{\su{{i=1}}{N}(1-\hat{\gamma}_{i})}&, & 
			\hat{\sigma}_{1} = \dfrac{\su{{i=1}}{N}(1-\hat{\gamma}_{i})(y_{i}-\hat{\mu}_{1})^{2}}{\su{{i=1}}{N}(1-\hat{\gamma}_{i})}\\
			\hat{\mu}_{2} = \dfrac{\su{{i=1}}{N}\hat{\gamma}_{i}y_{i}}{\su{{i=1}}{N}\hat{\gamma}_{i}}&, & 
			\hat{\sigma}_{2} = \dfrac{\su{{i=1}}{N}\hat{\gamma}_{i}(y_{i}-\hat{\mu}_{2})^{2}}{\su{{i=1}}{N}\hat{\gamma}_{i}}
		\end{align*}
	\item Iterate steps 2 and 3 until convergence
\end{enumerate}
and the mixing probability $\hat{\pi}=\su{{i=1}}{N}\dfrac{\hat{\gamma}_{i}}{N}$

\begin{python}
import sklearn
from sklearn.mixture import GaussianMixture

df = pd.read_csv('myFile.csv', sep=';')
y, X = df.iloc[:, 0], df.iloc[:, 1:]

model = GaussianMixture(n_components=2,
    init_params='random')
model.fit()
y_pred = model.predict(X)
\end{python}

\paragraph{The EM Algorithm in General}
Our observed data is $\bm{Z}$ having log-likelihood $l(\theta,\bm{Z})$. The latent or missing data
is $\bm{Z}^{m}$, so that the complete data is $\bm{T}=(\bm{Z},\bm{Z}^{m})$ with log-likelihood 
$l_{0}(\theta,\bm{T})$. In the mixture problem $(\bm{Z},\bm{Z}^{m}) = (\bm{y},\bm{\Delta})$
Since $\ProbC{\bm{Z},\theta'}{\bm{Z}^{m}}=\dfrac{\ProbC{\theta'}{\bm{Z}^{m},\bm{Z}}}{\ProbC{\theta'}{\bm{Z}}}$ we can write:
$\ProbC{\theta'}{\bm{Z}}=\dfrac{\ProbC{\theta'}{\bm{T}}}{\ProbC{\bm{Z},\theta'}{\bm{Z}^{m}}}$.
In terms of log-likelihoods, we have:
$$ l(\theta';\bm{Z}) = l_{0}(\theta'; \bm{T}) - l_{1}(\theta';\bm{Z}^{m}|bm{Z})$$ where $l_{1}$ is
based on the conditional density: $\ProbC{\bm{Z},\theta'}{\bm{Z}^{m}}$
\begin{enumerate}
	\item Start with initial guesses for the parameters $\hat{\theta}^{(0)}$
	\item Expectation Step: at the $j^{th}$ step, compute 
		$$ Q(\theta',\hat{\theta}^{(j)}) = \E{l_{0}(\theta':\bm{T})|\bm{Z},\hat{
		\theta}^{j)}}$$
	\item Maximization Step: determine the new estimate $\hat{\theta}^{(j+1)}$ as the 
		maximizer of $Q\left(\theta',\hat{\theta}^{(j)}\right)$ over $\theta'$
	\item Iterate steps 2 and 3 until convergence
\end{enumerate}
In the M step, the EM algorithm maximizes $Q(\theta', \theta)$ over $\theta'$ rather than
actual objective function $l(\theta':\bm{Z})$
\subparagraph{EM as a Maximization Procedure}
Here is a different view of the EM procedure, as a joint maximization algorithm. Consider 
function:
$$ F(\theta',\tilde{P})= \mathbb{E}_{\tilde{P}}\left(l_{0}(\theta':\bm{T})\right)-
\mathbb{E}_{\tilde{P}}\left(\tilde{P}(\bm{Z}^{m})\right)$$
Here $\tilde{P}(\bm{Z}^{m})$ is any distribution over the latent data $\bm{Z}^{m}$

The  function $F$ expands the domain of the log-likelihood, to facilitate its maximization.
The maximizer over $$\tilde{P}(\bm{Z})=\ProbC{\bm{Z},\theta'}{\bm{Z}}$$
\begin{figure}[H]
	\begin{center}
		\includegraphics[width=.5\textwidth]{./chap/1chap/6sec/images/3_emAlgo.PNG}
	\end{center}
	\caption{The contours of the (augmented) observed data log-likelihood $F(\theta,\tilde{P})$
	The $E$ step is equivalent to maximizing the log-likelihood over the parameters of the 
	latent data distribution. The $M$ step maximizes it over the parameters of the 
	log-likelihood. The red curve corresponds to the observed data log-likelihood a profile
	obtained by maximizing $F(\theta', \tilde{P})$ for each value of $\theta'$}
	\label{fig:3_emAlgo}
\end{figure}
\subparagraph{Gibbs Sampler}
\begin{enumerate}
	\item Take some initial values $U_{k}^{(0)}$ for $k\in\inter{1}{K}$
	\item Repeat for $t\in\inter{1}{?}$:
		For $k\in\inter{1}{K}$ generate $U_{k}^{(t)}$ from 
		$$ \Pr(U_{k}^{(t)}|U_{1}^{(t)}, \cdots U_{k-1}^{(t)}, U_{k+1}^{(t-1)},\cdots U_{K}^{(t-1)})$$
	\item Continue step 2 until the joint distribution of $\left( U_{j}^{(t)} \right)_{1\leq j\leq K}$
\end{enumerate}
\paragraph{MCMC (Markov chain Monte Carlo) for Sampling from the Posterior}
Gibbs sampling  is an MCMC procedure that is closely related to the EM Algorithm: the main 
difference  is that is it samples from the conditional distributions rather than maximizing over 
them. \\
More formally, Gibbs sampling produces a Markov chain whose stationary distribution is the 
true joint distribution, and hence the term of ``Markov Chain Monte Carlo''
\subparagraph{Gibbs sampling for mixtures}
\begin{enumerate}
	\item Take some initial values: $\theta^{(0)}=(\mu_{1}^{(0)}, \mu_{2}^{(0)})$
	\item Repeat for $t\in\inter{1}{?}$:
		\begin{enumerate}[label=(\alph*)]
			\item For $i\in\inter{1}{N}$ generate $\bm{\Delta}_{i}^{(t)}\in\left\{0,1
				\right\}$ with $\Prob{\bm{\Delta}_{i}^{(t)}=1}=\hat{\gamma}_{i}(
				\theta^{(t)})$
			\item Set
				\begin{align*}
					\hat{\mu}_{1} = & \dfrac{\su{{i=1}}{N}\left(1-\bm{\Delta}_{i}^{(t)}\right)y_{i}}{\su{{i=1}}{N}\left(1-\bm{\Delta}_{i}^{(t)}\right)}\\
					\hat{\mu}_{2} = & \dfrac{\su{{i=1}}{N}\bm{\Delta}_{i}^{(t)}y_{i}}{\su{{i=1}}{N}\bm{\Delta}_{i}^{(t)}}
				\end{align*}
					and generate $\mu_{1}^{(t)}\hookrightarrow\mathcal{N}(\hat{\mu}_{1}, \hat{\sigma}_{1}^{2})$ and $\mu_{1}^{(t)}\hookrightarrow\mathcal{N}(\hat{\mu}_{2},
					\hat{\sigma}_{2}^{2})$
			\item Continue until the joint distribution of $\left(\Delta^{(t)},\mu_{1}^{(t)}, \mu_{2}^{(t)}\right)$ does not change.
		\end{enumerate}
\end{enumerate}

\paragraph{Bagging}
We show how to use the bootstrap to improve the estimate or prediction itself.\\
For each bootstrap $\bm{Z}^{*b}$ for $b\in\inter{1}{B}$ we fit our model, giving prediction
$\hat{f}^{*b}(x)$. The bagging estimate is defined by:
$$ \hat{f}_{bag}(x) = \dfrac{1}{B}\su{{b=1}}{B}\hat{f}^{*b}(x)$$
It is a Monte Carlo estimate of the true bagging estimate, approaching it as $B\rightarrow \infty$.o
Suppose $\xi$
\paragraph{Model Averaging and Stacking}
The posterior distribution of $\xi$ is
$$ \ProbC{\bm{Z}}{\xi} = \su{{m=1}}{M}\ProbC{M_{m},\bm{Z}}{\xi}\ProbC{\bm{Z}}{M_{m}}$$
with posterior $\E{\xi|\bm{Z}} = \su{{m=1}}{M}\E{\xi|M_{m},\bm{Z}}\ProbC{\bm{Z}}{M_{m}}$.
This Bayesian prediction is a weighted average of the individual predictions with weights 
proportional to the posterior probability of each model. This formulation leads to a number
of different model-averaging strategies.

\paragraph{Stochastic Search: Bumping}
\emph{Bumping} uses bootstrap sampling to move randomly through model space. 
In detail we draw bootstrap samples $\left(\bm{Z}^{*j}\right)_{1\leq j\leq B}$ and fit our model
to each giving predictions $\hat{f}^{*b}(x), b\in\inter{1}{B}$. We then choose the model that 
produces the smallest prediction error, averaged over the \emph{original training set}.
$$ \hat{b}=\min\limits_{b}\su{{i=1}}{N}\left[y_{i}-\hat{f}^{*b}(x_{i})\right]^{2}$$

\subsection{lab: Non-linear modeling}
\input{./chap/1chap/6sec/8_labNonLinearModeling.tex}
\subsection{Exercises}
\input{./chap/1chap/6sec/9_exercises.tex}

\section{Tree-based-methods}
\begin{itemize}
	\item
 \end{itemize}
\subsection{The basics of decision trees}
\paragraph{Regression Trees}
\subparagraph{Predication via Stratification of the feature space }
\begin{itemize}
	\item \tB{We divide the predictor space (the set of possible 
		values for $\prth{X}{j}{p}$) into $J$ distinct and non
		overlapping regions $\prth{R}{k}{J}$.}
	\item \sB{For every observation that falls into the region
		$R_{j}$ we make the same prediction, which is simply
		the mean of the response values} for the training 
		observation in $R_{j}$
\end{itemize}
Theoretically the regions could have any shape, however we choose to
divide the predictor space into high-dimensional rectangles (boxes).
\tB{The goal is to find $\prth{R}{k}{J}$ that minimize the RSS given by
:}
\begin{center}
\encV{
$\su{{k=1}}{J}\su{{i\in R_{k}}}{}\left(y_{i}-\hat{y}_{R_{k}}\right)^{2}$}
\end{center}
where $\hat{y}_{R}$, is the mean response for the training observations
within the $k^{th}$ box.
It is computationally infeasible to consider every possible partition
of the feature space into $J$ boxes.\\
For this reason, we take an approach which is:
\begin{itemize}
	\item \textbf{\emph{top-down}}: \sB{it begins at the top of the
		tree} (at which point all observations belong to a 
		single region) \sB{and then successively splits the 
		predictor space.}
	\item \textbf{\emph{greedy}}: at each step of the tree-building
		process, the best split is made at that particular step
\end{itemize}
It is known as \tR{\emph{recursive binary splitting}}\\

We select the predictor $X_{j}$ and the cutpoint $s$ such that 
splitting the predictor space into the regions $\left\{X|X_{j}<s\right\}$ (``\emph{the region of predictor space in which} $X_{j}$ \emph{takes
on a value less than} $s$'') and $\left\{X|X_{j}>s\right\}$ leads to the greatest possible 
reduction in RSS.
For any $j$ and $s$ \sB{we define the pair of half-planes}:
\encB{$
\begin{cases}
	R_{1}(j,s)=\left\{ X|X_{j}<s \right\}\\
	R_{2}(j,s)=\left\{ X|X_{j}>s \right\}
\end{cases}$}
and \sB{we seek the value of $j$ and $s$ that minimize}:

\begin{center}
\enc{
	$\su{{i:x_{i}\in R_{1}(j,s)}}{}\left( y_{i}-\hat{y}_{R_{1}} \right)^{2} + \su{{i:x_{i}\in R_{2}(j,s)}}{}\left( y_{i}-\hat{y}_{R_{2}} \right)^{2}$}
\end{center}
where \tB{$\hat{y}_{R_{1}} = ave\left(y_{i}|x_{i}\in R_{1}(j,s)\right)$} is the mean response for the 
training observations in $R_{1}(j,s)$  

\subparagraph{Tree Pruning}
A smaller tree with fewer splits (fewer regions $\prth{R}{i}{J}$) might
lead to lower variance and better interpretation at the cost of a 
little bias.\\ A better strategy is to grow a very large tree $T_{0}$,
and then \emph{prune} it back in order to obtain a \emph{subtree}.\\
Rather than considering every possible subtree, \sB{we consider a
sequence of trees indexed by a nonnegative tuning parameter $\alpha$}.

\paragraph{Algorithm: Building a Regression Tree}
\begin{enumerate}
	\item \tB{Use recursive binary splitting} to grow a large tree on 
		the training data, \sB{stopping only when each terminal
		node has fewer than some minimum number of observations}
	\item Apply cost complexity pruning to the large tree in order
		to obtain a sequence of subtrees, as a function of 
		$\alpha$
	\item \tB{Use $K$-fold cross-validation to choose $\alpha$}.
		That is divide the training observations into $K$-folds.
		\begin{enumerate}[label=\Alph*]
			\item \sB{Repeat steps 1 and 2 on all but $k^{
				th}$ fold} of the training data.
			\item Evaluate the mean squared prediction 
				error on the data in the left-out 
				$k^{th}$, as a function of $\alpha$
		\end{enumerate}
		Average the results for each value of $\alpha$, and
		pick $\alpha$ to minimize the average error.
	\item Return the subtree from Step 2 that correspond to the
		chosen value of $\alpha$
\end{enumerate}

For each value of $\alpha$ there corresponds a subtree $T\subset T_{0}$
such that:
\begin{center}
	\enc{ $
	\su{{m=1}}{|T|}\su{{i:x_{i}\in R_{m}}}{}\left( y_{i}-\hat{y}_{Rm} \right)^{2} + \alpha|T| $}
\end{center}
is as small as possible. \tB{$|T|$ indicates the number of terminal nodes of the tree $T$.}

\begin{python}
import sklearn
from sklearn import tree

X = [[0, 0], [2, 2]] 
y = [0.5, 2.5]
reg = tree.DecisionTreeRegressor()
reg = reg.fit(X, y) 
print(reg.score(X, y))
\end{python}

\paragraph{Classification Trees}
The \tB{classification error rate} is simply the fraction of the training
observations in that region that do not belong to the most common 
class:
$E=1-\max\limits_{k}(\hat{p}_{mk})$
Here $\hat{p}_{mk}$ represents the proportion of training 
observation in the $m^{th}$ region are from the $k^{th}$ class, \tB{$\hat{p}_{mk}=\dfrac{1}{N_{m}}
\su{{x_{i}\in R_{m}}}{}I(y_{i}=k)$}.\\
It turns out that classification error is not sufficiently sensitive 
for tree-growing, and in practice 2 other measures are preferable.\\
The Gini index is defined by
$$
G=\su{{k=1}}{K}\hat{p}_{mk}(1-\hat{p}_{mk})
$$\\
The \tB{entropy} given by:
\encB{$D = -\su{{k=1}}{K}\hat{p}_{mk}\ln\left( \hat{p}_{mk} \right)$}\\
The entropy \sB{will take on a value near zero if the $\hat{p}_{mk}$'s are
all near zero or near one}.\\ Therefore, like the Gini index, the 
entropy will take on a small value if the $m^{th}$ node is pure.\\

Any of these three approaches might be used when \emph{pruning} \sB{the
tree, but the classification error rate is preferable if prediction
accuracy of the final pruned tree is the goal}.

\begin{python}
import sklearn
from sklearn import tree

X, y = load_iris(return_X_y=True)
clf = tree.DecisionTreeClassifier()
clf = clf.fit(X, y) 
print(clf.score(X, y))
tree.plot_tree(clf)
\end{python}
\paragraph{Advantages and Disadvantages of Trees}
\begin{itemize}
	\item[\tV{+}] They are very easy to explain to people.
		easily interpreted.
	\item[\tV{+}] Trees can be displayed graphically, and are 
		easily interpreted
	\item[\tV{+}] Trees can easily handle qualitative predictors 
		without the need to create dummy variables.
	\item[\tR{-}] Trees generally do not have the same level of
		predictive accuracy as some of the other regression
		and classification approaches.
	\item[\tR{-}] Trees can be non-robust, a small change in the 
		data can cause a large change in the final estimated
		tree.
\end{itemize}

\paragraph{PRIM: Bump Hunting}
Tree-based methods (for regression) partition the feature space into box-shaped regions, to try to
make the response averages in each box as different as possible.\\
The \tB{Patient Rule Introduction Method} (PRIM) also finds boxes in the feature space, but seeks 
boxes in which the response average is high. Hence it looks for maxima in the target function.
PRIM also differs from tree-based partitioning methods in that the box definition are not 
described by a binary tree.
This makes interpretation of the collection of rules more difficult; however, by removing the binary
tree constraint, the individual rules are often simpler.
\subparagraph{Example}
There are 200 data points uniformely distributed over the unit square. The color-coded plot indicates
the response $Y=
\begin{cases}
	1\text{ (red), }0.5<X_{1}<0.8\text{ and }0.4<X_{2}<0.6\\
	0\text{ (blue), otherwise}
\end{cases}$\\
The panel shows the successive boxes found by the top-down peeling procedure, peeling off a 
proportion $\alpha=0.1$ of the remaining data points at each stage.\\
After the top-down sequence is computed, PRIM reverses the process, expanding along any edges, if 
such an expansion increases the box mean, this called \emph{pasting}.
\begin{figure}[H]
	\begin{center}
		\includegraphics[width=.7\textwidth]{./chap/1chap/7sec/images/6_prim.png}
	\end{center}
	\caption{The procedure starts with a rectangle (broken black lines) surronding all of the 
	data, and then \tB{peels away points along one edge by a prespecified amout in order to
	maximize the mean of the points remaining in the box}. The iteration number is indicated at 
	the top of each panel.}
	\label{fig:6_prim}
\end{figure}
\subparagraph{Patient Rule Induction Method}
\begin{enumerate}
	\item \sB{Start with all of the training data}, and a maximal box containing all of the data
	\item Consider shrinking the box by compressing one face, so as to \sB{peel off the 
		proportion $\alpha$ of observations having either the highest values of a 
		predictor $X_{j}$ or the lowest}. Choose the peeling that produces the highest
		response mean in the remaining box. (Typically $\alpha=0.05$ or 0.10) 
	\item Repeat \sB{step 2 until some minimal number of observation remain in the box}, (say 10).
	\item Expand the box along any face, as long as the resulting box mean increases.
	\item Step 1-4 give a sequence of boxes, with different numbers of observation in each
		box. \sB{Use cross-validation to choose a member of the sequence.} Call the box
		$B_{1}$
	\item Remove the data in box $B_{1}$ from the dataset and repeat steps $2-5$ to obtain
		a second box, and continue to get as many boxes as desired.
\end{enumerate}
This produces a sequence of boxes $\prth{1}{l}{k}$, each box is defined by a set of rules involving
a subset of predictors like: 
$\begin{cases}
	a_{1}\leq X_{1}\leq b_{1}\\
	a_{3}\leq X_{3}\leq b_{3}
\end{cases}$

\paragraph{MARS: Multivariate Adaptive Regression Splines}
MARS is an \tB{adaptive procedure for regression, and is well suited for high-dimensional problems}.
MARS uses expansions in piecewise linear basis functions of the form $(x-t)_{+}$, $(t-x)_{+}$
\begin{center}
	$(x-t)_{+} = \begin{cases} x-t \Leftarrow x>t\\ 0\Leftarrow x\leq t\end{cases}$ and 
$(t-x)_{+} = \begin{cases} t-x \Leftarrow x<t\\ 0\Leftarrow x\geq t\end{cases}$
\end{center}
Therefore, the collection of basis function is: 
\begin{center}
	\encB{$ \mathcal{C}=\left\{(X_{j}-t)_{+}, (t-X_{j})_{+}|\in\left\{x_{ij}\right\}_{1\leq i\leq N}\right\}_{1\leq j\leq p}$}
\end{center}
Thus the model has the form 
\begin{center}
	\enc{$ f(X)=\beta_{0}+\su{{m=1}}{M}\beta_{m}h_{m}(X)$}
\end{center}
where each \sB{$h_{m}(X)$ is a function in $\mathcal{C}$} or a product of 2 or more such functions.\\
Given a choice for the $h_{m}$, the coefficient $\beta_{m}$ are estimated by minimizing the residual
sum-of-squares, that is, by standard linear regression.\\
At each stage we consider as a new basis function pari all products of a function $h_{m}$ in the 
model set $\mathcal{M}$ with one of the reflected pairs in $\mathcal{C}$\\
We add to the model $\mathcal{M}$ the term of the form:
\begin{center}
$\hat{\beta}_{M+1}h_{l}(X)\times(X_{j}-t)_{+}+\hat{\beta}_{M+2}h_{l}(X)\times(t-X_{j})_{+}, h_{l}\in
\mathcal{M}$
\end{center}
that produces the largest decrease in training error. Here $\hat{\beta}_{M+1}$ and $\hat{\beta}_{M+2
}$ are coefficients estimated by least squares, along with all other $M+1$ coefficient in the model.
\begin{python}
import pyearth
from pyearth import Earth

model_mars = Earth()
model_mars.fit(X, y)
\end{python}
\subparagraph{Example}
At the first stage we consider adding to the model a function of the form $\beta_{1}\left(X_{j}-t
\right)_{+}+\beta_{2}\left(t-X_{j}\right)_{+}; t\in\left\{x_{ijj}\right\}$ since multiplication by
the constant function just produces the functdion itself.\\
Suppose the best choice is $\hat{\beta}_{1}\left(X_{2}-x_{72}\right)_{+}+\hat{\beta}_{2}\left(x_{72}-X_{2}\right)_{+}$.\\
Then this pair of basis functions is added to the set $\mathcal{M}$ and the next stage we consider
including a pair of products the form:
$h_{m}(X)\times\left(X_{j}-t\right)_{+}\text{ and }h_{m}(X)\left(t-X_{j}\right)_{+}, t\in\left\{
x_{ij}\right\}$
where we have the choices:
\begin{align*}
	h_{0}(X) &= 1\\
	h_{1}(X) &= \left(X_{2}-x_{72}\right)_{+}\\
	h_{0}(X) &= \left( x_{72}-X_{2} \right)
\end{align*}

\subparagraph{Cross-validation}
One could use cross-validation to estimate the optimal value of $\lambda$, but for computational
savings the MARS procedure instead uses generalized cross-validation:
$$ GCV(\lambda) = \dfrac{\su{{i=1}}{N}\left(y_{i}-\hat{f}_{\lambda}(x_{i})\right)^{2}}{
\left(1-\frac{M(\lambda)}{N}\right)^{2}}$$
The $M(\lambda)$ is the effective number of parameters in the model: this accounts both for the
number of parameters used in selecting the optimal positions of the knots. 

\subsection{Bagging, RandomForest and Boosting}
\paragraph{Bagging}
\subparagraph{Bootstrap aggregation (bagging) definition}
It is a \tB{general-purpose procedure for reducing the variance of a
statistical learning method}; we introduce it here because it is
particularly useful and frequently used in the context of decision
trees.\\
In other words for estimating $\hat{f}(x)$, the prediction at input x, we could calculate $\left(f^{i
}(x)\right)_{1\leq i\leq B}$ using $B$ separate training sets, and average them in order to
obtain a single low-variance statistical learning model.\\
We then \sB{train our method on the $b^{th}$ bootstrapped training set in
order to get $\hat{f}^{*b}(x)$}, and finally average all the predictions
to obtain:
\begin{center}
	\encB{$\hat{f}_{bag}(x)=\dfrac{1}{B}\su{{b=1}}{B}\hat{f}^{*b}(x)$}
\end{center}
This is called bagging.

\emph{Python Code}
\begin{python}
import pandas as pd
import sklearn
from sklearn import tree
from sklearn.ensemble import BaggingClassifier,
    BaggingRegressor

y, X = df.iloc[:, 0], df.iloc[:, 1:]
bagging = BaggingClassifier(tree.DecisionTreeClassifier(),
    max_samples=0.5, max_features=0.5)
\end{python}

\subparagraph{Out-of-Bag Error Estimation}
The key to bagging is that trees are repeatedly fit to bootstrapped subsets of the observations.

\paragraph{Random Forest}
When building these decision trees, \sR{each time a split in a tree is 
considered, a random sample of $m$ predictors is chosen as split 
candidates from the full set of $p$ predictors}.\\
In building a random forest, at each split in the tree the algorithm
is \emph{not even allowed to consider} a majority of the available
predictors.

\subparagraph{Random Forest for Regression or Classification}
\begin{enumerate}
	\item For $b\in\inter{1}{B}$
		\begin{enumerate}[label=(\alph*)]
			\item \sB{Draw a bootstrap sample $\bm{Z}^{*}$ of size $N$} from the training
				data
			\item \sB{Grow a random-forest tree $T_{b}$ to the bootstrapped data}, by
				recursively repeating the following steps for each terminal mode
				of the tree, until the minimum node size $n_{min}$ is reached.
				\begin{enumerate}[label=\alph*]
					\item[i.] Select $m$ variables at random from the $p$ 
						variables
					\item[ii.] Pick the best variable/split-point among the $m$
					\item[iii.] Split the node into 2 daughter nodes.
				\end{enumerate}
		\end{enumerate}
	\item Output the ensemble of trees $\left\{T_{b}\right\}_{1}^{B}$
\end{enumerate}
\textit{Regression}: \enc{$\hat{f}_{rf}^{B}(x) = \dfrac{1}{B}\su{{b=1}}{B}T_{b}(x)$}\\
\textit{Classification}: Let $\hat{C}_{b}(x)$ be the class prediction of the $b^{th}$ 
random-forest tree, then $\hat{C}_{rf}^{B}(x)=majority\_vote\left\{\hat{C}_{b}(x)\right\}_{1}^{B}$

\subparagraph{Python Code}
\begin{python}
import pandas as pd
import sklearn
from sklearn.ensemble import RandomForestClassifier,
    RandomForestRegressor

y, X = df.iloc[:, 0], df.iloc[:, 1:]
clf = RandomForestClassifier(n_estimator=10)
\end{python}

\paragraph{Boosting}
\subparagraph{Boosting for Regression Trees}
The motivation for boosting was a procedure that combines the outputs of many ``weak'' classifiers
to produce a powerful ``committee'''. A \sB{weak classifier is one whose error rate is only slightly
better than random guessing}.
\begin{enumerate}
	\item Set $\hat{f}(x)=0$ and $r_{i}=y_{i}$ for all $i$ in the
		training set.
	\item For $b\in\inter{1}{B}$ repeat:
		\begin{enumerate}[label=\alph*]
			\item Fit a tree $\hat{f}^{b}$ with $d$ splits
				($d+1$ terminal nodes) to the training
				data (X,r)
			\item Update $\hat{f}$ by adding in a shrunken
				version of the new tree:
				$$
				\hat{f}(x)\leftarrow \hat{f}(x)+\lambda
				\hat{f}^{b}(x)
				$$
			\item Update the residuals,
				$r_{i}\leftarrow r_{i}-\lambda \hat{f}^{b}(x_{i})$
			\end{enumerate}
	\item Output the boosted model, $\hat{f}(x)=\su{{b=1}}{B}\lambda\hat{f}^{b}(x)$
\end{enumerate}

\subparagraph{Boosting methods}
Consider a two-class problem, with the output variable coded as $Y\in\left\{-1,1\right\}$. Given a
vector of predictor variables $X$, a classifier $G(X)$ produces a prediction taking one of the 
2 values $\{-1,1\}$
\begin{figure}[H]
	\begin{center}
		\includegraphics[width=.5\textwidth]{./chap/1chap/7sec/images/1_adaboost.PNG}
	\end{center}
	\caption{Schematic of AdaBoost. Classifiers are trained on weighted versions of the 
	dataset, and then combined to produce a final prediction.}
	\label{fig:1_adaboost}
\end{figure}
\begin{center}
	\encB{$ G(x) = sign\left( \su{{m=1}}{M}\alpha_{m}G_{m}(x) \right)$}
\end{center}
Here \sB{$\alpha_{m}$'s are computed by the boosting algorithm and weight the contribution of each 
respective $G_{m}(x)$}. Their effect is to give higher influence to the more accurate classifiers in
the sequence.

\subparagraph{AdaBoost.M1}
\begin{enumerate}
	\item Initialize the observation weights $w_{i}=\frac{1}{N}, i\in\inter{1}{N}$
	\item For $m\in\inter{1}{M}$:
		\begin{enumerate}[label=\alph*]
			\item Fit a classifier $G_{m}(x)$ to the training data using weights $w_{i}$
			\item Compute:
				$$ err_{m} = \dfrac{\su{{i=1}}{N}\omega_{i}I\left(y_{i}\neq G_{m}(x_{i})\right)}{\su{{i=1}}{N}\omega_{i}}$$
			\item Compute \tV{$\alpha_{m}=\log\left(\dfrac{1-err_{m}}{err_{m}}\right)$}
			\item Set \tV{$\omega_{i} \leftarrow \omega_{i}e^{\alpha_{m}I(y_{i}\neq G_{m}(x_{i}))}$ for $i\in\inter{1}{N}$}
		\end{enumerate}
	\item Output $G(x)=sign\left[\su{{m=1}}{M}\alpha_{m}G_{m}(x)\right]$
\end{enumerate}
\subparagraph{Python Code}
\begin{python}
import pandas as pd
import sklearn
from sklearn.ensemble import AdaBoostClassifier,
    AdaboostRegressor
from sklearn.model_selection import corss_val_score

y, X = df.iloc[:, 0], df.iloc[:, 1:]
clf = AdaBoostClassifier(n_estimator=100)
scores = corss_val_score(clf, X, y, cv=5)
\end{python}

\paragraph{Boosting Fits an Additive Model}
\begin{center}
\encB{$ f(x)=\su{{m=1}}{M}\beta_{m}b(x;\gamma_{m})$}
\end{center}
where $\beta_{m}$'s are the expansion coefficients and $b(x;\gamma)\in\mathbb{R}$ are usually
simple functions of the multivariate argument $x$, characterized by a set of parameters $\gamma$
\subparagraph{Forward Stagewise Additive Modeling}
\begin{enumerate}
	\item Initialize $f_{0}(x)=0$
	\item For $m\in\inter{1}{M}$
		\begin{enumerate}[label=\alph*]
			\item Compute \tV{$(\beta_{m},\gamma_{m}) = \min\limits_{\beta,\gamma}\su{{i=1}}{N}L(y_{i},f_{m-1}(x_{i})+\beta b(x_{i};\gamma))$}
			\item Set $f_{m}(x)=f_{m-1}(x)+\beta_{m}b(x;\gamma_{m})$
		\end{enumerate}
\end{enumerate}

\paragraph{Forward Stagewise Additive Modeling}
\sB{At each iteration $m$, one solves the optimal basis function $b(x;\gamma_{m})$ and corresponding
coefficient $\beta_{m}$ to add the current expansion $f_{m-1}(x)$.} This procedure $f_{m}(x)$,
and the process is repeated.
For the squared-error loss: $L(y,f(x))=\left(y-f(x)\right)^{2}$ one has 
\begin{align*}
L(y_{i},f_{m-1}(x_{i})+\beta b(x_{i};\gamma)) =& \left(y_{i}-f_{m-1}(x)-\beta 
b(x_{i};\gamma)\right)\\
=& \left(r_{im}-\beta b(x_{i},\gamma)\right)^{2}
\end{align*}
where $r_{im}=y_{i}-f_{m-1}(x_{i})$ is simply the residual of the current model on the $i^{th}$
observation.

\paragraph{Loss Functions and Robustness}
\subparagraph{Robust Loss Functions for Classification}
The minimizer of the corresponding risk on the population is :
$$ f^{*}(x)=\min\limits_{f(x)}\mathbb{E}_{Y|x}\left(Y-f(x)\right)^{2} = \E{Y|X}=2\ProbC{x}{Y=1}
-1$$
\begin{figure}[H]
	\begin{center}
		\includegraphics[width=.6\textwidth]{./chap/1chap/7sec/images/2_LossFunction.PNG}
	\end{center}
	\caption{The response is $y=\pm 1$ the prediction is $f$, with class prediction $sign(f)$.
	The losses are misclassification: $I(sign(f)\neq y)$; exponential: $exp(-yf)$; binomial
	deviance $\log(1+e^{-2yf})$; squared error $(y-f)^{2}$; and support vector: $(1-yf)_{+}$}
	\label{fig:2_LossFunction}
\end{figure}
\subparagraph{Python Code}
\begin{python}
import pandas as pd
import sklearn
from sklearn.ensemble import GradientBoostingClassifier
from sklearn.model_selection import train_test_split

y, X = df.iloc[:, 0], df.iloc[:, 1:]
X_train, X_test, y_train, y_test = train_test_split(
    X, y, random_state=0)
clf = GradientBoostingClassifier(
   loss = 'deviance', # {'deviance', 'exponential'}
   random_state = 0
)
print(clf.score(X_test, y_test))
\end{python}
\subparagraph{Robust Loss Functions for Regression}
One such criterion is the Huber loss criterion used for M-regression
$$ L(y,f(x))=
\begin{cases}
	\left(y-f(x)\right)^{2} \Leftarrow |y-f(x)|\leq \delta \\
	2\delta|y-f(x)|-\delta^{2} \Leftarrow |y-f(x)|> \delta
\end{cases}
$$
\begin{figure}[H]
	\begin{center}
		\includegraphics[width=.7\textwidth]{./chap/1chap/7sec/images/3_LossFunction_reg.PNG}
	\end{center}
	\caption{The Huber loss function combines the good properties of squared-error loss near
	zero and absolute error loss when $|y-f|$ is large}
	\label{fig:3_LossFunction_reg}
\end{figure}

\subparagraph{Python Code}
\begin{python}
import pandas as pd
import sklearn
from sklearn.ensemble import GradientBoostingRegressor
from sklearn.model_selection import train_test_split

y, X = df.iloc[:, 0], df.iloc[:, 1:]
X_train, X_test, y_train, y_test = train_test_split(
    X, y, random_state=0)
reg = GradientBoostingRegressor(
   loss = 'huber', # {'ls', 'lad', 'huber'}
   random_state = 0
)
print(reg.score(X_test, y_test))
\end{python}



\paragraph{Procedures for Data Mining}
\begin{figure}[H]
	\begin{center}
		\includegraphics[width=\textwidth]{./chap/1chap/7sec/images/4_compMeth.PNG}
	\end{center}
	\caption{Some characteristics of different learning methods.Green=good, Yellow=fair, 
	Red=poor}
	\label{fig:4_compMeth}
\end{figure}

\paragraph{Boosting Trees}
Regression and classification trees partition the space of all joint predictor variable values
into disjoint regions $R_{j},j\in\inter{1}{J}$ as represented by the terminal nodes of the tree.\\
A constant $\gamma_{j}$ is assigned to each such region and the predictive rule is :
$x\in R_{j} \Rightarrow f(x)=\gamma_{j}$. Thus a tree can be formally expressed as:
$$ T(x;\Theta)=\su{{j=1}}{J}\gamma_{j}I(x\in R_{J})$$
with parameters $\Theta=\left\{R_{j},\gamma_{j}\right\}_{1}^{J}$
The parameters are found by minimizing the empirical risk: 
$\Theta = \min\limits_{\Theta}\su{{j=1}}{J}\su{{x_{i}\in R_{j}}}{}L(y_{i},\gamma_{j})$

\paragraph{Numerical Optimization via Gradient Boosting}
The goal is to minimize $L(f)=\su{{i=1}}{N}L(y_{i},f(x_{i}))$ with respect to $f$, where here
$f(x)$ is constrained to be a sum of trees. Ignoring this constraint, minimizing $L(f)$ can be
viewed as a numerical optimization:
$$ \hat{\bm{f}} = \min\limits_{f}L(\bm{f})$$ where the parameters $\bm{f}\in\mathbb{R}^{N}$ are
the values of the approximating function $f(x_{i})$ such as $\bm{f} = \left(f(x_{i})
\right)_{1\leq i\leq N}^{T}$
Numerical optimization procedures solves the previous equation as a sum of component vectors:
$$ \bm{f}_{M} = \su{{m=0}}{M}\bm{h}_{m}, \bm{h}_{m}\in\mathbb{R}^{N}$$

\subparagraph{Steepest Descent}
Steepest descent chooses \tB{$\bm{h}_{m} = -\rho_{m}\bm{g}_{m}, (\rho_{m}, \bm{g}_{m})\in\mathbb{R}
\times\mathbb{R}^{N}$, $\bm{g}_{m}$ is the gradient of $L(\bm{f})$ evaluated at $\bm{f}=\bm{f}_{
m-1}$}. The components of the gradient $\bm{g}_{m}$ are: 
\begin{center}
\encB{$ g_{im} = \left[\dfrac{\partial L(y_{i},f(x_{i}))}{\partial f(x_{i})}\right]_{f(x_{i}=f_{m-1}(
x_{1}))}$} 
\end{center}
The step length $\rho_{m}$ is the solution to 
$ \rho_{m}=\min\limits_{\rho}L\left(\bm{f}_{m-1}-\rho\bm{g}_{m}\right)$, the current solution
is then updated 
\begin{center}
	\enc{$ \bm{f}_{m} = \bm{f}_{m-1} -\rho_{m}\bm{g}_{m}$}
\end{center}

\begin{figure}[H]
	\begin{center}
		\includegraphics[width=\textwidth]{./chap/1chap/7sec/images/5_common_gradients.PNG}
	\end{center}
	\caption{Gradients for commonly used loss functions}
	\label{fig:5_common_gradients}
\end{figure}

Gradient Tree Boosting Algorithm
\begin{enumerate}
	\item Initialize $f_{0}(x)=\min\limits_{\gamma}\su{{i=1}}{N}L(y_{i},\gamma)$
	\item For $m\in\inter{1}{M}$:
		\begin{enumerate}[label=(\alph*)]
			\item For $i\in\inter{1}{N}$ compute: $r_{im}=-\left[\dfrac{\partial L(y_{i},f(x_{i}))}{\partial f(x_{i})}\right]_{f=f_{m-1}}$
			\item  Fit a regression tree to the targets $r_{im}$ giving terminal 
				regions $R_{jm},j\in\inter{1}{J_{m}}$
			\item For $j\in\inter{1}{J_{m}}$ compute: $\gamma_{jm}=\min\limits_{
				\gamma}\su{{x_{i}}\in R_{jm}}{}L(y_{i},f_{m-1}(x_{i}) + \gamma)$
			\item Update $f_{m}(x)=f_{m-1}(x)+\su{{j=1}}{J_{m}}\gamma_{jm}I(x\in\mathbb{R}_{jm})$
		\end{enumerate}
	\item Output $\hat{f}(x)=f_{M}(x)$
\end{enumerate}


\subsection{Lab: decision trees}
\input{./chap/1chap/7sec/3_labDecisionTrees.tex}
\subsection{Exercises}
\begin{abstract}
<<>>=
print(x = 2^3)
@
\end{abstract}


\section{Support vector machines}
\subsection{Maximal margin classifier}
\paragraph{Definition}
It is \tB{the separating hyperplane that is farthest from the training
observations}. We can compute the (perpendicular) distance from each
training observation to a given separating hyperplane; the smallest
such distance is the minimal distance from the observations to the
hyperplane, and is known as the margin.\\
In a sense, the maximal margin hyperplane represents the mid-line of 
the widest ``slab'' that we can insert between 2 classes.

\begin{figure}[H]
	\begin{center}
		\includegraphics[width=\textwidth]{./chap/1chap/8sec/images/1margineSVM.png}
	\end{center}
	\caption{The margin is distance from the solid line to either
	of the dashed lines. The 2 points and the purple point that 
	lie on the dashed lines are the support vectors}
	\label{fig:8.1margineSVM}
\end{figure}

\paragraph{Construction of the Maximal Margin Classifier}
Set of $n$ observations $
\begin{pmatrix}
	x_{1}
	.
	.
	.
	x_{p}
\end{pmatrix}\in\mathbb{R}^{p}
$ and associated class labels $\prth{y}{i}{n}\in\{-1,1\}$
The maximal margin hyperplane is the solution to the optimization 
problem:
\begin{center}
$ \max\limits_{\prtH{\beta}{i}{0}{p},M}M 
\text{ subject to: }
\begin{cases}
	\su{{j=1}}{p}\beta_{j}^{2}=1~(1)\\
	\forall i\in\inter{1}{n},~y_{i}(\beta_{0}+\su{{j=1}}{p}\beta_{j}x_{ij})>M~(2)
\end{cases} $
\end{center}

(2) \sB{guarantees that each observation will be on the correct side of
the hyperplane, provided that $M\geq 0$}\\

\begin{figure}[H]
	\begin{center}
		\includegraphics[width=\textwidth]{./chap/1chap/8sec/images/21_svC.PNG}
	\end{center}
	\caption{The left panel shows the separable case. The decision boundary is the solid line,
	while broken lines bound the shaded maximal margin of width $2M=\frac{2}{\norm{\beta}}$.
	The right panel shows the \emph{non-separable} case. The point labeled $\xi_{j}^{*}$ are
	on the wrong side of their margin by an amount $\xi_{j}^{*}=M\xi_{j}$; points on the 
	correct side have $\xi_{j}^{*}=0$. The margin is maximized subject to a total budget
	$\su{}{}\xi_{i}\leq constant$. Hence $\su{}{}\xi_{i}^{*}$ is the total distance of points
	on the wrong side.}
	\label{fig:21_svC}
\end{figure}
We can drop the norm constraint on $\beta$, and instead define $M=\dfrac{1}{\norm{\beta}}$:
$$\min\norm{\beta}\text{ subject to }
\begin{cases}
	\forall i, y_{i}\left(x_{i}^{T}\beta+\beta_{0}\right)\geq M\left(1-\xi_{i}\right)\\
	\xi_{i}\geq 0, \su{}{}\xi_{i}\leq constant
\end{cases}
$$
This is the usual way the support vector classifier is defined from the non-separable case.
\subparagraph{Computing the Support Vector Classifier}
Computationally it is convenient to re-express previous equation:
\begin{center}
\tB{$ \min\limits_{\beta,\beta_{0}}\dfrac{1}{2}\norm{\beta}^{2} + C\su{{i=1}}{N}\xi_{i}\text{ subject
to }\forall i,\xi_{i}\geq 0, y_{i}\left(x_{i}^{T}\beta+\beta_{0}\right)\leq M\left(1-\xi_{i}
\right)$}
\end{center}
where the ``cost'' parameter $C$ replaces the constraint; the separable case corresponds
to $C=\infty$\\
We describe a quadratic programming solution using Lagrange multipliers.
The Lagrange (primal) function is:
\begin{center}
	\enc{$ L_{p}=\dfrac{1}{2}\norm{\beta}^{2} + C\su{{i=1}}{N}\xi_{i}-\su{{i=1}}{N}\alpha_{i}\left[y_{i}^{T}\beta+\beta_{0}-(1-\xi_{i})\right]-\su{{i=1}}{N}\mu_{i}\xi_{i}$}
\end{center}
Setting the respective derivatives to zero we get : 
$
\begin{cases}
	\beta = \su{{i=1}}{N}\alpha_{i}y_{i}x_{i}\\
	0 = \su{{i=1}}{N}\alpha_{i}y_{i}\\
	\alpha_{i} = C-\mu_{i}, \forall i
\end{cases}
$

By inserting the equation system in the Lagrange function we obtain:
$L_{D} = \su{{i=1}}{N}\alpha_{i} - \frac{1}{2}\su{{i=1}}{N}\su{{j=1}}{N} \alpha_{i}\alpha_{j}
y_{i}y_{j}x_{i}^{T}x_{j}$ which gives a lower bound on the objective function (computational 
re-expression) for any feasible point.


\subsection{Support vectors classifiers}
\paragraph{Aim}
Considering that the maximal margin hyperplane is extremely sensitive
to a change in a single observation, it may have overfit the training
data.\\
In this case we might be willing to consider a classifier based on a
hyperplane that does not perfectly separate the 2 classes therewith to
get 
\begin{itemize}
	\item \tB{Greater robustness} to individual observation
	\item \tB{Better classification} of most of the training observations
\end{itemize}
And the \emph{Support Vector Classifier} does exactly this.

\paragraph{Details}
The support vector classifier classifies a test observation depending
on which side of the hyperplane it lies.
\begin{figure}[H]
	\begin{center}
		\includegraphics[width=\textwidth]{./chap/1chap/8sec/images/2supportVectorClassifier.png}
	\end{center}
	\caption{Left:Purple observations: 3, 4, 5 and 6 are on the 
	correct side, 2 is on the margin and 1 is on the wrong side.\\
	Blue observations: 7 and 10 are on the correct side, 9 on the
	margin and 8 on the wrong side\\
	Right: same as left panel with two additional points, 11 and 12
	which are on the wrong side}
	\label{fig:8.1 supportVectorClassifier}
\end{figure}

It is the solution to the optimization problem
\begin{center}
\enc{$
\max\limits_{\prtH{\beta}{i}{0}{p}\prtH{\epsilon}{i}{1}{n},M}~M~
\text{ subject to }
\begin{cases}
	\su{{j=1}}{p}\beta_{j}^{2}=1\\
	y_{i}\left( \beta_{0}+\su{{j=1}}{p}\beta_{j}x_{ij} \right) >
	M(1-\epsilon_{i})\\
	\epsilon_{i}\geq 0, \su{{i=1}}{n}\epsilon_{i}\leq C
\end{cases}$}
\end{center}
where $C$ is a nonnegative tuning parameter, $M$ is the width of the 
margin.\\
$\prth{\epsilon}{i}{n}$ are \emph{slack variables} that allow 
individual observations to be on the wrong side of the margin or the
hyperplane.
We have:
$$
\begin{cases}
	\tB{\epsilon_{i}=0}\Rightarrow i^{th}\text{ observation is on the
	\sB{correct side of the margin}}\\
	\tB{\epsilon_{i}>0}\Rightarrow i^{th}\text{ observation is on the
	\sB{wrong side of the margin}}\\
	\tB{\epsilon_{i}>1}\Rightarrow i^{th}\text{ observation is on the
	\sB{wrong side of the hyperplane}}\\
\end{cases}
$$
$C$ determines the number and severity of the violations to the margin
that we will tolerate.\\

In practice $C$ is treated as a tuning parameter that is generally 
chosen via cross-validation.\\

It turns out an observation that only observations that lies strictly
on the correct side of the margin does not affect the support vector
classifier.

\subsection{Support vectors machines}
\paragraph{Classification with non-linear decision boundaries}
Rather than fitting a support vector classifier using $p$ features
we could instead fit a support vector classifier using $2p$ features,
then we get for $\left(X_{i},X_{i}^{2}\right)_{0\leq i\leq p}$:
$$
\max\limits_{\left(\beta_{i1}\right)_{0\leq i\leq p}\left(\beta_{i2}\right)_{1\leq i\leq p}\prtH{\epsilon}{i}{1}{n},M} M\\
\text{ subject to }
\begin{cases}
	\su{{j=1}}{p}\su{{k=1}}{2}\beta_{jk}^{2}=1\\
	y_{i}\left( \beta_{0}+\su{{j=1}}{p}\beta_{j1}x_{ij}+\su{{j=1}}{p}\beta_{j2}x_{ij}^{2}\right) >
	M(1-\epsilon_{i})\\
	\epsilon_{i}\geq 0, \su{{i=1}}{n}\epsilon_{i}\leq C
\end{cases}
$$

\paragraph{Support Vector Machine}
The linear \tB{support vector classifier} can be represented as :
\begin{center}
\enc{$
f(x)=\beta_{0} + \su{{i\in\mathcal{S}}}{}\alpha_{i}\sP{x_{i}}{x_{i'}}$}
\end{center}
where $\prth{\alpha}{i}{n}$ are parameters.\\
It turns out that $\alpha_{i}$ is nonzero only for the support vectors
in the solution. So \sB{$\mathcal{S}$ is the collection of indices of 
support vectors in the solution}.\\

We replace inner product with a generalization of the inner product
of the form:
$K(x_{i},x_{i'})$
Where $K$ is some function that we will refer to as a \emph{kernel}.
A kernel is a function that quantifies the similarity of 2 observations

\subparagraph{Linear Kernel}
$K(x_{i},x_{i'})=\tB{\su{{j=1}}{p}x_{ij}x_{i'j}}$ essentially \sB{quantifies
the similarity of a pair of observations} using Pearson (standard)
correlation.
\subparagraph{Polynomial Kernel}
$K(x_{i},x_{i'})=\tB{\left( 1 + \su{{j=1}}{p}x_{ij}x_{i'j} \right)^{d}}$
instead of the standard linear kernel in the support vector classifier
algorithm leads to a \sB{much more flexible decision boundary}.
\subparagraph{Radial Kernel}
$K(x_{i},x_{i'})=\tB{\exp\left( -\gamma\su{{j=1}}{p}(x_{ij}x_{i'j})^{2} \right)}$
with $\gamma\geq 0$, the radial kernel has \sB{very local behavior}, in the
sens that only nearby training observations have an effect on the class
label of a test observation.
\begin{figure}[H]
	\begin{center}
		\includegraphics[width=\textwidth]{./chap/1chap/8sec/images/3polynomialRadialKernel.png}
	\end{center}
	\caption{Left: a SVM with a polynomial kernel of degree 3 is
	applied to non-linear data.\\
	Right: a SVM  with a radial kernel is applied.}
	\label{fig:8.3polynomialRadialKernel}
\end{figure}

\subparagraph{Python Code}
\begin{python}
import pandas as pd
import sklearn
from sklearn import tree
from sklearn import svm

y, X = df.iloc[:, 0], df.iloc[:, 1:]
clf = svm.SVC(
   C = 1.0, # Regularization parameter
   kernel = 'linear', # {'linear', 'poly', 'rbf'}
)
clf.fit(X, y)

\end{python}

\subsection{SVMs with more than 2 classes}
\paragraph{One-Versus-One Classification}
Suppose that we would like to perform classification using SVMs, and
there are $K>2$ classes.\\
We classify a test observation using each of the $\binom{K}{2}$ 
classifier and we tally the number of times that the test observation
is assigned to each of the $K$ classes. The final classification is 
performed by assigning the test observation to the class to which it
was most frequently assigned in these $\binom{K}{2}$ pairwise
classifications.

\paragraph{One-Versus-All Classification}
We fit $K$ SVMs, each time comparing one of the $K$ classes to the 
remaining $K-1$ classes.

\subsection{Relationship to Ridge Regression}
\input{./chap/1chap/8sec/5_relationshipToLogisticRegression.tex}
%\subsection{lab: support vector machines}
%\input{./chap/1chap/8sec/6_labSupportVectorMachines.tex}
%\subsection{Exercises}
%\paragraph{Conceptual}
\begin{enumerate}
 	\item The null hypothesis corresponding to p-values of the 
		TABLE 3.4 is\\\emph{$H_{0}:\beta_{TV}=\beta_{radio}=
		\beta_{newspaper}=0$}.\\Newspaper contribution is 
		staticaly few sigificant in comparison with radio and
		TV.
	\item \emph{KNN classifier}: given a test observation $x_{0}$
		it identifies the K point in the training data that are
		closet to $x_{0}$ represented by $\mathcal{N}_{0}$.
		And then the conditional probability for class $j$ as
		the fraction of points in $\mathcal{N}_{0}$ whose
		response values equal $j$.\\\emph{KNN regression}:
		it estimates $f\left(x_{0}\right)$ using average of all
		the training responses in $\mathcal{N}_{0}$.\\Therefore
		\emph{KNN classifier} make a prediction on the class 
		$j$ of $x_{0}$, whereas \emph{KNN regression} make 
		prediction on the value on $f\left(x_{0}\right)$. 
	\item 
		\begin{itemize}
			\item[(a)]
				\begin{itemize}
					\item[i.] FALSE
					\item[i.] TRUE
					\item[i.] TRUE
					\item[i.] FALSE
				\end{itemize}
			\item[(b)] Salary prediction for a female with
				$IQ=110$ and $GPA=4.0$:\\
				$salary=50+20\times 4.0+0.07\times 110
				+35+0.01\times(110\times4.0)-10\times
				(4.0\times 1)=137.1$
			\item[(c)] We should rather observe the p-value
				for the GPA/IQ interaction term and
				after that decide if despite the low
				value of this estimate we can ignore
				this interaction effect.
		\end{itemize}
	\item 
		\begin{itemize}	
			\item[(a)] Knowing that the true relationship 
				between $X$ and $Y$ is linear, cubic 
				regression will have a unsuitable curve
				, because the real form of the data is
				a straight line.\\Thus the training RSS
				from cubic regression will be greater
				than this of simple regression.
			\item[(b)] The response remains the same.
			\item[(c)] This time we would expect that the
				training RSS from cubic regression
				have a lower value than that of linear
				regression.
			\item[(d)] The response remains the same.
		\end{itemize}	
	\item 
	\item 
	\item 
 \end{enumerate}


\section{Unsupervised learning}
\begin{itemize}
	\item
 \end{itemize}
\subsection{The challenge of unsupervised learning}
Unsupervised learning is often performed as part of an 
\emph{explaratory data analysis}.

\subsection{Principal compoments analysis}
Refers to the process by which principal components are computed, and
the subsequent use of these components in understanding the data.

\paragraph{Definition of principal components}
PCA finds a low-dimensional representation of a data set that contains
as much as possible of the variation. PCA seeks a small number of 
dimensions that are interesting as possible.\\

The \emph{first principal component} of a set of features 
$\prtH{X}{i}{1}{p}$ is the \emph{normalized} linear combination of the
features:
\begin{center}
\enc{$
Z_{1}=\su{{i=1}}{p}\phi_{i1}X_{i}
$}
\end{center}
that has the largest variance. \sB{By normalized}, we mean that 
\tB{$\su{{j=1}}{p}\phi_{j1}^{2}=1$} \\

The first principal component loading vector solves the optimization
problem:
\begin{center}
\enc{
$
\max\limits{\left(\phi_{i1}\right)_{1\leq i\leq n}}\left\{ \dfrac{1}{n}\su{{i=1}}{n}\left( \su{{j=1}}{p}\phi_{j1}x_{ij} \right)^{2} \right\}\text{ subject to }\su{{j=1}}{p}\phi_{j1}^{2}=1
$}
\end{center}
It turns out that constructing $Z_{2}$ to be uncorrelated with $Z_{1}$
is equivalent to constraining the direction $\phi_{2}$ to be orthogonal
to the direction $\phi_{1}$\\

The principal components of a set of data in $\mathbb{R}^{p}$ provide a sequence of best linear
approximations t that data, of all ranks $q\leq p$\\
Consider the rank-\emph{q} linear model:
$$ f(\lambda)=\mu + \bm{V}_{q}\lambda$$
where $\mu$ is a location vector in $\mathbb{R}^{p}$, $\bm{V}_{q}$ is a $p\times q$ orthogonal
unit vectors, and $\lambda$ is a $q$ vector of parameters. Fitting such a model to the data by
least squares amounts to minimizing the \textit{reconstruction error}:
$$ \min\limits_{\mu,\{\lambda_{i}\},\bm{V}_{q}}\su{{i=1}}{N}\norm{x_{i}-\mu-\bm{V}_{q}\lambda_{i}}
^{2}$$

\paragraph{Another interpretation of principal components}
An alternative interpretation for principal components can also be 
useful: principal components provide low-dimensional linear surfaces
that are \emph{closet} to the observations.\\
With the first principal component we seek a single dimension of the 
data that lies as close as possible to all of the data points.

\paragraph{More on PCA}
\subparagraph{Scaling the variables}
The results obtained when we perform PCA will also depend on whether
the variables have been individually scaled.
\subparagraph{Uniqueness of the principal component}
The sign has no effects as the direction does not change.
\subparagraph{The proportion of variance explained}
We are interested in knowing the \emph{Proportion of Variance
Explained} PVE by each principal component. \\
The total variance present in a daa set (assuming that the variables
have been centered to have mean zero) is defined as:
$$
\su{{j=1}}{p}\V{X_{j}}=\su{{j=1}}{p}\dfrac{1}{n}\su{{i=1}}{n}x_{ij}^{2}
$$

The variance explained by the $m^{th}$ principal component is:
$$
\dfrac{1}{n}\su{{i=1}}{n}z_{im}^{2}=\dfrac{1}{n}\su{{i=1}}{n}\left( \su{{j=1}}{p}\phi_{jm}x_{ij} \right)^{2}
$$

PVE:
$$
\dfrac{\su{{i=1}}{n}\left( \su{{j=1}}{p}\phi_{jm}x_{ij} \right)^{2}}{\su{{j=1}}{p}\su{{i=1}}{n}x_{ij}^{2}}
$$

\subparagraph{Deciding how many principal component to use}
We tipycally decide on the number of principal components required
to visualize the data by examining a \emph{scree pot}


\subparagraph{Python Code}
\begin{python}
import pandas as pd
import sklearn
from sklearn.decomposition import PCA

y, X = df.iloc[:, 0], df.iloc[:, 1:]
pca = PCA(
   n_components=2,
   svd_solver='full' # {'auto', 'full', 'arpack', 'randomized'}
   )
pca.fit(X)
print(pca.explained_variance_ratio_)
\end{python}

\subsection{Clustering methods}
\input{./chap/1chap/9sec/3_clusteringMethods.tex}
\subsection{Lab1: principal compoments analysis}
\input{./chap/1chap/9sec/4_lab1PrincipalCompomentsAnalysis.tex}
\subsection{Lab2: clustering}
\input{./chap/1chap/9sec/5_lab2Clustering.tex}
\subsection{Lab3: NCI60 data example}
\input{./chap/1chap/9sec/6_lab3NCI60DataExample.tex}
\subsection{Exercises}
\paragraph{Conceptual}
\begin{enumerate}
 	\item The null hypothesis corresponding to p-values of the 
		TABLE 3.4 is\\\emph{$H_{0}:\beta_{TV}=\beta_{radio}=
		\beta_{newspaper}=0$}.\\Newspaper contribution is 
		staticaly few sigificant in comparison with radio and
		TV.
	\item \emph{KNN classifier}: given a test observation $x_{0}$
		it identifies the K point in the training data that are
		closet to $x_{0}$ represented by $\mathcal{N}_{0}$.
		And then the conditional probability for class $j$ as
		the fraction of points in $\mathcal{N}_{0}$ whose
		response values equal $j$.\\\emph{KNN regression}:
		it estimates $f\left(x_{0}\right)$ using average of all
		the training responses in $\mathcal{N}_{0}$.\\Therefore
		\emph{KNN classifier} make a prediction on the class 
		$j$ of $x_{0}$, whereas \emph{KNN regression} make 
		prediction on the value on $f\left(x_{0}\right)$. 
	\item 
		\begin{itemize}
			\item[(a)]
				\begin{itemize}
					\item[i.] FALSE
					\item[i.] TRUE
					\item[i.] TRUE
					\item[i.] FALSE
				\end{itemize}
			\item[(b)] Salary prediction for a female with
				$IQ=110$ and $GPA=4.0$:\\
				$salary=50+20\times 4.0+0.07\times 110
				+35+0.01\times(110\times4.0)-10\times
				(4.0\times 1)=137.1$
			\item[(c)] We should rather observe the p-value
				for the GPA/IQ interaction term and
				after that decide if despite the low
				value of this estimate we can ignore
				this interaction effect.
		\end{itemize}
	\item 
		\begin{itemize}	
			\item[(a)] Knowing that the true relationship 
				between $X$ and $Y$ is linear, cubic 
				regression will have a unsuitable curve
				, because the real form of the data is
				a straight line.\\Thus the training RSS
				from cubic regression will be greater
				than this of simple regression.
			\item[(b)] The response remains the same.
			\item[(c)] This time we would expect that the
				training RSS from cubic regression
				have a lower value than that of linear
				regression.
			\item[(d)] The response remains the same.
		\end{itemize}	
	\item 
	\item 
	\item 
 \end{enumerate}


\section{Neural Networks}
\subsection{Projection Pursuit Regression}
Let $\omega_{m}$ with $m\in\inter{1}{M}$ be unit \emph{p}-vectors of unknown parameters.
The \tB{projection pursuit regression} (PPR):
\begin{center}
	\enc{$ f(X) = \su{{m=1}}{M}g_{m}(\omega_{m}^{T}X)$}
\end{center}
\sB{This is an additive model}, but in the derived features $V_{m}=\omega_{m}^{T}X$ rather than the
inputs themeselves.
\sB{The functions $g_{m}$ are unspecified and are estimated along with the directions $\omega_{m}$
using some flexible smoothing method.} The function $g_{m}(\omega_{m}^{T}X)$ is called a ridge
function in $\mathbb{R}^{p}$
The \sB{scalar variable $V_{m}=\omega_{m}X$ is the projection of $X$ onto the unit vector 
$\omega_{m}$} an we seek $\omega_{m}$ so that the model fits well, hence the name ``projection 
pursuit''.
\begin{figure}[H]
	\begin{center}
		\includegraphics[width=\textwidth]{./chap/1chap/99sec/images/1_ridge_fct.PNG}
	\end{center}
	\caption{
	Perspective plots of 2 ridge functions.\\ Left: $g(V)=\dfrac{1}{1+e^{-5(V-0.5)}}$,
	where $V=\frac{X_{1}+X_{2}}{\sqrt(2)}$\\ Right: $g(V)=(V+0.1)\sin\left(\dfrac{1}{
	\frac{1}{3}+0.1}\right)$, where $V=X_{1}$}
	\label{fig:1_ridge_fct}
\end{figure}
We seek to approximate minimzers of the error function:
\begin{center}
	\encB{$ \su{{i=1}}{N}\left[y_{i}-\su{{m=1}}{M}g_{m}(\omega_{m}^{T}x_{i})\right]^{2}$}
\end{center}
over functions $g_{m}$ and direction vectors $\omega_{m}$.\\
Let $\omega_{old}$ be the current estimate for $\omega$.
$$ g(\omega^{T}x_{i})\approx g(\omega_{old}x_{i}) + g'(\omega_{old}^{T}x_{i})(\omega-
\omega_{old})^{T}x_{i}$$

\subsection{Neural Networks}
They are a large class of nonlinear statistical models much like the projection pursuit regression
model.\\
A neural network is a two-stage regression or classification model, represented by a network diagram:
\begin{figure}[H]
	\begin{center}
		\includegraphics[width=.7\textwidth]{./chap/1chap/99sec/images/2_networkDiag.png}
	\end{center}
	\caption{Schematic of a single hidden layer, feed-forward neural network}
	\label{fig:99_2_networkDiag}
\end{figure}

For $K$-class classification, there are $K$ units at the top, with $k^{th}$ unit modeling the 
probability of class $k$. There are $K$ target measurements $\left\{Y_{k}:k\in\inter{1}{k}\right\}
\in {0, 1}^{K}$.\\
Derived features $Z_{m}$ are created from linear combinations of the $Z_{m}$:\\
$
\begin{cases}
	\forall m\in\inter{1}{M},~\bm{Z}_{m} = \sigma\left(\alpha_{0m}+\alpha_{m}^{T}\bm{X}\right)\\
	\forall k\in\inter{1}{K},~\bm{T}_{k} = \beta_{0k} + \beta_{k}^{T}\bm{Z}\\
	\forall k\in\inter{1}{K},~f_{k}(\bm{X}) = g_{k}(\bm{T})
\end{cases}
$\\
where 
$
Z =
\begin{pmatrix}
	Z_{1}\\
	\vdots\\
	Z_{M}
\end{pmatrix}
$ and 
$
T =
\begin{pmatrix}
	T_{1}\\
	\vdots\\
	T_{K}
\end{pmatrix}
$ 
The \tB{activation function} is usually chosen to be the sigmoid \encB{$\sigma(v) =
\dfrac{1}{1+e^{-v}}$}, the \tB{softmax} function is $g_{k}(T)=\dfrac{e^{T_{k}}}{\su{{l=1}}{K}e^{T_{l}}}$

\subsection{Fitting Neural Networks}
The neural network model has unknown parameters often called \emph{weights} and we seek values
for them that make the model fit the training data well. We denote the complete set of of 
weights by $$\theta=
\begin{cases}
	\left\{(\alpha_{0m},\alpha_{m})|m\in\inter{1}{M}\right\}\\
	\left\{(\beta_{0k},\beta_{k})|k\in\inter{1}{K}\right\}
\end{cases}
$$
A aour measure of fit we use:
$$
\begin{cases}
	R(\theta)=\su{{k=1}}{K}\su{{i=1}}{N}\left(y_{ik}-f_{k}(x_{i})\right)^{2}\text{ 
	regression}\\
	R(\theta)=-\su{{k=1}}{K}\su{{i=1}}{N}y_{ik}\log\left(f_{k}(x_{i})\right)\text{ 
	classification}
\end{cases}
$$
\tB{The generic approach to minimizing $R(\theta)$ is by gradient descent called 
\emph{back-propagation} in this setting.}\\
Here is back-propagation in detail for squared error loss.\\
Let $z_{mi}=\sigma(\alpha_{0m}+
\alpha_{m}^{T}x_{i})$ and let $z_{i}=(z_{1i},\cdots,z_{Mi})$
with derivatives:
$$
\begin{cases}
	\dfrac{\partial R_{i}}{\partial\beta_{km}} = -2\left(y_{ik}-f_{k}(x_{i})\right)
	g_{k}^{'}\left(\beta_{k}^{T}z_{i}\right)z_{mi}\\
	\dfrac{\partial R_{i}}{\partial\alpha_{ml}} = -\su{{k=1}}{K}2\left(y_{ik}-f_{k}(x_{i})
	\right)g_{k}^{'}\left(\beta_{k}^{T}z_{i}\right)\beta_{km}\sigma^{'}\left(\alpha_{m}^{T}
	x_{i}\right)x_{il}
\end{cases}
$$
Given these derivatives, a gradient descent update at the $(r+1)^{st}$ iteration has the form:
\begin{center}
\encB{$
\begin{cases}
	\beta_{km}^{(r+1)} = \beta_{km}^{(r)} - \gamma_{r}\su{{i=1}}{N}\dfrac{\partial R_{i}}{
	\partial\beta_{km}^{(r)}}\\
	\alpha_{lm}^{(r+1)} = \alpha_{lm}^{(r)} - \gamma_{r}\su{{i=1}}{N}\dfrac{\partial R_{i}}{
	\partial\alpha_{lm}^{(r)}}
\end{cases}
$}
\end{center}
where $\gamma_{r}$ is the \tB{\textit{learning rate}}. Now we write derivatives as:
$$
\begin{cases}
	\dfrac{\partial R_{i}}{\partial\beta_{km}}=\delta_{ki}z_{mi}\\
	\dfrac{\partial R_{i}}{\partial\alpha_{ml}}=s_{mi}x_{il}
\end{cases}
$$
The quantities $\delta_{ki}$ and $s_{mi}$ are ``errors'' from the current model at the output
and hidden layer units, respectively. These errors satisfy:
\begin{center}
	\enc{
$
s_{mi}=\sigma^{'}\left(\alpha_{m}^{T}x_{i}\right)\su{{k=1}}{K}\beta_{km}\delta_{ki}$}
\end{center}
known as the \tR{\emph{back-propagation equations}}.


\subparagraph{Python Code}
\begin{python}
import pandas as pd
import sklearn
from sklearn.model_selection import train_test_split
from sklearn.neural_network import MLPClassifier,
    MLPRegressor

y, X = df.iloc[:, 0], df.iloc[:, 1:]
clf = MLPClassifier(
   random_state=1,
   max_iter=300)
clf.fit(X_train, y_train)
print(clf.score(X_test, y_test))
\end{python}



\chapter{Deep Learning}
Consider $p_{data}(\bm{x})$ the distribution by which examples $\left\{x_{1},\cdots,
x_{n}\right\}$ are drawn.
Let $p_{model}(\bm{x}, \bm{\theta})$ be a parametric family of probability distribution 
over the same space indexed by $theta$.
from 
maps any configuration
\section{Deep Forward Networks}
\subsection{Gradient based learning}
\paragraph{Cost functions}
In most cases our parametric model defines a distribution $p(x\lvert x;\bm{\theta})$ and
we simply use the principle of maximum likelihood, namely the \textit{cross-entropy} 
between the training data and the model's prediction as the cost function.

\subparagraph{Learning conditional distributions with Maximum Likelihood}
The cost function is simply the negative log-likelihood:
\begin{center}
	$J(\theta) = -\mathbb{E}_{\bm{X},\bm{y}\sim\hat{p}_{data}}\left(
	\log\left(p_{model}(\bm{y}|\bm{x})\right)\right)$
\end{center}


\section{Regularization for Deep Learning}
\input{./chap/2chap/2_RegularizationForDeepLearning.tex}
\section{Optimization for training Deep Learning}
\input{./chap/2chap/3_OptimizationForTrainngDeepModels.tex}
\section{Convolution Networks}
\input{./chap/2chap/4_ConvolutionNetworks.tex}
\section{Recurrent and Recursive Nets}
\input{./chap/2chap/5_RecurrentAndRecursiveNets.tex}

\end{document}
