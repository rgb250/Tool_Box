\paragraph{High-Dimensional Data}
Data sets containing more features than observations are often referred
to as \emph{high-dimensional}. \sB{Classical approaches such as least 
squares linear regression are not appropriate in this setting.}

%\paragraph{What goes wrong in high dimensions?}
%Regression and classification when $p>n$, we begin by examining what
%can go wring if we apply a statistical technique not intended for the
%high-dimensional setting.

\paragraph{Regression in High Dimensions}
\sB{Ridge regression, the lasso and principal components regression, are
particularly useful} for performing regression in the high-dimensional
setting. Essentially these approaches avoid overfitting by using a 
less flexible fitting approach than least squares.

%\paragraph{Interpreting Results in High Dimensions}
%At most, we can hope to assign large regression coefficients to 
%variables that are correlated with the variables that truly are 
%predictive of the outcome.
