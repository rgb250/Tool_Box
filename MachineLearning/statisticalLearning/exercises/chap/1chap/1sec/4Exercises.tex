\documentclass[a4paper, 10pt]{scrartcl}\usepackage[]{graphicx}\usepackage[]{color}
%% maxwidth is the original width if it is less than linewidth
%% otherwise use linewidth (to make sure the graphics do not exceed the margin)
\makeatletter
\def\maxwidth{ %
  \ifdim\Gin@nat@width>\linewidth
    \linewidth
  \else
    \Gin@nat@width
  \fi
}
\makeatother

\definecolor{fgcolor}{rgb}{0.345, 0.345, 0.345}
\newcommand{\hlnum}[1]{\textcolor[rgb]{0.686,0.059,0.569}{#1}}%
\newcommand{\hlstr}[1]{\textcolor[rgb]{0.192,0.494,0.8}{#1}}%
\newcommand{\hlcom}[1]{\textcolor[rgb]{0.678,0.584,0.686}{\textit{#1}}}%
\newcommand{\hlopt}[1]{\textcolor[rgb]{0,0,0}{#1}}%
\newcommand{\hlstd}[1]{\textcolor[rgb]{0.345,0.345,0.345}{#1}}%
\newcommand{\hlkwa}[1]{\textcolor[rgb]{0.161,0.373,0.58}{\textbf{#1}}}%
\newcommand{\hlkwb}[1]{\textcolor[rgb]{0.69,0.353,0.396}{#1}}%
\newcommand{\hlkwc}[1]{\textcolor[rgb]{0.333,0.667,0.333}{#1}}%
\newcommand{\hlkwd}[1]{\textcolor[rgb]{0.737,0.353,0.396}{\textbf{#1}}}%
\let\hlipl\hlkwb

\usepackage{framed}
\makeatletter
\newenvironment{kframe}{%
 \def\at@end@of@kframe{}%
 \ifinner\ifhmode%
  \def\at@end@of@kframe{\end{minipage}}%
  \begin{minipage}{\columnwidth}%
 \fi\fi%
 \def\FrameCommand##1{\hskip\@totalleftmargin \hskip-\fboxsep
 \colorbox{shadecolor}{##1}\hskip-\fboxsep
     % There is no \\@totalrightmargin, so:
     \hskip-\linewidth \hskip-\@totalleftmargin \hskip\columnwidth}%
 \MakeFramed {\advance\hsize-\width
   \@totalleftmargin\z@ \linewidth\hsize
   \@setminipage}}%
 {\par\unskip\endMakeFramed%
 \at@end@of@kframe}
\makeatother

\definecolor{shadecolor}{rgb}{.97, .97, .97}
\definecolor{messagecolor}{rgb}{0, 0, 0}
\definecolor{warningcolor}{rgb}{1, 0, 1}
\definecolor{errorcolor}{rgb}{1, 0, 0}
\newenvironment{knitrout}{}{} % an empty environment to be redefined in TeX

\usepackage{alltt}  %livre

% Permet l'encodage avec le Unicode Transformation Format-8-bit
\usepackage[utf8]{inputenc}
% Permet la gestion des caractères accentués ainsi que la stabilité des impressions en P.D.F.
\usepackage[T1]{fontenc}
% Permet la stabilisation de l'écriture
\usepackage{lmodern, textcomp}
% Permet d'utiliser les liens hypertexte sans altérer la bibliothèque KOMA
\usepackage{scrhack}
\KOMAoptions{hyperref=false}
% Utilise les règles gramaticales françaises
\usepackage[french]{babel}

%Utilise les règles typographique de la bibliothéques KOMA
\setkomafont{author}{\scshape}
%\usepackage{blindtext}

% Package pour avoir plus d'arguments dans ses commandes.
\usepackage{xargs}
%Package pour plus de souplesse
\usepackage{tabularx}

%Package pour les listes indexées
\usepackage{enumitem}

%Bibliothèque mathématiques pour la police.
\usepackage{amsfonts}
%Bibliothèques mathématiques générale.
\usepackage{amsmath}
\usepackage{amssymb}
%Pour appliquer \mathbb à des nombres
\usepackage{bbm}
%Package pour l'indexation de matrices
\usepackage{blkarray}
%\usepackage{dsfont}
%Package pour spreadsheet
\usepackage{spreadtab}
\usepackage{fp}
\usepackage{xstring}
\renewcommand\STprintnum[1]{\numprint{#1}}
\STsetdecimalsep{,}
\STautoround{6}
\usepackage{eurosym}
%Bibliothèque pour l'affichage de nombre avec la typographie définie
\usepackage{numprint}
\newcommand{\np}[1]{\numprint{#1}}
% Pour la notation scientifique
\usepackage{siunitx}
\sisetup{locale= FR,exponent-product=., unit-mode = text}
\DeclareSIUnit\year{ann\'{e}ee}
%Positionnement des images
\usepackage{float}
\usepackage{subcaption}
%Utilisation des couleurs
\usepackage{xcolor}
% Package pour le soulignement
\usepackage{color,soulutf8}
\setulcolor{red}
% Package pour les annotations
\usepackage{todonotes}
%\usepackage[pygments]{pythontex} % Pour utiliser Python
%\usepackage{fvextra} % Utile pour la mise en forme du code source inséré
%Package pour la programmation
\usepackage{listings}
% Package pour les scripts en R
\lstset{language=R,
    basicstyle=\small\ttfamily,
    stringstyle=\color{DarkGreen},
    otherkeywords={0,1,2,3,4,5,6,7,8,9},
    morekeywords={TRUE,FALSE},
    deletekeywords={data,frame,length,as,character},
    keywordstyle=\color{blue},
    commentstyle=\color{DarkGreen},
    }

%Définition de couleurs:
\definecolor{bleudefrance}{rgb}{.19, .55, .91}
\definecolor{pakistangreen}{rgb}{.0, .4, .0}
\definecolor{rossocorsa}{rgb}{0.83, 0.0, 0.0}
\definecolor{persimmon}{rgb}{0.93, 0.35, 0.0}
\definecolor{bronze}{rgb}{0.8, 0.5, 0.2}
% Annotation auteur
\newcommand{\Moi}[1]{\todo[color = teal!40]{#1}}
\newcommand{\Cnsl}[1]{\todo[color = pakistangreen!40]{#1}}
\newcommand{\MeG}[1]{\todo[color = rossocorsa!40]{#1}}
%Notation récurrante:
\newcommandx{\hb}[1]{\widehat{\beta_{#1}}}
\newcommandx{\prth}[4]{\left( #1_{#2} \right)_{#3\leq #2 \leq #4}}
% Encadrement des résultats
\newcommand{\enc}[1]{\fcolorbox{rossocorsa}{white}{#1}}
\newcommand{\encB}[1]{\fcolorbox{bleudefrance}{white}{#1}}
\newcommand{\encV}[1]{\fcolorbox{pakistangreen}{white}{#1}}
\newcommand{\encN}[1]{\fcolorbox{black}{white}{#1}}
% Soulignement couleur
\newcommandx{\sN}[1]{\setulcolor{black}\ul{#1}}
\newcommandx{\sR}[1]{\setulcolor{rossocorsa}\ul{#1}}
\newcommandx{\sB}[1]{\setulcolor{bleudefrance}\ul{#1}}
\newcommandx{\sV}[1]{\setulcolor{pakistangreen}\ul{#1}}
\newcommandx{\sT}[1]{\setulcolor{teal}\ul{#1}}
\newcommandx{\sO}[1]{\setulcolor{persimmon}\ul{#1}}
\newcommandx{\sG}[1]{\setulcolor{tiger}\ul{#1}}
% Text de couleur
\newcommandx{\tR}[1]{\textcolor{rossocorsa}{#1}}
\newcommandx{\tB}[1]{\textcolor{bleudefrance}{#1}}
\newcommandx{\tV}[1]{\textcolor{pakistangreen}{#1}}
\newcommandx{\tT}[1]{\textcolor{teal}{#1}}
\newcommandx{\tO}[1]{\textcolor{persimmon}{#1}}
\newcommandx{\tBr}[1]{\textcolor{bronze}{#1}}
% Écriture intervalle
\newcommandx{\inter}[2]{\left[\![#1, #2]\!\right]}
% Écriture intégrale
\newcommandx{\Su}[2]{{\displaystyle \int_{#1}^{#2}}}
% Écriture somme sigma
\newcommandx{\su}[2]{{\displaystyle \sum_{#1}^{#2}}}
% Écriture produit pi
\newcommandx{\prd}[2]{{\displaystyle \prod_{#1}^{#2}}}
%Écriture limite lim
\newcommandx{\lm}[2]{{\displaystyle \lim_{#1\to #2}}}
% Écriture P(X)
\newcommandx{\prob}[1]{\mathbb{P}\left(#1\right)}
% Écriture P_{\{Y\}}({X})
\newcommandx{\probC}[2]{\mathbb{P}_{#1}\left(#2\right)}
% Écriture P({X})
\newcommandx{\Prob}[1]{\mathbb{P}\left(\left\{#1 \right\}\right)}
% Écriture P_{\{Y\}}({X})
\newcommandx{\ProbC}[2]{\mathbb{P}_{\left\{#1\right\}}\left(\left\{#2\right\}\right)}
% Ecriture Esperance et Variance
\newcommandx{\E}[1]{\mathbb{E}\left(#1\right)}
\newcommandx{\Ec}[2]{\mathbb{E}_{\left\{#1\right\}}\left(#2\right)}
\newcommandx{\V}[1]{\mathbb{V}\left(#1\right)}
\newcommandx{\Vc}[2]{\mathbb{V}_{\left\{#1\right\}}\left(#2\right)}
%Symbole de la norme
\newcommandx{\norm}[1]{\left\lVert#1\right\rVert}

\title{Summarise Course/Methods}
\author{Siger}
\IfFileExists{upquote.sty}{\usepackage{upquote}}{}
\begin{document}

\maketitle

Si Dieu est infini, alors je suis une partie de Dieu sinon je serai sa limite\ldots

\tableofcontents

\begin{abstract}
\begin{knitrout}
\definecolor{shadecolor}{rgb}{0.969, 0.969, 0.969}\color{fgcolor}\begin{kframe}
\begin{alltt}
\hlkwd{print}\hlstd{(}\hlkwd{paste}\hlstd{(}\hlstr{"Euclidean function creation"}\hlstd{))}
\end{alltt}
\begin{verbatim}
## [1] "Euclidean function creation"
\end{verbatim}
\begin{alltt}
\hlstd{EuclDist} \hlkwb{=} \hlkwa{function}\hlstd{(}\hlkwc{x}\hlstd{,} \hlkwc{y}\hlstd{)\{}
        \hlcom{#To acertain that we have 2 vectors of the same length.}
        \hlkwa{if} \hlstd{(}\hlkwd{length}\hlstd{(x)}\hlopt{==}\hlkwd{length}\hlstd{(y))\{}
                \hlcom{#s2 variable will be the sum of the squared difference of compoments}
                \hlstd{s2}\hlkwb{=}\hlnum{0}
                \hlkwa{for} \hlstd{(i} \hlkwa{in} \hlnum{1}\hlopt{:}\hlkwd{length}\hlstd{(x))\{}
                        \hlstd{s2}\hlkwb{=}\hlstd{s2}\hlopt{+}\hlstd{(x[i]}\hlopt{-}\hlstd{y[i])}\hlopt{^}\hlnum{2}
                \hlstd{\}}
                \hlcom{#s variable will be the euclidian distance}
                \hlstd{s}\hlkwb{=}\hlkwd{sqrt}\hlstd{(s2)}
                \hlkwd{return}\hlstd{(s)}
        \hlstd{\}}
        \hlkwa{else}\hlstd{\{}
                \hlkwd{print}\hlstd{(}\hlkwd{paste}\hlstd{(}\hlstr{"Given vectors have different length"}\hlstd{))}
        \hlstd{\}}
\hlstd{\}}

\hlkwd{print}\hlstd{(}\hlkwd{paste}\hlstd{(}\hlstr{"KNN function creation"}\hlstd{))}
\end{alltt}
\begin{verbatim}
## [1] "KNN function creation"
\end{verbatim}
\begin{alltt}
\hlstd{KNN} \hlkwb{=} \hlkwa{function}\hlstd{(}\hlkwc{x0}\hlstd{,} \hlkwc{M}\hlstd{,} \hlkwc{k}\hlstd{)\{}
        \hlkwa{if}\hlstd{(}\hlkwd{length}\hlstd{(x0)}\hlopt{==}\hlkwd{dim}\hlstd{(M)[}\hlnum{1}\hlstd{])\{}
                \hlcom{#l variable will contain all euclidean distance between x0 and}
                \hlcom{# vectors of cbind M}
                \hlstd{l}\hlkwb{=}\hlkwd{c}\hlstd{()}
                \hlkwa{for}\hlstd{(j} \hlkwa{in} \hlnum{1}\hlopt{:}\hlkwd{dim}\hlstd{(M)[}\hlnum{2}\hlstd{])\{}
                        \hlstd{l}\hlkwb{=}\hlkwd{c}\hlstd{(l,} \hlkwd{EuclDist}\hlstd{(x0,M[,j]))}
                \hlstd{\}}
                \hlcom{#names of M columns are the class}
                \hlkwd{names}\hlstd{(l)}\hlkwb{=}\hlkwd{colnames}\hlstd{(M)}
                \hlcom{#nearestN contains the neigborhood of x0}
                \hlstd{nearestN}\hlkwb{=}\hlkwd{sort}\hlstd{(l,} \hlkwc{decreasing}\hlstd{=}\hlnum{TRUE}\hlstd{)[}\hlnum{1}\hlopt{:}\hlstd{k]}
                \hlcom{#class contain the name (the class) wich is the most common name}
                \hlstd{class}\hlkwb{=}\hlkwd{sort}\hlstd{(}\hlkwd{table}\hlstd{(}\hlkwd{names}\hlstd{(nearestN)),}\hlkwc{decreasing}\hlstd{=}\hlnum{TRUE}\hlstd{)[}\hlnum{1}\hlstd{]}
                \hlkwd{return}\hlstd{(}\hlkwd{names}\hlstd{(class))}
        \hlstd{\}}
\hlstd{\}}

\hlstd{M}\hlkwb{=} \hlkwd{cbind}\hlstd{(}\hlkwd{c}\hlstd{(}\hlnum{0}\hlstd{,}\hlnum{3}\hlstd{,}\hlnum{0}\hlstd{),} \hlkwd{c}\hlstd{(}\hlnum{2}\hlstd{,}\hlnum{0}\hlstd{,}\hlnum{0}\hlstd{),} \hlkwd{c}\hlstd{(}\hlnum{0}\hlstd{,}\hlnum{1}\hlstd{,}\hlnum{3}\hlstd{),} \hlkwd{c}\hlstd{(}\hlnum{0}\hlstd{,}\hlnum{1}\hlstd{,}\hlnum{2}\hlstd{),} \hlkwd{c}\hlstd{(}\hlopt{-}\hlnum{1}\hlstd{,}\hlnum{0}\hlstd{,}\hlnum{1}\hlstd{),} \hlkwd{c}\hlstd{(}\hlnum{1}\hlstd{,}\hlnum{1}\hlstd{,}\hlnum{1}\hlstd{))}
\hlkwd{colnames}\hlstd{(M)}\hlkwb{=}\hlkwd{c}\hlstd{(}\hlstr{"Red"}\hlstd{,} \hlstr{"Red"}\hlstd{,} \hlstr{"Red"}\hlstd{,} \hlstr{"Green"}\hlstd{,} \hlstr{"Green"}\hlstd{,} \hlstr{"Red"}\hlstd{)}
\hlkwd{print}\hlstd{(}\hlkwd{paste}\hlstd{(}\hlstr{"The result for K=1"}\hlstd{,}\hlkwd{KNN}\hlstd{(}\hlkwd{c}\hlstd{(}\hlnum{0}\hlstd{,}\hlnum{0}\hlstd{,}\hlnum{0}\hlstd{), M,} \hlnum{1}\hlstd{)))}
\end{alltt}
\begin{verbatim}
## [1] "The result for K=1 Red"
\end{verbatim}
\begin{alltt}
\hlkwd{print}\hlstd{(}\hlkwd{paste}\hlstd{(}\hlstr{"The result for K=3"}\hlstd{,}\hlkwd{KNN}\hlstd{(}\hlkwd{c}\hlstd{(}\hlnum{0}\hlstd{,}\hlnum{0}\hlstd{,}\hlnum{0}\hlstd{), M,} \hlnum{3}\hlstd{)))}
\end{alltt}
\begin{verbatim}
## [1] "The result for K=3 Red"
\end{verbatim}
\end{kframe}
\end{knitrout}
\end{abstract}

\end{document}
